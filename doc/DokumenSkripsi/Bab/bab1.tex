%versi 2 (8-10-2016) 
\chapter{Pendahuluan}
\label{chap:pendahuluan}
   
\section{Latar Belakang}
\label{sec:latarbelakang}
Terdapat pengakuan yang berkembang secara global bahwa orang-orang disabilitas memiliki hak yang sama dengan orang lain untuk mengakses teknologi informasi. Pengakuan ini diwujudkan dengan berlakunya undang-undang di Amerika Serikat yang bertujuan untuk membuat web dan teknologi informasi lainnya dapat diakses oleh orang-orang disabilitas. Undang-undang semacam itu telah menyebabkan terciptanya suatu standar, pedoman, dan daftar centang untuk aksesibilitas. Tujuannya adalah untuk mengembangkan pemahaman umum tentang apa yang diperlukan untuk membuat halaman web menjadi mudah diakses oleh orang-orang disabilitas sehingga memungkinkan desainer dan pembuat web untuk memenuhi standar dan pedoman yang telah diciptakan. Untuk membantu para desainer dan pembuat web memenuhi standar dan pedoman ini tersedia alat validasi aksesibilitas web dan pemeriksa aksesibilitas situs web profesional \cite{WebAccessibility-ABroderView}.

BlueTape merupakan aplikasi berbasis web yang dibuat untuk memudahkan berbagai urusan administrasi di Fakultas Teknologi Informasi dan Sains Universitas Katolik Parahyangan \cite{BlueTape}. Konsep aplikasi ini yaitu membuat urusan-urusan administrasi dapat dikerjakan melalui situs web sehingga mengurangi penggunaan kertas. Aplikasi ini disediakan untuk digunakan oleh mahasiswa, staf tata usaha, dan dosen. Fitur-fitur yang tersedia pada BlueTape yaitu manajemen cetak transkrip dan manajemen perubahan jadwal kuliah.

\textit{Web Content Accessibility Guidelines (WCAG)} adalah panduan yang berisi rekomendasi-rekomendasi untuk membuat konten web lebih mudah diakses dan digunakan oleh orang-orang, termasuk mereka yang memiliki keterbatasan \cite{WCAG:2.1}. Keterbatasan yang tercakup dalam panduan ini yaitu keterbatasan visual, keterbatasan pendengaran, keterbatasan gerak, keterbatasan berbicara dan berbahasa, keterbatasan belajar, fotosensitif, keterbatasan kognitif, dan kombinasi dari beberapa keterbatasan yang telah disebutkan. Dalam \textit{WCAG} terdapat 3 tingkat kriteria sukses yaitu A, AA, dan AAA. Kriteria sukses adalah pernyataan-pernyataan yang dapat diuji yang dijadikan acuan untuk menilai tingkat kepatuhan sebuah situs web terhadap \textit{WCAG}. Kepatuhan tingkat A adalah tingkat kepatuhan terendah yang diperoleh jika seluruh kriteria sukses tingkat A terpenuhi atau versi alternatif yang sesuai tersedia. Kepatuhan tingkat AA adalah tingkat kepatuhan yang diperoleh jika seluruh kriteria sukses tingkat A dan tingkat AA terpenuhi atau versi alternatif tingkat AA yang sesuai tersedia. Kepatuhan tingkat AAA adalah tingkat kepatuhan tertinggi yang diperoleh jika seluruh kriteria sukses tingkat A, tingkat AA, dan tingkat AAA terpenuhi atau versi alternatif tingkat AAA yang sesuai tersedia.

Pada skripsi ini, akan dilihat sejauh mana tingkat kepatuhan situs web BlueTape terhadap \textit{WCAG} 2.1 dan rekomendasi apa saja yang perlu dilakukan untuk menaikkan tingkat kepatuhannya. Selain itu, akan dilakukan pengujian pada situs web tersebut dengan beberapa kondisi keterbatasan yang terdapat dalam \textit{WCAG} 2.1 seperti keterbatasan visual, keterbatasan gerak, keterbatasan pendengaran, dan keterbatasan bahasa.

\section{Rumusan Masalah}
\label{sec:rumusan}
Rumusan masalah yang akan dibahas dalam penelitian ini adalah: 
\begin{itemize}
	\item Bagaimana tingkat kepatuhan situs web BlueTape terhadap \textit{WCAG} 2.1?
	\item Bagaimana meningkatkan tingkat kepatuhan situs web BlueTape terhadap \textit{WCAG} 2.1?  
	\item Bagaimana pengalaman menggunakan situs web BlueTape yang telah diperbarui dengan berbagai kondisi keterbatasan seperti yang terdapat dalam \textit{WCAG} 2.1?
\end{itemize}

\section{Tujuan}
\label{sec:tujuan}
Tujuan yang ingin dicapai dalam penelitian ini adalah:
\begin{itemize}
	\item Mendapatkan tingkat kepatuhan situs web BlueTape terhadap \textit{WCAG} 2.1.
	\item Meningkatkan tingkat kepatuhan situs web BlueTape terhadap \textit{WCAG} 2.1.
	\item Mendapatkan pengalaman menggunakan situs web BlueTape yang tingkat kepatuhannya telah ditingkatkan dengan kondisi keterbatasan tertentu seperti yang terdapat dalam \textit{WCAG} 2.1.
\end{itemize}

\section{Batasan Masalah}
\label{sec:batasan}
Pada penelitian ini ditetapkan batasan-batasan masalah sebagai berikut:
\begin{enumerate}
	\item Perangkat yang digunakan hanya \textit{Personal Computer} seperti \textit{laptop}.
	\item Karena keterbatasan waktu, alat, dan sumber daya, kondisi keterbatasan yang diujikan hanya meliputi keterbatasan visual.
	\item Karena keterbatasan waktu, tingkat kepatuhan BlueTape hanya ditingkatkan menjadi tingkat A.
\end{enumerate}

\section{Metodologi}
\label{sec:metlit}
Metode penelitian yang digunakan dalam skripsi ini adalah:
\begin{enumerate}
	\item Menganalisis fitur-fitur yang ada pada BlueTape.
	\item Melakukan studi literatur mengenai \textit{WCAG} 2.1.
	\item Menganalisis tingkat kepatuhan situs web BlueTape terhadap \textit{WCAG} 2.1
	\item Memperbaiki situs web BlueTape sehingga tingkat kepatuhan terhadap \textit{WCAG} 2.1 meningkat.
	\item Melakukan pengujian dan eksperimen pada situs web BlueTape yang telah diperbarui dengan kondisi keterbatasan visual.
	\item Menulis dokumen hasil pengujian.
\end{enumerate}

\section{Sistematika Pembahasan}
\label{sec:sispem}
Skrpsi ini akan memiliki sistematika pembahasan sebagai berikut: 
\begin{enumerate}
	\item Bab 1: Pendahuluan, akan membahas gambaran umum dari skripsi ini. Bab ini berisi latar belakang, rumusan masalah, tujuan, batasan masalah, metode penelitian, dan sistematika pembahasan.
	\item Bab 2: Landasan Teori, akan membahas dasar teori yang menjadi acuan dalam pembuatan skripsi ini. Dasar teori yang digunakan yaitu \textit{WCAG} 2.1 dan BlueTape.
	\item Bab 3: Analisis, akan membahas hasil analisis mengenai tingkat kepatuhan situs web BlueTape terhadap \textit{WCAG} 2.1.
	\item Bab 4: Perancangan, akan membahas mengenai perubahan-perubahan yang dapat dilakukan untuk meningkatkan kepatuhan situs web BlueTape terhadap \textit{WCAG} 2.1.
	\item Bab 5: Implementasi dan Pengujian, akan membahas hasil implementasi dan pengujian yang telah dilakukan pada situs web BlueTape.
	\item Bab 6: Kesimpulan dan saran, akan berisi kesimpulan dari hasil penelitian yang telah dilakukan dan saran yang dapat diberikan untuk penelitian berikutnya.
\end{enumerate}