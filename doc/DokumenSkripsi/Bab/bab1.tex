%versi 2 (8-10-2016) 
\chapter{Pendahuluan}
\label{chap:intro}
   
\section{Latar Belakang}
\label{sec:latarbelakang}
BlueTape merupakan aplikasi berbasis web yang dibuat untuk memudahkan berbagai urusan administrasi di Fakultas Teknologi Informasi dan Sains Universitas Katolik Parahyangan. Konsep aplikasi ini yaitu membuat urusan-urusan administrasi dapat dikerjakan melalui situs web sehingga mengurangi penggunaan kertas. Aplikasi ini disediakan untuk digunakan oleh mahasiswa, staf tata usaha, dan dosen. Fitur-fitur yang tersedia pada BlueTape yaitu manajemen cetak transkrip dan manajemen perubahan jadwal kuliah.

Aplikasi BlueTape yang berbasis web mungkin saja digunakan oleh orang-orang yang memiliki keterbatasan. Oleh karena itu perlu dipastikan bahwa aplikasi ini dapat digunakan dengan mudah oleh setiap subjek yang mengaksesnya. W3C (World Wide Web Consortium) memiliki rekomendasi bernama \textit{WCAG} 2.1 (\textit{Web Content Accessibility Guidelines}) yang merupakan panduan untuk membuat konten web yang dapat diakses dengan mudah oleh semua orang termasuk mereka yang memiliki keterbatasan.

Pada skripsi ini, akan dilihat sejauh mana tingkat kepatuhan situs web BlueTape terhadap \textit{WCAG} 2.1 dan rekomendasi apa saja yang perlu dilakukan untuk menaikkan tingkat kepatuhannya. Selain itu, akan dilakukan pengujian pada situs web tersebut dengan beberapa kondisi keterbatasan yang terdapat dalam \textit{WCAG} 2.1 seperti keterbatasan visual, keterbatasan gerak, keterbatasan pendengaran, dan keterbatasan bahasa.

\section{Rumusan Masalah}
\label{sec:rumusan}
Rumusan masalah yang akan dibahas dalam penelitian ini adalah: 
\begin{itemize}
	\item Bagaimana tingkat kepatuhan situs web BlueTape terhadap \textit{WCAG} 2.1?
	\item Rekomendasi apa saja (dalam bentuk perubahan kode) yang perlu dilakukan terhadap situs web BlueTape untuk menaikkan tingkat kepatuhannya?  
	\item Bagaimana pengalaman menggunakan situs web BlueTape dengan berbagai kondisi keterbatasan seperti yang terdapat dalam \textit{WCAG} 2.1?
\end{itemize}

\section{Tujuan}
\label{sec:tujuan}
Tujuan yang ingin dicapai dalam penelitian ini adalah:
\begin{itemize}
	\item Mendapatkan tingkat kepatuhan situs web Bluetape terhadap \textit{WCAG} 2.1.
	\item Meningkatkan level kepatuhan situs web BlueTape terhadap \textit{WCAG} 2.1.
	\item Mendapatkan pengalaman menggunakan situs web BlueTape dengan berbagai kondisi keterbatasan seperti yang terdapat dalam \textit{WCAG} 2.1.
\end{itemize}

\section{Batasan Masalah}
\label{sec:batasan}
Pada penelitian ini ditetapkan batasan-batasan masalah sebagai berikut:
\begin{enumerate}
	\item Pengujian aplikasi hanya dilakukan pada perangkat laptop.
	\item Kondisi keterbatasan yang diujikan hanya meliputi keterbatasan visual, keterbatasan gerak, keterbatasan pendengaran, dan keterbatasan bahasa.
	\item Kondisi keterbatasan yang diujikan tidak meliputi kombinasi dari beberapa keterbatasan.
\end{enumerate}

\section{Metodologi}
\label{sec:metlit}
Metode penelitian yang digunakan dalam skripsi ini adalah:
\begin{enumerate}
	\item Menganalisis situs web BlueTape saat ini. 
	\item Melakukan studi literatur mengenai \textit{WCAG} 2.1.
	\item Menganalisis tingkat kepatuhan situs web BlueTape terhadap \textit{WCAG} 2.1
	\item Memodifikasi situs web BlueTape sehingga level kepatuhan terhadap \textit{WCAG} 2.1 meningkat.
	\item Melakukan pengujian dan eksperimen pada situs web BlueTape yang sudah dimodifikasi dengan kondisi keterbatasan visual, keterbatasan gerak, keterbatasan pendengaran, dan keterbatasan bahasa.
	\item Menulis dokumentasi hasil pengujian.
\end{enumerate}

\section{Sistematika Pembahasan}
\label{sec:sispem}
Skrpsi ini akan memiliki sistematika pembahasan sebagai berikut: 
\begin{enumerate}
	\item Bab 1: Pendahuluan, akan membahas gambaran umum dari skripsi ini. Bab ini berisi latar belakang, rumusan masalah, tujuan, batasan masalah, metode penelitian, dan sistematika pembahasan.
	\item Bab 2: Landasan Teori, akan membahas dasar teori yang menjadi acuan dalam pembuatan skripsi ini. Dasar teori yang digunakan yaitu \textit{WCAG} 2.1.
	\item Bab 3: Analisis, akan membahas hasil analisis mengenai tingkat kepatuhan situs web BlueTape terhadap \textit{WCAG} 2.1.
	\item Bab 4: Perancangan, akan membahas mengenai perubahan-perubahan yang dapat dilakukan untuk meningkatkan kepatuhan situs web BlueTape terhadap \textit{WCAG} 2.1.
	\item Bab 5: Implementasi dan Pengujian, akan membahas hasil implementasi dan pengujian yang telah dilakukan pada situs web BlueTape.
	\item Bab 6: Kesimpulan dan saran, akan berisi kesimpulan dari hasil penelitian yang telah dilakukan dan saran yang dapat diberikan untuk penelitian berikutnya.
\end{enumerate}