\chapter{Implementasi dan Pengujian}
\label{chap:implementasi_dan_pengujian}

\section{Implementasi}
\label{sec:implementasi}
Pada subbab ini dibahas perbaikan apa saja yang dilakukan untuk membuat kriteria sukses pada subbab \ref{sec:kepatuhan_bluetape_terhadap_wcag_2.1}
yang belum sukses dipatuhi menjadi sukses dipatuhi. Setiap perubahan kode yang terdapat pada subbab ini ditampilkan dalam format \textit{diff} dan \textit{indentation} pada kode dihilangkan karena keterbatasan tempat. Tampilan dalam format \textit{diff} akan menunjukkan perbedaan yang terdapat pada kode sebelum dan sesudah dilakukan perubahan. Untuk bagian yang dihilangkan akan ditampilkan dengan warna merah dan untuk bagian yang ditambahkan akan diberi warna hijau.

\subsection{Perbaikan Kriteria Sukses 1.1.1 \textit{Non-text Content}}
\label{subsec:perbaikan_kriteria_sukses_1.1.1}
Pada bagian ini dilakukan perbaikan sebagai berikut:

Pada konten berupa ikon yang tidak memiliki nama di halaman cetak transkrip, manajemen cetak transkrip, perubahan kuliah, dan manajemen perubahan kuliah diberikan atribut \textit{aria-label} yang diisi dengan nilai yang sesuai dengan fungsi setiap konten ikon. Perbaikan yang dilakukan dapat dilihat pada \textit{Listing} \ref{lst:perbaikan_1.1.1_ikon_tanpa_nama}.
\begin{lstlisting}[frame=single, label={lst:perbaikan_1.1.1_ikon_tanpa_nama}, language=diff, caption=Perbaikan Kriteria Sukses 1.1.1]
diff --git a/www/application/views/PerubahanKuliahManage/main.php b/www/application/views/PerubahanKuliahManage/main.php
index 657260b..43c215f 100644
--- a/www/application/views/PerubahanKuliahManage/main.php
+++ b/www/application/views/PerubahanKuliahManage/main.php
@@ -30,11 +30,11 @@ defined('BASEPATH') OR exit('No direct script access allowed');
    <td><?= $request->mataKuliahCode ?></td>
    <td><?= PerubahanKuliah_model::CHANGETYPE_TYPES[$request->changeType] ?></td>
    <td>
_   <a data-open="detail<?= $request->id ?>"><i class="fi-eye"></i></a>
_   <a target="_blank" href="/PerubahanKuliahManage/printview/<?= $request->id ?>"><i class="fi-print"></i></a>
_   <a data-open="konfirmasi<?= $request->id ?>"><i class="fi-like"></i></a>                                    
_   <a data-open="tolak<?= $request->id ?>"><i class="fi-dislike"></i></a>
_   <a data-open="hapus<?= $request->id ?>"><i class="fi-trash"></i></a>
+   <a aria-label="lihat detail permohonan <?= $request->requestByName ?>" data-open="detail<?= $request->id ?>"><i class="fi-eye"></i></a>
+   <a aria-label="cetak permohonan <?= $request->requestByName ?>" target="_blank" href="/PerubahanKuliahManage/printview/<?= $request->id ?>"><i class="fi-print"></i></a>
+   <a aria-label="konfirmasi permohonan <?= $request->requestByName ?>" data-open="konfirmasi<?= $request->id ?>"><i class="fi-like"></i></a>
+   <a aria-label="tolak permohonan <?= $request->requestByName ?>" data-open="tolak<?= $request->id ?>"><i class="fi-dislike"></i></a>
+   <a aria-label="hapus permohonan <?= $request->requestByName ?>" data-open="hapus<?= $request->id ?>"><i class="fi-trash"></i></a>
    </td>
    </tr>
    <?php endforeach; ?>
diff --git a/www/application/views/PerubahanKuliahRequest/main.php b/www/application/views/PerubahanKuliahRequest/main.php
index 72457a7..cd47c17 100644
--- a/www/application/views/PerubahanKuliahRequest/main.php
+++ b/www/application/views/PerubahanKuliahRequest/main.php
@@ -118,7 +118,7 @@ defined('BASEPATH') OR exit('No direct script access allowed');
    <td><time datetime="<?= $request->answeredDateTime ?>"><?= $request->answeredDateString ?></time></td>
    <td><?= $request->answeredMessage ?></td>
    <td>
_   <a data-open="detail<?= $request->id ?>"><i class="fi-eye"></i></a>
+   <a aria-label="lihat detail permohonan <?= $request->requestByName ?>" data-open="detail<?= $request->id ?>"><i class="fi-eye"></i></a>
    </td>
    </tr>
    <?php endforeach; ?>
diff --git a/www/application/views/TranskripManage/main.php b/www/application/views/TranskripManage/main.php
index 0970fc2..9f24954 100644
--- a/www/application/views/TranskripManage/main.php
+++ b/www/application/views/TranskripManage/main.php
@@ -85,7 +85,7 @@ defined('BASEPATH') OR exit('No direct script access allowed');
    <span aria-hidden="true">&times;</span>
    </button>                                        
    </div>
_   <a data-open="detail<?= $request->id ?>"><i class="fi-eye"></i></a>
+   <a aria-label="lihat detail permohonan <?= $request->requestByName ?>" data-open="detail<?= $request->id ?>"><i class="fi-eye"></i></a>
    <div class="reveal" id="tolak<?= $request->id ?>" data-reveal>
    <h5>Tolak Permohonan</h5>
    <form method="POST" action="/TranskripManage/answer">
@@ -105,7 +105,7 @@ defined('BASEPATH') OR exit('No direct script access allowed');
    <span aria-hidden="true">&times;</span>
    </button>
    </div>
_   <a data-open="tolak<?= $request->id ?>"><i class="fi-dislike"></i></a>
+   <a aria-label="tolak permohonan <?= $request->requestByName ?>" data-open="tolak<?= $request->id ?>"><i class="fi-dislike"></i></a>
    <div class="reveal" id="cetak<?= $request->id ?>" data-reveal>
    <h5>Cetak Permohonan</h5>
    <?php if ($request->requestByNPM !== NULL): ?>
@@ -131,7 +131,7 @@ defined('BASEPATH') OR exit('No direct script access allowed');
    <span aria-hidden="true">&times;</span>
    </button>
    </div>
_   <a data-open="cetak<?= $request->id ?>"><i class="fi-print"></i></a>
+   <a aria-label="cetak permohonan <?= $request->requestByName ?>" data-open="cetak<?= $request->id ?>"><i class="fi-print"></i></a>
    <div class="reveal" id="hapus<?= $request->id ?>" data-reveal>
    <h5>Hapus Permohonan</h5>
    <form method="POST" action="/TranskripManage/remove">
@@ -146,7 +146,7 @@ defined('BASEPATH') OR exit('No direct script access allowed');
    <span aria-hidden="true">&times;</span>
    </button>
    </div>
_   <a data-open="hapus<?= $request->id ?>"><i class="fi-trash"></i></a>
+   <a aria-label="hapus permohonan <?= $request->requestByName ?>" data-open="hapus<?= $request->id ?>"><i class="fi-trash"></i></a>
    </td>
    </tr>
    <?php endforeach; ?>
diff --git a/www/application/views/TranskripRequest/main.php b/www/application/views/TranskripRequest/main.php
index ada0254..44598cc 100644
--- a/www/application/views/TranskripRequest/main.php
+++ b/www/application/views/TranskripRequest/main.php
@@ -126,7 +126,7 @@ defined('BASEPATH') OR exit('No direct script access allowed');
    <span aria-hidden="true">&times;</span>
    </button>
    </div>
_   <a data-open="detail<?= $request->id ?>"><i class="fi-eye"></i></a>
+   <a aria-label="lihat detail permohonan <?= $request->requestByName ?>" data-open="detail<?= $request->id ?>"><i class="fi-eye"></i></a>
    </td>
    </tr>
    <?php endforeach; ?>
\end{lstlisting}

\subsection{Perbaikan Kriteria Sukses 1.3.1 \textit{Info and Relationships}}
\label{subsec:perbaikan_kriteria_sukses_1.3.1}
Pada bagian ini dilakukan perbaikan sebagai berikut:

\begin{itemize}
\item Pada penggunaan \textit{tag heading} yang tidak tepat, judul yang menggunakan \texttt{<h5>} diubah menggunakan \textit{tag heading} yang sesuai struktur \textit{HTML} secara berurutan dimulai dari \texttt{<h1>} hingga \texttt{<h6>}. Perbaikan yang dilakukan dapat dilihat pada \textit{Listing} \ref{lst:perbaikan_1.3.1_tag_heading}.
\begin{lstlisting}[frame=single, label={lst:perbaikan_1.3.1_tag_heading}, language=diff, caption=Perbaikan Kriteria Sukses 1.3.1 pada Bagian \textit{Heading}]
diff --git a/www/application/views/EntriJadwalDosen/main.php b/www/application/views/EntriJadwalDosen/main.php
index 6392b95..7af8ded 100644
--- a/www/application/views/EntriJadwalDosen/main.php
+++ b/www/application/views/EntriJadwalDosen/main.php
@@ -12,7 +12,7 @@ defined('BASEPATH') OR exit('No direct script access allowed');
    <div class="row">

    <div class="large-12 column callout">
_   <h5>Tambah Jadwal</h5>
+   <h1>Tambah Jadwal</h1>
    <div class="large-4 columns">
    <form method="POST" action="/EntriJadwalDosen/add">
    <input type="hidden" name="<?= $this->security->get_csrf_token_name() ?>" value="<?= $this->security->get_csrf_hash() ?>" />
@@ -59,7 +59,7 @@ defined('BASEPATH') OR exit('No direct script access allowed');
    <!-- ===================================================================== Pembentukan Tabel ============================================================================= -->

    <div class="large-12 column callout">
_   <h5>Daftar Jadwal</h5>
+   <h1>Daftar Jadwal</h1>
    <div class="table-scroll" id="jadwal_table">
    <table border=1 style="border-color:black ; border-collapse:separate">
    <tr> 
@@ -169,7 +169,7 @@ defined('BASEPATH') OR exit('No direct script access allowed');
    <button class="close-button" data-close aria-label="Close modal" type="button">
    <span aria-hidden="true">&times;</span>
    </button>
_   <h5> Edit Jadwal </h5>
+   <h2> Edit Jadwal </h2>
    <form name="form<?php echo $dataHariIni->id ?>" method="POST" action="/EntriJadwalDosen/update/<?php echo $dataHariIni->id ?>">
    <input type="hidden" name="<?= $this->security->get_csrf_token_name() ?>" value="<?= $this->security->get_csrf_hash() ?>" />
    <input type="hidden" name="id_jadwal_parameter" value="<?php echo $dataHariIni->id ?>"> </a> <br>
diff --git a/www/application/views/PerubahanKuliahManage/main.php b/www/application/views/PerubahanKuliahManage/main.php
index 657260b..428510f 100644
--- a/www/application/views/PerubahanKuliahManage/main.php
+++ b/www/application/views/PerubahanKuliahManage/main.php
@@ -9,7 +9,7 @@ defined('BASEPATH') OR exit('No direct script access allowed');

    <div class="row">
    <div class="callout">
_   <h5>Permohonan Perubahan Kuliah</h5>
+   <h1>Permohonan Perubahan Kuliah</h1>
    <table class="stack">
    <thead>
    <tr>
@@ -56,7 +56,7 @@ defined('BASEPATH') OR exit('No direct script access allowed');
    <?php foreach ($requests as $request): ?>

    <div class="reveal" id="detail<?= $request->id ?>" data-reveal>
_   <h5>Detail Permohonan #<?= $request->id ?></h5>
+   <h2>Detail Permohonan #<?= $request->id ?></h2>
    <table class="stack">
    <tbody>
    <tr>
@@ -128,7 +128,7 @@ defined('BASEPATH') OR exit('No direct script access allowed');
    </button>
    </div>
    <div class="reveal" id="konfirmasi<?= $request->id ?>" data-reveal>
_   <h5>Konfirmasi Permohonan</h5>
+   <h2>Konfirmasi Permohonan</h2>
    <form method="POST" action="/PerubahanKuliahManage/answer">
    <input type="hidden" name="<?= $this->security->get_csrf_token_name() ?>" value="<?= $this->security->get_csrf_hash() ?>" />
    <input type="hidden" name="id" value="<?= $request->id ?>"/>
@@ -147,7 +147,7 @@ defined('BASEPATH') OR exit('No direct script access allowed');
    </button>
    </div>        
    <div class="reveal" id="tolak<?= $request->id ?>" data-reveal>
_   <h5>Tolak Permohonan</h5>
+   <h2>Tolak Permohonan</h2>
    <form method="POST" action="/PerubahanKuliahManage/answer">
    <input type="hidden" name="<?= $this->security->get_csrf_token_name() ?>" value="<?= $this->security->get_csrf_hash() ?>" />
    <input type="hidden" name="id" value="<?= $request->id ?>"/>
@@ -166,7 +166,7 @@ defined('BASEPATH') OR exit('No direct script access allowed');
    </button>
    </div>
    <div class="reveal" id="hapus<?= $request->id ?>" data-reveal>
_   <h5>Hapus Permohonan</h5>
+   <h2>Hapus Permohonan</h2>
    <form method="POST" action="/PerubahanKuliahManage/remove">
    <input type="hidden" name="<?= $this->security->get_csrf_token_name() ?>" value="<?= $this->security->get_csrf_hash() ?>" />
    <input type="hidden" name="id" value="<?= $request->id ?>"/>
diff --git a/www/application/views/PerubahanKuliahRequest/main.php b/www/application/views/PerubahanKuliahRequest/main.php
index 72457a7..b6b0b94 100644
--- a/www/application/views/PerubahanKuliahRequest/main.php
+++ b/www/application/views/PerubahanKuliahRequest/main.php
@@ -10,7 +10,7 @@ defined('BASEPATH') OR exit('No direct script access allowed');
    <div class="row">
    <div class="large-12 column">
    <div class="callout">
_   <h5>Permohonan Baru</h5>
+   <h1>Permohonan Baru</h1>
    <form method="POST" action="/PerubahanKuliahRequest/add">
    <input type="hidden" name="<?= $this->security->get_csrf_token_name() ?>" value="<?= $this->security->get_csrf_hash() ?>" />
    <div class="row">
@@ -93,7 +93,7 @@ defined('BASEPATH') OR exit('No direct script access allowed');
    </form>
    </div>
    <div class="callout">
_   <h5>Histori Permohonan</h5>
+   <h1>Histori Permohonan</h1>
    <table class="stack">
    <thead>
    <tr>
@@ -130,7 +130,7 @@ defined('BASEPATH') OR exit('No direct script access allowed');
    <?php foreach ($requests as $request): ?>

    <div class="reveal" id="detail<?= $request->id ?>" data-reveal>
_   <h5>Detail Permohonan #<?= $request->id ?></h5>
+   <h2>Detail Permohonan #<?= $request->id ?></h2>
    <table class="stack">
    <tbody>
    <tr>
diff --git a/www/application/views/TranskripManage/main.php b/www/application/views/TranskripManage/main.php
index 0970fc2..72ff740 100644
--- a/www/application/views/TranskripManage/main.php
+++ b/www/application/views/TranskripManage/main.php
@@ -9,7 +9,7 @@ defined('BASEPATH') OR exit('No direct script access allowed');

    <div class="row">
    <div class="callout">
_   <h5>Permintaan Transkrip</h5>
+   <h1>Permintaan Transkrip</h1>
    <form method="GET" action="/TranskripManage">
    <div class="input-group">
    <span class="input-group-label">Cari NPM:</span>
@@ -40,7 +40,7 @@ defined('BASEPATH') OR exit('No direct script access allowed');
    <td><?= isset($request->requestByNPM) ? $request->requestByNPM : '-' ?></td>
    <td>
    <div class="reveal" id="detail<?= $request->id ?>" data-reveal>
_   <h5>Detail Permohonan #<?= $request->id ?></h5>
+   <h2>Detail Permohonan #<?= $request->id ?></h2>
    <table class="stack">
    <tbody>
    <tr>
@@ -87,7 +87,7 @@ defined('BASEPATH') OR exit('No direct script access allowed');
    </div>
    <a data-open="detail<?= $request->id ?>"><i class="fi-eye"></i></a>
    <div class="reveal" id="tolak<?= $request->id ?>" data-reveal>
_   <h5>Tolak Permohonan</h5>
+   <h2>Tolak Permohonan</h2>
    <form method="POST" action="/TranskripManage/answer">
    <input type="hidden" name="<?= $this->security->get_csrf_token_name() ?>" value="<?= $this->security->get_csrf_hash() ?>" />
    <input type="hidden" name="id" value="<?= $request->id ?>"/>
@@ -107,7 +107,7 @@ defined('BASEPATH') OR exit('No direct script access allowed');
    </div>
    <a data-open="tolak<?= $request->id ?>"><i class="fi-dislike"></i></a>
    <div class="reveal" id="cetak<?= $request->id ?>" data-reveal>
_   <h5>Cetak Permohonan</h5>
+   <h2>Cetak Permohonan</h2>
    <?php if ($request->requestByNPM !== NULL): ?>
    <a target="_blank" href="<?= sprintf($transkripURLs[$request->requestType], $request->requestByNPM) ?>">Klik untuk membuka DPS/LHS</a>
    <?php else: ?>
@@ -133,7 +133,7 @@ defined('BASEPATH') OR exit('No direct script access allowed');
    </div>
    <a data-open="cetak<?= $request->id ?>"><i class="fi-print"></i></a>
    <div class="reveal" id="hapus<?= $request->id ?>" data-reveal>
_   <h5>Hapus Permohonan</h5>
+   <h2>Hapus Permohonan</h2>
    <form method="POST" action="/TranskripManage/remove">
    <input type="hidden" name="<?= $this->security->get_csrf_token_name() ?>" value="<?= $this->security->get_csrf_hash() ?>" />
    <input type="hidden" name="id" value="<?= $request->id ?>"/>
diff --git a/www/application/views/TranskripRequest/main.php b/www/application/views/TranskripRequest/main.php
index ada0254..e1a87f0 100644
--- a/www/application/views/TranskripRequest/main.php
+++ b/www/application/views/TranskripRequest/main.php
@@ -10,7 +10,7 @@ defined('BASEPATH') OR exit('No direct script access allowed');
    <div class="row">
    <div class="medium-12 column">
    <div class="callout">
_   <h5>Permohonan Baru</h5>
+   <h1>Permohonan Baru</h1>
    <?php if (is_array($forbiddenTypes)): ?>
    <form method="POST" action="/TranskripRequest/add">
    <input type="hidden" name="<?= $this->security->get_csrf_token_name() ?>" value="<?= $this->security->get_csrf_hash() ?>" />
@@ -57,7 +57,7 @@ defined('BASEPATH') OR exit('No direct script access allowed');
    <?php endif ?>
    </div>
    <div class="callout">
_   <h5>Histori Permohonan</h5>
+   <h1>Histori Permohonan</h1>
    <table class="stack">
    <thead>
    <tr>
@@ -81,7 +81,7 @@ defined('BASEPATH') OR exit('No direct script access allowed');
    <td><?= $request->answeredMessage ?></td>
    <td>
    <div class="reveal" id="detail<?= $request->id ?>" data-reveal>
_   <h5>Detail Permohonan #<?= $request->id ?></h5>
+   <h2>Detail Permohonan #<?= $request->id ?></h2>
    <table class="stack">
    <tbody>
    <tr>
\end{lstlisting}

\item Pada halaman manajemen cetak transkrip, kolom \textit{input} NPM diberi atribut \textit{id} yang diisi dengan nilai "npm" lalu label "Cari NPM" diubah dari \texttt{<span>} menjadi \texttt{<label>} dan diberi atribut \textit{for} dengan nilai "npm". Perbaikan yang dilakukan dapat dilihat pada \textit{Listing} \ref{lst:perbaikan_1.3.1_label_masukan_manajemen_cetak_transkrip}.
\begin{lstlisting}[frame=single, label={lst:perbaikan_1.3.1_label_masukan_manajemen_cetak_transkrip}, language=diff, caption=Perbaikan Kriteria Sukses 1.3.1 pada Kolom \textit{Input} di Halaman Manajemen Cetak Transkrip]
diff --git a/www/application/views/TranskripManage/main.php b/www/application/views/TranskripManage/main.php
index 0970fc2..6dffa7d 100644
--- a/www/application/views/TranskripManage/main.php
+++ b/www/application/views/TranskripManage/main.php
@@ -12,8 +12,8 @@ defined('BASEPATH') OR exit('No direct script access allowed');
    <h5>Permintaan Transkrip</h5>
    <form method="GET" action="/TranskripManage">
    <div class="input-group">
_   <span class="input-group-label">Cari NPM:</span>
_   <input name="npm" class="input-group-field" type="text" placeholder="2013730013" maxlength="10" minlength="10"<?= $npmQuery === NULL ? '' : " value='$npmQuery'" ?>/>
+   <label for="npm" class="input-group-label">Cari NPM:</label>
+   <input id="npm" name="npm" class="input-group-field" type="text" placeholder="2013730013" maxlength="10" minlength="10"<?= $npmQuery === NULL ? '' : " value='$npmQuery'" ?>/>
<div class="input-group-button">
<input class="button" type="submit" value="Cari"/>
</div>
\end{lstlisting}

\item Pada halaman entri jadwal dosen, setiap kolom \textit{input} diberi atribut \textit{id} dengan nilai yang sesuai dengan isi kolom \textit{input} lalu setiap judul kolom dibuat sebagai label dengan menggunakan \texttt{<label>} dan dihubungkan dengan kolom \textit{input} yang bersangkutan dengan menggunakan atribut \textit{for}. Perbaikan yang dilakukan dapat dilihat pada \textit{Listing} \ref{lst:perbaikan_1.3.1_label_masukan_entri_jadwal_dosen}.
\begin{lstlisting}[frame=single, label={lst:perbaikan_1.3.1_label_masukan_entri_jadwal_dosen}, language=diff, caption=Perbaikan Kriteria Sukses 1.3.1 pada Kolom \textit{Input} di Halaman Entri Jadwal Dosen]
diff --git a/www/application/views/EntriJadwalDosen/main.php b/www/application/views/EntriJadwalDosen/main.php
index 6392b95..f8471cb 100644
--- a/www/application/views/EntriJadwalDosen/main.php
+++ b/www/application/views/EntriJadwalDosen/main.php
@@ -16,8 +16,8 @@ defined('BASEPATH') OR exit('No direct script access allowed');
    <div class="large-4 columns">
    <form method="POST" action="/EntriJadwalDosen/add">
    <input type="hidden" name="<?= $this->security->get_csrf_token_name() ?>" value="<?= $this->security->get_csrf_hash() ?>" />
_   Hari
_   <select name="hari"> 
+   <label for="hari">Hari</label>
+   <select id="hari" name="hari">
    <?php
    $hariValue = 0;
    foreach ($namaHari as $hari) {
@@ -29,30 +29,30 @@ defined('BASEPATH') OR exit('No direct script access allowed');
    }
    ?>
    </select><br>
_   Jam Mulai
_   <select name="jam_mulai"> 
+   <label for="jam_mulai">Jam Mulai</label>
+   <select id="jam_mulai" name="jam_mulai"> 
    <?php for ($i = 7; $i <= 16; $i++) { ?>
    <option value="<?php echo $i ?>"> <?php echo $i ?>:00 </option>
    <?php } ?>
    </select><br>
    </div>
    <div class=" large-4 columns">
_   Durasi
_   <select name="durasi"> 
+   <label for="durasi">Durasi</label>
+   <select id="durasi" name="durasi"> 
    <?php for ($i = 1; $i <= 9; $i++) { ?>
    <option value="<?php echo $i ?>"> <?php echo $i ?> jam </option>
    <?php } ?>
    </select><br>
_   Jenis  
_   <select name="jenis_jadwal"> 
+   <label for="jenis_jadwal">Jenis</label>
+   <select id="jenis_jadwal" name="jenis_jadwal"> 
    <option value="konsultasi" style="background-color:yellow"> Konsultasi </option>
    <option value="terjadwal" style="background-color:green;color:white"> Terjadwal</option>
    <option value="kelas" style="background-color:white"> Kelas </option>
    </select>
    </div>
    <div class="large-4 columns">
_   Label <input type="text" name="label_jadwal"><br>
+   <label for="label_jadwal">Label</label>
+   <input id="label_jadwal" type="text" name="label_jadwal"><br>
    <input type="submit" class="button" value="Tambah">
    </form>
    </div>
    </div>
@@ -173,8 +174,8 @@ defined('BASEPATH') OR exit('No direct script access allowed');
    <form name="form<?php echo $dataHariIni->id ?>" method="POST" action="/EntriJadwalDosen/update/<?php echo $dataHariIni->id ?>">
    <input type="hidden" name="<?= $this->security->get_csrf_token_name() ?>" value="<?= $this->security->get_csrf_hash() ?>" />
    <input type="hidden" name="id_jadwal_parameter" value="<?php echo $dataHariIni->id ?>"> </a> <br>
_   Hari 
_   <select name="hari"> 
+   <label for="edit_hari">Hari</label>
+   <select id="edit_hari" name="hari"> 
    <?php
    $hariValue = 0;
    foreach ($namaHari as $hari) {
@@ -185,14 +186,14 @@ defined('BASEPATH') OR exit('No direct script access allowed');
    }
    ?>
    </select><br>
_   Jam Mulai
_   <select name="jam_mulai"> 
+   <label for="edit_jam_mulai">Jam Mulai</label>
+   <select id="edit_jam_mulai" name="jam_mulai"> 
    <?php
    for ($i = 7; $i <= 16; $i++) {
    if ($i == $dataHariIni->jam_mulai) {
@@ -207,8 +208,8 @@ defined('BASEPATH') OR exit('No direct script access allowed');
    }
    ?>
    </select><br>
_   Durasi
_   <select name="durasi"> 
+   <label for="edit_durasi">Durasi</label>
+   <select id="edit_durasi" name="durasi"> 
    <?php
    for ($i = 1; $i <= 9; $i++) {
    if ($i == $dataHariIni->durasi) {
@@ -223,13 +224,14 @@ defined('BASEPATH') OR exit('No direct script access allowed');
    }
    ?>
    </select><br>
_   Jenis  
_   <select name="jenis_jadwal"> 
+   <label for="edit_jenis_jadwal">Jenis</label>
+   <select id="edit_jenis_jadwal" name="jenis_jadwal"> 
    <option style="background-color:yellow" value="konsultasi" <?php if ($dataHariIni->jenis == 'konsultasi') echo "selected"; ?> > Konsultasi </option>
    <option style="background-color:green" value="terjadwal" <?php if ($dataHariIni->jenis == 'terjadwal') echo "selected"; ?>> Terjadwal</option>
    <option style="background-color:white" value="kelas" <?php if ($dataHariIni->jenis == 'kelas') echo "selected"; ?>> Kelas </option>
    </select>
_   Label <input type="text" name="label_jadwal" value="<?php echo $dataHariIni->label; ?>"><br> 
+   <label for="edit_label_jadwal">Label</label>
+   <input type="text" id="edit_label_jadwal" name="label_jadwal" value="<?php echo $dataHariIni->label; ?>"><br> 
    <div class="row large-4 column">
    <div class="large-2 column">
    <input type="submit" name="submitId<?php echo $dataHariIni->id ?>" class="button" value="Save  ">
\end{lstlisting}
\end{itemize}

\subsection{Perbaikan Kriteria Sukses 2.1.1 \textit{Keyboard}}
\label{subsec:perbaikan_kriteria_sukses_2.1.1}
Pada bagian ini dilakukan perbaikan sebagai berikut:

\begin{itemize}
\item Pada bagian navigasi menu, atribut \textit{data-responsive-menu="dropdown"} dihilangkan dari elemen \textit{ul} karena isi dari elemen tersebut bukanlah sebuah menu \textit{dropdown}. Perbaikan yang dilakukan dapat dilihat pada \textit{Listing} \ref{lst:perbaikan_2.1.1_keyboard_menu_navigasi}.
\begin{lstlisting}[frame=single, label={lst:perbaikan_2.1.1_keyboard_menu_navigasi}, language=diff, caption=Perbaikan Kriteria Sukses 2.1.1 pada Menu Navigasi]
diff --git a/www/application/views/templates/topbar_loggedin.php b/www/application/views/templates/topbar_loggedin.php
index 51fc11e..bc9f2c0 100644
--- a/www/application/views/templates/topbar_loggedin.php
+++ b/www/application/views/templates/topbar_loggedin.php
@@ -7,7 +7,7 @@ defined('BASEPATH') OR exit('No direct script access allowed');

    <div class="top-bar" id="navigation-menu">
    <div class="top-bar-left">
_   <ul class="menu" data-responsive-menu="dropdown">
+   <ul class="menu">
    <li class="menu-text"><img src="/public/img/logo.png" class="textsized" alt="B"/></li>
    <?php foreach ($this->Auth_model->getUserInfo()['modules'] as $module): ?>
    <li<?= $module === $currentModule ? ' class="menu-active"' : '' ?>><a href="/<?= $module ?>"><?= $this->config->item('module-names')[$module] ?></a></li>
\end{lstlisting}

\item Pada bagian tabel "Daftar Jadwal" di halaman entri jadwal dosen, setiap \textit{cell} dalam tabel yang berisi jadwal dosen diberi atribut \textit{tabindex} yang diisi dengan nilai nol. Selanjutnya dibuat kode \textit{JavaScript} untuk memicu aksi membuka kotak dialog ketika tombol "Enter" ditekan pada \textit{cell} yang berisi jadwal dosen. Perbaikan yang dilakukan dapat dilihat pada \textit{Listing} \ref{lst:perbaikan_2.1.1_keyboard_entri_jadwal_dosen}.
\begin{lstlisting}[frame=single, label={lst:perbaikan_2.1.1_keyboard_entri_jadwal_dosen}, language=diff, caption=Perbaikan Kriteria Sukses 2.1.1 pada Halaman Entri Jadwal Dosen]
diff --git a/www/application/views/EntriJadwalDosen/main.php b/www/application/views/EntriJadwalDosen/main.php
index 6392b95..6c0e935 100644
--- a/www/application/views/EntriJadwalDosen/main.php
+++ b/www/application/views/EntriJadwalDosen/main.php
@@ -110,12 +110,21 @@ defined('BASEPATH') OR exit('No direct script access allowed');
    $("#cell" + i + "-" +<?php echo $colIdx; ?>).remove();
    }
    $($cellLocation).html("<?php echo $dataHariIni->label ?>");
+   $($cellLocation).attr("tabindex", "0");

    //membuat cell-cell yang telah diwarnai jadi memunculkan pop-up untuk mengedit jadwal ketika diklik oleh mouse
    $(document).on("click", $cellLocation, function () {
      var $menuName = "#edit_menu<?php echo $dataHariIni->id ?>";
      $($menuName).foundation('open');
    });
+
+   // membuat cell-cell dalam tabel dapat diakses menggunakan keyboard
+   $($cellLocation).keyup(function(e){
+   if(e.keyCode == 13){
+   var $menuName = "#edit_menu<?php echo $dataHariIni->id ?>";
+   $($menuName).foundation('open');
+   }
+   })
    </script>
    <?php
    }
\end{lstlisting}
\end{itemize}

\subsection{Perbaikan Kriteria Sukses 2.4.1 \textit{Bypass Blocks}}
\label{subsec:perbaikan_kriteria_sukses_2.4.1}
Pada bagian ini dilakukan perbaikan dengan membuat sebuah mekanisme untuk melompati bagian menu navigasi ketika pengguna sedang bernavigasi menggunakan \textit{keyboard}. Mekanisme yang dimaksud adalah sebuah elemen yang akan membuat fokus \textit{keyboard} berpindah ke area konten utama ketika pengguna mengaktifkan elemen tersebut. Elemen ini dapat dikenali oleh teknologi alat bantu dan hanya dapat diakses oleh pengguna yang bernavigasi dengan menggunakan \textit{keyboard}. Saat halaman web selesai dimuat, elemen ini akan menjadi yang pertama menerima fokus \textit{keyboard} ketika pengguna menekan tombol \textit{tab}. Perbaikan yang dilakukan dapat dilihat pada \textit{Listing} \ref{lst:perbaikan_2.4.1_bypass_blocks}.
\begin{lstlisting}[frame=single, label={lst:perbaikan_2.4.1_bypass_blocks}, language=diff, caption=Perbaikan Kriteria Sukses 2.4.1]
diff --git a/www/application/views/EntriJadwalDosen/main.php b/www/application/views/EntriJadwalDosen/main.php
index 6392b95..a62cce4 100644
--- a/www/application/views/EntriJadwalDosen/main.php
+++ b/www/application/views/EntriJadwalDosen/main.php
@@ -9,7 +9,7 @@ defined('BASEPATH') OR exit('No direct script access allowed');
    <body>
    <?php $this->load->view('templates/topbar_loggedin'); ?>

_   <div class="row">
+   <div id="mainContent" class="row">

    <div class="large-12 column callout">
    <h5>Tambah Jadwal</h5>
diff --git a/www/application/views/LihatJadwalDosen/main.php b/www/application/views/LihatJadwalDosen/main.php
index ef1be4f..5df2507 100644
--- a/www/application/views/LihatJadwalDosen/main.php
+++ b/www/application/views/LihatJadwalDosen/main.php
@@ -10,7 +10,7 @@ defined('BASEPATH') OR exit('No direct script access allowed');
    <?php //$this->load->view('templates/flashmessage'); ?>


_   <div class="row">
+   <div id="mainContent" class="row">

    <div class="large-12 column">
    <ul class="tabs" data-tabs id="tab_jadwal">
diff --git a/www/application/views/PerubahanKuliahManage/main.php b/www/application/views/PerubahanKuliahManage/main.php
index 657260b..a64fbf2 100644
--- a/www/application/views/PerubahanKuliahManage/main.php
+++ b/www/application/views/PerubahanKuliahManage/main.php
@@ -7,7 +7,7 @@ defined('BASEPATH') OR exit('No direct script access allowed');
    <?php $this->load->view('templates/topbar_loggedin'); ?>
    <?php $this->load->view('templates/flashmessage'); ?>

_   <div class="row">
+   <div id="mainContent" class="row">
    <div class="callout">
    <h5>Permohonan Perubahan Kuliah</h5>
    <table class="stack">
diff --git a/www/application/views/PerubahanKuliahRequest/main.php b/www/application/views/PerubahanKuliahRequest/main.php
index 72457a7..dbb9113 100644
--- a/www/application/views/PerubahanKuliahRequest/main.php
+++ b/www/application/views/PerubahanKuliahRequest/main.php
@@ -7,7 +7,7 @@ defined('BASEPATH') OR exit('No direct script access allowed');
    <?php $this->load->view('templates/topbar_loggedin'); ?>
    <?php $this->load->view('templates/flashmessage'); ?>

_   <div class="row">
+   <div id="mainContent" class="row">
    <div class="large-12 column">
    <div class="callout">
    <h5>Permohonan Baru</h5>
diff --git a/www/application/views/TranskripManage/main.php b/www/application/views/TranskripManage/main.php
index 0970fc2..4d94b34 100644
--- a/www/application/views/TranskripManage/main.php
+++ b/www/application/views/TranskripManage/main.php
@@ -7,7 +7,7 @@ defined('BASEPATH') OR exit('No direct script access allowed');
    <?php $this->load->view('templates/topbar_loggedin'); ?>
    <?php $this->load->view('templates/flashmessage'); ?>

_   <div class="row">
+   <div id="mainContent" class="row">
    <div class="callout">
    <h5>Permintaan Transkrip</h5>
    <form method="GET" action="/TranskripManage">
diff --git a/www/application/views/TranskripRequest/main.php b/www/application/views/TranskripRequest/main.php
index ada0254..f5beb85 100644
--- a/www/application/views/TranskripRequest/main.php
+++ b/www/application/views/TranskripRequest/main.php
@@ -7,7 +7,7 @@ defined('BASEPATH') OR exit('No direct script access allowed');
    <?php $this->load->view('templates/topbar_loggedin'); ?>
    <?php $this->load->view('templates/flashmessage'); ?>

_   <div class="row">
+   <div id="mainContent" class="row">
    <div class="medium-12 column">
    <div class="callout">
    <h5>Permohonan Baru</h5>
diff --git a/www/application/views/templates/script_foundation.php b/www/application/views/templates/script_foundation.php
index a93c72a..e5b9251 100644
--- a/www/application/views/templates/script_foundation.php
+++ b/www/application/views/templates/script_foundation.php
@@ -5,3 +5,10 @@ defined('BASEPATH') OR exit('No direct script access allowed');
    <script src="/public/js/foundation.min.js"></script>
    <script src="/public/js/app.js"></script>
    <script src="/public/lib/xdan-datetimepicker/jquery.datetimepicker.full.min.js"></script>
+   <script>
+   $(document).ready(function(){
+   var skipToContent = document.createElement("div");
+   skipToContent.innerHTML += '<a class="skip-navbar" href="#mainContent">Lompat ke menu utama</a>';
+   document.body.insertBefore(skipToContent, document.body.firstChild);
+   })
+   </script>
    \ No newline at end of file
diff --git a/www/public/css/app.css b/www/public/css/app.css
index f283c52..b6af0a7 100644
--- a/www/public/css/app.css
+++ b/www/public/css/app.css
@@ -62,3 +62,27 @@ i[class^="fi-"] {
    margin-left: 10px;
    margin-right: 10px;    
    }
+   
+   /* Skip Navbar */
+   a.skip-navbar {
+   position: absolute;
+   left: -99px;
+   top: 0;
+   opacity: 0;
+   width: 1px;
+   height: 1px;
+   z-index: -99;
+   }
+
+   a.skip-navbar:focus, a.skip-navbar:active {
+   opacity: 1;
+   left: 0;
+   background:white;
+   color:#000;
+   width: auto;
+   height: auto;
+   margin: 10px;
+   padding: 5px;
+   font-size: 1.4rem;
+   z-index: 99;
+   }
    \ No newline at end of file
\end{lstlisting}

\subsection{Perbaikan Kriteria Sukses 2.4.4 \textit{Link Purpose (In Context)}}
\label{subsec:perbaikan_kriteria_sukses_2.4.4}
Pada bagian ini dilakukan perbaikan sebagai berikut:

Pada tautan berisi ikon yang tidak memiliki teks untuk menjelaskan tujuan tautan tersebut yang terdapat di halaman cetak transkrip, manajemen cetak transkrip, perubahan kuliah, dan manajemen perubahan kuliah diberikan atribut \textit{aria-label} yang diisi dengan nilai yang sesuai dengan fungsi setiap konten ikon. Perbaikan yang dilakukan dapat dilihat pada \textit{Listing} \ref{lst:perbaikan_2.4.4_tautan_tanpa_keterangan}.
\begin{lstlisting}[frame=single, label={lst:perbaikan_2.4.4_tautan_tanpa_keterangan}, language=diff, caption=Perbaikan Kriteria Sukses 2.4.4]
diff --git a/www/application/views/PerubahanKuliahManage/main.php b/www/application/views/PerubahanKuliahManage/main.php
index 657260b..43c215f 100644
--- a/www/application/views/PerubahanKuliahManage/main.php
+++ b/www/application/views/PerubahanKuliahManage/main.php
@@ -30,11 +30,11 @@ defined('BASEPATH') OR exit('No direct script access allowed');
    <td><?= $request->mataKuliahCode ?></td>
    <td><?= PerubahanKuliah_model::CHANGETYPE_TYPES[$request->changeType] ?></td>
    <td>
_   <a data-open="detail<?= $request->id ?>"><i class="fi-eye"></i></a>
_   <a target="_blank" href="/PerubahanKuliahManage/printview/<?= $request->id ?>"><i class="fi-print"></i></a>
_   <a data-open="konfirmasi<?= $request->id ?>"><i class="fi-like"></i></a>                                    
_   <a data-open="tolak<?= $request->id ?>"><i class="fi-dislike"></i></a>
_   <a data-open="hapus<?= $request->id ?>"><i class="fi-trash"></i></a>
+   <a aria-label="lihat detail permohonan <?= $request->requestByName ?>" data-open="detail<?= $request->id ?>"><i class="fi-eye"></i></a>
+   <a aria-label="cetak permohonan <?= $request->requestByName ?>" target="_blank" href="/PerubahanKuliahManage/printview/<?= $request->id ?>"><i class="fi-print"></i></a>
+   <a aria-label="konfirmasi permohonan <?= $request->requestByName ?>" data-open="konfirmasi<?= $request->id ?>"><i class="fi-like"></i></a>
+   <a aria-label="tolak permohonan <?= $request->requestByName ?>" data-open="tolak<?= $request->id ?>"><i class="fi-dislike"></i></a>
+   <a aria-label="hapus permohonan <?= $request->requestByName ?>" data-open="hapus<?= $request->id ?>"><i class="fi-trash"></i></a>
    </td>
    </tr>
    <?php endforeach; ?>
diff --git a/www/application/views/PerubahanKuliahRequest/main.php b/www/application/views/PerubahanKuliahRequest/main.php
index 72457a7..cd47c17 100644
--- a/www/application/views/PerubahanKuliahRequest/main.php
+++ b/www/application/views/PerubahanKuliahRequest/main.php
@@ -118,7 +118,7 @@ defined('BASEPATH') OR exit('No direct script access allowed');
    <td><time datetime="<?= $request->answeredDateTime ?>"><?= $request->answeredDateString ?></time></td>
    <td><?= $request->answeredMessage ?></td>
    <td>
_   <a data-open="detail<?= $request->id ?>"><i class="fi-eye"></i></a>
+   <a aria-label="lihat detail permohonan <?= $request->requestByName ?>" data-open="detail<?= $request->id ?>"><i class="fi-eye"></i></a>
    </td>
    </tr>
    <?php endforeach; ?>
diff --git a/www/application/views/TranskripManage/main.php b/www/application/views/TranskripManage/main.php
index 0970fc2..9f24954 100644
--- a/www/application/views/TranskripManage/main.php
+++ b/www/application/views/TranskripManage/main.php
@@ -85,7 +85,7 @@ defined('BASEPATH') OR exit('No direct script access allowed');
    <span aria-hidden="true">&times;</span>
    </button>                                        
    </div>
_   <a data-open="detail<?= $request->id ?>"><i class="fi-eye"></i></a>
+   <a aria-label="lihat detail permohonan <?= $request->requestByName ?>" data-open="detail<?= $request->id ?>"><i class="fi-eye"></i></a>
    <div class="reveal" id="tolak<?= $request->id ?>" data-reveal>
    <h5>Tolak Permohonan</h5>
    <form method="POST" action="/TranskripManage/answer">
@@ -105,7 +105,7 @@ defined('BASEPATH') OR exit('No direct script access allowed');
    <span aria-hidden="true">&times;</span>
    </button>
    </div>
_   <a data-open="tolak<?= $request->id ?>"><i class="fi-dislike"></i></a>
+   <a aria-label="tolak permohonan <?= $request->requestByName ?>" data-open="tolak<?= $request->id ?>"><i class="fi-dislike"></i></a>
    <div class="reveal" id="cetak<?= $request->id ?>" data-reveal>
    <h5>Cetak Permohonan</h5>
    <?php if ($request->requestByNPM !== NULL): ?>
@@ -131,7 +131,7 @@ defined('BASEPATH') OR exit('No direct script access allowed');
    <span aria-hidden="true">&times;</span>
    </button>
    </div>
_   <a data-open="cetak<?= $request->id ?>"><i class="fi-print"></i></a>
+   <a aria-label="cetak permohonan <?= $request->requestByName ?>" data-open="cetak<?= $request->id ?>"><i class="fi-print"></i></a>
    <div class="reveal" id="hapus<?= $request->id ?>" data-reveal>
    <h5>Hapus Permohonan</h5>
    <form method="POST" action="/TranskripManage/remove">
@@ -146,7 +146,7 @@ defined('BASEPATH') OR exit('No direct script access allowed');
    <span aria-hidden="true">&times;</span>
    </button>
    </div>
_   <a data-open="hapus<?= $request->id ?>"><i class="fi-trash"></i></a>
+   <a aria-label="hapus permohonan <?= $request->requestByName ?>" data-open="hapus<?= $request->id ?>"><i class="fi-trash"></i></a>
    </td>
    </tr>
    <?php endforeach; ?>
diff --git a/www/application/views/TranskripRequest/main.php b/www/application/views/TranskripRequest/main.php
index ada0254..44598cc 100644
--- a/www/application/views/TranskripRequest/main.php
+++ b/www/application/views/TranskripRequest/main.php
@@ -126,7 +126,7 @@ defined('BASEPATH') OR exit('No direct script access allowed');
    <span aria-hidden="true">&times;</span>
    </button>
    </div>
_   <a data-open="detail<?= $request->id ?>"><i class="fi-eye"></i></a>
+   <a aria-label="lihat detail permohonan <?= $request->requestByName ?>" data-open="detail<?= $request->id ?>"><i class="fi-eye"></i></a>
    </td>
    </tr>
    <?php endforeach; ?>
\end{lstlisting}

\subsection{Perbaikan Kriteria Sukses 2.5.3 \textit{Label in Name}}
\label{subsec:perbaikan_kriteria_sukses_2.5.3}
Pada bagian ini dilakukan perbaikan yaitu untuk setiap kolom \textit{input} di halaman manajemen cetak transkrip dan entri jadwal dosen diberi atribut \textit{id} dengan nilai yang sesuai dengan isi kolom \textit{input} lalu setiap judul kolom dibuat sebagai label dengan menggunakan \texttt{<label>} dan dihubungkan dengan kolom \textit{input} yang bersangkutan dengan menggunakan atribut \textit{for}. Perbaikan yang dilakukan dapat dilihat pada \textit{Listing} \ref{lst:perbaikan_2.5.3_label_dan_nama_pada_komponen_masukan}.

\begin{lstlisting}[frame=single, label={lst:perbaikan_2.5.3_label_dan_nama_pada_komponen_masukan}, language=diff, caption=Perbaikan Kriteria Sukses 2.5.3]
diff --git a/www/application/views/EntriJadwalDosen/main.php b/www/application/views/EntriJadwalDosen/main.php
index 6392b95..43af850 100644
--- a/www/application/views/EntriJadwalDosen/main.php
+++ b/www/application/views/EntriJadwalDosen/main.php
@@ -16,8 +16,8 @@ defined('BASEPATH') OR exit('No direct script access allowed');
    <div class="large-4 columns">
    <form method="POST" action="/EntriJadwalDosen/add">
    <input type="hidden" name="<?= $this->security->get_csrf_token_name() ?>" value="<?= $this->security->get_csrf_hash() ?>" />
_   Hari
_   <select name="hari"> 
+   <label for="hari">Hari</label>
+   <select id="hari" name="hari"> 
    <?php
    $hariValue = 0;
    foreach ($namaHari as $hari) {
@@ -29,30 +29,30 @@ defined('BASEPATH') OR exit('No direct script access allowed');
    }
    ?>
    </select><br>
_   Jam Mulai
_   <select name="jam_mulai"> 
+   <label for="jam_mulai">Jam Mulai</label>
+   <select id="jam_mulai" name="jam_mulai">  
    <?php for ($i = 7; $i <= 16; $i++) { ?>
    <option value="<?php echo $i ?>"> <?php echo $i ?>:00 </option>
    <?php } ?>
    </select><br>
    </div>
    <div class=" large-4 columns">
_   Durasi
_   <select name="durasi"> 
+   <label for="durasi">Durasi</label>
+   <select id="durasi" name="durasi">  
    <?php for ($i = 1; $i <= 9; $i++) { ?>
    <option value="<?php echo $i ?>"> <?php echo $i ?> jam </option>
    <?php } ?>
    </select><br>
_   Jenis  
_   <select name="jenis_jadwal"> 
+   <label for="jenis_jadwal">Jenis</label>
+   <select id="jenis_jadwal" name="jenis_jadwal">  
    <option value="konsultasi" style="background-color:yellow"> Konsultasi </option>
    <option value="terjadwal" style="background-color:green;color:white"> Terjadwal</option>
    <option value="kelas" style="background-color:white"> Kelas </option>
    </select>
    </div>
    <div class="large-4 columns">
_   Label <input type="text" name="label_jadwal"><br>
+   <label for="label_jadwal">Label</label>
+   <input id="label_jadwal" type="text" name="label_jadwal"><br>
    <input type="submit" class="button" value="Tambah">
    </form>
    </div>
    </div>
@@ -173,8 +174,8 @@ defined('BASEPATH') OR exit('No direct script access allowed');
    <form name="form<?php echo $dataHariIni->id ?>" method="POST" action="/EntriJadwalDosen/update/<?php echo $dataHariIni->id ?>">
    <input type="hidden" name="<?= $this->security->get_csrf_token_name() ?>" value="<?= $this->security->get_csrf_hash() ?>" />
    <input type="hidden" name="id_jadwal_parameter" value="<?php echo $dataHariIni->id ?>"> </a> <br>
_   Hari 
_   <select name="hari"> 
+   <label for="edit_hari">Hari</label>
+   <select id="edit_hari" name="hari"> 
    <?php
    $hariValue = 0;
    foreach ($namaHari as $hari) {
@@ -185,14 +186,14 @@ defined('BASEPATH') OR exit('No direct script access allowed');
    }
    ?>
    </select><br>
_   Jam Mulai
_   <select name="jam_mulai"> 
+   <label for="edit_jam_mulai">Jam Mulai</label>
+   <select id="edit_jam_mulai" name="jam_mulai"> 
    <?php
    for ($i = 7; $i <= 16; $i++) {
    if ($i == $dataHariIni->jam_mulai) {
@@ -207,8 +208,8 @@ defined('BASEPATH') OR exit('No direct script access allowed');
    }
    ?>
    </select><br>
_   Durasi
_   <select name="durasi"> 
+   <label for="edit_durasi">Durasi</label>
+   <select id="edit_durasi" name="durasi"> 
    <?php
    for ($i = 1; $i <= 9; $i++) {
    if ($i == $dataHariIni->durasi) {
@@ -223,13 +224,14 @@ defined('BASEPATH') OR exit('No direct script access allowed');
    }
    ?>
    </select><br>
_   Jenis  
_   <select name="jenis_jadwal"> 
+   <label for="edit_jenis_jadwal">Jenis</label>
+   <select id="edit_jenis_jadwal" name="jenis_jadwal"> 
    <option style="background-color:yellow" value="konsultasi" <?php if ($dataHariIni->jenis == 'konsultasi') echo "selected"; ?> > Konsultasi </option>
    <option style="background-color:green" value="terjadwal" <?php if ($dataHariIni->jenis == 'terjadwal') echo "selected"; ?>> Terjadwal</option>
    <option style="background-color:white" value="kelas" <?php if ($dataHariIni->jenis == 'kelas') echo "selected"; ?>> Kelas </option>
    </select>
_   Label <input type="text" name="label_jadwal" value="<?php echo $dataHariIni->label; ?>"><br> 
+   <label for="edit_label_jadwal">Label</label>
+   <input type="text" id="edit_label_jadwal" name="label_jadwal" value="<?php echo $dataHariIni->label; ?>"><br> 
    <div class="row large-4 column">
    <div class="large-2 column">
    <input type="submit" name="submitId<?php echo $dataHariIni->id ?>" class="button" value="Save  ">
diff --git a/www/application/views/TranskripManage/main.php b/www/application/views/TranskripManage/main.php
index 0970fc2..6dffa7d 100644
--- a/www/application/views/TranskripManage/main.php
+++ b/www/application/views/TranskripManage/main.php
@@ -12,8 +12,8 @@ defined('BASEPATH') OR exit('No direct script access allowed');
    <h5>Permintaan Transkrip</h5>
    <form method="GET" action="/TranskripManage">
    <div class="input-group">
_   <span class="input-group-label">Cari NPM:</span>
_   <input name="npm" class="input-group-field" type="text" placeholder="2013730013" maxlength="10" minlength="10"<?= $npmQuery === NULL ? '' : " value='$npmQuery'" ?>/>
+   <label for="npm" class="input-group-label">Cari NPM:</label>
+   <input id="npm" name="npm" class="input-group-field" type="text" placeholder="2013730013" maxlength="10" minlength="10"<?= $npmQuery === NULL ? '' : " value='$npmQuery'" ?>/>
    <div class="input-group-button">
    <input class="button" type="submit" value="Cari"/>
    </div>
\end{lstlisting} 

\subsection{Perbaikan Kriteria Sukses 3.1.1 \textit{Language of Page}}
\label{subsec:perbaikan_kriteria_sukses_3.1.1}
Pada bagian ini dilakukan perbaikan dengan mengubah setelan bahasa dari bahasa Inggris menjadi bahasa Indonesia untuk setiap halaman yang ditampilkan kepada pengguna. Perbaikan yang dilakukan dapat dilihat pada \textit{Listing} \ref{lst:perbaikan_3.1.1_bahasa_halaman}.

\begin{lstlisting}[frame=single, label={lst:perbaikan_3.1.1_bahasa_halaman}, language=diff, caption=Perbaikan Kriteria Sukses 3.1.1]
diff --git a/www/application/views/EntriJadwalDosen/main.php b/www/application/views/EntriJadwalDosen/main.php
index 6392b95..6d86231 100644
--- a/www/application/views/EntriJadwalDosen/main.php
+++ b/www/application/views/EntriJadwalDosen/main.php
@@ -1,7 +1,7 @@
    <?php
    defined('BASEPATH') OR exit('No direct script access allowed');
    ?><!doctype html>
_   <html class="no-js" lang="en">
+   <html class="no-js" lang="id">
    <?php $this->load->view('templates/script_foundation'); ?>
    <?php $this->load->view('templates/head_loggedin'); ?>
    <?php $this->load->view('templates/flashmessage'); ?>
diff --git a/www/application/views/LihatJadwalDosen/main.php b/www/application/views/LihatJadwalDosen/main.php
index ef1be4f..aff813f 100644
--- a/www/application/views/LihatJadwalDosen/main.php
+++ b/www/application/views/LihatJadwalDosen/main.php
@@ -1,7 +1,7 @@
    <?php
    defined('BASEPATH') OR exit('No direct script access allowed');
    ?><!doctype html>
_   <html class="no-js" lang="en">
+   <html class="no-js" lang="id">
    <?php $this->load->view('templates/head_loggedin'); ?>
    <body>
    <?php $this->load->view('templates/topbar_loggedin'); ?>
diff --git a/www/application/views/PerubahanKuliahManage/main.php b/www/application/views/PerubahanKuliahManage/main.php
index 657260b..6c9f7ee 100644
--- a/www/application/views/PerubahanKuliahManage/main.php
+++ b/www/application/views/PerubahanKuliahManage/main.php
@@ -1,7 +1,7 @@
    <?php
    defined('BASEPATH') OR exit('No direct script access allowed');
    ?><!doctype html>
_   <html class="no-js" lang="en">
+   <html class="no-js" lang="id">
    <?php $this->load->view('templates/head_loggedin'); ?>
    <body>
    <?php $this->load->view('templates/topbar_loggedin'); ?>
diff --git a/www/application/views/PerubahanKuliahManage/printview.php b/www/application/views/PerubahanKuliahManage/printview.php
index 7c4034c..1d80622 100644
--- a/www/application/views/PerubahanKuliahManage/printview.php
+++ b/www/application/views/PerubahanKuliahManage/printview.php
@@ -2,7 +2,7 @@
    defined('BASEPATH') OR exit('No direct script access allowed');
    setlocale(LC_TIME, 'ind');
    ?><!doctype html>
_   <html class="no-js" lang="en">
+   <html class="no-js" lang="id">
    <head>
    <meta charset="utf-8" />
    <meta http-equiv="x-ua-compatible" content="ie=edge">
diff --git a/www/application/views/PerubahanKuliahRequest/main.php b/www/application/views/PerubahanKuliahRequest/main.php
index 72457a7..2a0c874 100644
--- a/www/application/views/PerubahanKuliahRequest/main.php
+++ b/www/application/views/PerubahanKuliahRequest/main.php
@@ -1,7 +1,7 @@
    <?php
    defined('BASEPATH') OR exit('No direct script access allowed');
    ?><!doctype html>
_   <html class="no-js" lang="en">
+   <html class="no-js" lang="id">
    <?php $this->load->view('templates/head_loggedin'); ?>
    <body>
    <?php $this->load->view('templates/topbar_loggedin'); ?>
diff --git a/www/application/views/TranskripManage/main.php b/www/application/views/TranskripManage/main.php
index 0970fc2..9faf620 100644
--- a/www/application/views/TranskripManage/main.php
+++ b/www/application/views/TranskripManage/main.php
@@ -1,7 +1,7 @@
    <?php
    defined('BASEPATH') OR exit('No direct script access allowed');
    ?><!doctype html>
_   <html class="no-js" lang="en">
+   <html class="no-js" lang="id">
    <?php $this->load->view('templates/head_loggedin'); ?>
    <body>
    <?php $this->load->view('templates/topbar_loggedin'); ?>
diff --git a/www/application/views/TranskripRequest/main.php b/www/application/views/TranskripRequest/main.php
index ada0254..e81549e 100644
--- a/www/application/views/TranskripRequest/main.php
+++ b/www/application/views/TranskripRequest/main.php
@@ -1,7 +1,7 @@
    <?php
    defined('BASEPATH') OR exit('No direct script access allowed');
    ?><!doctype html>
_   <html class="no-js" lang="en">
+   <html class="no-js" lang="id">
    <?php $this->load->view('templates/head_loggedin'); ?>
    <body>
    <?php $this->load->view('templates/topbar_loggedin'); ?>
diff --git a/www/application/views/auth/login.php b/www/application/views/auth/login.php
index e4daa52..b2089af 100644
--- a/www/application/views/auth/login.php
+++ b/www/application/views/auth/login.php
@@ -1,7 +1,7 @@
    <?php
    defined('BASEPATH') OR exit('No direct script access allowed');
    ?><!doctype html>
_   <html class="no-js" lang="en">
+   <html class="no-js" lang="id">
    <head>
    <meta charset="utf-8" />
    <meta http-equiv="x-ua-compatible" content="ie=edge">
\end{lstlisting}

\subsection{Perbaikan Kriteria Sukses 3.3.2 \textit{Labels or Instructions}}
\label{subsec:perbaikan_kriteria_sukses_3.3.2}
Pada bagian ini dilakukan perbaikan yaitu untuk setiap kolom \textit{input} di halaman manajemen cetak transkrip dan entri jadwal dosen diberi atribut \textit{id} dengan nilai yang sesuai dengan isi kolom \textit{input} lalu setiap judul kolom dibuat sebagai label dengan menggunakan \texttt{<label>} dan dihubungkan dengan kolom \textit{input} yang bersangkutan dengan menggunakan atribut \textit{for}. Perbaikan yang dilakukan dapat dilihat pada \textit{Listing} \ref{lst:perbaikan_3.3.2_label_masukan}.

\begin{lstlisting}[frame=single, label={lst:perbaikan_3.3.2_label_masukan}, language=diff, caption=Perbaikan Kriteria Sukses 3.3.2]
diff --git a/www/application/views/EntriJadwalDosen/main.php b/www/application/views/EntriJadwalDosen/main.php
index 6392b95..43af850 100644
--- a/www/application/views/EntriJadwalDosen/main.php
+++ b/www/application/views/EntriJadwalDosen/main.php
@@ -16,8 +16,8 @@ defined('BASEPATH') OR exit('No direct script access allowed');
    <div class="large-4 columns">
    <form method="POST" action="/EntriJadwalDosen/add">
    <input type="hidden" name="<?= $this->security->get_csrf_token_name() ?>" value="<?= $this->security->get_csrf_hash() ?>" />
_   Hari
_   <select name="hari"> 
+   <label for="hari">Hari</label>
+   <select id="hari" name="hari"> 
    <?php
    $hariValue = 0;
    foreach ($namaHari as $hari) {
@@ -29,30 +29,30 @@ defined('BASEPATH') OR exit('No direct script access allowed');
    }
    ?>
    </select><br>
_   Jam Mulai
_   <select name="jam_mulai"> 
+   <label for="jam_mulai">Jam Mulai</label>
+   <select id="jam_mulai" name="jam_mulai">  
    <?php for ($i = 7; $i <= 16; $i++) { ?>
    <option value="<?php echo $i ?>"> <?php echo $i ?>:00 </option>
    <?php } ?>
    </select><br>
    </div>
    <div class=" large-4 columns">
_   Durasi
_   <select name="durasi"> 
+   <label for="durasi">Durasi</label>
+   <select id="durasi" name="durasi">  
    <?php for ($i = 1; $i <= 9; $i++) { ?>
    <option value="<?php echo $i ?>"> <?php echo $i ?> jam </option>
    <?php } ?>
    </select><br>
_   Jenis  
_   <select name="jenis_jadwal"> 
+   <label for="jenis_jadwal">Jenis</label>
+   <select id="jenis_jadwal" name="jenis_jadwal">  
    <option value="konsultasi" style="background-color:yellow"> Konsultasi </option>
    <option value="terjadwal" style="background-color:green;color:white"> Terjadwal</option>
    <option value="kelas" style="background-color:white"> Kelas </option>
    </select>
    </div>
    <div class="large-4 columns">
_   Label <input type="text" name="label_jadwal"><br>
+   <label for="label_jadwal">Label</label>
+   <input id="label_jadwal" type="text" name="label_jadwal"><br>
    <input type="submit" class="button" value="Tambah">
    </form>
    </div>
    </div>
@@ -173,8 +174,8 @@ defined('BASEPATH') OR exit('No direct script access allowed');
    <form name="form<?php echo $dataHariIni->id ?>" method="POST" action="/EntriJadwalDosen/update/<?php echo $dataHariIni->id ?>">
    <input type="hidden" name="<?= $this->security->get_csrf_token_name() ?>" value="<?= $this->security->get_csrf_hash() ?>" />
    <input type="hidden" name="id_jadwal_parameter" value="<?php echo $dataHariIni->id ?>"> </a> <br>
_   Hari 
_   <select name="hari"> 
+   <label for="edit_hari">Hari</label>
+   <select id="edit_hari" name="hari"> 
    <?php
    $hariValue = 0;
    foreach ($namaHari as $hari) {
@@ -185,14 +186,14 @@ defined('BASEPATH') OR exit('No direct script access allowed');
    }
    ?>
    </select><br>
_   Jam Mulai
_   <select name="jam_mulai"> 
+   <label for="edit_jam_mulai">Jam Mulai</label>
+   <select id="edit_jam_mulai" name="jam_mulai"> 
    <?php
    for ($i = 7; $i <= 16; $i++) {
    if ($i == $dataHariIni->jam_mulai) {
@@ -207,8 +208,8 @@ defined('BASEPATH') OR exit('No direct script access allowed');
    }
    ?>
    </select><br>
_   Durasi
_   <select name="durasi"> 
+   <label for="edit_durasi">Durasi</label>
+   <select id="edit_durasi" name="durasi"> 
    <?php
    for ($i = 1; $i <= 9; $i++) {
    if ($i == $dataHariIni->durasi) {
@@ -223,13 +224,14 @@ defined('BASEPATH') OR exit('No direct script access allowed');
    }
    ?>
    </select><br>
_   Jenis  
_   <select name="jenis_jadwal"> 
+   <label for="edit_jenis_jadwal">Jenis</label>
+   <select id="edit_jenis_jadwal" name="jenis_jadwal"> 
    <option style="background-color:yellow" value="konsultasi" <?php if ($dataHariIni->jenis == 'konsultasi') echo "selected"; ?> > Konsultasi </option>
    <option style="background-color:green" value="terjadwal" <?php if ($dataHariIni->jenis == 'terjadwal') echo "selected"; ?>> Terjadwal</option>
    <option style="background-color:white" value="kelas" <?php if ($dataHariIni->jenis == 'kelas') echo "selected"; ?>> Kelas </option>
    </select>
_   Label <input type="text" name="label_jadwal" value="<?php echo $dataHariIni->label; ?>"><br> 
+   <label for="edit_label_jadwal">Label</label>
+   <input type="text" id="edit_label_jadwal" name="label_jadwal" value="<?php echo $dataHariIni->label; ?>"><br> 
    <div class="row large-4 column">
    <div class="large-2 column">
    <input type="submit" name="submitId<?php echo $dataHariIni->id ?>" class="button" value="Save  ">
diff --git a/www/application/views/TranskripManage/main.php b/www/application/views/TranskripManage/main.php
index 0970fc2..6dffa7d 100644
--- a/www/application/views/TranskripManage/main.php
+++ b/www/application/views/TranskripManage/main.php
@@ -12,8 +12,8 @@ defined('BASEPATH') OR exit('No direct script access allowed');
    <h5>Permintaan Transkrip</h5>
    <form method="GET" action="/TranskripManage">
    <div class="input-group">
_   <span class="input-group-label">Cari NPM:</span>
_   <input name="npm" class="input-group-field" type="text" placeholder="2013730013" maxlength="10" minlength="10"<?= $npmQuery === NULL ? '' : " value='$npmQuery'" ?>/>
+   <label for="npm" class="input-group-label">Cari NPM:</label>
+   <input id="npm" name="npm" class="input-group-field" type="text" placeholder="2013730013" maxlength="10" minlength="10"<?= $npmQuery === NULL ? '' : " value='$npmQuery'" ?>/>
    <div class="input-group-button">
    <input class="button" type="submit" value="Cari"/>
    </div>
\end{lstlisting} 

\subsection{Perbaikan Kriteria Sukses 4.1.1 \textit{Parsing}}
\label{subsec:perbaikan_kriteria_sukses_4.1.1}
Pada bagian ini dilakukan perbaikan sebagai berikut:

\begin{itemize}
\item Pada penggunaan \textit{tag "time"} buka yang tidak disertai dengan \textit{tag "time"} tutup di halaman cetak transkrip, bagian yang tidak lengkap ditambahkan \texttt{</time>}. Perbaikan yang dilakukan dapat dilihat pada \textit{Listing} \ref{lst:perbaikan_4.1.1_parsing_halaman_cetak_transkrip}.
\begin{lstlisting}[frame=single, label={lst:perbaikan_4.1.1_parsing_halaman_cetak_transkrip}, language=diff, caption=Perbaikan Kriteria Sukses 4.1.1 pada Halaman Cetak Transkrip]
diff --git a/www/application/views/TranskripRequest/main.php b/www/application/views/TranskripRequest/main.php
index ada0254..ed45720 100644
--- a/www/application/views/TranskripRequest/main.php
+++ b/www/application/views/TranskripRequest/main.php
@@ -77,7 +77,7 @@ defined('BASEPATH') OR exit('No direct script access allowed');
    <td><span class="<?= $request->labelClass ?> label"><?= $request->status ?></span></td>
    <td><time datetime="<?= $request->requestDateTime ?>"><?= $request->requestDateString ?></time></td>
    <td><?= $request->requestType ?></td>
_   <td><time datetime="<?= $request->answeredDateTime ?>"><?= $request->answeredDateString ?></td>
+   <td><time datetime="<?= $request->answeredDateTime ?>"><?= $request->answeredDateString ?></time></td>
    <td><?= $request->answeredMessage ?></td>
    <td>
    <div class="reveal" id="detail<?= $request->id ?>" data-reveal>
\end{lstlisting} 

\item Pada penempatan elemen \textit{"HTML"} yang tidak tepat di halaman entri jadwal dosen, \textit{tag "div"} dipindahkan agar berada sesudah \textit{tag "body"}. \textit{Tag "div"} yang bermasalah ini muncul dari \textit{file} "flashmessage.php" yang dimuat di tempat yang salah. Oleh karena perbaikan yang dilakukan adalah dengan memuat \textit{file} "flashmessage.php" di tempat yang seharusnya. Perbaikan yang dilakukan dapat dilihat pada \textit{Listing} \ref{lst:perbaikan_4.1.1_parsing_halaman_entri_jadwal_dosen}.
\begin{lstlisting}[frame=single, label={lst:perbaikan_4.1.1_parsing_halaman_entri_jadwal_dosen}, language=diff, caption=Perbaikan Kriteria Sukses 4.1.1 pada Halaman Entri Jadwal Dosen]
diff --git a/www/application/views/EntriJadwalDosen/main.php b/www/application/views/EntriJadwalDosen/main.php
index 6392b95..40d0c73 100644
--- a/www/application/views/EntriJadwalDosen/main.php
+++ b/www/application/views/EntriJadwalDosen/main.php
@@ -4,10 +4,10 @@ defined('BASEPATH') OR exit('No direct script access allowed');
    <html class="no-js" lang="en">
    <?php $this->load->view('templates/script_foundation'); ?>
    <?php $this->load->view('templates/head_loggedin'); ?>
-	<?php $this->load->view('templates/flashmessage'); ?>
    <?php $this->load->helper('url'); ?>
    <body>
    <?php $this->load->view('templates/topbar_loggedin'); ?>
+   <?php $this->load->view('templates/flashmessage'); ?>
 
    <div class="row">
\end{lstlisting} 
\end{itemize}

\section{Pengujian}
\label{sec:pengujian}
Pada subbab sebelumnya telah dilakukan perbaikan-perbaikan untuk membuat halaman web BlueTape menjadi lebih mudah diakses untuk orang berkebutuhan khusus dalam pengelihatan. Pada subbab ini berisi skenario pengujian dan hasil yang didapatkan dari setiap skenario pengujian yang telah dibuat. Setiap skenario pengujian merupakan langkah-langkah yang dituliskan dalam bentuk poin-poin yang menjelaskan cara untuk menggunakan fitur-fitur pada halaman web BlueTape. Hasil pengujian dituliskan dalam bentuk tabel yang berisi langkah skenario pengujian, hasil pengujian, dan tindakan yang dilakukan dalam pengujian.

Pengujian dilakukan dengan kondisi seakan-akan halaman web BlueTape digunakan oleh orang yang tidak bisa melihat untuk menilai apakah perbaikan-perbaikan yang telah dilakukan di subbab \ref{sec:implementasi} berhasil membuat halaman web BlueTape lebih mudah diakses atau tidak. Pengujian dilakukan dengan menggunakan perangkat komputer berupa laptop dengan sistem operasi Windows, \textit{browser} Google Chrome, dan pembaca layar ChromeVox sebagai teknologi alat bantu. Pengujian dilakukan pada server lokal dan akun yang digunakan memiliki hak akses tak terbatas sehingga dapat menggunakan semua fitur yang terdapat pada halaman web BlueTape.

\subsection{Skenario Login dan Hasilnya}
\label{subsec:skenario_login}
Langkah-langkah untuk \textit{login} pada halaman web BlueTape adalah sebagai berikut:

\begin{enumerate}
    \item Buka halaman web BlueTape dengan menggunakan alamat \url{http://bluetape.localhost/}.
    \item Tekan tombol "Login dengan Google".
    \item Pilih akun 7315020@student.unpar.ac.id.
    \item Masuk ke dalam halaman web BlueTape.
\end{enumerate}

Berikut adalah tabel hasil pengujian untuk skenario \textit{login}.

\begin{table}[H]
    \centering 
    \caption{Hasil Pengujian untuk Skenario \textit{Login}}
    \label{tab:hasil_pengujian_login}
    \begin{tabular}{|c|c|p{10cm}|}
        \toprule
        Langkah ke & Hasil (sukses/tidak) & Tindakan \\

        \midrule
        1 & Sukses & Mengetik alamat \url{http://bluetape.localhost/} ketika ChromeVox menyebutkan \textit{"new tab, tab"}, lalu menekan \textit{"enter"}. \\
        2 & Sukses & Menekan \textit{"enter"} pada saat ChromeVox menyebutkan \textit{"login, Login dengan Google, link"}. \\
        3 & Sukses & Menekan \textit{"tab"} hingga ChromeVox menyebutkan "Hizkia Steven, 7315020@student.unpar.ac.id", lalu menekan \textit{"enter"}. \\
        4 & Sukses & Mendengar ChromeVox menyebutkan "cetak transkrip, cetak transkrip, \textit{link list item}". \\

        \bottomrule

    \end{tabular}
\end{table}

\subsection{Skenario Membuat Permohonan Cetak Transkrip dan Hasilnya}
\label{subsec:skenario_membuat_permohonan_cetak_transkrip}
Langkah-langkah untuk membuat permohonan cetak transkrip pada halaman cetak transkrip adalah sebagai berikut:

\begin{enumerate}
    \item Lakukan \textit{login} pada halaman web BlueTape.
    \item Masuk ke halaman cetak transkrip dengan memilih menu "Cetak Transkrip".
    \item Pilih opsi "DPS (Seluruh Semester, Bilingual)" pada kolom "Tipe Transkrip".
    \item Isi kolom "Keperluan" dengan "Tes Permohonan Baru".
    \item Tekan tombol "Kirim Permohonan" untuk mengirim permohonan.
\end{enumerate}

Berikut adalah tabel hasil pengujian untuk skenario membuat permohonan cetak transkrip.

\begin{table}[H]
    \centering 
    \caption{Hasil Pengujian untuk Skenario Membuat Permohonan Cetak Transkrip}
    \label{tab:hasil_pengujian_membuat_permohonan_cetak_transkrip}
    \begin{tabular}{|c|c|p{10cm}|}
        \toprule
        Langkah ke & Hasil (sukses/tidak) & Tindakan \\

        \midrule
        1 & Sukses & Mengikuti langkah-langkah pada Tabel \ref{tab:hasil_pengujian_login} \\
        2 & Sukses & Tidak ada karena halaman awal yang ditampilkan setelah berhasil melakukan \textit{login} adalah halaman cetak transkrip. \\
        3 & Sukses & Menekan \textit{"shift + alt + N"} lalu \textit{"H"} sebanyak satu kali hingga ChromeVox menyebutkan "permohonan baru, \textit{heading} satu". Setelah itu menekan \textit{"tab"} sebanyak empat kali hingga ChromeVox menyebutkan "tipe transkrip, DPS (seluruh semester bilingual), \textit{combo box satu of satu}". \\
        4 & Sukses & Menekan \textit{"tab"} sebanyak satu kali hingga ChromeVox menyebutkan "keperluan, \textit{edit text}" lalu mengetik "Tes Permohonan Baru" untuk mengisi kolom tersebut. \\
        5 & Sukses & Menekan \textit{"tab"} sebanyak satu kali hingga ChromeVox menyebutkan "kirim permohonan, \textit{button}" lalu menekan \textit{"enter"}. \\

        \bottomrule

    \end{tabular}
\end{table}

\subsection{Skenario Memeriksa Detail Permohonan Cetak Transkrip di Halaman Cetak Transkrip dan Hasilnya}
\label{subsec:skenario_memeriksa_detail_permohonan_cetak_transkrip_di_halaman_cetak_transkrip}
Langkah-langkah untuk memeriksa detail permohonan cetak transkrip pada halaman cetak transkrip adalah sebagai berikut:

\begin{enumerate}
    \item Lakukan \textit{login} pada halaman web BlueTape.
    \item Masuk ke halaman cetak transkrip dengan memilih menu "Cetak Transkrip".
    \item Tekan tombol dengan ikon mata yang terdapat pada kolom "Aksi" di tabel "Histori Permohonan".
    \item Periksa detail permohonan.
\end{enumerate}

Berikut adalah tabel hasil pengujian untuk skenario memeriksa detail permohonan cetak transkrip di halaman cetak transkrip.

\begin{table}[H]
    \centering 
    \caption{Hasil Pengujian untuk Skenario Memeriksa Detail Permohonan Cetak Transkrip di Halaman Cetak Transkrip}
    \label{tab:hasil_pengujian_memeriksa_detail_permohonan_cetak_transkrip_di_halaman_cetak_transkrip}
    \begin{tabular}{|c|c|p{10cm}|}
        \toprule
        Langkah ke & Hasil (sukses/tidak) & Tindakan \\

        \midrule
        1 & Sukses & Mengikuti langkah-langkah pada Tabel \ref{tab:hasil_pengujian_login} \\
        2 & Sukses & Tidak ada karena halaman awal yang ditampilkan setelah berhasil melakukan \textit{login} adalah halaman cetak transkrip. \\
        3 & Sukses & Menekan \textit{"shift + alt + N"} lalu \textit{"H"}, dilakukan sebanyak dua kali hingga ChromeVox menyebutkan "histori permohonan \textit{heading} satu". Setelah itu menekan \textit{"tab"} sebanyak satu kali hingga ChromeVox menyebutkan "lihat detail permohonan Hizkia Steven, \textit{link has popup}" lalu menekan \textit{"enter"} hingga ChromeVox menyebutkan "\textit{entered dialog}". \\
        4 & Sukses & Menekan \textit{"shift + alt + right arrow"} hingga ChromeVox menyebutkan "tabel, \textit{email} pemohon, 7315020@student.unpar.ac.id" lalu menekan \textit{"shift + alt + R"} sehingga ChromeVox membacakan seluruh isi detail permohonan yang dipilih. \\

        \bottomrule

    \end{tabular}
\end{table}

\subsection{Skenario Menyaring Permohonan Cetak Transkrip Berdasarkan NPM dan Hasilnya}
\label{subsec:skenario_menyaring_permohonan_cetak_transkrip_berdasarkan_npm}
Langkah-langkah untuk menyaring permohonan cetak transkrip berdasarkan NPM pada halaman manajemen cetak transkrip adalah sebagai berikut:

\begin{enumerate}
    \item Lakukan \textit{login} pada halaman web BlueTape.
    \item Masuk ke halaman manajemen cetak transkrip dengan memilih menu "Manajemen Cetak Transkrip".
    \item Isi kolom "Cari NPM" dengan "7315020@student.unpar.ac.id".
    \item Tekan tombol "Cari" untuk melakukan penyaringan.
\end{enumerate}

Berikut adalah tabel hasil pengujian untuk skenario menyaring permohonan cetak transkrip berdasarkan NPM pada halaman manajemen cetak transkrip.

\begin{table}[H]
    \centering 
    \caption{Hasil Pengujian untuk Skenario Menyaring Permohonan Cetak Transkrip Berdasarkan NPM}
    \label{tab:hasil_pengujian_menyaring_permohonan_cetak_transkrip_berdasarkan_npm}
    \begin{tabular}{|c|c|p{10cm}|}
        \toprule
        Langkah ke & Hasil (sukses/tidak) & Tindakan \\

        \midrule
        1 & Sukses & Mengikuti langkah-langkah pada Tabel \ref{tab:hasil_pengujian_login} \\
        2 & Sukses & Menekan \textit{"tab"} sebanyak satu kali hingga ChromeVox menyebutkan "manajemen cetak transkrip, \textit{link list item}" lalu menekan \textit{"enter"}. \\
        3 & Sukses & Menekan \textit{"shift + alt + N"} lalu \textit{"H"} sebanyak satu kali hingga ChromeVox menyebutkan "permintaan transkrip, \textit{heading} satu" lalu menekan \textit{"tab"} sebanyak satu kali hingga ChromeVox menyebutkan "cari NPM \textit{within} 2013730013, \textit{edit text}" dilanjutkan dengan mengetik "2015730020". \\
        4 & Sukses & Menekan \textit{"tab"} sebanyak satu kali hingga ChromeVox menyebutkan "cari, \textit{button}" kemudian menekan \textit{"enter"}. \\

        \bottomrule

    \end{tabular}
\end{table}

\subsection{Skenario Memeriksa Detail Permohonan Cetak Transkrip di Halaman Manajemen Cetak Transkrip dan Hasilnya}
\label{subsec:skenario_memeriksa_detail_permohonan_cetak_transkrip_di_halaman_manajemen_cetak_transkrip}
Langkah-langkah untuk memeriksa detail permohonan cetak transkrip di halaman manajemen cetak transkrip adalah sebagai berikut:

\begin{enumerate}
    \item Lakukan \textit{login} pada halaman web BlueTape.
    \item Masuk ke halaman manajemen cetak transkrip dengan memilih menu "Manajemen Cetak Transkrip".
    \item Tekan tombol dengan ikon mata yang terdapat pada kolom "Aksi" di tabel "Permintaan Transkrip".
    \item Periksa detail permohonan.
\end{enumerate}

Berikut adalah tabel hasil pengujian untuk skenario memeriksa detail permohonan cetak transkrip di halaman manajemen cetak transkrip.

\begin{table}[H]
    \centering 
    \caption{Hasil Pengujian untuk Skenario Memeriksa Detail Permohonan Cetak Transkrip di Halaman Manajemen Cetak Transkrip}
    \label{tab:hasil_pengujian_memeriksa_detail_permohonan_cetak_transkrip_di_halaman_manajemen_cetak_transkrip}
    \begin{tabular}{|c|c|p{10cm}|}
        \toprule
        Langkah ke & Hasil (sukses/tidak) & Tindakan \\

        \midrule
        1 & Sukses & Mengikuti langkah-langkah pada Tabel \ref{tab:hasil_pengujian_login} \\
        2 & Sukses & Menekan \textit{"tab"} sebanyak satu kali hingga ChromeVox menyebutkan "manajemen cetak transkrip, \textit{link list item}" lalu menekan \textit{"enter"}. \\
        3 & Sukses & Menekan \textit{"shift + alt + N"} lalu \textit{"H"} sebanyak satu kali hingga ChromeVox menyebutkan "permintaan transkrip, \textit{heading} satu" lalu menekan \textit{"tab"} sebanyak tiga kali hingga ChromeVox menyebutkan "lihat detail permohonan Hizkia Steven, \textit{link has popup}" lalu menekan \textit{"enter"} hingga ChromeVox menyebutkan "\textit{entered dialog}". \\
        4 & Sukses & Menekan \textit{"shift + alt + right arrow"} hingga ChromeVox menyebutkan "tabel, \textit{email} pemohon, 7315020@student.unpar.ac.id" lalu menekan \textit{"shift + alt + R"} sehingga ChromeVox membacakan seluruh isi detail permohonan yang dipilih. \\

        \bottomrule

    \end{tabular}
\end{table}

\subsection{Skenario Menolak Permohonan Cetak Transkrip di Halaman Manajemen Cetak Transkrip dan Hasilnya}
\label{subsec:skenario_menolak_permohonan_cetak_transkrip_di_halaman_manajemen_cetak_transkrip}
Langkah-langkah untuk menolak permohonan cetak transkrip di halaman manajemen cetak transkrip adalah sebagai berikut:

\begin{enumerate}
    \item Lakukan \textit{login} pada halaman web BlueTape.
    \item Masuk ke halaman manajemen cetak transkrip dengan memilih menu "Manajemen Cetak Transkrip".
    \item Tekan tombol dengan ikon ibu jari mengarah ke bawah yang terdapat pada tabel "Permintaan Transkrip".
    \item Isi kolom "Alasan Penolakan" dengan "Tes Tolak Permohonan".
    \item Tekan tombol "Tolak".
\end{enumerate}

Berikut adalah tabel hasil pengujian untuk skenario menolak permohonan cetak transkrip di halaman manajemen cetak transkrip.

\begin{table}[H]
    \centering 
    \caption{Hasil Pengujian untuk Skenario Menolak Permohonan Cetak Transkrip di Halaman Manajemen Cetak Transkrip}
    \label{tab:hasil_pengujian_menolak_permohonan_cetak_transkrip_di_halaman_manajemen_cetak_transkrip}
    \begin{tabular}{|c|c|p{10cm}|}
        \toprule
        Langkah ke & Hasil (sukses/tidak) & Tindakan \\

        \midrule
        1 & Sukses & Mengikuti langkah-langkah pada Tabel \ref{tab:hasil_pengujian_login} \\
        2 & Sukses & Menekan \textit{"tab"} sebanyak satu kali hingga ChromeVox menyebutkan "manajemen cetak transkrip, \textit{link list item}" lalu menekan \textit{"enter"}. \\
        3 & Sukses & Menekan \textit{"shift + alt + N"} lalu \textit{"H"} sebanyak satu kali hingga ChromeVox menyebutkan "permintaan transkrip, \textit{heading} satu" lalu menekan \textit{"tab"} sebanyak empat kali hingga ChromeVox menyebutkan "tolak permohonan Hizkia Steven, \textit{link has popup}" lalu menekan \textit{"enter"} hingga ChromeVox menyebutkan "\textit{entered dialog}". \\
        4 & Sukses & Menekan \textit{"shift + alt + right arrow"} sebanyak tiga kali hingga ChromeVox menyebutkan "alasan penolakan, \textit{edit text}" dilanjutkan dengan mengetik "Tes Tolak Permohonan". \\
        5 & Sukses & Menekan \textit{"shift + alt + right arrow"} sebanyak satu kali hingga ChromeVox menyebutkan "tolak, \textit{button}" lalu menekan \textit{"enter"}. \\ 

        \bottomrule

    \end{tabular}
\end{table}

\subsection{Skenario Menyetujui Permohonan Cetak Transkrip di Halaman Manajemen Cetak Transkrip dan Hasilnya}
\label{subsec:skenario_menyetujui_permohonan_cetak_transkrip_di_halaman_manajemen_cetak_transkrip}
Langkah-langkah untuk menyetujui permohonan cetak transkrip di halaman manajemen cetak transkrip adalah sebagai berikut:

\begin{enumerate}
    \item Lakukan \textit{login} pada halaman web BlueTape.
    \item Masuk ke halaman manajemen cetak transkrip dengan memilih menu "Manajemen Cetak Transkrip".
    \item Tekan tombol dengan ikon mesin cetak yang terdapat pada tabel "Permintaan Transkrip".
    \item Isi kolom "Keterangan Tambahan" dengan "Tes Cetak Permohonan".
    \item Tekan tombol "Sudah dicetak".
\end{enumerate}

Berikut adalah tabel hasil pengujian untuk skenario menyetujui permohonan cetak transkrip di halaman manajemen cetak transkrip.

\begin{table}[H]
    \centering 
    \caption{Hasil Pengujian untuk Skenario Menyetujui Permohonan Cetak Transkrip di Halaman Manajemen Cetak Transkrip}
    \label{tab:hasil_pengujian_menyetujui_permohonan_cetak_transkrip_di_halaman_manajemen_cetak_transkrip}
    \begin{tabular}{|c|c|p{10cm}|}
        \toprule
        Langkah ke & Hasil (sukses/tidak) & Tindakan \\

        \midrule
        1 & Sukses & Mengikuti langkah-langkah pada Tabel \ref{tab:hasil_pengujian_login} \\
        2 & Sukses & Menekan \textit{"tab"} sebanyak satu kali hingga ChromeVox menyebutkan "manajemen cetak transkrip, \textit{link list item}" lalu menekan \textit{"enter"}. \\
        3 & Sukses & Menekan \textit{"shift + alt + N"} lalu \textit{"H"} sebanyak satu kali hingga ChromeVox menyebutkan "permintaan transkrip, \textit{heading} satu" lalu menekan \textit{"tab"} sebanyak lima kali hingga ChromeVox menyebutkan "cetak permohonan Hizkia Steven, \textit{link has popup}" lalu menekan \textit{"enter"} hingga ChromeVox menyebutkan "\textit{entered dialog}". \\
        4 & Sukses & Menekan \textit{"shift + alt + right arrow"} sebanyak empat kali hingga ChromeVox menyebutkan "keterangan tambahan, \textit{edit text}" dilanjutkan dengan mengetik "Tes Cetak Permohonan". \\
        5 & Sukses & Menekan \textit{"shift + alt + right arrow"} sebanyak satu kali hingga ChromeVox menyebutkan "sudah dicetak, \textit{button}" lalu menekan \textit{"enter"}. \\ 

        \bottomrule

    \end{tabular}
\end{table}

\subsection{Skenario Menghapus Permohonan Cetak Transkrip di Halaman Manajemen Cetak Transkrip dan Hasilnya}
\label{subsec:skenario_menghapus_permohonan_cetak_transkrip_di_halaman_manajemen_cetak_transkrip}
Langkah-langkah untuk menghapus permohonan cetak transkrip di halaman manajemen cetak transkrip adalah sebagai berikut:

\begin{enumerate}
    \item Lakukan \textit{login} pada halaman web BlueTape.
    \item Masuk ke halaman manajemen cetak transkrip dengan memilih menu "Manajemen Cetak Transkrip".
    \item Tekan tombol dengan ikon keranjang yang terdapat pada tabel "Permintaan Transkrip".
    \item Tekan tombol "Hapus".
\end{enumerate}

Berikut adalah tabel hasil pengujian untuk skenario menghapus permohonan cetak transkrip di halaman manajemen cetak transkrip.

\begin{table}[H]
    \centering 
    \caption{Hasil Pengujian untuk Skenario Menghapus Permohonan Cetak Transkrip di Halaman Manajemen Cetak Transkrip}
    \label{tab:hasil_pengujian_menghapus_permohonan_cetak_transkrip_di_halaman_manajemen_cetak_transkrip}
    \begin{tabular}{|c|c|p{10cm}|}
        \toprule
        Langkah ke & Hasil (sukses/tidak) & Tindakan \\

        \midrule
        1 & Sukses & Mengikuti langkah-langkah pada Tabel \ref{tab:hasil_pengujian_login} \\
        2 & Sukses & Menekan \textit{"tab"} sebanyak satu kali hingga ChromeVox menyebutkan "manajemen cetak transkrip, \textit{link list item}" lalu menekan \textit{"enter"}. \\
        3 & Sukses & Menekan \textit{"shift + alt + N"} lalu \textit{"H"} sebanyak satu kali hingga ChromeVox menyebutkan "permintaan transkrip, \textit{heading} satu" lalu menekan \textit{"tab"} sebanyak enam kali hingga ChromeVox menyebutkan "hapus permohonan Hizkia Steven, \textit{link has popup}" lalu menekan \textit{"enter"} hingga ChromeVox menyebutkan "\textit{entered dialog}". \\
        4 & Sukses & Menekan \textit{"shift + alt + right arrow"} sebanyak empat kali hingga ChromeVox menyebutkan "hapus, \textit{button}" lalu menekan \textit{"enter"}. \\ 

        \bottomrule

    \end{tabular}
\end{table}

\subsection{Skenario Membuat Permohonan Perubahan Kuliah di Halaman Perubahan Kuliah dan Hasilnya}
\label{subsec:skenario_membuat_permohonan_perubahan_kuliah_di_halaman_perubahan_kuliah}
Langkah-langkah untuk membuat permohonan perubahan kuliah di halaman perubahan kuliah adalah sebagai berikut:

\begin{enumerate}
    \item Lakukan \textit{login} pada halaman web BlueTape.
    \item Masuk ke halaman perubahan kuliah dengan memilih menu "Perubahan Kuliah".
    \item Isi kolom "Kode MK" dengan "TES123".
    \item Isi kolom "Nama Mata Kuliah" dengan "Tes Mata Kuliah".
    \item Isi kolom "Kelas" dengan "A".
    \item Pilih opsi "Diganti" pada kolom "Jenis Perubahan".
    \item Isi kolom "Dari Hari \& Jam" dengan "2020-04-10 09:00".
    \item Isi kolom "Dari Ruang" dengan "10316".
    \item Isi kolom "Keterangan Tambahan" dengan "Tes Permohonan Perubahan Kuliah".
    \item Isi kolom "Menjadi Hari \& Jam" dengan "2020-04-10 10:00".
    \item Isi kolom "Menjadi Ruang" dengan "10317".
    \item Tekan tombol "Kirim Permohonan".
\end{enumerate}

Berikut adalah tabel hasil pengujian untuk skenario membuat permohonan perubahan kuliah di halaman perubahan kuliah.

\begin{table}[H]
    \centering 
    \caption{Hasil Pengujian untuk Skenario Membuat Permohonan Perubahan Kuliah di Halaman Perubahan Kuliah}
    \label{tab:hasil_pengujian_membuat_permohonan_perubahan_kuliah_di_halaman_perubahan_kuliah}
    \begin{tabular}{|c|c|p{10cm}|}
        \toprule
        Langkah ke & Hasil (sukses/tidak) & Tindakan \\

        \midrule
        1 & Sukses & Mengikuti langkah-langkah pada Tabel \ref{tab:hasil_pengujian_login} \\
        2 & Sukses & Menekan \textit{"tab"} sebanyak dua kali hingga ChromeVox menyebutkan "perubahan kuliah, \textit{link list item}" lalu menekan \textit{"enter"}. \\
        3 & Sukses & Menekan \textit{"shift + alt + N"} lalu \textit{"H"} sebanyak satu kali hingga ChromeVox menyebutkan "permohonan baru, \textit{heading} satu" lalu menekan \textit{"tab"} sebanyak tiga kali hingga ChromeVox menyebutkan "kode mk, \textit{edit text}" dan dilanjutkan dengan mengetik "TES123" untuk mengisi kolom tersebut. \\
        4 & Sukses & Menekan \textit{"tab"} sebanyak satu kali hingga ChromeVox menyebutkan "nama mata kuliah, \textit{edit text}" dan dilanjutkan dengan mengetik "TES123" untuk mengisi kolom tersebut. \\
        5 & Sukses & Menekan \textit{"tab"} sebanyak satu kali hingga ChromeVox menyebutkan "kelas, \textit{edit text}" dan dilanjutkan dengan mengetik "A" untuk mengisi kolom tersebut. \\
        6 & Sukses & Menekan \textit{"tab"} sebanyak satu kali hingga ChromeVox menyebutkan "jenis perubahan, diganti, \textit{combo box satu of tiga}". \\
        7 & Sukses & Menekan \textit{"tab"} sebanyak satu kali hingga ChromeVox menyebutkan "dari hari dan jam, \textit{edit text}" dan dilanjutkan dengan mengetik "2020-04-10 09:00" untuk mengisi kolom tersebut. \\
        8 & Sukses & Menekan \textit{"tab"} sebanyak satu kali hingga ChromeVox menyebutkan "dari ruang, \textit{edit text}" dan dilanjutkan dengan mengetik "10316" untuk mengisi kolom tersebut. \\
        9 & Sukses & Menekan \textit{"tab"} sebanyak satu kali hingga ChromeVox menyebutkan "keterangan tambahan, \textit{edit text}" dan dilanjutkan dengan mengetik "Tes Permohonan Perubahan Kuliah" untuk mengisi kolom tersebut. \\
        10 & Sukses & Menekan \textit{"tab"} sebanyak satu kali hingga ChromeVox menyebutkan "menjadi hari dan jam, \textit{edit text}" dan dilanjutkan dengan mengetik "2020-04-10 10:00" untuk mengisi kolom tersebut. \\
        11 & Sukses & Menekan \textit{"tab"} sebanyak satu kali hingga ChromeVox menyebutkan "menjadi ruang, \textit{edit text}" dan dilanjutkan dengan mengetik "10317" untuk mengisi kolom tersebut. \\
        12 & Sukses & Menekan \textit{"tab"} sebanyak satu kali hingga ChromeVox menyebutkan "kirim permohonan, \textit{button}" lalu menekan \textit{"enter"}. \\

        \bottomrule

    \end{tabular}
\end{table}

\subsection{Skenario Memeriksa Detail Permohonan Perubahan Kuliah di Halaman Perubahan Kuliah dan Hasilnya}
\label{subsec:skenario_memeriksa_detail_permohonan_perubahan_kuliah_di_halaman_perubahan_kuliah}
Langkah-langkah untuk memeriksa detail permohonan perubahan kuliah di halaman perubahan kuliah adalah sebagai berikut:

\begin{enumerate}
    \item Lakukan \textit{login} pada halaman web BlueTape.
    \item Masuk ke halaman perubahan kuliah dengan memilih menu "Perubahan Kuliah".
    \item Tekan tombol dengan ikon mata yang terdapat pada kolom "Aksi" di tabel "Histori Permohonan".
    \item Periksa detail permohonan.
\end{enumerate}

Berikut adalah tabel hasil pengujian untuk skenario memeriksa detail permohonan perubahan kuliah di halaman perubahan kuliah.

\begin{table}[H]
    \centering 
    \caption{Hasil Pengujian untuk Skenario Memeriksa Detail Permohonan Perubahan Kuliah di Halaman Perubahan Kuliah}
    \label{tab:hasil_pengujian_memeriksa_detail_permohonan_perubahan_kuliah_di_halaman_perubahan_kuliah}
    \begin{tabular}{|c|c|p{10cm}|}
        \toprule
        Langkah ke & Hasil (sukses/tidak) & Tindakan \\

        \midrule
        1 & Sukses & Mengikuti langkah-langkah pada Tabel \ref{tab:hasil_pengujian_login} \\
        2 & Sukses & Menekan \textit{"tab"} sebanyak dua kali hingga ChromeVox menyebutkan "perubahan kuliah, \textit{link list item}" lalu menekan \textit{"enter"}. \\
        3 & Sukses & Menekan \textit{"shift + alt + N"} lalu \textit{"H"}, dilakukan sebanyak dua kali hingga ChromeVox menyebutkan "histori permohonan, \textit{heading} satu". Setelah itu menekan \textit{"tab"} sebanyak satu kali hingga ChromeVox menyebutkan "lihat detail permohonan Hizkia Steven, \textit{link has popup}" lalu menekan \textit{"enter"} hingga ChromeVox menyebutkan "\textit{entered dialog}". \\
        4 & Sukses & Menekan \textit{"shift + alt + right arrow"} hingga ChromeVox menyebutkan "tabel, \textit{email} pemohon, 7315020@student.unpar.ac.id" lalu menekan \textit{"shift + alt + R"} sehingga ChromeVox membacakan seluruh isi detail permohonan yang dipilih. \\

        \bottomrule

    \end{tabular}
\end{table}

\subsection{Skenario Memeriksa Detail Permohonan Perubahan Kuliah di Halaman Manajemen Perubahan Kuliah dan Hasilnya}
\label{subsec:skenario_memeriksa_detail_permohonan_perubahan_kuliah_di_halaman_manajemen_perubahan_kuliah}
Langkah-langkah untuk memeriksa detail permohonan perubahan kuliah di halaman manajemen perubahan kuliah adalah sebagai berikut:

\begin{enumerate}
    \item Lakukan \textit{login} pada halaman web BlueTape.
    \item Masuk ke halaman manajemen perubahan kuliah dengan memilih menu "Manajemen Perubahan Kuliah".
    \item Tekan tombol dengan ikon mata yang terdapat pada tabel "Permohonan Perubahan Kuliah".
    \item Periksa detail permohonan.
\end{enumerate}

Berikut adalah tabel hasil pengujian untuk skenario memeriksa detail permohonan perubahan kuliah di halaman manajemen perubahan kuliah.

\begin{table}[H]
    \centering 
    \caption{Hasil Pengujian untuk Skenario Memeriksa Detail Permohonan Perubahan Kuliah di Halaman Manajemen Perubahan Kuliah}
    \label{tab:hasil_pengujian_memeriksa_detail_permohonan_perubahan_kuliah_di_halaman_manajemen_perubahan_kuliah}
    \begin{tabular}{|c|c|p{10cm}|}
        \toprule
        Langkah ke & Hasil (sukses/tidak) & Tindakan \\

        \midrule
        1 & Sukses & Mengikuti langkah-langkah pada Tabel \ref{tab:hasil_pengujian_login} \\
        2 & Sukses & Menekan \textit{"tab"} sebanyak tiga kali hingga ChromeVox menyebutkan "manajemen perubahan kuliah, \textit{link list item}" lalu menekan \textit{"enter"}. \\
        3 & Sukses & Menekan \textit{"shift + alt + N"} lalu \textit{"H"} sebanyak satu kali hingga ChromeVox menyebutkan "permohonan perubahan kuliah, \textit{heading} satu" lalu menekan \textit{"tab"} sebanyak satu kali hingga ChromeVox menyebutkan "lihat detail permohonan Hizkia Steven, \textit{link has popup}" lalu menekan \textit{"enter"} hingga ChromeVox menyebutkan "\textit{entered dialog}". \\
        4 & Sukses & Menekan \textit{"shift + alt + right arrow"} hingga ChromeVox menyebutkan "tabel, \textit{email} pemohon, 7315020@student.unpar.ac.id" lalu menekan \textit{"shift + alt + R"} sehingga ChromeVox membacakan seluruh isi detail permohonan yang dipilih. \\

        \bottomrule

    \end{tabular}
\end{table}

\subsection{Skenario Membuka dan Memeriksa Halaman \textit{Print-Out} Pengumuman Perubahan Kuliah dan Hasilnya}
\label{subsec:skenario_membuka_dan_memeriksa_halaman_print_out_pengumuman_perubahan_kuliah}
Langkah-langkah untuk membuka dan memeriksa halaman \textit{print-out} pengumuman perubahan kuliah adalah sebagai berikut:

\begin{enumerate}
    \item Lakukan \textit{login} pada halaman web BlueTape.
    \item Masuk ke halaman manajemen perubahan kuliah dengan memilih menu "Manajemen Perubahan Kuliah".
    \item Tekan tombol dengan ikon mesin cetak yang terdapat pada tabel "Permohonan Perubahan Kuliah".
    \item Periksa isi \textit{print-out} pengumuman perubahan kuliah.
\end{enumerate}

Berikut adalah tabel hasil pengujian untuk skenario membuka dan memeriksa halaman \textit{print-out} pengumuman perubahan kuliah.

\begin{table}[H]
    \centering 
    \caption{Hasil Pengujian untuk Skenario Membuka dan Memeriksa Halaman \textit{Print-Out} Pengumuman Perubahan Kuliah}
    \label{tab:hasil_pengujian_membuka_dan_memeriksa_halaman_print_out_pengumuman_perubahan_kuliah}
    \begin{tabular}{|c|c|p{10cm}|}
        \toprule
        Langkah ke & Hasil (sukses/tidak) & Tindakan \\

        \midrule
        1 & Sukses & Mengikuti langkah-langkah pada Tabel \ref{tab:hasil_pengujian_login} \\
        2 & Sukses & Menekan \textit{"tab"} sebanyak tiga kali hingga ChromeVox menyebutkan "manajemen perubahan kuliah, \textit{link list item}" lalu menekan \textit{"enter"}. \\
        3 & Sukses & Menekan \textit{"shift + alt + N"} lalu \textit{"H"} sebanyak satu kali hingga ChromeVox menyebutkan "permohonan perubahan kuliah, \textit{heading} satu" lalu menekan \textit{"tab"} sebanyak dua kali hingga ChromeVox menyebutkan "cetak permohonan Hizkia Steven, \textit{link}" lalu menekan \textit{"enter"} sehingga berpindah halaman dan mendengar ChromeVox menyebutkan "pengumuman jadwal kuliah, pengumuman, \textit{heading} satu". \\
        4 & Sukses & Menekan \textit{"shift + alt + R"} sehingga ChromeVox membacakan seluruh isi \textit{print-out} pengumuman perubahan kuliah. \\ 

        \bottomrule

    \end{tabular}
\end{table}

\subsection{Skenario Mengonfirmasi Permohonan Perubahan Kuliah di Halaman Manajemen Perubahan Kuliah dan Hasilnya}
\label{subsec:skenario_mengonfirmasi_permohonan_perubahan_kuliah_di_halaman_manajemen_perubahan_kuliah}
Langkah-langkah untuk mengonfirmasi permohonan perubahan kuliah di halaman manajemen perubahan kuliah adalah sebagai berikut:

\begin{enumerate}
    \item Lakukan \textit{login} pada halaman web BlueTape.
    \item Masuk ke halaman manajemen perubahan kuliah dengan memilih menu "Manajemen Perubahan Kuliah".
    \item Tekan tombol dengan ikon ibu jari mengarah ke atas yang terdapat pada tabel "Permohonan Perubahan Kuliah".
    \item Isi kolom "Keterangan" dengan "Tes Kofirmasi Permohonan".
    \item Tekan tombol "Konfirmasi".
\end{enumerate}

Berikut adalah tabel hasil pengujian untuk skenario mengonfirmasi permohonan perubahan kuliah di halaman manajemen perubahan kuliah.

\begin{table}[H]
    \centering 
    \caption{Hasil Pengujian untuk Skenario Mengonfirmasi Permohonan Perubahan Kuliah di Halaman Manajemen Perubahan Kuliah}
    \label{tab:hasil_pengujian_mengonfirmasi_permohonan_perubahan_kuliah_di_halaman_manajemen_perubahan_kuliah}
    \begin{tabular}{|c|c|p{10cm}|}
        \toprule
        Langkah ke & Hasil (sukses/tidak) & Tindakan \\

        \midrule
        1 & Sukses & Mengikuti langkah-langkah pada Tabel \ref{tab:hasil_pengujian_login} \\
        2 & Sukses & Menekan \textit{"tab"} sebanyak tiga kali hingga ChromeVox menyebutkan "manajemen perubahan kuliah, \textit{link list item}" lalu menekan \textit{"enter"}. \\
        3 & Sukses & Menekan \textit{"shift + alt + N"} lalu \textit{"H"} sebanyak satu kali hingga ChromeVox menyebutkan "permohonan perubahan kuliah, \textit{heading} satu" lalu menekan \textit{"tab"} sebanyak tiga kali hingga ChromeVox menyebutkan "konfirmasi permohonan Hizkia Steven, \textit{link has popup}" lalu menekan \textit{"enter"} hingga ChromeVox menyebutkan "\textit{entered dialog}". \\
        4 & Sukses & Menekan \textit{"shift + alt + right arrow"} sebanyak tiga kali hingga ChromeVox menyebutkan "keterangan, \textit{edit text}" dilanjutkan dengan mengetik "Tes Konfirmasi Permohonan". \\
        5 & Sukses & Menekan \textit{"shift + alt + right arrow"} sebanyak satu kali hingga ChromeVox menyebutkan "konfirmasi, \textit{button}" lalu menekan \textit{"enter"}. \\ 

        \bottomrule

    \end{tabular}
\end{table}

\subsection{Skenario Menolak Permohonan Perubahan Kuliah di Halaman Manajemen Perubahan Kuliah dan Hasilnya}
\label{subsec:skenario_menolak_permohonan_perubahan_kuliah_di_halaman_manajemen_perubahan_kuliah}
Langkah-langkah untuk menolak permohonan perubahan kuliah di halaman manajemen perubahan kuliah adalah sebagai berikut:

\begin{enumerate}
    \item Lakukan \textit{login} pada halaman web BlueTape.
    \item Masuk ke halaman manajemen perubahan kuliah dengan memilih menu "Manajemen Perubahan Kuliah".
    \item Tekan tombol dengan ikon ibu jari mengarah ke bawah yang terdapat pada tabel "Permohonan Perubahan Kuliah".
    \item Isi kolom "Alasan penolakan" dengan "Tes Tolak Permohonan".
    \item Tekan tombol "Tolak".
\end{enumerate}

Berikut adalah tabel hasil pengujian untuk skenario menolak permohonan perubahan kuliah di halaman manajemen perubahan kuliah.

\begin{table}[H]
    \centering 
    \caption{Hasil Pengujian untuk Skenario Menolak Permohonan Perubahan Kuliah di Halaman Manajemen Perubahan Kuliah}
    \label{tab:hasil_pengujian_menolak_permohonan_perubahan_kuliah_di_halaman_manajemen_perubahan_kuliah}
    \begin{tabular}{|c|c|p{10cm}|}
        \toprule
        Langkah ke & Hasil (sukses/tidak) & Tindakan \\

        \midrule
        1 & Sukses & Mengikuti langkah-langkah pada Tabel \ref{tab:hasil_pengujian_login} \\
        2 & Sukses & Menekan \textit{"tab"} sebanyak tiga kali hingga ChromeVox menyebutkan "manajemen perubahan kuliah, \textit{link list item}" lalu menekan \textit{"enter"}. \\
        3 & Sukses & Menekan \textit{"shift + alt + N"} lalu \textit{"H"} sebanyak satu kali hingga ChromeVox menyebutkan "permohonan perubahan kuliah, \textit{heading} satu" lalu menekan \textit{"tab"} sebanyak empat kali hingga ChromeVox menyebutkan "tolak permohonan Hizkia Steven, \textit{link has popup}" lalu menekan \textit{"enter"} hingga ChromeVox menyebutkan "\textit{entered dialog}". \\
        4 & Sukses & Menekan \textit{"shift + alt + right arrow"} sebanyak tiga kali hingga ChromeVox menyebutkan "alasan penolakan, \textit{edit text}" dilanjutkan dengan mengetik "Tes Tolak Permohonan". \\
        5 & Sukses & Menekan \textit{"shift + alt + right arrow"} sebanyak satu kali hingga ChromeVox menyebutkan "tolak, \textit{button}" lalu menekan \textit{"enter"}. \\ 

        \bottomrule

    \end{tabular}
\end{table}

\subsection{Skenario Menghapus Permohonan Perubahan Kuliah di Halaman Manajemen Perubahan Kuliah dan Hasilnya}
\label{subsec:skenario_menghapus_permohonan_perubahan_kuliah_di_halaman_manajemen_perubahan_kuliah}
Langkah-langkah untuk menghapus permohonan perubahan kuliah di halaman manajemen perubahan kuliah adalah sebagai berikut:

\begin{enumerate}
    \item Lakukan \textit{login} pada halaman web BlueTape.
    \item Masuk ke halaman manajemen perubahan kuliah dengan memilih menu "Manajemen Perubahan Kuliah".
    \item Tekan tombol dengan ikon keranjang yang terdapat pada tabel "Permohonan Perubahan Kuliah".
    \item Tekan tombol "Hapus".
\end{enumerate}

Berikut adalah tabel hasil pengujian untuk skenario menghapus permohonan perubahan kuliah di halaman manajemen perubahan kuliah.

\begin{table}[H]
    \centering 
    \caption{Hasil Pengujian untuk Skenario Menghapus Permohonan Perubahan Kuliah di Halaman Manajemen Perubahan Kuliah}
    \label{tab:hasil_pengujian_menghapus_permohonan_perubahan_kuliah_di_halaman_manajemen_perubahan_kuliah}
    \begin{tabular}{|c|c|p{10cm}|}
        \toprule
        Langkah ke & Hasil (sukses/tidak) & Tindakan \\

        \midrule
        1 & Sukses & Mengikuti langkah-langkah pada Tabel \ref{tab:hasil_pengujian_login} \\
        2 & Sukses & Menekan \textit{"tab"} sebanyak tiga kali hingga ChromeVox menyebutkan "manajemen perubahan kuliah, \textit{link list item}" lalu menekan \textit{"enter"}. \\
        3 & Sukses & Menekan \textit{"shift + alt + N"} lalu \textit{"H"} sebanyak satu kali hingga ChromeVox menyebutkan "permohonan perubahan kuliah, \textit{heading} satu" lalu menekan \textit{"tab"} sebanyak lima kali hingga ChromeVox menyebutkan "hapus permohonan Hizkia Steven, \textit{link has popup}" lalu menekan \textit{"enter"} hingga ChromeVox menyebutkan "\textit{entered dialog}". \\
        4 & Sukses & Menekan \textit{"shift + alt + right arrow"} sebanyak lima kali hingga ChromeVox menyebutkan "hapus, \textit{button}" lalu menekan \textit{"enter"}. \\ 

        \bottomrule

    \end{tabular}
\end{table}

\subsection{Skenario Menambah Jadwal Dosen di Halaman Entri Jadwal Dosen dan Hasilnya}
\label{subsec:skenario_menambah_jadwal_dosen_di_halaman_entri_jadwal_dosen}
Langkah-langkah untuk menambah jadwal dosen di halaman entri jadwal dosen adalah sebagai berikut:

\begin{enumerate}
    \item Lakukan \textit{login} pada halaman web BlueTape.
    \item Masuk ke halaman entri jadwal dosen dengan memilih menu "Entri Jadwal Dosen".
    \item Pilih opsi "Selasa" pada kolom "Hari".
    \item Pilih opsi "8:00" pada kolom "Jam Mulai".
    \item Pilih opsi "2 jam" pada kolom "Durasi".
    \item Pilih opsi "Kelas" pada kolom "Jenis".
    \item Isi kolom "Label" dengan "Tes Tambah Jadwal".
    \item Tekan tombol "Tambah" untuk menambahkan jadwal.
\end{enumerate}

Berikut adalah tabel hasil pengujian untuk skenario menambah jadwal dosen di halaman entri jadwal dosen.

\begin{table}[H]
    \centering 
    \caption{Hasil Pengujian untuk Skenario Menambah Jadwal Dosen di Halaman Entri Jadwal Dosen}
    \label{tab:hasil_pengujian_menambah_jadwal_dosen_di_halaman_entri_jadwal_dosen}
    \begin{tabular}{|c|c|p{10cm}|}
        \toprule
        Langkah ke & Hasil (sukses/tidak) & Tindakan \\

        \midrule
        1 & Sukses & Mengikuti langkah-langkah pada Tabel \ref{tab:hasil_pengujian_login} \\
        2 & Sukses & Menekan \textit{"tab"} sebanyak empat kali hingga ChromeVox menyebutkan "entri jadwal dosen, \textit{link list item}" lalu menekan \textit{"enter"}. \\
        3 & Sukses & Menekan \textit{"shift + alt + N"} lalu \textit{"H"} sebanyak satu kali hingga ChromeVox menyebutkan "tambah jadwal, \textit{heading} satu" lalu menekan \textit{"tab"} sebanyak satu kali hingga ChromeVox menyebutkan "hari, senin, \textit{combo box satu of lima}" lalu menekan \textit{"down arrow"} sebanyak satu kali hingga ChromeVox menyebutkan "selasa, \textit{dua of lima}". \\
        4 & Sukses & Menekan \textit{"tab"} sebanyak satu kali hingga ChromeVox menyebutkan "jam mulai, pukul tujuh, \textit{combo box satu of sepuluh}" lalu menekan \textit{"down arrow"} sebanyak satu kali hingga ChromeVox menyebutkan "pukul delapan, \textit{dua of sepuluh}. \\ 
        5 & Sukses & Menekan \textit{"tab"} sebanyak satu kali hingga ChromeVox menyebutkan "durasi, satu jam, \textit{combo box satu of sembilan}" lalu menekan \textit{"down arrow"} sebanyak satu kali hingga ChromeVox menyebutkan "dua jam, \textit{dua of sembilan}. \\ 
        6 & Sukses & Menekan \textit{"tab"} sebanyak satu kali hingga ChromeVox menyebutkan "jenis, konsultasi, \textit{combo box satu of tiga}" lalu menekan \textit{"down arrow"} sebanyak dua kali hingga ChromeVox menyebutkan "kelas, \textit{tiga of tiga}. \\ 
        7 & Sukses & Menekan \textit{"tab"} sebanyak satu kali hingga ChromeVox menyebutkan "label, \textit{edit text}" lalu mengetik "Tes Tambah Jadwal". \\ 
        8 & Sukses & Menekan \textit{"tab"} sebanyak satu kali hingga ChromeVox menyebutkan "tambah, \textit{button}" lalu menekan \textit{"enter"}. \\ 

        \bottomrule

    \end{tabular}
\end{table}

\subsection{Skenario Mengubah Jadwal Dosen di Halaman Entri Jadwal Dosen dan Hasilnya}
\label{subsec:skenario_mengubah_jadwal_dosen_di_halaman_entri_jadwal_dosen}
Langkah-langkah untuk mengubah jadwal dosen di halaman entri jadwal dosen adalah sebagai berikut:

\begin{enumerate}
    \item Lakukan \textit{login} pada halaman web BlueTape.
    \item Masuk ke halaman entri jadwal dosen dengan memilih menu "Entri Jadwal Dosen".
    \item Tekan jadwal dosen pada tabel "Daftar Jadwal".
    \item Pilih opsi "Senin" pada kolom "Hari".
    \item Pilih opsi "7:00" pada kolom "Jam Mulai".
    \item Pilih opsi "1 jam" pada kolom "Durasi".
    \item Pilih opsi "Konsultasi" pada kolom "Jenis".
    \item Isi kolom "Label" dengan "Tes Ubah Jadwal".
    \item Tekan tombol "\textit{Save}" untuk menyimpan perubahan.
\end{enumerate}

Berikut adalah tabel hasil pengujian untuk skenario mengubah jadwal dosen di halaman entri jadwal dosen.

\begin{table}[H]
    \centering 
    \caption{Hasil Pengujian untuk Skenario Mengubah Jadwal Dosen di Halaman Entri Jadwal Dosen}
    \label{tab:hasil_pengujian_mengubah_jadwal_dosen_di_halaman_entri_jadwal_dosen}
    \begin{tabular}{|c|c|p{10cm}|}
        \toprule
        Langkah ke & Hasil (sukses/tidak) & Tindakan \\

        \midrule
        1 & Sukses & Mengikuti langkah-langkah pada Tabel \ref{tab:hasil_pengujian_login} \\
        2 & Sukses & Menekan \textit{"tab"} sebanyak empat kali hingga ChromeVox menyebutkan "entri jadwal dosen, \textit{link list item}" lalu menekan \textit{"enter"}. \\
        3 & Sukses & Menekan \textit{"shift + alt + N"} lalu \textit{"H"}, dilakukan sebanyak dua kali hingga ChromeVox menyebutkan "daftar jadwal, \textit{heading} satu" lalu menekan \textit{"tab"} sebanyak satu kali hingga ChromeVox menyebutkan "tabel, tes tambah jadwal" lalu menekan \textit{"enter"} sebanyak satu kali hingga ChromeVox menyebutkan "\textit{entered dialog}". \\
        4 & Sukses & Menekan \textit{"shift + alt + right arrow"} sebanyak tiga kali hingga ChromeVox menyebutkan "hari, selasa, \textit{combo box dua of lima}" lalu menekan \textit{"up arrow"} sebanyak satu kali hingga ChromeVox menyebutkan "senin, \textit{satu of lima}. \\
        5 & Sukses & Menekan \textit{"shift + alt + right arrow"} sebanyak satu kali hingga ChromeVox menyebutkan "jam mulai, pukul delapan, \textit{combo box dua of sepuluh}" lalu menekan \textit{"up arrow"} sebanyak satu kali hingga ChromeVox menyebutkan "pukul tujuh, \textit{satu of sepuluh}. \\
        6 & Sukses & Menekan \textit{"shift + alt + right arrow"} sebanyak satu kali hingga ChromeVox menyebutkan "durasi, dua jam, \textit{combo box dua of sembilan}" lalu menekan \textit{"up arrow"} sebanyak satu kali hingga ChromeVox menyebutkan "satu jam, \textit{satu of sembilan}. \\
        7 & Sukses & Menekan \textit{"shift + alt + right arrow"} sebanyak satu kali hingga ChromeVox menyebutkan "jenis, kelas, \textit{combo box tiga of tiga}" lalu menekan \textit{"up arrow"} sebanyak dua kali hingga ChromeVox menyebutkan "konsultasi, \textit{satu of tiga}. \\
        8 & Sukses & Menekan \textit{"shift + alt + right arrow"} sebanyak satu kali hingga ChromeVox menyebutkan "label, tes tambah jadwal, \textit{edit text}" lalu mengetik "Tes Ubah Jadwal". \\ 
        9 & Sukses & Menekan \textit{"shift + alt + right arrow"} sebanyak satu kali hingga ChromeVox menyebutkan "\textit{save, button}" lalu menekan \textit{"enter"}. \\ 

        \bottomrule

    \end{tabular}
\end{table}

\subsection{Skenario Menghapus Jadwal Dosen di Halaman Entri Jadwal Dosen dan Hasilnya}
\label{subsec:skenario_menghapus_jadwal_dosen_di_halaman_entri_jadwal_dosen}
Langkah-langkah untuk menghapus jadwal dosen di halaman entri jadwal dosen adalah sebagai berikut:

\begin{enumerate}
    \item Lakukan \textit{login} pada halaman web BlueTape.
    \item Masuk ke halaman entri jadwal dosen dengan memilih menu "Entri Jadwal Dosen".
    \item Tekan jadwal dosen pada tabel "Daftar Jadwal".
    \item Tekan tombol "\textit{Delete}".
\end{enumerate}

Berikut adalah tabel hasil pengujian untuk skenario menghapus jadwal dosen di halaman entri jadwal dosen.

\begin{table}[H]
    \centering 
    \caption{Hasil Pengujian untuk Skenario Menghapus Jadwal Dosen di Halaman Entri Jadwal Dosen}
    \label{tab:hasil_pengujian_menghapus_jadwal_dosen_di_halaman_entri_jadwal_dosen}
    \begin{tabular}{|c|c|p{10cm}|}
        \toprule
        Langkah ke & Hasil (sukses/tidak) & Tindakan \\

        \midrule
        1 & Sukses & Mengikuti langkah-langkah pada Tabel \ref{tab:hasil_pengujian_login} \\
        2 & Sukses & Menekan \textit{"tab"} sebanyak empat kali hingga ChromeVox menyebutkan "entri jadwal dosen, \textit{link list item}" lalu menekan \textit{"enter"}. \\
        3 & Sukses & Menekan \textit{"shift + alt + N"} lalu \textit{"H"}, dilakukan sebanyak dua kali hingga ChromeVox menyebutkan "daftar jadwal, \textit{heading} satu" lalu menekan \textit{"tab"} sebanyak satu kali hingga ChromeVox menyebutkan "tabel, tes ubah jadwal" lalu menekan \textit{"enter"} sebanyak satu kali hingga ChromeVox menyebutkan "\textit{entered dialog}". \\
        4 & Sukses & Menekan \textit{"shift + alt + right arrow"} sebanyak sepuluh kali hingga ChromeVox menyebutkan "\textit{delete, button}" lalu menekan \textit{"enter"}. \\

        \bottomrule

    \end{tabular}
\end{table}

\subsection{Skenario Menghapus Seluruh Jadwal Dosen di Halaman Entri Jadwal Dosen dan Hasilnya}
\label{subsec:skenario_menghapus_seluruh_jadwal_dosen_di_halaman_entri_jadwal_dosen}
Langkah-langkah untuk menghapus seluruh jadwal dosen di halaman entri jadwal dosen adalah sebagai berikut:

\begin{enumerate}
    \item Lakukan \textit{login} pada halaman web BlueTape.
    \item Masuk ke halaman entri jadwal dosen dengan memilih menu "Entri Jadwal Dosen".
    \item Tekan tombol "\textit{Delete All}".
    \item Periksa pesan konfirmasi penghapusan jadwal pada dialog peringatan yang muncul.
    \item Lakukan konfirmasi penghapusan dengan menekan tombol "OK".
\end{enumerate}

Berikut adalah tabel hasil pengujian untuk skenario menghapus seluruh jadwal dosen di halaman entri jadwal dosen.

\begin{table}[H]
    \centering 
    \caption{Hasil Pengujian untuk Skenario Menghapus Seluruh Jadwal Dosen di Halaman Entri Jadwal Dosen}
    \label{tab:hasil_pengujian_menghapus_seluruh_jadwal_dosen_di_halaman_entri_jadwal_dosen}
    \begin{tabular}{|c|c|p{10cm}|}
        \toprule
        Langkah ke & Hasil (sukses/tidak) & Tindakan \\

        \midrule
        1 & Sukses & Mengikuti langkah-langkah pada Tabel \ref{tab:hasil_pengujian_login} \\
        2 & Sukses & Menekan \textit{"tab"} sebanyak empat kali hingga ChromeVox menyebutkan "entri jadwal dosen, \textit{link list item}" lalu menekan \textit{"enter"}. \\
        3 & Sukses & Menekan \textit{"shift + alt + N"} lalu \textit{"H"}, dilakukan sebanyak dua kali hingga ChromeVox menyebutkan "daftar jadwal, \textit{heading} satu" lalu menekan \textit{"shift + alt + N"} lalu \textit{"L"} sebanyak satu kali hingga ChromeVox menyebutkan "\textit{delete all, link}" lalu menekan \textit{"enter"}. \\
        4 & Sukses & Menekan \textit{"shift + alt + R"} sebanyak satu kali sehingga ChromeVox membacakan pesan "anda yakin mau menghapus semua data jadwal? aksi ini tidak dapat dibatalkan" pada dialog peringatan. \\
        5 & Sukses & Menekan \textit{"enter"}. \\

        \bottomrule

    \end{tabular}
\end{table}

\subsection{Skenario Membuat Dokumen XLS yang Berisi Jadwal Dosen di Halaman Entri Jadwal Dosen dan Hasilnya}
\label{subsec:skenario_membuat_dokumen_xls_yang_berisi_jadwal_dosen_di_halaman_entri_jadwal_dosen}
Langkah-langkah untuk membuat dokumen XLS yang berisi jadwal dosen di halaman entri jadwal dosen adalah sebagai berikut:

\begin{enumerate}
    \item Lakukan \textit{login} pada halaman web BlueTape.
    \item Masuk ke halaman entri jadwal dosen dengan memilih menu "Entri Jadwal Dosen".
    \item Tekan tombol "Ekspor ke XLS".
\end{enumerate}

Berikut adalah tabel hasil pengujian untuk skenario membuat dokumen XLS yang berisi jadwal dosen di halaman entri jadwal dosen.

\begin{table}[H]
    \centering 
    \caption{Hasil Pengujian untuk Skenario Membuat Dokumen XLS yang Berisi Jadwal Dosen di Halaman Entri Jadwal Dosen}
    \label{tab:hasil_pengujian_membuat_dokumen_xls_yang_berisi_jadwal_dosen_di_halaman_entri_jadwal_dosen}
    \begin{tabular}{|c|c|p{10cm}|}
        \toprule
        Langkah ke & Hasil (sukses/tidak) & Tindakan \\

        \midrule
        1 & Sukses & Mengikuti langkah-langkah pada Tabel \ref{tab:hasil_pengujian_login} \\
        2 & Sukses & Menekan \textit{"tab"} sebanyak empat kali hingga ChromeVox menyebutkan "entri jadwal dosen, \textit{link list item}" lalu menekan \textit{"enter"}. \\
        3 & Sukses & Menekan \textit{"shift + alt + N"} lalu \textit{"H"}, dilakukan sebanyak dua kali hingga ChromeVox menyebutkan "daftar jadwal, \textit{heading} satu" lalu menekan \textit{"shift + alt + N"} lalu \textit{"L"}, dilakukan sebanyak dua kali hingga ChromeVox menyebutkan "\textit{ekspor ke XLS, link}" lalu menekan \textit{"enter"}. \\

        \bottomrule

    \end{tabular}
\end{table}

\subsection{Skenario Memeriksa Jadwal Dosen di Halaman Lihat Jadwal Dosen dan Hasilnya}
\label{subsec:skenario_memeriksa_jadwal_dosen_di_halaman_lihat_jadwal_dosen}
Langkah-langkah untuk memeriksa jadwal dosen di halaman lihat jadwal dosen adalah sebagai berikut:

\begin{enumerate}
    \item Lakukan \textit{login} pada halaman web BlueTape.
    \item Masuk ke halaman lihat jadwal dosen dengan memilih menu "Lihat Jadwal Dosen".
    \item Pilih nama dosen yang ingin diketahui jadwalnya.
    \item Telusuri tabel yang menampilkan jadwal dosen yang telah dipilih.
    \item Periksa detail jadwal dengan nama "Tes Jadwal".
\end{enumerate}

Berikut adalah tabel hasil pengujian untuk skenario memeriksa jadwal dosen di halaman lihat jadwal dosen.

\begin{table}[H]
    \centering 
    \caption{Hasil Pengujian untuk Skenario Memeriksa Jadwal Dosen di Halaman Lihat Jadwal Dosen}
    \label{tab:hasil_pengujian_memeriksa_jadwal_dosen_di_halaman_lihat_jadwal_dosen}
    \begin{tabular}{|c|c|p{10cm}|}
        \toprule
        Langkah ke & Hasil (sukses/tidak) & Tindakan \\

        \midrule
        1 & Sukses & Mengikuti langkah-langkah pada Tabel \ref{tab:hasil_pengujian_login} \\
        2 & Sukses & Menekan \textit{"tab"} sebanyak lima kali hingga ChromeVox menyebutkan "lihat jadwal dosen, \textit{link list item}" lalu menekan \textit{"enter"}. \\
        3 & Sukses & Menekan \textit{"shift + alt + L"} lalu \textit{"L"} sebanyak satu kali hingga ChromeVox menyebutkan "\textit{link list, use up and down arrow keys to browse or type to search}" lalu mengetik "hizkia steven", dilanjutkan dengan menekan \textit{"enter"} hingga ChromeVox menyebutkan "\textit{exited, link list, list with dua items, hizkia steven, tab selected list item}" lalu menekan \textit{"enter"}. \\
        4 & Suskes & Menekan \textit{"shift + alt + N"} lalu \textit{"T"} sebanyak satu kali hingga ChromeVox menyebutkan "tabel, senin" lalu menekan \textit{"shift + alt + R"} sehingga ChromeVox membacakan isi tabel. \\
        5 & Sukses & Menekan \textit{"ctrl"} saat ChromeVox menyebutkan "Tes Jadwal" agar ChromeVox berhenti membacakan isi tabel dan berhenti di \textit{cell} "Tes Jadwal". Berikutnya menekan \textit{"shift + alt + back slash"} untuk masuk ke mode tabel, dilanjutkan dengan menekan \textit{"shift + alt + T"} lalu \textit{"H"} sebanyak satu kali hingga ChromeVox menyebutkan "\textit{row header tujuh delapan, column header senin}". \\

        \bottomrule

    \end{tabular}
\end{table}

\subsection{Skenario Membuat Dokumen XLS yang Berisi Jadwal Dosen di Halaman Lihat Jadwal Dosen dan Hasilnya}
\label{subsec:skenario_membuat_dokumen_xls_yang_berisi_jadwal_dosen_di_halaman_lihat_jadwal_dosen}
Langkah-langkah untuk membuat dokumen XLS yang berisi jadwal dosen di halaman lihat jadwal dosen adalah sebagai berikut:

\begin{enumerate}
    \item Lakukan \textit{login} pada halaman web BlueTape.
    \item Masuk ke halaman lihat jadwal dosen dengan memilih menu "Lihat Jadwal Dosen".
    \item Tekan tombol "Ekspor ke XLS".
\end{enumerate}

Berikut adalah tabel hasil pengujian untuk skenario membuat dokumen XLS yang berisi jadwal dosen di halaman lihat jadwal dosen.

\begin{table}[H]
    \centering 
    \caption{Hasil Pengujian untuk Skenario Membuat Dokumen XLS yang Berisi Jadwal Dosen di Halaman Lihat Jadwal Dosen}
    \label{tab:hasil_pengujian_membuat_dokumen_xls_yang_berisi_jadwal_dosen_di_halaman_lihat_jadwal_dosen}
    \begin{tabular}{|c|c|p{10cm}|}
        \toprule
        Langkah ke & Hasil (sukses/tidak) & Tindakan \\

        \midrule
        1 & Sukses & Mengikuti langkah-langkah pada Tabel \ref{tab:hasil_pengujian_login} \\
        2 & Sukses & Menekan \textit{"tab"} sebanyak lima kali hingga ChromeVox menyebutkan "lihat jadwal dosen, \textit{link list item}" lalu menekan \textit{"enter"}. \\
        3 & Sukses & Menekan \textit{"shift + alt + L"} lalu \textit{"L"} sebanyak satu kali hingga ChromeVox menyebutkan "\textit{link list, use up and down arrow keys to browse or type to search}" lalu mengetik "ekspor ke XLS", dilanjutkan dengan menekan \textit{"enter"} hingga ChromeVox menyebutkan "\textit{exited, link list, ekspor ke XLS, link}" lalu menekan \textit{"enter"}. \\

        \bottomrule

    \end{tabular}
\end{table}