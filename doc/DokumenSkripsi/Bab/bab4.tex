\chapter{Implementasi dan Pengujian}
\label{chap:implementasi_dan_pengujian}

\section{Implementasi}
\label{sec:implementasi}
Pada bagian ini akan dibahas perbaikan apa saja yang dilakukan untuk membuat kriteria sukses pada bagian 3.1
yang belum sukses dipatuhi menjadi sukses dipatuhi. Setiap perubahan kode yang terdapat pada bagian ini ditampilkan dalam format \textit{diff} dengan jarak \textit{indentation} sebanyak empat spasi atau karakter dari batas ujung paling kiri.

\subsection{Perbaikan Kriteria Sukses 1.1.1 \textit{Non-text Content}}
\label{subsec:perbaikan_kriteria_sukses_1.1.1}
Pada bagian ini dilakukan perbaikan sebagai berikut:

Pada konten berupa ikon yang tidak memiliki nama di halaman cetak transkrip, manajemen cetak transkrip, perubahan kuliah, dan manajemen perubahan kuliah diberikan atribut \textit{aria-label} yang diisi dengan nilai yang sesuai dengan fungsi setiap konten ikon. Perbaikan yang dilakukan dapat dilihat di \ref{lst:perbaikan_1.1.1_ikon_tanpa_nama}.
\begin{lstlisting}[frame=single, label={lst:perbaikan_1.1.1_ikon_tanpa_nama}, language=diff, caption=Perbaikan Kriteria Sukses 1.1.1 - Ikon Tanpa Nama]
diff --git a/www/application/views/PerubahanKuliahManage/main.php b/www/application/views/PerubahanKuliahManage/main.php
index 657260b..43c215f 100644
--- a/www/application/views/PerubahanKuliahManage/main.php
+++ b/www/application/views/PerubahanKuliahManage/main.php
@@ -30,11 +30,11 @@ defined('BASEPATH') OR exit('No direct script access allowed');
    <td><?= $request->mataKuliahCode ?></td>
    <td><?= PerubahanKuliah_model::CHANGETYPE_TYPES[$request->changeType] ?></td>
    <td>
-   <a data-open="detail<?= $request->id ?>"><i class="fi-eye"></i></a>
-   <a target="_blank" href="/PerubahanKuliahManage/printview/<?= $request->id ?>"><i class="fi-print"></i></a>
-   <a data-open="konfirmasi<?= $request->id ?>"><i class="fi-like"></i></a>                                    
-   <a data-open="tolak<?= $request->id ?>"><i class="fi-dislike"></i></a>
-   <a data-open="hapus<?= $request->id ?>"><i class="fi-trash"></i></a>
+   <a aria-label="lihat detail permohonan <?= $request->requestByName ?>" data-open="detail<?= $request->id ?>"><i class="fi-eye"></i></a>
+   <a aria-label="cetak permohonan <?= $request->requestByName ?>" target="_blank" href="/PerubahanKuliahManage/printview/<?= $request->id ?>"><i class="fi-print"></i></a>
+   <a aria-label="konfirmasi permohonan <?= $request->requestByName ?>" data-open="konfirmasi<?= $request->id ?>"><i class="fi-like"></i></a>
+   <a aria-label="tolak permohonan <?= $request->requestByName ?>" data-open="tolak<?= $request->id ?>"><i class="fi-dislike"></i></a>
+   <a aria-label="hapus permohonan <?= $request->requestByName ?>" data-open="hapus<?= $request->id ?>"><i class="fi-trash"></i></a>
    </td>
    </tr>
    <?php endforeach; ?>
diff --git a/www/application/views/PerubahanKuliahRequest/main.php b/www/application/views/PerubahanKuliahRequest/main.php
index 72457a7..cd47c17 100644
--- a/www/application/views/PerubahanKuliahRequest/main.php
+++ b/www/application/views/PerubahanKuliahRequest/main.php
@@ -118,7 +118,7 @@ defined('BASEPATH') OR exit('No direct script access allowed');
    <td><time datetime="<?= $request->answeredDateTime ?>"><?= $request->answeredDateString ?></time></td>
    <td><?= $request->answeredMessage ?></td>
    <td>
-   <a data-open="detail<?= $request->id ?>"><i class="fi-eye"></i></a>
+   <a aria-label="lihat detail permohonan <?= $request->requestByName ?>" data-open="detail<?= $request->id ?>"><i class="fi-eye"></i></a>
    </td>
    </tr>
    <?php endforeach; ?>
diff --git a/www/application/views/TranskripManage/main.php b/www/application/views/TranskripManage/main.php
index 0970fc2..9f24954 100644
--- a/www/application/views/TranskripManage/main.php
+++ b/www/application/views/TranskripManage/main.php
@@ -85,7 +85,7 @@ defined('BASEPATH') OR exit('No direct script access allowed');
    <span aria-hidden="true">&times;</span>
    </button>                                        
    </div>
-   <a data-open="detail<?= $request->id ?>"><i class="fi-eye"></i></a>
+   <a aria-label="lihat detail permohonan <?= $request->requestByName ?>" data-open="detail<?= $request->id ?>"><i class="fi-eye"></i></a>
    <div class="reveal" id="tolak<?= $request->id ?>" data-reveal>
    <h5>Tolak Permohonan</h5>
    <form method="POST" action="/TranskripManage/answer">
@@ -105,7 +105,7 @@ defined('BASEPATH') OR exit('No direct script access allowed');
    <span aria-hidden="true">&times;</span>
    </button>
    </div>
-   <a data-open="tolak<?= $request->id ?>"><i class="fi-dislike"></i></a>
+   <a aria-label="tolak permohonan <?= $request->requestByName ?>" data-open="tolak<?= $request->id ?>"><i class="fi-dislike"></i></a>
    <div class="reveal" id="cetak<?= $request->id ?>" data-reveal>
    <h5>Cetak Permohonan</h5>
    <?php if ($request->requestByNPM !== NULL): ?>
@@ -131,7 +131,7 @@ defined('BASEPATH') OR exit('No direct script access allowed');
    <span aria-hidden="true">&times;</span>
    </button>
    </div>
-   <a data-open="cetak<?= $request->id ?>"><i class="fi-print"></i></a>
+   <a aria-label="cetak permohonan <?= $request->requestByName ?>" data-open="cetak<?= $request->id ?>"><i class="fi-print"></i></a>
    <div class="reveal" id="hapus<?= $request->id ?>" data-reveal>
    <h5>Hapus Permohonan</h5>
    <form method="POST" action="/TranskripManage/remove">
@@ -146,7 +146,7 @@ defined('BASEPATH') OR exit('No direct script access allowed');
    <span aria-hidden="true">&times;</span>
    </button>
    </div>
-   <a data-open="hapus<?= $request->id ?>"><i class="fi-trash"></i></a>
+   <a aria-label="hapus permohonan <?= $request->requestByName ?>" data-open="hapus<?= $request->id ?>"><i class="fi-trash"></i></a>
    </td>
    </tr>
    <?php endforeach; ?>
diff --git a/www/application/views/TranskripRequest/main.php b/www/application/views/TranskripRequest/main.php
index ada0254..44598cc 100644
--- a/www/application/views/TranskripRequest/main.php
+++ b/www/application/views/TranskripRequest/main.php
@@ -126,7 +126,7 @@ defined('BASEPATH') OR exit('No direct script access allowed');
    <span aria-hidden="true">&times;</span>
    </button>
    </div>
-   <a data-open="detail<?= $request->id ?>"><i class="fi-eye"></i></a>
+   <a aria-label="lihat detail permohonan <?= $request->requestByName ?>" data-open="detail<?= $request->id ?>"><i class="fi-eye"></i></a>
    </td>
    </tr>
    <?php endforeach; ?>
\end{lstlisting}

\subsection{Perbaikan Kriteria Sukses 1.3.1 \textit{Info and Relationships}}
\label{subsec:perbaikan_kriteria_sukses_1.3.1}
Pada bagian ini dilakukan perbaikan sebagai berikut:

\begin{itemize}
\item Pada penggunaan \textit{tag heading} yang tidak tepat, judul yang menggunakan \textit{<h5>} diubah menggunakan \textit{tag heading} yang sesuai struktur \textit{HTML} secara berurutan dimulai dari \textit{<h1>} hingga \textit{<h6>}. Perbaikan yang dilakukan dapat dilihat di \ref{lst:perbaikan_1.3.1_tag_heading}.
\begin{lstlisting}[frame=single, label={lst:perbaikan_1.3.1_tag_heading}, language=diff, caption=Perbaikan Kriteria Sukses 1.3.1 - Penggunaan \textit{Heading} Tidak Tepat]
diff --git a/www/application/views/EntriJadwalDosen/main.php b/www/application/views/EntriJadwalDosen/main.php
index 6392b95..7af8ded 100644
--- a/www/application/views/EntriJadwalDosen/main.php
+++ b/www/application/views/EntriJadwalDosen/main.php
@@ -12,7 +12,7 @@ defined('BASEPATH') OR exit('No direct script access allowed');
    <div class="row">

    <div class="large-12 column callout">
-   <h5>Tambah Jadwal</h5>
+   <h1>Tambah Jadwal</h1>
    <div class="large-4 columns">
    <form method="POST" action="/EntriJadwalDosen/add">
    <input type="hidden" name="<?= $this->security->get_csrf_token_name() ?>" value="<?= $this->security->get_csrf_hash() ?>" />
@@ -59,7 +59,7 @@ defined('BASEPATH') OR exit('No direct script access allowed');
    <!-- ===================================================================== Pembentukan Tabel ============================================================================= -->

    <div class="large-12 column callout">
-   <h5>Daftar Jadwal</h5>
+   <h1>Daftar Jadwal</h1>
    <div class="table-scroll" id="jadwal_table">
    <table border=1 style="border-color:black ; border-collapse:separate">
    <tr> 
@@ -169,7 +169,7 @@ defined('BASEPATH') OR exit('No direct script access allowed');
    <button class="close-button" data-close aria-label="Close modal" type="button">
    <span aria-hidden="true">&times;</span>
    </button>
-   <h5> Edit Jadwal </h5>
+   <h2> Edit Jadwal </h2>
    <form name="form<?php echo $dataHariIni->id ?>" method="POST" action="/EntriJadwalDosen/update/<?php echo $dataHariIni->id ?>">
    <input type="hidden" name="<?= $this->security->get_csrf_token_name() ?>" value="<?= $this->security->get_csrf_hash() ?>" />
    <input type="hidden" name="id_jadwal_parameter" value="<?php echo $dataHariIni->id ?>"> </a> <br>
diff --git a/www/application/views/PerubahanKuliahManage/main.php b/www/application/views/PerubahanKuliahManage/main.php
index 657260b..428510f 100644
--- a/www/application/views/PerubahanKuliahManage/main.php
+++ b/www/application/views/PerubahanKuliahManage/main.php
@@ -9,7 +9,7 @@ defined('BASEPATH') OR exit('No direct script access allowed');

    <div class="row">
    <div class="callout">
-   <h5>Permohonan Perubahan Kuliah</h5>
+   <h1>Permohonan Perubahan Kuliah</h1>
    <table class="stack">
    <thead>
    <tr>
@@ -56,7 +56,7 @@ defined('BASEPATH') OR exit('No direct script access allowed');
    <?php foreach ($requests as $request): ?>

    <div class="reveal" id="detail<?= $request->id ?>" data-reveal>
-   <h5>Detail Permohonan #<?= $request->id ?></h5>
+   <h2>Detail Permohonan #<?= $request->id ?></h2>
    <table class="stack">
    <tbody>
    <tr>
@@ -128,7 +128,7 @@ defined('BASEPATH') OR exit('No direct script access allowed');
    </button>
    </div>
    <div class="reveal" id="konfirmasi<?= $request->id ?>" data-reveal>
-   <h5>Konfirmasi Permohonan</h5>
+   <h2>Konfirmasi Permohonan</h2>
    <form method="POST" action="/PerubahanKuliahManage/answer">
    <input type="hidden" name="<?= $this->security->get_csrf_token_name() ?>" value="<?= $this->security->get_csrf_hash() ?>" />
    <input type="hidden" name="id" value="<?= $request->id ?>"/>
@@ -147,7 +147,7 @@ defined('BASEPATH') OR exit('No direct script access allowed');
    </button>
    </div>        
    <div class="reveal" id="tolak<?= $request->id ?>" data-reveal>
-   <h5>Tolak Permohonan</h5>
+   <h2>Tolak Permohonan</h2>
    <form method="POST" action="/PerubahanKuliahManage/answer">
    <input type="hidden" name="<?= $this->security->get_csrf_token_name() ?>" value="<?= $this->security->get_csrf_hash() ?>" />
    <input type="hidden" name="id" value="<?= $request->id ?>"/>
@@ -166,7 +166,7 @@ defined('BASEPATH') OR exit('No direct script access allowed');
    </button>
    </div>
    <div class="reveal" id="hapus<?= $request->id ?>" data-reveal>
-   <h5>Hapus Permohonan</h5>
+   <h2>Hapus Permohonan</h2>
    <form method="POST" action="/PerubahanKuliahManage/remove">
    <input type="hidden" name="<?= $this->security->get_csrf_token_name() ?>" value="<?= $this->security->get_csrf_hash() ?>" />
    <input type="hidden" name="id" value="<?= $request->id ?>"/>
diff --git a/www/application/views/PerubahanKuliahRequest/main.php b/www/application/views/PerubahanKuliahRequest/main.php
index 72457a7..b6b0b94 100644
--- a/www/application/views/PerubahanKuliahRequest/main.php
+++ b/www/application/views/PerubahanKuliahRequest/main.php
@@ -10,7 +10,7 @@ defined('BASEPATH') OR exit('No direct script access allowed');
    <div class="row">
    <div class="large-12 column">
    <div class="callout">
-   <h5>Permohonan Baru</h5>
+   <h1>Permohonan Baru</h1>
    <form method="POST" action="/PerubahanKuliahRequest/add">
    <input type="hidden" name="<?= $this->security->get_csrf_token_name() ?>" value="<?= $this->security->get_csrf_hash() ?>" />
    <div class="row">
@@ -93,7 +93,7 @@ defined('BASEPATH') OR exit('No direct script access allowed');
    </form>
    </div>
    <div class="callout">
-   <h5>Histori Permohonan</h5>
+   <h1>Histori Permohonan</h1>
    <table class="stack">
    <thead>
    <tr>
@@ -130,7 +130,7 @@ defined('BASEPATH') OR exit('No direct script access allowed');
    <?php foreach ($requests as $request): ?>

    <div class="reveal" id="detail<?= $request->id ?>" data-reveal>
-   <h5>Detail Permohonan #<?= $request->id ?></h5>
+   <h2>Detail Permohonan #<?= $request->id ?></h2>
    <table class="stack">
    <tbody>
    <tr>
diff --git a/www/application/views/TranskripManage/main.php b/www/application/views/TranskripManage/main.php
index 0970fc2..72ff740 100644
--- a/www/application/views/TranskripManage/main.php
+++ b/www/application/views/TranskripManage/main.php
@@ -9,7 +9,7 @@ defined('BASEPATH') OR exit('No direct script access allowed');

    <div class="row">
    <div class="callout">
-   <h5>Permintaan Transkrip</h5>
+   <h1>Permintaan Transkrip</h1>
    <form method="GET" action="/TranskripManage">
    <div class="input-group">
    <span class="input-group-label">Cari NPM:</span>
@@ -40,7 +40,7 @@ defined('BASEPATH') OR exit('No direct script access allowed');
    <td><?= isset($request->requestByNPM) ? $request->requestByNPM : '-' ?></td>
    <td>
    <div class="reveal" id="detail<?= $request->id ?>" data-reveal>
-   <h5>Detail Permohonan #<?= $request->id ?></h5>
+   <h2>Detail Permohonan #<?= $request->id ?></h2>
    <table class="stack">
    <tbody>
    <tr>
@@ -87,7 +87,7 @@ defined('BASEPATH') OR exit('No direct script access allowed');
    </div>
    <a data-open="detail<?= $request->id ?>"><i class="fi-eye"></i></a>
    <div class="reveal" id="tolak<?= $request->id ?>" data-reveal>
-   <h5>Tolak Permohonan</h5>
+   <h2>Tolak Permohonan</h2>
    <form method="POST" action="/TranskripManage/answer">
    <input type="hidden" name="<?= $this->security->get_csrf_token_name() ?>" value="<?= $this->security->get_csrf_hash() ?>" />
    <input type="hidden" name="id" value="<?= $request->id ?>"/>
@@ -107,7 +107,7 @@ defined('BASEPATH') OR exit('No direct script access allowed');
    </div>
    <a data-open="tolak<?= $request->id ?>"><i class="fi-dislike"></i></a>
    <div class="reveal" id="cetak<?= $request->id ?>" data-reveal>
-   <h5>Cetak Permohonan</h5>
+   <h2>Cetak Permohonan</h2>
    <?php if ($request->requestByNPM !== NULL): ?>
    <a target="_blank" href="<?= sprintf($transkripURLs[$request->requestType], $request->requestByNPM) ?>">Klik untuk membuka DPS/LHS</a>
    <?php else: ?>
@@ -133,7 +133,7 @@ defined('BASEPATH') OR exit('No direct script access allowed');
    </div>
    <a data-open="cetak<?= $request->id ?>"><i class="fi-print"></i></a>
    <div class="reveal" id="hapus<?= $request->id ?>" data-reveal>
-   <h5>Hapus Permohonan</h5>
+   <h2>Hapus Permohonan</h2>
    <form method="POST" action="/TranskripManage/remove">
    <input type="hidden" name="<?= $this->security->get_csrf_token_name() ?>" value="<?= $this->security->get_csrf_hash() ?>" />
    <input type="hidden" name="id" value="<?= $request->id ?>"/>
diff --git a/www/application/views/TranskripRequest/main.php b/www/application/views/TranskripRequest/main.php
index ada0254..e1a87f0 100644
--- a/www/application/views/TranskripRequest/main.php
+++ b/www/application/views/TranskripRequest/main.php
@@ -10,7 +10,7 @@ defined('BASEPATH') OR exit('No direct script access allowed');
    <div class="row">
    <div class="medium-12 column">
    <div class="callout">
-   <h5>Permohonan Baru</h5>
+   <h1>Permohonan Baru</h1>
    <?php if (is_array($forbiddenTypes)): ?>
    <form method="POST" action="/TranskripRequest/add">
    <input type="hidden" name="<?= $this->security->get_csrf_token_name() ?>" value="<?= $this->security->get_csrf_hash() ?>" />
@@ -57,7 +57,7 @@ defined('BASEPATH') OR exit('No direct script access allowed');
    <?php endif ?>
    </div>
    <div class="callout">
-   <h5>Histori Permohonan</h5>
+   <h1>Histori Permohonan</h1>
    <table class="stack">
    <thead>
    <tr>
@@ -81,7 +81,7 @@ defined('BASEPATH') OR exit('No direct script access allowed');
    <td><?= $request->answeredMessage ?></td>
    <td>
    <div class="reveal" id="detail<?= $request->id ?>" data-reveal>
-   <h5>Detail Permohonan #<?= $request->id ?></h5>
+   <h2>Detail Permohonan #<?= $request->id ?></h2>
    <table class="stack">
    <tbody>
    <tr>
\end{lstlisting}

\item Pada halaman manajemen cetak transkrip, kolom masukan NPM diberi atribut \textit{id} yang diisi dengan nilai "npm" lalu label "Cari NPM" diubah dari \textit{<span>} menjadi \textit{<label>} dan diberi atribut \textit{for} dengan nilai "npm". Perbaikan yang dilakukan dapat dilihat di \ref{lst:perbaikan_1.3.1_label_masukan_manajemen_cetak_transkrip}.
\begin{lstlisting}[frame=single, label={lst:perbaikan_1.3.1_label_masukan_manajemen_cetak_transkrip}, language=diff, caption=Perbaikan Kriteria Sukses 1.3.1 - Tidak Terdapat Label pada Kolom Masukan di Halaman Manajemen Cetak Transkrip]
diff --git a/www/application/views/TranskripManage/main.php b/www/application/views/TranskripManage/main.php
index 0970fc2..6dffa7d 100644
--- a/www/application/views/TranskripManage/main.php
+++ b/www/application/views/TranskripManage/main.php
@@ -12,8 +12,8 @@ defined('BASEPATH') OR exit('No direct script access allowed');
    <h5>Permintaan Transkrip</h5>
    <form method="GET" action="/TranskripManage">
    <div class="input-group">
-   <span class="input-group-label">Cari NPM:</span>
-   <input name="npm" class="input-group-field" type="text" placeholder="2013730013" maxlength="10" minlength="10"<?= $npmQuery === NULL ? '' : " value='$npmQuery'" ?>/>
+   <label for="npm" class="input-group-label">Cari NPM:</label>
+   <input id="npm" name="npm" class="input-group-field" type="text" placeholder="2013730013" maxlength="10" minlength="10"<?= $npmQuery === NULL ? '' : " value='$npmQuery'" ?>/>
<div class="input-group-button">
<input class="button" type="submit" value="Cari"/>
</div>
\end{lstlisting}

\item Pada halaman entri jadwal dosen, setiap kolom masukan diberi atribut \textit{id} dengan nilai yang sesuai dengan isi kolom masukan lalu setiap judul kolom dibuat sebagai label dengan menggunakan \textit{<label>} dan dihubungkan dengan kolom masukan yang bersangkutan dengan menggunakan atribut \textit{for}. Perbaikan yang dilakukan dapat dilihat di \ref{lst:perbaikan_1.3.1_label_masukan_entri_jadwal_dosen}.
\begin{lstlisting}[frame=single, label={lst:perbaikan_1.3.1_label_masukan_entri_jadwal_dosen}, language=diff, caption=Perbaikan Kriteria Sukses 1.3.1 - Tidak Terdapat Label pada Kolom Masukan di Halaman Entri Jadwal Dosen]
diff --git a/www/application/views/EntriJadwalDosen/main.php b/www/application/views/EntriJadwalDosen/main.php
index 6392b95..f8471cb 100644
--- a/www/application/views/EntriJadwalDosen/main.php
+++ b/www/application/views/EntriJadwalDosen/main.php
@@ -16,8 +16,8 @@ defined('BASEPATH') OR exit('No direct script access allowed');
    <div class="large-4 columns">
    <form method="POST" action="/EntriJadwalDosen/add">
    <input type="hidden" name="<?= $this->security->get_csrf_token_name() ?>" value="<?= $this->security->get_csrf_hash() ?>" />
-   Hari
-   <select name="hari"> 
+   <label for="hari">Hari</label>
+   <select id="hari" name="hari">
    <?php
    $hariValue = 0;
    foreach ($namaHari as $hari) {
    @@ -29,30 +29,30 @@ defined('BASEPATH') OR exit('No direct script access allowed');
    }
    ?>
    </select><br>
-   Jam Mulai
-   <select name="jam_mulai"> 
+   <label for="jam_mulai">Jam Mulai</label>
+   <select id="jam_mulai" name="jam_mulai"> 
    <?php for ($i = 7; $i <= 16; $i++) { ?>
    <option value="<?php echo $i ?>"> <?php echo $i ?>:00 </option>
    <?php } ?>
    </select><br>
    </div>
    <div class=" large-4 columns">
-   Durasi
-   <select name="durasi"> 
+   <label for="durasi">Durasi</label>
+   <select id="durasi" name="durasi"> 
    <?php for ($i = 1; $i <= 9; $i++) { ?>
    <option value="<?php echo $i ?>"> <?php echo $i ?> jam </option>
    <?php } ?>
    </select><br>
-   Jenis  
-   <select name="jenis_jadwal"> 
+   <label for="jenis_jadwal">Jenis</label>
+   <select id="jenis_jadwal" name="jenis_jadwal"> 
    <option value="konsultasi" style="background-color:yellow"> Konsultasi </option>
    <option value="terjadwal" style="background-color:green;color:white"> Terjadwal</option>
    <option value="kelas" style="background-color:white"> Kelas </option>
    </select>
    </div>
    <div class="large-4 columns">
-   Label <input type="text" name="label_jadwal"><br>
-   <input type="submit" class="button" value="Tambah">
+   <label for="label_jadwal">Label</label>
+   <input id="label_jadwal" type="text" name="label_jadwal"><br>
    </form>
    </div>
    </div>
\end{lstlisting}
\end{itemize}

\subsection{Perbaikan Kriteria Sukses 2.1.1 \textit{Keyboard}}
\label{subsec:perbaikan_kriteria_sukses_2.1.1}
Pada bagian ini dilakukan perbaikan sebagai berikut:

\begin{itemize}
\item Pada bagian navigasi menu, atribut \textit{data-responsive-menu="dropdown"} dihilangkan dari elemen \textit{ul} karena isi dari elemen tersebut bukanlah sebuah menu \textit{dropdown}. Perbaikan yang dilakukan dapat dilihat di \ref{lst:perbaikan_2.1.1_keyboard_menu_navigasi}.
\begin{lstlisting}[frame=single, label={lst:perbaikan_2.1.1_keyboard_menu_navigasi}, language=diff, caption=Perbaikan Kriteria Sukses 2.1.1 - Penggunaan \textit{Keyboard} pada Menu Navigasi]
diff --git a/www/application/views/templates/topbar_loggedin.php b/www/application/views/templates/topbar_loggedin.php
index 51fc11e..bc9f2c0 100644
--- a/www/application/views/templates/topbar_loggedin.php
+++ b/www/application/views/templates/topbar_loggedin.php
@@ -7,7 +7,7 @@ defined('BASEPATH') OR exit('No direct script access allowed');

    <div class="top-bar" id="navigation-menu">
    <div class="top-bar-left">
-   <ul class="menu" data-responsive-menu="dropdown">
+   <ul class="menu">
    <li class="menu-text"><img src="/public/img/logo.png" class="textsized" alt="B"/></li>
    <?php foreach ($this->Auth_model->getUserInfo()['modules'] as $module): ?>
    <li<?= $module === $currentModule ? ' class="menu-active"' : '' ?>><a href="/<?= $module ?>"><?= $this->config->item('module-names')[$module] ?></a></li>
\end{lstlisting}

\item Pada bagian tabel "Daftar Jadwal" di halaman entri jadwal dosen, setiap \textit{cell} dalam tabel yang berisi jadwal dosen diberi atribut \textit{tabindex} yang diisi dengan nilai nol. Selanjutnya dibuat kode \textit{JavaScript} untuk memicu aksi membuka kotak dialog ketika tombol "Enter" ditekan pada \textit{cell} yang berisi jadwal dosen. Perbaikan yang dilakukan dapat dilihat di \ref{lst:perbaikan_2.1.1_keyboard_entri_jadwal_dosen}.
\begin{lstlisting}[frame=single, label={lst:perbaikan_2.1.1_keyboard_entri_jadwal_dosen}, language=diff, caption=Perbaikan Kriteria Sukses 2.1.1 - Penggunaan \textit{Keyboard} pada Halaman Entri Jadwal Dosen]
diff --git a/www/application/views/EntriJadwalDosen/main.php b/www/application/views/EntriJadwalDosen/main.php
index 6392b95..6c0e935 100644
--- a/www/application/views/EntriJadwalDosen/main.php
+++ b/www/application/views/EntriJadwalDosen/main.php
@@ -110,12 +110,21 @@ defined('BASEPATH') OR exit('No direct script access allowed');
    $("#cell" + i + "-" +<?php echo $colIdx; ?>).remove();
    }
    $($cellLocation).html("<?php echo $dataHariIni->label ?>");
+   $($cellLocation).attr("tabindex", "0");

    //membuat cell-cell yang telah diwarnai jadi memunculkan pop-up untuk mengedit jadwal ketika diklik oleh mouse
    $(document).on("click", $cellLocation, function () {
      var $menuName = "#edit_menu<?php echo $dataHariIni->id ?>";
      $($menuName).foundation('open');
    });
+
+   // membuat cell-cell dalam tabel dapat diakses menggunakan keyboard
+   $($cellLocation).keyup(function(e){
+   if(e.keyCode == 13){
+   var $menuName = "#edit_menu<?php echo $dataHariIni->id ?>";
+   $($menuName).foundation('open');
+   }
+   })
    </script>
    <?php
    }
\end{lstlisting}
\end{itemize}

\subsection{Perbaikan Kriteria Sukses 2.4.1 \textit{Bypass Blocks}}
\label{subsec:perbaikan_kriteria_sukses_2.4.1}
Pada bagian ini dilakukan perbaikan dengan membuat sebuah mekanisme untuk melompati bagian menu navigasi ketika pengguna sedang bernavigasi menggunakan \textit{keyboard}. Mekanisme yang dimaksud adalah sebuah elemen yang akan membuat fokus \textit{keyboard} berpindah ke area konten utama ketika pengguna mengaktifkan elemen tersebut. Elemen ini dapat dikenali oleh teknologi alat bantu dan hanya dapat diakses oleh pengguna yang bernavigasi dengan menggunakan \textit{keyboard}. Saat halaman web selesai dimuat, elemen ini akan menjadi yang pertama menerima fokus \textit{keyboard} ketika pengguna menekan tombol \textit{tab}. Perbaikan yang dilakukan dapat dilihat di \ref{lst:perbaikan_2.4.1_bypass_blocks}.
\begin{lstlisting}[frame=single, label={lst:perbaikan_2.4.1_bypass_blocks}, language=diff, caption=Perbaikan Kriteria Sukses 2.4.1 - Mekanisme untuk Melompati Area Konten yang Berulang]
diff --git a/www/application/views/EntriJadwalDosen/main.php b/www/application/views/EntriJadwalDosen/main.php
index 6392b95..a62cce4 100644
--- a/www/application/views/EntriJadwalDosen/main.php
+++ b/www/application/views/EntriJadwalDosen/main.php
@@ -9,7 +9,7 @@ defined('BASEPATH') OR exit('No direct script access allowed');
    <body>
    <?php $this->load->view('templates/topbar_loggedin'); ?>

-   <div class="row">
+   <div id="mainContent" class="row">

    <div class="large-12 column callout">
    <h5>Tambah Jadwal</h5>
diff --git a/www/application/views/LihatJadwalDosen/main.php b/www/application/views/LihatJadwalDosen/main.php
index ef1be4f..5df2507 100644
--- a/www/application/views/LihatJadwalDosen/main.php
+++ b/www/application/views/LihatJadwalDosen/main.php
@@ -10,7 +10,7 @@ defined('BASEPATH') OR exit('No direct script access allowed');
    <?php //$this->load->view('templates/flashmessage'); ?>


-   <div class="row">
+   <div id="mainContent" class="row">

    <div class="large-12 column">
    <ul class="tabs" data-tabs id="tab_jadwal">
diff --git a/www/application/views/PerubahanKuliahManage/main.php b/www/application/views/PerubahanKuliahManage/main.php
index 657260b..a64fbf2 100644
--- a/www/application/views/PerubahanKuliahManage/main.php
+++ b/www/application/views/PerubahanKuliahManage/main.php
@@ -7,7 +7,7 @@ defined('BASEPATH') OR exit('No direct script access allowed');
    <?php $this->load->view('templates/topbar_loggedin'); ?>
    <?php $this->load->view('templates/flashmessage'); ?>

-   <div class="row">
+   <div id="mainContent" class="row">
    <div class="callout">
    <h5>Permohonan Perubahan Kuliah</h5>
    <table class="stack">
diff --git a/www/application/views/PerubahanKuliahRequest/main.php b/www/application/views/PerubahanKuliahRequest/main.php
index 72457a7..dbb9113 100644
--- a/www/application/views/PerubahanKuliahRequest/main.php
+++ b/www/application/views/PerubahanKuliahRequest/main.php
@@ -7,7 +7,7 @@ defined('BASEPATH') OR exit('No direct script access allowed');
    <?php $this->load->view('templates/topbar_loggedin'); ?>
    <?php $this->load->view('templates/flashmessage'); ?>

-   <div class="row">
+   <div id="mainContent" class="row">
    <div class="large-12 column">
    <div class="callout">
    <h5>Permohonan Baru</h5>
diff --git a/www/application/views/TranskripManage/main.php b/www/application/views/TranskripManage/main.php
index 0970fc2..4d94b34 100644
--- a/www/application/views/TranskripManage/main.php
+++ b/www/application/views/TranskripManage/main.php
@@ -7,7 +7,7 @@ defined('BASEPATH') OR exit('No direct script access allowed');
    <?php $this->load->view('templates/topbar_loggedin'); ?>
    <?php $this->load->view('templates/flashmessage'); ?>

-   <div class="row">
+   <div id="mainContent" class="row">
    <div class="callout">
    <h5>Permintaan Transkrip</h5>
    <form method="GET" action="/TranskripManage">
diff --git a/www/application/views/TranskripRequest/main.php b/www/application/views/TranskripRequest/main.php
index ada0254..f5beb85 100644
--- a/www/application/views/TranskripRequest/main.php
+++ b/www/application/views/TranskripRequest/main.php
@@ -7,7 +7,7 @@ defined('BASEPATH') OR exit('No direct script access allowed');
    <?php $this->load->view('templates/topbar_loggedin'); ?>
    <?php $this->load->view('templates/flashmessage'); ?>

-   <div class="row">
+   <div id="mainContent" class="row">
    <div class="medium-12 column">
    <div class="callout">
    <h5>Permohonan Baru</h5>
diff --git a/www/application/views/templates/script_foundation.php b/www/application/views/templates/script_foundation.php
index a93c72a..e5b9251 100644
--- a/www/application/views/templates/script_foundation.php
+++ b/www/application/views/templates/script_foundation.php
@@ -5,3 +5,10 @@ defined('BASEPATH') OR exit('No direct script access allowed');
    <script src="/public/js/foundation.min.js"></script>
    <script src="/public/js/app.js"></script>
    <script src="/public/lib/xdan-datetimepicker/jquery.datetimepicker.full.min.js"></script>
+   <script>
+   $(document).ready(function(){
+   var skipToContent = document.createElement("div");
+   skipToContent.innerHTML += '<a class="skip-navbar" href="#mainContent">Lompat ke menu utama</a>';
+   document.body.insertBefore(skipToContent, document.body.firstChild);
+   })
+   </script>
    \ No newline at end of file
diff --git a/www/public/css/app.css b/www/public/css/app.css
index f283c52..b6af0a7 100644
--- a/www/public/css/app.css
+++ b/www/public/css/app.css
@@ -62,3 +62,27 @@ i[class^="fi-"] {
    margin-left: 10px;
    margin-right: 10px;    
    }
+
+   /* Skip Navbar */
+   a.skip-navbar {
+   position: absolute;
+   left: -99px;
+   top: 0;
+   opacity: 0;
+   width: 1px;
+   height: 1px;
+   z-index: -99;
+   }
+
+   a.skip-navbar:focus, a.skip-navbar:active {
+   opacity: 1;
+   left: 0;
+   background:white;
+   color:#000;
+   width: auto;
+   height: auto;
+   margin: 10px;
+   padding: 5px;
+   font-size: 1.4rem;
+   z-index: 99;
+   }
    \ No newline at end of file
\end{lstlisting}

\subsection{Perbaikan Kriteria Sukses 2.4.4 \textit{Link Purpose (In Context)}}
\label{subsec:perbaikan_kriteria_sukses_2.4.4}
Pada bagian ini dilakukan perbaikan sebagai berikut:

Pada tautan berisi ikon yang tidak memiliki teks untuk menjelaskan tujuan tautan tersebut yang terdapat di halaman cetak transkrip, manajemen cetak transkrip, perubahan kuliah, dan manajemen perubahan kuliah diberikan atribut \textit{aria-label} yang diisi dengan nilai yang sesuai dengan fungsi setiap konten ikon. Perbaikan yang dilakukan dapat dilihat di \ref{lst:perbaikan_2.4.4_tautan_tanpa_keterangan}.
\begin{lstlisting}[frame=single, label={lst:perbaikan_2.4.4_tautan_tanpa_keterangan}, language=diff, caption=Perbaikan Kriteria Sukses 2.4.4 - Tautan Tanpa Keterangan]
diff --git a/www/application/views/PerubahanKuliahManage/main.php b/www/application/views/PerubahanKuliahManage/main.php
index 657260b..43c215f 100644
--- a/www/application/views/PerubahanKuliahManage/main.php
+++ b/www/application/views/PerubahanKuliahManage/main.php
@@ -30,11 +30,11 @@ defined('BASEPATH') OR exit('No direct script access allowed');
    <td><?= $request->mataKuliahCode ?></td>
    <td><?= PerubahanKuliah_model::CHANGETYPE_TYPES[$request->changeType] ?></td>
    <td>
-   <a data-open="detail<?= $request->id ?>"><i class="fi-eye"></i></a>
-   <a target="_blank" href="/PerubahanKuliahManage/printview/<?= $request->id ?>"><i class="fi-print"></i></a>
-   <a data-open="konfirmasi<?= $request->id ?>"><i class="fi-like"></i></a>                                    
-   <a data-open="tolak<?= $request->id ?>"><i class="fi-dislike"></i></a>
-   <a data-open="hapus<?= $request->id ?>"><i class="fi-trash"></i></a>
+   <a aria-label="lihat detail permohonan <?= $request->requestByName ?>" data-open="detail<?= $request->id ?>"><i class="fi-eye"></i></a>
+   <a aria-label="cetak permohonan <?= $request->requestByName ?>" target="_blank" href="/PerubahanKuliahManage/printview/<?= $request->id ?>"><i class="fi-print"></i></a>
+   <a aria-label="konfirmasi permohonan <?= $request->requestByName ?>" data-open="konfirmasi<?= $request->id ?>"><i class="fi-like"></i></a>
+   <a aria-label="tolak permohonan <?= $request->requestByName ?>" data-open="tolak<?= $request->id ?>"><i class="fi-dislike"></i></a>
+   <a aria-label="hapus permohonan <?= $request->requestByName ?>" data-open="hapus<?= $request->id ?>"><i class="fi-trash"></i></a>
    </td>
    </tr>
    <?php endforeach; ?>
diff --git a/www/application/views/PerubahanKuliahRequest/main.php b/www/application/views/PerubahanKuliahRequest/main.php
index 72457a7..cd47c17 100644
--- a/www/application/views/PerubahanKuliahRequest/main.php
+++ b/www/application/views/PerubahanKuliahRequest/main.php
@@ -118,7 +118,7 @@ defined('BASEPATH') OR exit('No direct script access allowed');
    <td><time datetime="<?= $request->answeredDateTime ?>"><?= $request->answeredDateString ?></time></td>
    <td><?= $request->answeredMessage ?></td>
    <td>
-   <a data-open="detail<?= $request->id ?>"><i class="fi-eye"></i></a>
+   <a aria-label="lihat detail permohonan <?= $request->requestByName ?>" data-open="detail<?= $request->id ?>"><i class="fi-eye"></i></a>
    </td>
    </tr>
    <?php endforeach; ?>
diff --git a/www/application/views/TranskripManage/main.php b/www/application/views/TranskripManage/main.php
index 0970fc2..9f24954 100644
--- a/www/application/views/TranskripManage/main.php
+++ b/www/application/views/TranskripManage/main.php
@@ -85,7 +85,7 @@ defined('BASEPATH') OR exit('No direct script access allowed');
    <span aria-hidden="true">&times;</span>
    </button>                                        
    </div>
-   <a data-open="detail<?= $request->id ?>"><i class="fi-eye"></i></a>
+   <a aria-label="lihat detail permohonan <?= $request->requestByName ?>" data-open="detail<?= $request->id ?>"><i class="fi-eye"></i></a>
    <div class="reveal" id="tolak<?= $request->id ?>" data-reveal>
    <h5>Tolak Permohonan</h5>
    <form method="POST" action="/TranskripManage/answer">
@@ -105,7 +105,7 @@ defined('BASEPATH') OR exit('No direct script access allowed');
    <span aria-hidden="true">&times;</span>
    </button>
    </div>
-   <a data-open="tolak<?= $request->id ?>"><i class="fi-dislike"></i></a>
+   <a aria-label="tolak permohonan <?= $request->requestByName ?>" data-open="tolak<?= $request->id ?>"><i class="fi-dislike"></i></a>
    <div class="reveal" id="cetak<?= $request->id ?>" data-reveal>
    <h5>Cetak Permohonan</h5>
    <?php if ($request->requestByNPM !== NULL): ?>
@@ -131,7 +131,7 @@ defined('BASEPATH') OR exit('No direct script access allowed');
    <span aria-hidden="true">&times;</span>
    </button>
    </div>
-   <a data-open="cetak<?= $request->id ?>"><i class="fi-print"></i></a>
+   <a aria-label="cetak permohonan <?= $request->requestByName ?>" data-open="cetak<?= $request->id ?>"><i class="fi-print"></i></a>
    <div class="reveal" id="hapus<?= $request->id ?>" data-reveal>
    <h5>Hapus Permohonan</h5>
    <form method="POST" action="/TranskripManage/remove">
@@ -146,7 +146,7 @@ defined('BASEPATH') OR exit('No direct script access allowed');
    <span aria-hidden="true">&times;</span>
    </button>
    </div>
-   <a data-open="hapus<?= $request->id ?>"><i class="fi-trash"></i></a>
+   <a aria-label="hapus permohonan <?= $request->requestByName ?>" data-open="hapus<?= $request->id ?>"><i class="fi-trash"></i></a>
    </td>
    </tr>
    <?php endforeach; ?>
diff --git a/www/application/views/TranskripRequest/main.php b/www/application/views/TranskripRequest/main.php
index ada0254..44598cc 100644
--- a/www/application/views/TranskripRequest/main.php
+++ b/www/application/views/TranskripRequest/main.php
@@ -126,7 +126,7 @@ defined('BASEPATH') OR exit('No direct script access allowed');
    <span aria-hidden="true">&times;</span>
    </button>
    </div>
-   <a data-open="detail<?= $request->id ?>"><i class="fi-eye"></i></a>
+   <a aria-label="lihat detail permohonan <?= $request->requestByName ?>" data-open="detail<?= $request->id ?>"><i class="fi-eye"></i></a>
    </td>
    </tr>
    <?php endforeach; ?>
\end{lstlisting}

\subsection{Perbaikan Kriteria Sukses 2.5.3 \textit{Label in Name}}
\label{subsec:perbaikan_kriteria_sukses_2.5.3}
Pada bagian ini dilakukan perbaikan yaitu untuk setiap kolom masukan di halaman manajemen cetak transkrip dan entri jadwal dosen diberi atribut \textit{id} dengan nilai yang sesuai dengan isi kolom masukan lalu setiap judul kolom dibuat sebagai label dengan menggunakan \textit{<label>} dan dihubungkan dengan kolom masukan yang bersangkutan dengan menggunakan atribut \textit{for}. Perbaikan yang dilakukan dapat dilihat di \ref{lst:perbaikan_2.5.3_label_dan_nama_pada_komponen_masukan}.

\begin{lstlisting}[frame=single, label={lst:perbaikan_2.5.3_label_dan_nama_pada_komponen_masukan}, language=diff, caption=Perbaikan Kriteria Sukses 2.5.3 - Label dan Nama Pada Komponen Masukan]
diff --git a/www/application/views/EntriJadwalDosen/main.php b/www/application/views/EntriJadwalDosen/main.php
index 6392b95..43af850 100644
--- a/www/application/views/EntriJadwalDosen/main.php
+++ b/www/application/views/EntriJadwalDosen/main.php
@@ -16,8 +16,8 @@ defined('BASEPATH') OR exit('No direct script access allowed');
    <div class="large-4 columns">
    <form method="POST" action="/EntriJadwalDosen/add">
    <input type="hidden" name="<?= $this->security->get_csrf_token_name() ?>" value="<?= $this->security->get_csrf_hash() ?>" />
-   Hari
-   <select name="hari"> 
+   <label for="hari">Hari</label>
+   <select id="hari" name="hari"> 
    <?php
    $hariValue = 0;
    foreach ($namaHari as $hari) {
@@ -29,30 +29,30 @@ defined('BASEPATH') OR exit('No direct script access allowed');
    }
    ?>
    </select><br>
-   Jam Mulai
-   <select name="jam_mulai"> 
+   <label for="jam_mulai">Jam Mulai</label>
+   <select id="jam_mulai" name="jam_mulai">  
    <?php for ($i = 7; $i <= 16; $i++) { ?>
    <option value="<?php echo $i ?>"> <?php echo $i ?>:00 </option>
    <?php } ?>
    </select><br>
    </div>
    <div class=" large-4 columns">
-   Durasi
-   <select name="durasi"> 
+   <label for="durasi">Durasi</label>
+   <select id="durasi" name="durasi">  
    <?php for ($i = 1; $i <= 9; $i++) { ?>
    <option value="<?php echo $i ?>"> <?php echo $i ?> jam </option>
    <?php } ?>
    </select><br>
-   Jenis  
-   <select name="jenis_jadwal"> 
+   <label for="jenis_jadwal">Jenis</label>
+   <select id="jenis_jadwal" name="jenis_jadwal">  
    <option value="konsultasi" style="background-color:yellow"> Konsultasi </option>
    <option value="terjadwal" style="background-color:green;color:white"> Terjadwal</option>
    <option value="kelas" style="background-color:white"> Kelas </option>
    </select>
    </div>
    <div class="large-4 columns">
-   Label <input type="text" name="label_jadwal"><br>
-   <input type="submit" class="button" value="Tambah">
+   <label for="label_jadwal">Label</label>
+   <input id="label_jadwal" type="text" name="label_jadwal"><br>
    </form>
    </div>
    </div>
diff --git a/www/application/views/TranskripManage/main.php b/www/application/views/TranskripManage/main.php
index 0970fc2..6dffa7d 100644
--- a/www/application/views/TranskripManage/main.php
+++ b/www/application/views/TranskripManage/main.php
@@ -12,8 +12,8 @@ defined('BASEPATH') OR exit('No direct script access allowed');
    <h5>Permintaan Transkrip</h5>
    <form method="GET" action="/TranskripManage">
    <div class="input-group">
-   <span class="input-group-label">Cari NPM:</span>
-   <input name="npm" class="input-group-field" type="text" placeholder="2013730013" maxlength="10" minlength="10"<?= $npmQuery === NULL ? '' : " value='$npmQuery'" ?>/>
+   <label for="npm" class="input-group-label">Cari NPM:</label>
+   <input id="npm" name="npm" class="input-group-field" type="text" placeholder="2013730013" maxlength="10" minlength="10"<?= $npmQuery === NULL ? '' : " value='$npmQuery'" ?>/>
    <div class="input-group-button">
    <input class="button" type="submit" value="Cari"/>
    </div>
\end{lstlisting} 

\subsection{Perbaikan Kriteria Sukses 3.1.1 \textit{Language of Page}}
\label{subsec:perbaikan_kriteria_sukses_3.1.1}
Pada bagian ini dilakukan perbaikan dengan mengubah setelan bahasa dari bahasa Inggris menjadi bahasa Indonesia untuk setiap halaman yang ditampilkan kepada pengguna. Perbaikan yang dilakukan dapat dilihat di \ref{lst:perbaikan_3.1.1_bahasa_halaman}.

\begin{lstlisting}[frame=single, label={lst:perbaikan_3.1.1_bahasa_halaman}, language=diff, caption=Perbaikan Kriteria Sukses 3.1.1 - Bahasa yang Tidak Sesuai]
diff --git a/www/application/views/EntriJadwalDosen/main.php b/www/application/views/EntriJadwalDosen/main.php
index 6392b95..6d86231 100644
--- a/www/application/views/EntriJadwalDosen/main.php
+++ b/www/application/views/EntriJadwalDosen/main.php
@@ -1,7 +1,7 @@
    <?php
    defined('BASEPATH') OR exit('No direct script access allowed');
    ?><!doctype html>
-   <html class="no-js" lang="en">
+   <html class="no-js" lang="id">
    <?php $this->load->view('templates/script_foundation'); ?>
    <?php $this->load->view('templates/head_loggedin'); ?>
    <?php $this->load->view('templates/flashmessage'); ?>
diff --git a/www/application/views/LihatJadwalDosen/main.php b/www/application/views/LihatJadwalDosen/main.php
index ef1be4f..aff813f 100644
--- a/www/application/views/LihatJadwalDosen/main.php
+++ b/www/application/views/LihatJadwalDosen/main.php
@@ -1,7 +1,7 @@
    <?php
    defined('BASEPATH') OR exit('No direct script access allowed');
    ?><!doctype html>
-   <html class="no-js" lang="en">
+   <html class="no-js" lang="id">
    <?php $this->load->view('templates/head_loggedin'); ?>
    <body>
    <?php $this->load->view('templates/topbar_loggedin'); ?>
diff --git a/www/application/views/PerubahanKuliahManage/main.php b/www/application/views/PerubahanKuliahManage/main.php
index 657260b..6c9f7ee 100644
--- a/www/application/views/PerubahanKuliahManage/main.php
+++ b/www/application/views/PerubahanKuliahManage/main.php
@@ -1,7 +1,7 @@
    <?php
    defined('BASEPATH') OR exit('No direct script access allowed');
    ?><!doctype html>
-   <html class="no-js" lang="en">
+   <html class="no-js" lang="id">
    <?php $this->load->view('templates/head_loggedin'); ?>
    <body>
    <?php $this->load->view('templates/topbar_loggedin'); ?>
diff --git a/www/application/views/PerubahanKuliahManage/printview.php b/www/application/views/PerubahanKuliahManage/printview.php
index 7c4034c..1d80622 100644
--- a/www/application/views/PerubahanKuliahManage/printview.php
+++ b/www/application/views/PerubahanKuliahManage/printview.php
@@ -2,7 +2,7 @@
    defined('BASEPATH') OR exit('No direct script access allowed');
    setlocale(LC_TIME, 'ind');
    ?><!doctype html>
-   <html class="no-js" lang="en">
+   <html class="no-js" lang="id">
    <head>
    <meta charset="utf-8" />
    <meta http-equiv="x-ua-compatible" content="ie=edge">
diff --git a/www/application/views/PerubahanKuliahRequest/main.php b/www/application/views/PerubahanKuliahRequest/main.php
index 72457a7..2a0c874 100644
--- a/www/application/views/PerubahanKuliahRequest/main.php
+++ b/www/application/views/PerubahanKuliahRequest/main.php
@@ -1,7 +1,7 @@
    <?php
    defined('BASEPATH') OR exit('No direct script access allowed');
    ?><!doctype html>
-   <html class="no-js" lang="en">
+   <html class="no-js" lang="id">
    <?php $this->load->view('templates/head_loggedin'); ?>
    <body>
    <?php $this->load->view('templates/topbar_loggedin'); ?>
diff --git a/www/application/views/TranskripManage/main.php b/www/application/views/TranskripManage/main.php
index 0970fc2..9faf620 100644
--- a/www/application/views/TranskripManage/main.php
+++ b/www/application/views/TranskripManage/main.php
@@ -1,7 +1,7 @@
    <?php
    defined('BASEPATH') OR exit('No direct script access allowed');
    ?><!doctype html>
-   <html class="no-js" lang="en">
+   <html class="no-js" lang="id">
    <?php $this->load->view('templates/head_loggedin'); ?>
    <body>
    <?php $this->load->view('templates/topbar_loggedin'); ?>
diff --git a/www/application/views/TranskripRequest/main.php b/www/application/views/TranskripRequest/main.php
index ada0254..e81549e 100644
--- a/www/application/views/TranskripRequest/main.php
+++ b/www/application/views/TranskripRequest/main.php
@@ -1,7 +1,7 @@
    <?php
    defined('BASEPATH') OR exit('No direct script access allowed');
    ?><!doctype html>
-   <html class="no-js" lang="en">
+   <html class="no-js" lang="id">
    <?php $this->load->view('templates/head_loggedin'); ?>
    <body>
    <?php $this->load->view('templates/topbar_loggedin'); ?>
diff --git a/www/application/views/auth/login.php b/www/application/views/auth/login.php
index e4daa52..b2089af 100644
--- a/www/application/views/auth/login.php
+++ b/www/application/views/auth/login.php
@@ -1,7 +1,7 @@
    <?php
    defined('BASEPATH') OR exit('No direct script access allowed');
    ?><!doctype html>
-   <html class="no-js" lang="en">
+   <html class="no-js" lang="id">
    <head>
    <meta charset="utf-8" />
    <meta http-equiv="x-ua-compatible" content="ie=edge">
\end{lstlisting}

\subsection{Perbaikan Kriteria Sukses 3.3.2 \textit{Labels or Instructions}}
\label{subsec:perbaikan_kriteria_sukses_3.3.2}
Pada bagian ini dilakukan perbaikan yaitu untuk setiap kolom masukan di halaman manajemen cetak transkrip dan entri jadwal dosen diberi atribut \textit{id} dengan nilai yang sesuai dengan isi kolom masukan lalu setiap judul kolom dibuat sebagai label dengan menggunakan \textit{<label>} dan dihubungkan dengan kolom masukan yang bersangkutan dengan menggunakan atribut \textit{for}. Perbaikan yang dilakukan dapat dilihat di \ref{lst:perbaikan_3.3.2_label_masukan}.

\begin{lstlisting}[frame=single, label={lst:perbaikan_3.3.2_label_masukan}, language=diff, caption=Perbaikan Kriteria Sukses 3.3.2 - Tidak Terdapat Label pada Kolom Masukan]
diff --git a/www/application/views/EntriJadwalDosen/main.php b/www/application/views/EntriJadwalDosen/main.php
index 6392b95..43af850 100644
--- a/www/application/views/EntriJadwalDosen/main.php
+++ b/www/application/views/EntriJadwalDosen/main.php
@@ -16,8 +16,8 @@ defined('BASEPATH') OR exit('No direct script access allowed');
    <div class="large-4 columns">
    <form method="POST" action="/EntriJadwalDosen/add">
    <input type="hidden" name="<?= $this->security->get_csrf_token_name() ?>" value="<?= $this->security->get_csrf_hash() ?>" />
-   Hari
-   <select name="hari"> 
+   <label for="hari">Hari</label>
+   <select id="hari" name="hari"> 
    <?php
    $hariValue = 0;
    foreach ($namaHari as $hari) {
@@ -29,30 +29,30 @@ defined('BASEPATH') OR exit('No direct script access allowed');
    }
    ?>
    </select><br>
-   Jam Mulai
-   <select name="jam_mulai"> 
+   <label for="jam_mulai">Jam Mulai</label>
+   <select id="jam_mulai" name="jam_mulai">  
    <?php for ($i = 7; $i <= 16; $i++) { ?>
    <option value="<?php echo $i ?>"> <?php echo $i ?>:00 </option>
    <?php } ?>
    </select><br>
    </div>
    <div class=" large-4 columns">
-   Durasi
-   <select name="durasi"> 
+   <label for="durasi">Durasi</label>
+   <select id="durasi" name="durasi">  
    <?php for ($i = 1; $i <= 9; $i++) { ?>
    <option value="<?php echo $i ?>"> <?php echo $i ?> jam </option>
    <?php } ?>
    </select><br>
-   Jenis  
-   <select name="jenis_jadwal"> 
+   <label for="jenis_jadwal">Jenis</label>
+   <select id="jenis_jadwal" name="jenis_jadwal">  
    <option value="konsultasi" style="background-color:yellow"> Konsultasi </option>
    <option value="terjadwal" style="background-color:green;color:white"> Terjadwal</option>
    <option value="kelas" style="background-color:white"> Kelas </option>
    </select>
    </div>
    <div class="large-4 columns">
-   Label <input type="text" name="label_jadwal"><br>
-   <input type="submit" class="button" value="Tambah">
+   <label for="label_jadwal">Label</label>
+   <input id="label_jadwal" type="text" name="label_jadwal"><br>
    </form>
    </div>
    </div>
diff --git a/www/application/views/TranskripManage/main.php b/www/application/views/TranskripManage/main.php
index 0970fc2..6dffa7d 100644
--- a/www/application/views/TranskripManage/main.php
+++ b/www/application/views/TranskripManage/main.php
@@ -12,8 +12,8 @@ defined('BASEPATH') OR exit('No direct script access allowed');
    <h5>Permintaan Transkrip</h5>
    <form method="GET" action="/TranskripManage">
    <div class="input-group">
-   <span class="input-group-label">Cari NPM:</span>
-   <input name="npm" class="input-group-field" type="text" placeholder="2013730013" maxlength="10" minlength="10"<?= $npmQuery === NULL ? '' : " value='$npmQuery'" ?>/>
+   <label for="npm" class="input-group-label">Cari NPM:</label>
+   <input id="npm" name="npm" class="input-group-field" type="text" placeholder="2013730013" maxlength="10" minlength="10"<?= $npmQuery === NULL ? '' : " value='$npmQuery'" ?>/>
    <div class="input-group-button">
    <input class="button" type="submit" value="Cari"/>
    </div>
\end{lstlisting} 

\subsection{Perbaikan Kriteria Sukses 4.1.1 \textit{Parsing}}
\label{subsec:perbaikan_kriteria_sukses_4.1.1}
Pada bagian ini dilakukan perbaikan sebagai berikut:

\begin{itemize}
\item Pada penggunaan \textit{tag "time"} buka yang tidak disertai dengan \textit{tag "time"} tutup di halaman cetak transkrip, bagian yang tidak lengkap ditambahkan \textit{</time>}. Perbaikan yang dilakukan dapat dilihat di \ref{lst:perbaikan_4.1.1_parsing_halaman_cetak_transkrip}.
\begin{lstlisting}[frame=single, label={lst:perbaikan_4.1.1_parsing_halaman_cetak_transkrip}, language=diff, caption=Perbaikan Kriteria Sukses 4.1.1 - Kesalahan Elemen pada Halaman Cetak Transkrip]
diff --git a/www/application/views/TranskripRequest/main.php b/www/application/views/TranskripRequest/main.php
index ada0254..ed45720 100644
--- a/www/application/views/TranskripRequest/main.php
+++ b/www/application/views/TranskripRequest/main.php
@@ -77,7 +77,7 @@ defined('BASEPATH') OR exit('No direct script access allowed');
    <td><span class="<?= $request->labelClass ?> label"><?= $request->status ?></span></td>
    <td><time datetime="<?= $request->requestDateTime ?>"><?= $request->requestDateString ?></time></td>
    <td><?= $request->requestType ?></td>
-   <td><time datetime="<?= $request->answeredDateTime ?>"><?= $request->answeredDateString ?></td>
+   <td><time datetime="<?= $request->answeredDateTime ?>"><?= $request->answeredDateString ?></time></td>
    <td><?= $request->answeredMessage ?></td>
    <td>
    <div class="reveal" id="detail<?= $request->id ?>" data-reveal>
\end{lstlisting} 

\item Pada penempatan elemen \textit{"HTML"} yang tidak tepat di halaman entri jadwal dosen, \textit{tag "div"} dipindahkan agar berada sesudah \textit{tag "body"}. \textit{Tag "div"} yang bermasalah ini muncul dari \textit{file} "flashmessage.php" yang dimuat di tempat yang salah. Oleh karena perbaikan yang dilakukan adalah dengan memuat \textit{file} "flashmessage.php" di tempat yang seharusnya. Perbaikan yang dilakukan dapat dilihat di \ref{lst:perbaikan_4.1.1_parsing_halaman_entri_jadwal_dosen}.
\begin{lstlisting}[frame=single, label={lst:perbaikan_4.1.1_parsing_halaman_entri_jadwal_dosen}, language=diff, caption=Perbaikan Kriteria Sukses 4.1.1 - Kesalahan Elemen pada Halaman Entri Jadwal Dosen]
diff --git a/www/application/views/EntriJadwalDosen/main.php b/www/application/views/EntriJadwalDosen/main.php
index 6392b95..40d0c73 100644
--- a/www/application/views/EntriJadwalDosen/main.php
+++ b/www/application/views/EntriJadwalDosen/main.php
@@ -4,10 +4,10 @@ defined('BASEPATH') OR exit('No direct script access allowed');
    <html class="no-js" lang="en">
    <?php $this->load->view('templates/script_foundation'); ?>
    <?php $this->load->view('templates/head_loggedin'); ?>
-	<?php $this->load->view('templates/flashmessage'); ?>
    <?php $this->load->helper('url'); ?>
    <body>
    <?php $this->load->view('templates/topbar_loggedin'); ?>
+   <?php $this->load->view('templates/flashmessage'); ?>
 
    <div class="row">
\end{lstlisting} 
\end{itemize}

\section{Pengujian}
\label{sec:pengujian}
Pada bagian ini akan diisi dengan skenario pengujian dan hasil yang didapatkan dari setiap skenario pengujian yang telah dibuat. Tujuannya adalah untuk melihat apakah perbaikan-perbaikan yang telah dilakukan di bagian \ref{sec:implementasi} berhasil atau tidak. Pengujian dilakukan dengan menggunakan perangkat komputer berupa laptop dengan sistem operasi Windows, \textit{browser} Google Chrome, dan \textit{screen reader} ChromeVox sebagai teknologi alat bantu. Pengujian dilakukan pada server lokal milik penguji dan akun yang digunakan oleh penguji memiliki hak akses tak terbatas sehingga dapat menggunakan semua fitur yang terdapat pada halaman web BlueTape.

\subsection{Skenario Pengujian}
\label{subsec:skenario_pengujian}
Pada bagian ini akan dituliskan skenario pengujian yang digunakan untuk menguji perbaikan-perbaikan yang telah dilakukan. Setiap skenario pengujian merupakan langkah-langkah yang dituliskan dalam bentuk poin-poin yang menjelaskan cara untuk menggunakan fitur-fitur pada halaman web BlueTape.

\subsubsection{Login}
\label{subsubsec:skenario_login}
Langkah-langkah untuk \textit{login} pada halaman web BlueTape adalah sebagai berikut:

\begin{enumerate}
    \item Membuka halaman web BlueTape dengan menggunakan alamat \url{http://bluetape.localhost/}.
    \item Menekan tombol "Login dengan Google".
    \item Memilih akun yang sudah terdaftar pada aplikasi BlueTape.
\end{enumerate}

\subsubsection{Membuat Permohonan Cetak Transkrip}
\label{subsubsec:skenario_membuat_permohonan_cetak_transkrip}
Langkah-langkah untuk membuat permohonan cetak transkrip pada halaman web BlueTape adalah sebagai berikut:

\begin{enumerate}
    \item Melakukan \textit{login} pada halaman web BlueTape.
    \item Masuk ke halaman cetak transkrip.
    \item Mengisi form permohonan cetak transkrip.
    \item Mengirim permohonan.
\end{enumerate}

\subsubsection{Melihat Detail Permohonan di Halaman Cetak Transkrip}
\label{subsubsec:skenario_melihat_detail_permohonan_di_halaman_cetak_transkrip}
Langkah-langkah untuk melihat detail permohonan di halaman cetak transkrip pada halaman web BlueTape adalah sebagai berikut:

\begin{enumerate}
    \item Melakukan \textit{login} pada halaman web BlueTape.
    \item Masuk ke halaman cetak transkrip.
    \item Mengakses ikon mata yang terdapat pada tabel "Histori Permohonan" sesuai dengan detail permohonan yang ingin dilihat.
\end{enumerate}

\subsection{Hasil Pengujian}
\label{subsec:hasil_pengujian}
Pada bagian ini akan dituliskan hasil dari pengujian yang telah dilakukan berdasarkan skenario pengujian pada bagian \ref{subsubsec:skenario_login}. Hasil pengujian akan dituliskan dalam bentuk tabel yang berisi langkah skenario pengujian, hasil pengujian, dan tindakan yang dilakukan dalam pengujian.

\subsubsection{Login}
\label{subsubsec:hasil_pengujian_login}

\begin{table}[H]
    \centering 
    \caption{Tabel Hasil Pengujian}
    \label{tab:hasil_pengujian_login}
    \begin{tabular}{|c|c|p{10cm}|}
        \toprule
        Langkah ke & Hasil (sukses/tidak) & Tindakan \\

        \midrule
        1 & Sukses & Mengetik alamat \url{http://bluetape.localhost/} ketika ChromeVox menyebutkan \textit{"new tab, tab"}, lalu menekan \textit{"enter"}. \\
        2 & Sukses & Menekan \textit{"enter"} pada saat ChromeVox menyebutkan \textit{"login, Login dengan Google, link"}. \\
        3 & Sukses & Menekan \textit{"tab"} sebanyak dua kali hingga ChromeVox menyebutkan "Hizkia Steven, 7315020@student.unpar.ac.id", lalu menekan \textit{"enter"}. \\

        \bottomrule

    \end{tabular}
\end{table}

\subsubsection{Membuat Permohonan Cetak Transkrip}
\label{subsubsec:hasil_pengujian_membuat_permohonan_cetak_transkrip}

\begin{table}[H]
    \centering 
    \caption{Tabel Hasil Pengujian}
    \label{tab:hasil_pengujian_membuat_permohonan_cetak_transkrip}
    \begin{tabular}{|c|c|p{10cm}|}
        \toprule
        Langkah ke & Hasil (sukses/tidak) & Tindakan \\

        \midrule
        1 & Sukses & Mengikuti langkah-langkah pada bagian \ref{subsubsec:hasil_pengujian_login}. \\
        2 & Sukses & Tidak ada karena halaman awal yang ditampilkan setelah berhasil melakukan \textit{login} adalah halaman cetak transkrip. \\
        3 & Sukses & Menekan \textit{"shift + tab"} sebanyak satu kali hingga ChromeVox menyebutkan "lompat ke menu utama" lalu menekan \textit{"enter"}. Setelah ChromeVox menyebutkan "permohonan baru, \textit{heading} satu", menekan \textit{"tab"} sebanyak lima kali hingga ChromeVox menyebutkan "keperluan, \textit{edit text}" lalu mengetik "Tes Permohonan Baru" untuk mengisi bagian tersebut. \\
        4 & Sukses & Menekan \textit{"tab"} sebanyak satu kali hingga ChromeVox menyebutkan "kirim permohonan, \textit{button}" lalu menekan \textit{"enter"}. \\

        \bottomrule

    \end{tabular}
\end{table}

\subsubsection{Melihat Detail Permohonan di Halaman Cetak Transkrip}
\label{subsubsec:hasil_pengujian_melihat_detail_permohonan_di_halaman_cetak_transkrip}

\begin{table}[H]
    \centering 
    \caption{Tabel Hasil Pengujian}
    \label{tab:hasil_pengujian_melihat_detail_permohonan_di_halaman_cetak_transkrip}
    \begin{tabular}{|c|c|p{10cm}|}
        \toprule
        Langkah ke & Hasil (sukses/tidak) & Tindakan \\

        \midrule
        1 & Sukses & Mengikuti langkah-langkah pada bagian \ref{subsubsec:hasil_pengujian_login}. \\
        2 & Sukses & Tidak ada karena halaman awal yang ditampilkan setelah berhasil melakukan \textit{login} adalah halaman cetak transkrip. \\
        3 &  &  \\

        \bottomrule

    \end{tabular}
\end{table}