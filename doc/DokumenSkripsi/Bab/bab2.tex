%versi 2 (8-10-2016)
\setcounter{secnumdepth}{4}
\titleformat{\paragraph}{\normalfont\normalsize\bfseries}{\theparagraph}{1em}{}
\titlespacing*{\paragraph}{0pt}{3.25ex plus 1ex minus .2ex}{1.5ex plus .2ex}

\chapter{Landasan Teori}
\label{chap:teori}

%2.1 WCAG 2.1%
\section{\textit{WCAG 2.1}}
\label{sec:wcag_2.1} 
\textit{WCAG} 2.1 adalah versi ketiga dari \textit{WCAG} yang dirilis pada tanggal 5 Juni 2018. \textit{WCAG} 2.1 dibuat untuk meningkatkan versi sebelumnya yaitu \textit{WCAG} 2.0. Pada \textit{WCAG} 2.1 terdapat penambahan kriteria sukses baru beserta definisi-definisi pendukungnya, pedoman untuk mengatur penambahan, dan beberapa tambahan pada bagian kesesuaian. Dalam \textit{WCAG} 2.1 terdapat 3 level kriteria sukses yaitu A, AA, dan AAA yang digunakan sebagai acuan untuk menilai tingkat kepatuhan sebuah situs web terhadap \textit{WCAG} 2.1.

%2.1.1 Perceivable%
\subsection{\textit{Perceivable}}
\label{sec:perceivable}
Pada subbab ini dijabarkan poin-poin kriteria sukses yang berkenaan dengan komponen informasi dan antarmuka pengguna sehingga dapat ditampilkan dengan cara yang dapat dimengerti oleh pengguna.

%2.1.1.1 Text Alternatives%
\subsubsection{\textit{Text Alternatives}}
\label{sec:text_alternatives}
Pada subbab ini dijabarkan poin-poin kriteria sukses yang berkenaan dengan penggunaan teks alternatif untuk setiap konten yang bukan merupakan teks.

\paragraph{Kriteria Sukses 1.1.1 \textit{Non-text Content}}
\label{sec:kriteria_sukses_1.1.1}
Lorem ipsum dolor sit amet
%End of 2.1.1.1 Text Alternatives%

%2.1.1.2 Time-based Media%
\subsubsection{\textit{Time-based Media}}
\label{sec:time_based_media}
Lorem ipsum dolor sit amet

\paragraph{Kriteria Sukses 1.2.1 \textit{Audio-only and Video-only (Prerecorded)}}
\label{sec:kriteria_sukses_1.2.1}
Lorem ipsum dolor sit amet

\paragraph{Kriteria Sukses 1.2.2 \textit{Captions (Prerecorded)}}
\label{sec:kriteria_sukses_1.2.2}
Lorem ipsum dolor sit amet

\paragraph{Kriteria Sukses 1.2.3 \textit{Audio Description or Media Alternative (Prerecorded)}}
\label{sec:kriteria_sukses_1.2.3}
Lorem ipsum dolor sit amet

\paragraph{Kriteria Sukses 1.2.4 \textit{Captions (Live)}}
\label{sec:kriteria_sukses_1.2.4}
Lorem ipsum dolor sit amet

\paragraph{Kriteria Sukses 1.2.5 \textit{Audio Description (Prerecorded)}}
\label{sec:kriteria_sukses_1.2.5}
Lorem ipsum dolor sit amet

\paragraph{Kriteria Sukses 1.2.6 \textit{Sign Language (Prerecorded)}}
\label{sec:kriteria_sukses_1.2.6}
Lorem ipsum dolor sit amet

\paragraph{Kriteria Sukses 1.2.7 \textit{Extended Audio Description (Prerecorded)}}
\label{sec:kriteria_sukses_1.2.7}
Lorem ipsum dolor sit amet

\paragraph{Kriteria Sukses 1.2.8 \textit{Media Alternative (Prerecorded)}}
\label{sec:kriteria_sukses_1.2.8}
Lorem ipsum dolor sit amet

\paragraph{Kriteria Sukses 1.2.9 \textit{Audio-only (Live)}}
\label{sec:kriteria_sukses_1.2.9}
Lorem ipsum dolor sit amet
%End of 2.1.1.2 Time-based Media%

%2.1.1.3 Adaptable%
\subsubsection{\textit{Adaptable}}
\label{sec:adaptable}
Lorem ipsum dolor sit amet

\paragraph{Kriteria Sukses 1.3.1 \textit{Info and Relationships}}
\label{sec:kriteria_sukses_1.3.1}
Lorem ipsum dolor sit amet

\paragraph{Kriteria Sukses 1.3.2 \textit{Meaningful Sequence}}
\label{sec:kriteria_sukses_1.3.2}
Lorem ipsum dolor sit amet

\paragraph{Kriteria Sukses 1.3.3 \textit{Sensory Characteristics}}
\label{sec:kriteria_sukses_1.3.3}
Lorem ipsum dolor sit amet

\paragraph{Kriteria Sukses 1.3.4 \textit{Orientation}}
\label{sec:kriteria_sukses_1.3.4}
Lorem ipsum dolor sit amet

\paragraph{Kriteria Sukses 1.3.5 \textit{Identify Input Purpose}}
\label{sec:kriteria_sukses_1.3.5}
Lorem ipsum dolor sit amet

\paragraph{Kriteria Sukses 1.3.6 \textit{Identify Purpose}}
\label{sec:kriteria_sukses_1.3.6}
Lorem ipsum dolor sit amet
%End of 2.1.1.3 Adaptable%

%2.1.1.4 Distinguishable%
\subsubsection{\textit{Distinguishable}}
\label{sec:distinguishable}
Lorem ipsum dolor sit amet

\paragraph{Kriteria Sukses 1.4.1 \textit{Use of Color}}
\label{sec:kriteria_sukses_1.4.1}
Lorem ipsum dolor sit amet

\paragraph{Kriteria Sukses 1.4.2 \textit{Audio Control}}
\label{sec:kriteria_sukses_1.4.2}
Lorem ipsum dolor sit amet

\paragraph{Kriteria Sukses 1.4.3 \textit{Contrast (Minimum)}}
\label{sec:kriteria_sukses_1.4.3}
Lorem ipsum dolor sit amet

\paragraph{Kriteria Sukses 1.4.4 \textit{Resize text}}
\label{sec:kriteria_sukses_1.4.4}
Lorem ipsum dolor sit amet

\paragraph{Kriteria Sukses 1.4.5 \textit{Images of Text}}
\label{sec:kriteria_sukses_1.4.5}
Lorem ipsum dolor sit amet

\paragraph{Kriteria Sukses 1.4.6 \textit{Contrast (Enhanced)}}
\label{sec:kriteria_sukses_1.4.6}
Lorem ipsum dolor sit amet

\paragraph{Kriteria Sukses 1.4.7 \textit{Low or No Background Audio}}
\label{sec:kriteria_sukses_1.4.7}
Lorem ipsum dolor sit amet

\paragraph{Kriteria Sukses 1.4.8 \textit{Visual Presentation}}
\label{sec:kriteria_sukses_1.4.8}
Lorem ipsum dolor sit amet

\paragraph{Kriteria Sukses 1.4.9 \textit{Images of Text (No Exception)}}
\label{sec:kriteria_sukses_1.4.9}
Lorem ipsum dolor sit amet

\paragraph{Kriteria Sukses 1.4.10 \textit{Reflow}}
\label{sec:kriteria_sukses_1.4.10}
Lorem ipsum dolor sit amet

\paragraph{Kriteria Sukses 1.4.11 \textit{Non-text Contrast}}
\label{sec:kriteria_sukses_1.4.11}
Lorem ipsum dolor sit amet

\paragraph{Kriteria Sukses 1.4.12 \textit{Text Spacing}}
\label{sec:kriteria_sukses_1.4.12}
Lorem ipsum dolor sit amet

\paragraph{Kriteria Sukses 1.4.13 \textit{Content on Hover or Focus}}
\label{sec:kriteria_sukses_1.4.13}
Lorem ipsum dolor sit amet
%End of 2.1.1.4 Distinguishable%

%End of 2.1.1 Perceivable%

%2.1.2 Operable%
\subsection{\textit{Operable}}
\label{sec:operable}
Lorem ipsum dolor sit amet

%2.1.2.1 Keyboard Accessible%
\subsubsection{\textit{Keyboard Accessible}}
\label{sec:keyboard_accessible}
Lorem ipsum dolor sit amet

\paragraph{Kriteria Sukses 2.1.1 \textit{Keyboard}}
\label{sec:kriteria_sukses_2.1.1}
Lorem ipsum dolor sit amet

\paragraph{Kriteria Sukses 2.1.2 \textit{No Keyboard Trap}}
\label{sec:kriteria_sukses_2.1.2}
Lorem ipsum dolor sit amet

\paragraph{Kriteria Sukses 2.1.3 \textit{Keyboard (No Exception)}}
\label{sec:kriteria_sukses_2.1.3}
Lorem ipsum dolor sit amet

\paragraph{Kriteria Sukses 2.1.4 \textit{Character Key Shortcuts}}
\label{sec:kriteria_sukses_2.1.4}
Lorem ipsum dolor sit amet
%End of 2.1.2.1 Keyboard Accessible%

%2.1.2.2 Enough Time%
\subsubsection{\textit{Enough Time}}
\label{sec:enough_time}
Lorem ipsum dolor sit amet

\paragraph{Kriteria Sukses 2.2.1 \textit{Timing Adjustable}}
\label{sec:kriteria_sukses_2.2.1}
Lorem ipsum dolor sit amet

\paragraph{Kriteria Sukses 2.2.2 \textit{Pause, Stop, Hide}}
\label{sec:kriteria_sukses_2.2.2}
Lorem ipsum dolor sit amet

\paragraph{Kriteria Sukses 2.2.3 \textit{No Timing}}
\label{sec:kriteria_sukses_2.2.3}
Lorem ipsum dolor sit amet

\paragraph{Kriteria Sukses 2.2.4 \textit{Interruptions}}
\label{sec:kriteria_sukses_2.2.4}
Lorem ipsum dolor sit amet

\paragraph{Kriteria Sukses 2.2.5 \textit{Re-authenticating}}
\label{sec:kriteria_sukses_2.2.5}
Lorem ipsum dolor sit amet

\paragraph{Kriteria Sukses 2.2.6 \textit{Timeouts}}
\label{sec:kriteria_sukses_2.2.6}
Lorem ipsum dolor sit amet
%End of 2.1.2.2 Enough Time%

%2.1.2.3 Seizures and Physical Reactions%
\subsubsection{\textit{Seizures and Physical Reactions}}
\label{sec:seizures_and_physical_reactions}
Lorem ipsum dolor sit amet

\paragraph{Kriteria Sukses 2.3.1 \textit{Three Flashes or Below Threshold}}
\label{sec:kriteria_sukses_2.3.1}
Lorem ipsum dolor sit amet

\paragraph{Kriteria Sukses 2.3.2 \textit{Three Flashes}}
\label{sec:kriteria_sukses_2.3.2}
Lorem ipsum dolor sit amet

\paragraph{Kriteria Sukses 2.3.3 \textit{Animation from Interactions}}
\label{sec:kriteria_sukses_2.3.3}
Lorem ipsum dolor sit amet
%End of 2.1.2.3 Seizures and Physical Reactions%

%2.1.2.4 Navigable%
\subsubsection{\textit{Navigable}}
\label{sec:navigable}
Lorem ipsum dolor sit amet

\paragraph{Kriteria Sukses 2.4.1 \textit{Bypass Blocks}}
\label{sec:kriteria_sukses_2.4.1}
Lorem ipsum dolor sit amet

\paragraph{Kriteria Sukses 2.4.2 \textit{Page Titled}}
\label{sec:kriteria_sukses_2.4.2}
Lorem ipsum dolor sit amet

\paragraph{Kriteria Sukses 2.4.3 \textit{Focus Order}}
\label{sec:kriteria_sukses_2.4.3}
Lorem ipsum dolor sit amet

\paragraph{Kriteria Sukses 2.4.4 \textit{Link Purpose (In Context)}}
\label{sec:kriteria_sukses_2.4.4}
Lorem ipsum dolor sit amet

\paragraph{Kriteria Sukses 2.4.5 \textit{Multiple Ways}}
\label{sec:kriteria_sukses_2.4.5}
Lorem ipsum dolor sit amet

\paragraph{Kriteria Sukses 2.4.6 \textit{Headings and Labels}}
\label{sec:kriteria_sukses_2.4.6}
Lorem ipsum dolor sit amet

\paragraph{Kriteria Sukses 2.4.7 \textit{Focus Visible}}
\label{sec:kriteria_sukses_2.4.7}
Lorem ipsum dolor sit amet

\paragraph{Kriteria Sukses 2.4.8 \textit{Location}}
\label{sec:kriteria_sukses_2.4.8}
Lorem ipsum dolor sit amet

\paragraph{Kriteria Sukses 2.4.9 \textit{Link Purpose (Link Only)}}
\label{sec:kriteria_sukses_2.4.9}
Lorem ipsum dolor sit amet

\paragraph{Kriteria Sukses 2.4.10 \textit{Section Headings}}
\label{sec:kriteria_sukses_2.4.10}
Lorem ipsum dolor sit amet
%End of 2.1.2.4 Navigable%

%2.1.2.5 Input Modalities%
\subsubsection{\textit{Input Modalities}}
\label{sec:input_modalities}
Lorem ipsum dolor sit amet

\paragraph{Kriteria Sukses 2.5.1 \textit{Pointer Gestures}}
\label{sec:kriteria_sukses_2.5.1}
Lorem ipsum dolor sit amet

\paragraph{Kriteria Sukses 2.5.2 \textit{Pointer Cancellation}}
\label{sec:kriteria_sukses_2.5.2}
Lorem ipsum dolor sit amet

\paragraph{Kriteria Sukses 2.5.3 \textit{Label in Name}}
\label{sec:kriteria_sukses_2.5.3}
Lorem ipsum dolor sit amet

\paragraph{Kriteria Sukses 2.5.4 \textit{Motion Actuation}}
\label{sec:kriteria_sukses_2.5.4}
Lorem ipsum dolor sit amet

\paragraph{Kriteria Sukses 2.5.5 \textit{Target Size}}
\label{sec:kriteria_sukses_2.5.5}
Lorem ipsum dolor sit amet

\paragraph{Kriteria Sukses 2.5.6 \textit{Concurrent Input Mechanisms}}
\label{sec:kriteria_sukses_2.5.6}
Lorem ipsum dolor sit amet
%End of 2.1.2.5 Input Modalities%

%End of 2.1.2 Operable%

%2.1.3 Understandable%
\subsection{\textit{Understandable}}
\label{sec:understandable}
Lorem ipsum dolor sit amet

%2.1.3.1 Readable%
\subsubsection{\textit{Readable}}
\label{sec:readable}
Lorem ipsum dolor sit amet

\paragraph{Kriteria Sukses 3.1.1 \textit{Language of Page}}
\label{sec:kriteria_sukses_3.1.1}
Lorem ipsum dolor sit amet

\paragraph{Kriteria Sukses 3.1.2 \textit{Language of Parts}}
\label{sec:kriteria_sukses_3.1.2}
Lorem ipsum dolor sit amet

\paragraph{Kriteria Sukses 3.1.3 \textit{Unusual Words}}
\label{sec:kriteria_sukses_3.1.3}
Lorem ipsum dolor sit amet

\paragraph{Kriteria Sukses 3.1.4 \textit{Abbreviations}}
\label{sec:kriteria_sukses_3.1.4}
Lorem ipsum dolor sit amet

\paragraph{Kriteria Sukses 3.1.5 \textit{Reading Level}}
\label{sec:kriteria_sukses_3.1.5}
Lorem ipsum dolor sit amet

\paragraph{Kriteria Sukses 3.1.6 \textit{Pronunciation}}
\label{sec:kriteria_sukses_3.1.6}
Lorem ipsum dolor sit amet
%End of 2.1.3.1 Readable%

%2.1.3.2 Predictable%
\subsubsection{\textit{Predictable}}
\label{sec:predictable}
Lorem ipsum dolor sit amet

\paragraph{Kriteria Sukses 3.2.1 \textit{On Focus}}
\label{sec:kriteria_sukses_3.2.1}
Lorem ipsum dolor sit amet

\paragraph{Kriteria Sukses 3.2.2 \textit{On Input}}
\label{sec:kriteria_sukses_3.2.2}
Lorem ipsum dolor sit amet

\paragraph{Kriteria Sukses 3.2.3 \textit{Consistent Navigation}}
\label{sec:kriteria_sukses_3.2.3}
Lorem ipsum dolor sit amet

\paragraph{Kriteria Sukses 3.2.4 \textit{Consistent Identification}}
\label{sec:kriteria_sukses_3.2.4}
Lorem ipsum dolor sit amet

\paragraph{Kriteria Sukses 3.2.5 \textit{Change on Request}}
\label{sec:kriteria_sukses_3.2.5}
Lorem ipsum dolor sit amet
%End of 2.1.3.2 Predictable%

%2.1.3.3 Input Assistance%
\subsubsection{\textit{Input Assistance}}
\label{sec:input_assistance}
Lorem ipsum dolor sit amet

\paragraph{Kriteria Sukses 3.3.1 \textit{Error Identification}}
\label{sec:kriteria_sukses_3.3.1}
Lorem ipsum dolor sit amet

\paragraph{Kriteria Sukses 3.3.2 \textit{Labels or Instructions}}
\label{sec:kriteria_sukses_3.3.2}
Lorem ipsum dolor sit amet

\paragraph{Kriteria Sukses 3.3.3 \textit{Error Suggestion}}
\label{sec:kriteria_sukses_3.3.3}
Lorem ipsum dolor sit amet

\paragraph{Kriteria Sukses 3.3.4 \textit{Error Prevention (Legal, Financial, Data)}}
\label{sec:kriteria_sukses_3.3.4}
Lorem ipsum dolor sit amet

\paragraph{Kriteria Sukses 3.3.5 \textit{Help}}
\label{sec:kriteria_sukses_3.3.5}
Lorem ipsum dolor sit amet

\paragraph{Kriteria Sukses 3.3.6 \textit{Error Prevention (All)}}
\label{sec:kriteria_sukses_3.3.6}
Lorem ipsum dolor sit amet
%End of 2.1.3.3 Input Assistance%

%End of 2.1.3 Understandable%

%2.1.4 Robust%
\subsection{\textit{Robust}}
\label{sec:robust}
Lorem ipsum dolor sit amet

%2.1.4.1 Compatible%
\subsubsection{\textit{Compatible}}
\label{sec:compatible}
Lorem ipsum dolor sit amet

\paragraph{Kriteria Sukses 4.1.1 \textit{Parsing}}
\label{sec:kriteria_sukses_4.1.1}
Lorem ipsum dolor sit amet

\paragraph{Kriteria Sukses 4.1.2 \textit{Name, Role, Value}}
\label{sec:kriteria_sukses_4.1.2}
Lorem ipsum dolor sit amet

\paragraph{Kriteria Sukses 4.1.3 \textit{Status Messages}}
\label{sec:kriteria_sukses_4.1.3}
Lorem ipsum dolor sit amet
%End of 2.1.4.1 Compatible%

%End of 2.1.4 Robust%

%End of WCAG 2.1%

\section{BlueTape}
\label{sec:bluetape}


\section{Template Skripsi FTIS UNPAR}
\label{sec:template}
 
Akan dipaparkan bagaimana menggunakan template ini, termasuk petunjuk singkat membuat referensi, gambar dan tabel.
Juga hal-hal lain yang belum terpikir sampai saat ini. 
 
\dtext{15-16}

\subsection{Tabel}  
Berikut adalah contoh pembuatan tabel. 
Penempatan tabel dan gambar secara umum diatur secara otomatis oleh \LaTeX{}, perhatikan contoh di file bab2.tex untuk melihat bagaimana cara memaksa tabel ditempatkan sesuai keinginan kita.

Perhatikan bawa berbeda dengan penempatan judul gambar gambar, keterangan tabel harus diletakkan di atas tabel!!
Lihat Tabel~\ref{tab:contoh1} berikut ini:

\begin{table}[H] %atau h saja untuk "kira kira di sini"
	\centering 
	\caption{Tabel contoh}
	\label{tab:contoh1}
	\begin{tabular}{cccc}
		\toprule
		& $v_{start}$ & $\mathcal{S}_{1}$ & $v_{end}$\\

		\midrule
		$\tau_{1}$ & 1 & 12& 20\\
		$\tau_{2}$ & 1 &  & 20\\
		$\tau_{3}$ & 1 & 9 & 20\\
		$\tau_{4}$ & 1 &  & 20\\

		\bottomrule
		
	\end{tabular} 
\end{table}
Tabel~\ref{tab:cthwarna1} dan Tabel~\ref{tab:cthwarna2} berikut ini adalah tabel dengan sel yang berwarna dan ada dua tabel yang bersebelahan. 
\begin{table}[H]
	\begin{minipage}[c]{0.49\linewidth}
		\centering
		\caption{Tabel bewarna(1)}
		\label{tab:cthwarna1}
		\begin{tabular}{ccccc}
			\toprule
			 & $v_{start}$ & $\mathcal{S}_{2}$ & $\mathcal{S}_{1}$ & $v_{end}$\\
			
			\midrule
			$\tau_{1}$ & 1 & 5 \cellcolor{green}& 12& 20\\
			$\tau_{2}$ & 1 & 8 \cellcolor{green}& & 20\\
			$\tau_{3}$ & 1 & 2/8/17 \cellcolor{green}& 9 & 20\\
			$\tau_{4}$ & 1 & \cellcolor{red}& & 20\\
			
			\bottomrule

		\end{tabular}
	\end{minipage}
	\begin{minipage}[c]{0.49\linewidth}
		
		\centering 
		\caption{Tabel bewarna(2)}
		\label{tab:cthwarna2}
		\begin{tabular}{ccccc}
			\toprule
			 & $v_{start}$ & $\mathcal{S}_{1}$ & $\mathcal{S}_{2}$ & $v_{end}$\\
			
			\midrule
			$\tau_{1}$ & 1 & 12& 5 \cellcolor{red} &20\\
			$\tau_{2}$ & 1 &  &  8 \cellcolor{green} &20\\
			$\tau_{3}$ & 1 & 9 & 2/8/17 \cellcolor{green} &20\\
			$\tau_{4}$ & 1 &   & \cellcolor{red} &20\\
			
			\bottomrule
		
		\end{tabular}
	\end{minipage}
\end{table}

 
\subsection{Kutipan}
\label{subs:kutipan} 
Berikut contoh kutipan dari berbagai sumber, untuk keterangan lebih lengkap, silahkan membaca file referensi.bib yang disediakan juga di template ini.
Contoh kutipan:
\begin{itemize}
	\item Buku:~\cite{berg:08:compgeom} 
	\item Bab dalam buku:~\cite{kreveld:04:GIS}
	\item Artikel dari Jurnal:~\cite{buchin:13:median}
	\item Artikel dari prosiding seminar/konferensi:~\cite{kreveld:11:median}
	\item Skripsi/Thesis/Disertasi:~\cite{lionov:02:animasi}~\cite{wiratma:10:following}~\cite{wiratma:22:later}
	\item Technical/Scientific Report:~\cite{kreveld:07:watertight}
	\item RFC (Request For Comments):~\cite{RFC1654}
	\item Technical Documentation/Technical Manual:~\cite{Z.500}~\cite{unicode:16:stdv9}~\cite{google:16:and7}
	\item Paten:~\cite{webb:12:comm}
	\item Tidak dipublikasikan:~\cite{wiratma:09:median}~\cite{lionov:11:cpoly}
	\item Laman web:~\cite{erickson:03:cgmodel}  
	\item Lain-lain:~\cite{agung:12:tango}
\end{itemize}    
  
\subsection{Gambar}

Pada hampir semua editor, penempatan gambar di dalam dokumen \LaTeX{} tidak dapat dilakukan melalui proses {\it drag and drop}.
Perhatikan contoh pada file bab2.tex untuk melihat bagaimana cara menempatkan gambar.
Beberapa hal yang harus diperhatikan pada saat menempatkan gambar:
\begin{itemize}
	\item Setiap gambar {\bf harus} diacu di dalam teks (gunakan {\it field} {\sc label})
	\item {\it Field} {\sc caption} digunakan untuk teks pengantar pada gambar. Terdapat dua bagian yaitu yang ada di antara tanda $[$ dan $]$ dan yang ada di antara tanda $\{$ dan $\}$. Yang pertama akan muncul di Daftar Gambar, sedangkan yang kedua akan muncul di teks pengantar gambar. Untuk skripsi ini, samakan isi keduanya.
	\item Jenis file yang dapat digunakan sebagai gambar cukup banyak, tetapi yang paling populer adalah tipe {\sc png} (lihat Gambar~\ref{fig:ularpng}), tipe {\sc jpg} (Gambar~\ref{fig:ularjpg}) dan tipe {\sc pdf} (Gambar~\ref{fig:ularpdf})
	\item Besarnya gambar dapat diatur dengan {\it field} {\sc scale}.
	\item Penempatan gambar diatur menggunakan {\it placement specifier} (di antara tanda  $[$ dan $]$ setelah deklarasi gambar.
	Yang umum digunakan adalah {\bf H} untuk menempatkan gambar {\bf sesuai} penempatannya di file .tex atau  {\bf h} yang berarti "kira-kira" di sini. \\
	Jika tidak menggunakan {\it placement specifier}, \LaTeX{} akan menempatkan gambar secara otomatis untuk menghindari bagian kosong pada dokumen anda.
	Walaupun cara ini sangat mudah, hindarkan terjadinya penempatan dua gambar secara berurutan. 	
	\begin{itemize}
		\item Gambar~\ref{fig:ularpng} ditempatkan di bagian atas halaman, walaupun penempatannya dilakukan setelah penulisan 3 paragraf setelah penjelasan ini.
		\item Gambar~\ref{fig:ularjpg} dengan skala 0.5 ditempatkan di antara dua buah paragraf. Perhatikan penulisannya di dalam file bab2.tex!
		\item Gambar~\ref{fig:ularpdf} ditempatkan menggunakan {\it specifier} {\bf h}.
	\end{itemize}
\end{itemize}
 
\dtext{17-18}
\begin{figure} 
	\centering  
	\includegraphics[scale=1]{ular-png}  
	\caption[Gambar {\it Serpentes} dalam format png]{Gambar {\it Serpentes} dalam format png} 
	\label{fig:ularpng} 
\end{figure} 

\dtext{19-20}
\begin{figure}[H]
	\centering  
	\includegraphics[scale=0.5]{ular-jpg}  
	\caption[Ular kecil]{Ular kecil} 
	\label{fig:ularjpg} 
\end{figure} 
\dtext{21-22}

\begin{figure}[ht] 
	\centering  
	\includegraphics[scale=1]{ular-pdf}  
	\caption[ {\it Serpentes} betina]{ {\it Serpentes} jantan} 
	\label{fig:ularpdf} 
\end{figure} 
 
