%versi 2 (8-10-2016)
\setcounter{secnumdepth}{3}

\chapter{Landasan Teori}
\label{chap:teori}

%2.1 WCAG 2.1%
\section{\textit{WCAG 2.1}}
\label{sec:wcag_2.1} 
\textit{Web Content Accessibility Guidelines (WCAG)} 2.1 adalah versi ketiga dari \textit{WCAG} yang dirilis pada tanggal 5 Juni 2018. Versi pertama dari \textit{WCAG} adalah \textit{WCAG} 1.0 yang dirilis pada tanggal 5 Mei 1999 dan versi kedua adalah \textit{WCAG} 2.0 yang dirilis pada tanggal 11 Desember 2008. \textit{WCAG} 2.1 dikembangkan oleh World Wide Web Consortium (W3C) melalui kerja sama dengan individu dan organisasi di seluruh dunia, dengan tujuan memberikan standar bersama untuk aksesibilitas konten web yang memenuhi kebutuhan individu, organisasi, dan pemerintah internasional. 

\textit{WCAG} 2.1 dibuat untuk meningkatkan versi sebelumnya yaitu \textit{WCAG} 2.0. Pada \textit{WCAG} 2.1 terdapat penambahan kriteria sukses baru beserta definisi-definisi pendukungnya, pedoman untuk mengatur penambahan, dan beberapa tambahan pada bagian tingkat kepatuhan. Dalam \textit{WCAG} 2.1 terdapat 3 level kriteria sukses yaitu A, AA, dan AAA yang digunakan sebagai acuan untuk menilai tingkat kepatuhan sebuah situs web terhadap \textit{WCAG} 2.1.

%2.1.1 Perceivable%
\subsection{\textit{Perceivable}}
\label{sec:perceivable}
Pada subbab ini dijabarkan poin-poin kriteria sukses yang berkenaan dengan komponen informasi dan antarmuka pengguna sehingga dapat ditampilkan dengan cara yang dapat dimengerti oleh pengguna.

%2.1.1.1 Text Alternatives%
\subsubsection{\textit{Text Alternatives}}
\label{sec:text_alternatives}
Untuk setiap konten yang bukan merupakan teks perlu disediakan teks alternatif.

\paragraph{Kriteria Sukses 1.1.1 \textit{Non-text Content}}
\label{sec:kriteria_sukses_1.1.1}
(Level A)\\

Semua konten bukan teks yang disajikan ke pengguna mempunyai alternatif berupa teks yang berperan sebagai ekuivalen, kecuali untuk situasi-situasi yang dicantumkan di bawah:
\begin{itemize}
	\item Kontrol, masukan: Bila konten bukan teks merupakan semacam kontrol atau bila konten tersebut menerima masukan dari pengguna, maka konten tersebut harus mempunyai nama yang menjelaskan tujuannya.
	\item Media berbasis waktu: Jika konten bukan teks merupakan media berbasis waktu, maka alternatif berupa teks paling tidak menyediakan identifikasi deskriptif dari konten bukan teks.
	\item Tes: Jika konten bukan teks merupakan tes atau latihan, yang akan mengungkap jawabannya jika disajikan dalam bentuk teks, maka alternatif berupa teks, paling tidak harus menyajikan identifikasi deskriptif dari konten bukan teks.
	\item Indra: Jika konten bukan teks utamanya dibuat untuk mencapai semacam pengalaman untuk indra tertentu, maka setidaknya alternatif berupa teks yang disediakan harus menyediakan identifikasi deskriptif untuk konten tersebut.
	\item \textit{CAPTCHA}: Jika tujuan dari konten bukan teks adalah untuk mengonfirmasi bahwa konten sedang diakses oleh orang dan bukannya komputer, maka alternatif berupa teks yang mengidentifikasi dan menjabarkan tujuan dari konten bukan teks disediakan, dan bentuk alternatif dari \textit{CAPTCHA} menggunakan mode keluaran untuk berbagai jenis persepsi sensoris disediakan untuk mengakomodasi berbagai disabilitas.
	\item Dekorasi, pemformatan, tak kentara: Jika konten bukan teks merupakan dekorasi semata, digunakan hanya untuk format visual, tidak disajikan kepada pengguna, maka konten itu diterapkan dalam cara yang dapat diacuhkan oleh teknologi alat bantu.
\end{itemize}
%End of 2.1.1.1 Text Alternatives%

%2.1.1.2 Time-based Media%
\subsubsection{\textit{Time-based Media}}
\label{sec:time_based_media}
Untuk media berbasis waktu perlu disediakan alternatif.

\paragraph{Kriteria Sukses 1.2.1 \textit{Audio-only and Video-only (Prerecorded)}}
\label{sec:kriteria_sukses_1.2.1}
(Level A)\\

Untuk media rekaman berupa audio saja dan video saja, ketentuan berikut ini berlaku, kecuali bila audio atau video tersebut merupakan alternatif media untuk teks dan dilabeli dengan jelas:
\begin{itemize}
	\item Rekaman audio saja: Tersedia alternatif untuk media berbasis waktu dan isinya mewakili informasi yang sama dengan konten rekaman audio saja.
	\item Rekaman video saja:  Tersedia alternatif untuk media berbasis waktu atau trek audio dan isinya mewakili informasi yang sama dengan konten rekaman video saja.
\end{itemize}

\paragraph{Kriteria Sukses 1.2.2 \textit{Captions (Prerecorded)}}
\label{sec:kriteria_sukses_1.2.2}
(Level A)\\

Keterangan disediakan untuk semua konten rekaman audio dalam media yang disinkronkan, kecuali bila media tersebut merupakan alternatif media untuk teks dan dilabeli dengan jelas.

\paragraph{Kriteria Sukses 1.2.3 \textit{Audio Description or Media Alternative (Prerecorded)}}
\label{sec:kriteria_sukses_1.2.3}
(Level A)\\

Tersedia alternatif untuk media berbasis waktu atau deskripsi audio dari konten video rekaman, untuk media yang disinkronkan, kecuali bila media tersebut merupakan alternatif media untuk teks dan dilabeli dengan jelas.

\paragraph{Kriteria Sukses 1.2.4 \textit{Captions (Live)}}
\label{sec:kriteria_sukses_1.2.4}
(Level AA)\\

Keterangan tersedia untuk semua konten audio yang disiarkan langsung pada media yang disinkronkan.

\paragraph{Kriteria Sukses 1.2.5 \textit{Audio Description (Prerecorded)}}
\label{sec:kriteria_sukses_1.2.5}
(Level AA)\\

Deskripsi audio disediakan untuk semua konten rekaman video pada media yang disinkronkan.

\paragraph{Kriteria Sukses 1.2.6 \textit{Sign Language (Prerecorded)}}
\label{sec:kriteria_sukses_1.2.6}
(Level AAA)\\

Penafsiran bahasa isyarat disediakan untuk semua konten rekaman audio pada media yang disinkronkan. 

\paragraph{Kriteria Sukses 1.2.7 \textit{Extended Audio Description (Prerecorded)}}
\label{sec:kriteria_sukses_1.2.7}
(Level AAA)\\

Ketika jeda di audio latar depan tidak memadai bagi deskripsi audio untuk menyampaikan maksud video, deskripsi audio tambahan disediakan untuk semua konten rekaman video pada media yang disinkronkan.

\paragraph{Kriteria Sukses 1.2.8 \textit{Media Alternative (Prerecorded)}}
\label{sec:kriteria_sukses_1.2.8}
(Level AAA)\\

Tersedia alternatif untuk media berbasis waktu untuk semua rekaman media yang disinkronkan dan semua rekaman media video saja.

\paragraph{Kriteria Sukses 1.2.9 \textit{Audio-only (Live)}}
\label{sec:kriteria_sukses_1.2.9}
(Level AAA)\\

Tersedia alternatif untuk media berbasis waktu yang menyajikan informasi yang sama dengan konten siaran langsung audio saja.
%End of 2.1.1.2 Time-based Media%

%2.1.1.3 Adaptable%
\subsubsection{\textit{Adaptable}}
\label{sec:adaptable}
Dalam membuat konten, sebaiknya konten dapat disajikan dalam berbagai cara (misalnya tata letak yang lebih sederhana) tanpa kehilangan informasi atau struktur konten tersebut.

\paragraph{Kriteria Sukses 1.3.1 \textit{Info and Relationships}}
\label{sec:kriteria_sukses_1.3.1}
(Level A)\\

Informasi, struktur, dan hubungan yang disampaikan melalui presentasi dapat ditentukan lewat pemrograman atau tersedia dalam bentuk teks. 

\paragraph{Kriteria Sukses 1.3.2 \textit{Meaningful Sequence}}
\label{sec:kriteria_sukses_1.3.2}
(Level A)\\

Ketika urutan konten yang disajikan memengaruhi maknanya, urutan membaca yang benar dapat ditentukan lewat pemrograman.

\paragraph{Kriteria Sukses 1.3.3 \textit{Sensory Characteristics}}
\label{sec:kriteria_sukses_1.3.3}
(Level A)\\

Instruksi yang disediakan untuk memahami maupun mengoperasikan konten, tidak hanya mengandalkan satu komponen karakteristik indra seperti bentuk, ukuran, lokasi visual, orientasi, atau suara.

\paragraph{Kriteria Sukses 1.3.4 \textit{Orientation}}
\label{sec:kriteria_sukses_1.3.4}
(Level AA)\\

Konten tidak membatasi tampilan dan operasinya hanya untuk satu orientasi tampilan, seperti \textit{portrait} atau \textit{landscape}, kecuali jika orientasi tampilan tertentu sangat penting.

\paragraph{Kriteria Sukses 1.3.5 \textit{Identify Input Purpose}}
\label{sec:kriteria_sukses_1.3.5}
(Level AA)\\

Tujuan dari setiap bidang masukan yang mengumpulkan informasi tentang pengguna dapat ditentukan melalui pemrograman ketika:
\begin{itemize}
	\item Bidang masukan menyajikan tujuan yang diidentifikasi di bagian tujuan masukan untuk komponen antarmuka pengguna.
	\item Konten diimplementasikan menggunakan teknologi dengan dukungan untuk mengidentifikasi makna yang diharapkan untuk masukan data formulir.
\end{itemize}

\paragraph{Kriteria Sukses 1.3.6 \textit{Identify Purpose}}
\label{sec:kriteria_sukses_1.3.6}
(Level AAA)\\

Pada konten yang diimplementasikan menggunakan bahasa markah, tujuan komponen antarmuka pengguna, ikon, dan bidang dapat ditentukan melalui pemrograman.
%End of 2.1.1.3 Adaptable%

%2.1.1.4 Distinguishable%
\subsubsection{\textit{Distinguishable}}
\label{sec:distinguishable}
Beri kemudahan bagi pengguna untuk melihat dan mendengar konten, termasuk memisahkan latar depan dari latar belakang.

\paragraph{Kriteria Sukses 1.4.1 \textit{Use of Color}}
\label{sec:kriteria_sukses_1.4.1}
(Level A)\\

Warna tidak digunakan sebagai satu-satunya cara visual untuk menyampaikan informasi, menandai tindakan yang wajib ditindaklanjuti, meminta respons, atau membedakan elemen visual.

\paragraph{Kriteria Sukses 1.4.2 \textit{Audio Control}}
\label{sec:kriteria_sukses_1.4.2}
(Level A)\\

Jika ada audio apa pun di halaman web yang diputar otomatis selama lebih dari 3 detik, maka harus tersedia mekanisme untuk memberi jeda atau memberhentikan audio tersebut, atau mengendalikan volume audio yang terpisah dari level volume sistem secara keseluruhan.

\paragraph{Kriteria Sukses 1.4.3 \textit{Contrast (Minimum)}}
\label{sec:kriteria_sukses_1.4.3}
(Level AA)\\

Lorem ipsum dolor sit amet

\paragraph{Kriteria Sukses 1.4.4 \textit{Resize text}}
\label{sec:kriteria_sukses_1.4.4}
(Level AA)\\

Kecuali untuk keterangan dan teks berupa gambar, teks dapat diubah ukurannya tanpa teknologi alat bantu sampai dengan 200 persen, tanpa mengorbankan fungsionalitas atau menghilangkan sebagian konten.

\paragraph{Kriteria Sukses 1.4.5 \textit{Images of Text}}
\label{sec:kriteria_sukses_1.4.5}
Lorem ipsum dolor sit amet

\paragraph{Kriteria Sukses 1.4.6 \textit{Contrast (Enhanced)}}
\label{sec:kriteria_sukses_1.4.6}
Lorem ipsum dolor sit amet

\paragraph{Kriteria Sukses 1.4.7 \textit{Low or No Background Audio}}
\label{sec:kriteria_sukses_1.4.7}
Lorem ipsum dolor sit amet

\paragraph{Kriteria Sukses 1.4.8 \textit{Visual Presentation}}
\label{sec:kriteria_sukses_1.4.8}
Lorem ipsum dolor sit amet

\paragraph{Kriteria Sukses 1.4.9 \textit{Images of Text (No Exception)}}
\label{sec:kriteria_sukses_1.4.9}
(Level AAA)\\

Teks berupa gambar hanya digunakan untuk dekorasi semata atau ketika wujud tertentu dari teks sangat penting dalam menyampaikan informasi.

\paragraph{Kriteria Sukses 1.4.10 \textit{Reflow}}
\label{sec:kriteria_sukses_1.4.10}
Lorem ipsum dolor sit amet

\paragraph{Kriteria Sukses 1.4.11 \textit{Non-text Contrast}}
\label{sec:kriteria_sukses_1.4.11}
Lorem ipsum dolor sit amet

\paragraph{Kriteria Sukses 1.4.12 \textit{Text Spacing}}
\label{sec:kriteria_sukses_1.4.12}
Lorem ipsum dolor sit amet

\paragraph{Kriteria Sukses 1.4.13 \textit{Content on Hover or Focus}}
\label{sec:kriteria_sukses_1.4.13}
Lorem ipsum dolor sit amet
%End of 2.1.1.4 Distinguishable%

%End of 2.1.1 Perceivable%

%2.1.2 Operable%
\subsection{\textit{Operable}}
\label{sec:operable}
Komponen antarmuka pengguna dan navigasi harus bisa dioperasikan.

%2.1.2.1 Keyboard Accessible%
\subsubsection{\textit{Keyboard Accessible}}
\label{sec:keyboard_accessible}
Pastikan semua fungsionalitas bisa diakses dengan \textit{keyboard}.

\paragraph{Kriteria Sukses 2.1.1 \textit{Keyboard}}
\label{sec:kriteria_sukses_2.1.1}
(Level A)\\

Semua fungsionalitas konten dapat dioperasikan melalui antarmuka \textit{keyboard} tanpa perlu mengatur jeda antar ketukan tombol, kecuali bila fungsi tersebut membutuhkan masukan yang bergantung pada jalur gerakan pengguna dan bukan hanya pada titik akhir.

\paragraph{Kriteria Sukses 2.1.2 \textit{No Keyboard Trap}}
\label{sec:kriteria_sukses_2.1.2}
(Level A)\\

Jika fokus \textit{keyboard} dapat dipindahkan ke komponen tertentu dengan menggunakan antarmuka \textit{keyboard}, maka fokus dapat dipindahkan dari komponen tersebut hanya dengan menggunakan antarmuka \textit{keyboard}. Jika dibutuhkan tindakan yang lebih dari sekadar menekan tombol panah atau \textit{tab} atau metode-metode keluar standar lainnya, maka pengguna akan diberi tahu tentang metode tersebut.

\paragraph{Kriteria Sukses 2.1.3 \textit{Keyboard (No Exception)}}
\label{sec:kriteria_sukses_2.1.3}
(Level AAA)\\

Semua fungsionalitas konten dapat dioperasikan melalui antarmuka \textit{keyboard} tanpa perlu mengatur jeda antar ketukan tombol.

\paragraph{Kriteria Sukses 2.1.4 \textit{Character Key Shortcuts}}
\label{sec:kriteria_sukses_2.1.4}
(Level A)\\

Jika pintasan \textit{keyboard} diterapkan dalam konten hanya menggunakan huruf (termasuk huruf besar dan kecil), tanda baca, angka, atau karakter simbol, maka setidaknya salah satu dari ketentuan berikut ini berlaku:
\begin{itemize}
	\item Dapat dinonaktifkan: Tersedia mekanisme untuk menonaktifkan pintasan;
	\item Dipetakan kembali: Tersedia mekanisme untuk memetakan kembali pintasan untuk menggunakan satu atau lebih karakter \textit{keyboard} yang tidak dapat dicetak (misalnya: \textit(CTRL, ALT, SHIFT));
	\item Hanya aktif saat mendapat fokus: Pintasan \textit{keyboard} untuk komponen antarmuka pengguna hanya aktif ketika komponen tersebut mendapat fokus.
\end{itemize}
%End of 2.1.2.1 Keyboard Accessible%

%2.1.2.2 Enough Time%
\subsubsection{\textit{Enough Time}}
\label{sec:enough_time}
Sediakan cukup waktu agar pengguna bisa membaca dan memanfaatkan konten.

\paragraph{Kriteria Sukses 2.2.1 \textit{Timing Adjustable}}
\label{sec:kriteria_sukses_2.2.1}
(Level A)\\

Untuk setiap batas waktu yang ditentukan oleh konten, setidaknya salah satu dari ketentuan berikut berlaku:
\begin{itemize}
	\item Dapat dinonaktifkan: Pengguna dapat menonaktifkan batas waktu sebelum mencapai batas tersebut; atau
	\item Dapat disesuaikan: Pengguna diizinkan untuk menyesuaikan batas waktu sebelum mencapai batas tersebut, dengan waktu tambahan setidaknya sepuluh kali dari setelan batas waktu; atau
	\item Dapat diperpanjang: Pengguna diberi peringatan ketika batas waktu hampir habis dan diberikan waktu setidaknya 20 detik untuk memperpanjang batas waktu tersebut dengan tindakan yang sederhana (misalnya menekan tombol spasi), dan pengguna diizinkan untuk menambah batas waktu tersebut setidaknya sepuluh kali lipat; atau
	\item Perkecualian waktu riil: Batas waktu merupakan bagian yang wajib dari kejadian waktu riil (misalnya pelelangan), dan mustahil untuk menyediakan alternatif untuk batas waktu tersebut; atau
	\item Perkecualian penting: Batas waktu bersifat esensial dan jika diperpanjang maka akan menyalahi inti dari kegiatan tersebut; atau
	\item Perkecualian 20 Jam: Batas waktu yang diberikan lebih dari 20 jam.
\end{itemize}

\paragraph{Kriteria Sukses 2.2.2 \textit{Pause, Stop, Hide}}
\label{sec:kriteria_sukses_2.2.2}
(Level A)\\

Untuk informasi yang bergerak, berkelip, bergulir, atau diperbarui otomatis, semua ketentuan berikut berlaku:
\begin{itemize}
	\item Bergerak, berkelip, bergulir: Untuk informasi apa pun yang bergerak, berkelip, atau bergulir yang (1) mulainya otomatis, (2) terjadi lebih dari lima detik, dan (3) disajikan paralel dengan konten lain, tersedia mekanisme bagi pengguna untuk memberi jeda, memberhentikan, atau menyembunyikan informasi tersebut; kecuali bila aktivitas bergerak, berkelip, atau bergulir tersebut merupakan bagian dari aktivitas yang esensial; dan
	\item Diperbarui otomatis: Untuk informasi apa pun yang diperbarui otomatis, yaitu yang (1) mulainya otomatis dan (2) disajikan paralel dengan konten lain, tersedia mekanisme bagi pengguna untuk memberi jeda, memberhentikan, atau menyembunyikan informasi tersebut; atau terdapat cara untuk mengendalikan frekuensi pembaruan tersebut, kecuali jika pembaruan otomatis tersebut merupakan bagian dari aktivitas yang esensial.
\end{itemize}

\paragraph{Kriteria Sukses 2.2.3 \textit{No Timing}}
\label{sec:kriteria_sukses_2.2.3}
(Level AAA)\\

Waktu bukanlah bagian esensial dari kejadian atau aktivitas yang disajikan oleh konten, kecuali untuk \textit(synchronized media) yang tidak interaktif dan kejadian waktu riil.

\paragraph{Kriteria Sukses 2.2.4 \textit{Interruptions}}
\label{sec:kriteria_sukses_2.2.4}
(Level AAA)\\

Interupsi dapat ditunda atau dihentikan oleh pengguna, kecuali bila interupsi melibatkan keadaan darurat.

\paragraph{Kriteria Sukses 2.2.5 \textit{Re-authenticating}}
\label{sec:kriteria_sukses_2.2.5}
(Level AAA)\\

Ketika sesi autentikasi berakhir, pengguna dapat melanjutkan aktivitas tanpa kehilangan data setelah melakukan autentikasi ulang.

\paragraph{Kriteria Sukses 2.2.6 \textit{Timeouts}}
\label{sec:kriteria_sukses_2.2.6}
(Level AAA)\\

Pengguna diberi peringatan mengenai durasi ketidakaktifan pengguna yang dapat menyebabkan data hilang, kecuali jika data tersebut disimpan lebih dari 20 jam ketika pengguna tidak melakukan tindakan apa pun.
%End of 2.1.2.2 Enough Time%

%2.1.2.3 Seizures and Physical Reactions%
\subsubsection{\textit{Seizures and Physical Reactions}}
\label{sec:seizures_and_physical_reactions}
Jangan merancang konten yang diketahui bisa menyebabkan kejang atau reaksi fisik.

\paragraph{Kriteria Sukses 2.3.1 \textit{Three Flashes or Below Threshold}}
\label{sec:kriteria_sukses_2.3.1}
(Level A)\\

Halaman web tidak mengandung apa pun yang berkelip lebih dari tiga kali dalam jangka waktu satu detik, atau kelipan di bawah ambang batas kelipan biasa dan kelipan merah.

\paragraph{Kriteria Sukses 2.3.2 \textit{Three Flashes}}
\label{sec:kriteria_sukses_2.3.2}
(Level AAA)\\

Halaman web tidak mengandung apa pun yang berkelip lebih dari tiga kali dalam jangka waktu satu detik

\paragraph{Kriteria Sukses 2.3.3 \textit{Animation from Interactions}}
\label{sec:kriteria_sukses_2.3.3}
(Level AAA)\\

Animasi gerak yang dipicu oleh interaksi dapat dinonaktifkan, kecuali jika animasi itu penting untuk fungsionalitas atau informasi yang disampaikan.
%End of 2.1.2.3 Seizures and Physical Reactions%

%2.1.2.4 Navigable%
\subsubsection{\textit{Navigable}}
\label{sec:navigable}
Sediakan cara yang mudah untuk membantu pengguna bernavigasi, menemukan konten, dan menentukan mereka ada di mana.

\paragraph{Kriteria Sukses 2.4.1 \textit{Bypass Blocks}}
\label{sec:kriteria_sukses_2.4.1}
(Level A)\\

Tersedia mekanisme untuk meloncati beberapa area konten yang diulang-ulang pada berbagai halaman web.

\paragraph{Kriteria Sukses 2.4.2 \textit{Page Titled}}
\label{sec:kriteria_sukses_2.4.2}
(Level A)\\

Halaman web memiliki judul yang menjelaskan topik atau tujuan.

\paragraph{Kriteria Sukses 2.4.3 \textit{Focus Order}}
\label{sec:kriteria_sukses_2.4.3}
(Level A)\\

Bila halaman web dapat dinavigasi berurutan dan urutan navigasi berdampak pada makna atau operasi, maka komponen yang memang dapat difokus akan dijadikan fokus sesuai urutan yang mempertahankan makna dan pengoperasian.

\paragraph{Kriteria Sukses 2.4.4 \textit{Link Purpose (In Context)}}
\label{sec:kriteria_sukses_2.4.4}
(Level A)\\

Tujuan tiap tautan dapat ditentukan semata-mata dari teks pada tautan atau dari kombinasi teks pada tautan dengan konteks tautan yang ditentukan melalui pemrograman, kecuali bila tujuan tautan akan bersifat ambigu bagi pengguna secara umum.

\paragraph{Kriteria Sukses 2.4.5 \textit{Multiple Ways}}
\label{sec:kriteria_sukses_2.4.5}
(Level AA)\\

Ada berbagai cara untuk menemukan halaman web dalam sekumpulan halaman web kecuali bila halaman web merupakan hasil dari, atau langkah ke-sekian dari suatu proses.

\paragraph{Kriteria Sukses 2.4.6 \textit{Headings and Labels}}
\label{sec:kriteria_sukses_2.4.6}
(Level AA)\\
Judul dan label menjabarkan topik atau tujuan.

\paragraph{Kriteria Sukses 2.4.7 \textit{Focus Visible}}
\label{sec:kriteria_sukses_2.4.7}
(Level AA)\\

Antarmuka pengguna mana pun yang dapat dioperasikan dengan \textit{board} mempunyai mode pengoperasian yang memungkinkan indikator fokus dari \textit{keyboard} tampak dengan jelas.

\paragraph{Kriteria Sukses 2.4.8 \textit{Location}}
\label{sec:kriteria_sukses_2.4.8}
(Level AAA)\\

Informasi mengenai lokasi persis pengguna pada sekumpulan halaman web selalu tersedia.

\paragraph{Kriteria Sukses 2.4.9 \textit{Link Purpose (Link Only)}}
\label{sec:kriteria_sukses_2.4.9}
(Level AAA)\\

Tersedia mekanisme untuk mengizinkan pengidentifikasian tujuan tiap tautan semata-mata dari teks pada tautan, kecuali bila tujuan tautan tersebut akan bersifat ambigu bagi pengguna secara umum.

\paragraph{Kriteria Sukses 2.4.10 \textit{Section Headings}}
\label{sec:kriteria_sukses_2.4.10}
(Level AAA)\\

Judul per bagian digunakan untuk mengatur konten.
%End of 2.1.2.4 Navigable%

%2.1.2.5 Input Modalities%
\subsubsection{\textit{Input Modalities}}
\label{sec:input_modalities}
Permudah pengguna untuk mengoperasikan fungsionalitas melalui berbagai masukan di luar \textit{keyboard}.

\paragraph{Kriteria Sukses 2.5.1 \textit{Pointer Gestures}}
\label{sec:kriteria_sukses_2.5.1}
(Level A)\\

Lorem ipsum dolor sit amet

\paragraph{Kriteria Sukses 2.5.2 \textit{Pointer Cancellation}}
\label{sec:kriteria_sukses_2.5.2}
Lorem ipsum dolor sit amet

\paragraph{Kriteria Sukses 2.5.3 \textit{Label in Name}}
\label{sec:kriteria_sukses_2.5.3}
Lorem ipsum dolor sit amet

\paragraph{Kriteria Sukses 2.5.4 \textit{Motion Actuation}}
\label{sec:kriteria_sukses_2.5.4}
Lorem ipsum dolor sit amet

\paragraph{Kriteria Sukses 2.5.5 \textit{Target Size}}
\label{sec:kriteria_sukses_2.5.5}
Lorem ipsum dolor sit amet

\paragraph{Kriteria Sukses 2.5.6 \textit{Concurrent Input Mechanisms}}
\label{sec:kriteria_sukses_2.5.6}
Lorem ipsum dolor sit amet
%End of 2.1.2.5 Input Modalities%

%End of 2.1.2 Operable%

%2.1.3 Understandable%
\subsection{\textit{Understandable}}
\label{sec:understandable}
Informasi dan pengoperasian antarmuka pengguna harus dapat dimengerti.

%2.1.3.1 Readable%
\subsubsection{\textit{Readable}}
\label{sec:readable}
Buat konten teks mudah dibaca dan dimengerti.

\paragraph{Kriteria Sukses 3.1.1 \textit{Language of Page}}
\label{sec:kriteria_sukses_3.1.1}
(Level A)\\

Bahasa manusia \textit{default} untuk setiap halaman web dapat ditentukan melalui pemrograman.

\paragraph{Kriteria Sukses 3.1.2 \textit{Language of Parts}}
\label{sec:kriteria_sukses_3.1.2}
(Level AA)\\

Bahasa manusia pada setiap bagian atau frasa yang terdapat dalam konten dapat ditentukan secara terprogram kecuali untuk nama diri, istilah teknis, kata dari bahasa yang tidak tentu, dan kata atau frasa yang telah menjadi bagian dari bahasa daerah dari teks yang ada di sekelilingnya.

\paragraph{Kriteria Sukses 3.1.3 \textit{Unusual Words}}
\label{sec:kriteria_sukses_3.1.3}
(Level AAA)\\

Tersedia mekanisme untuk mengidentifikasi definisi spesifik dari kata atau frasa yang digunakan dengan cara yang tidak lazim atau terbatas, termasuk idiom dan jargon.

\paragraph{Kriteria Sukses 3.1.4 \textit{Abbreviations}}
\label{sec:kriteria_sukses_3.1.4}
(Level AAA)\\

Tersedia mekanisme untuk mengidentifikasi kepanjangan dari singkatan.

\paragraph{Kriteria Sukses 3.1.5 \textit{Reading Level}}
\label{sec:kriteria_sukses_3.1.5}
(Level AAA)\\

Ketika teks yang tersaji cukup kompleks dan membutuhkan kemampuan membaca yang lebih tinggi dari rata-rata, versi konten yang lebih mudah dimengerti haruslah tersedia bagi pengguna.

\paragraph{Kriteria Sukses 3.1.6 \textit{Pronunciation}}
\label{sec:kriteria_sukses_3.1.6}
(Level AAA)\\

Tersedia mekanisme untuk mengidentifikasi pengucapan suatu kata apabila makna kata tersebut bersifat ambigu ketika cara mengucapkannya tidak diketahui.
%End of 2.1.3.1 Readable%

%2.1.3.2 Predictable%
\subsubsection{\textit{Predictable}}
\label{sec:predictable}
Pastikan halaman situs web tampak dan dapat dioperasikan dengan cara-cara yang mudah ditebak.

\paragraph{Kriteria Sukses 3.2.1 \textit{On Focus}}
\label{sec:kriteria_sukses_3.2.1}
(Level A)\\

Saat komponen antarmuka pengguna menerima fokus, komponen tersebut tidak menyebabkan perubahan konteks.

\paragraph{Kriteria Sukses 3.2.2 \textit{On Input}}
\label{sec:kriteria_sukses_3.2.2}
(Level A)\\

Mengubah setelan komponen antarmuka pengguna tidak otomatis menyebabkan perubahan konteks kecuali bila pengguna telah diperingati akan perilaku semacam ini sebelum menggunakan komponen tersebut.

\paragraph{Kriteria Sukses 3.2.3 \textit{Consistent Navigation}}
\label{sec:kriteria_sukses_3.2.3}
(Level AA)\\

Mekanisme navigasi yang muncul berulang pada tiap halaman web dalam sekumpulan halaman web, muncul dalam urutan relatif yang sama setiap kali tampak, kecuali jika ada perubahan yang dilakukan pengguna.

\paragraph{Kriteria Sukses 3.2.4 \textit{Consistent Identification}}
\label{sec:kriteria_sukses_3.2.4}
(Level AA)\\

Komponen-komponen yang memiliki fungsionalitas yang sama dalam sekumpulan halaman web diidentifikasikan secara konsisten.

\paragraph{Kriteria Sukses 3.2.5 \textit{Change on Request}}
\label{sec:kriteria_sukses_3.2.5}
(Level AAA)\\

Perubahan konteks hanya terjadi bila dilakukan oleh pengguna atau ada mekanisme yang tersedia untuk menonaktifkan perubahan tersebut.
%End of 2.1.3.2 Predictable%

%2.1.3.3 Input Assistance%
\subsubsection{\textit{Input Assistance}}
\label{sec:input_assistance}
Bantu pengguna menghindari kesalahan dan mengoreksi kesalahan tersebut.

\paragraph{Kriteria Sukses 3.3.1 \textit{Error Identification}}
\label{sec:kriteria_sukses_3.3.1}
(Level A)\\

Jika eror masukan terdeteksi otomatis, \textit{item} yang eror harus diidentifikasi dan eror harus dijabarkan kepada pengguna dalam bentuk teks.

\paragraph{Kriteria Sukses 3.3.2 \textit{Labels or Instructions}}
\label{sec:kriteria_sukses_3.3.2}
(Level A)\\

Label atau instruksi tersedia ketika konten membutuhkan masukan dari pengguna.

\paragraph{Kriteria Sukses 3.3.3 \textit{Error Suggestion}}
\label{sec:kriteria_sukses_3.3.3}
(Level AA)\\

Jika eror masukan terdeteksi otomatis dan saran untuk mengoreksi eror tersebut diketahui, maka saran disajikan kepada pengguna, kecuali bila saran tersebut akan mengacaukan keamanan atau tujuan dari konten.

\paragraph{Kriteria Sukses 3.3.4 \textit{Error Prevention (Legal, Financial, DatA)\\}}
\label{sec:kriteria_sukses_3.3.4}
(Level AA)\\

Untuk halaman web yang mengirim tanggapan pengguna atau yang menyebabkan terjadinya komitmen hukum atau transaksi keuangan bagi pengguna, setidaknya salah satu dari ketentuan berikut berlaku:
\begin{itemize}
	\item Bisa dibatalkan: Data yang akan dikirim bisa dibatalkan.
	\item Diperiksa: Data yang dimasukkan oleh pengguna diperiksa apa ada eror masukan atau tidak dan pengguna dipersilakan untuk mengoreksinya.
	\item Dikonfirmasi: Tersedia mekanisme untuk meninjau, mengonfirmasi, dan mengoreksi informasi sebelum informasi tersebut dikirim.
\end{itemize}

\paragraph{Kriteria Sukses 3.3.5 \textit{Help}}
\label{sec:kriteria_sukses_3.3.5}
(Level AAA)\\

Tersedia bantuan terkait konteks yang sedang berjalan.

\paragraph{Kriteria Sukses 3.3.6 \textit{Error Prevention (All)}}
\label{sec:kriteria_sukses_3.3.6}
(Level AAA)\\

Untuk halaman web yang mewajibkan pengguna mengirim informasi, setidaknya salah satu dari ketentuan berikut berlaku:
\begin{itemize}
	\item Bisa dibatalkan: Data yang akan dikirim bisa dibatalkan.
	\item Diperiksa: Data yang dimasukkan oleh pengguna diperiksa apa ada eror masukan atau tidak dan pengguna dipersilakan untuk mengoreksinya.
	\item Dikonfirmasi: Tersedia mekanisme untuk meninjau, mengonfirmasi, dan mengoreksi informasi sebelum informasi tersebut dikirim.
\end{itemize}
%End of 2.1.3.3 Input Assistance%

%End of 2.1.3 Understandable%

%2.1.4 Robust%
\subsection{\textit{Robust}}
\label{sec:robust}
Konten harus cukup andal sehingga dapat ditafsirkan oleh berbagai agen pengguna, termasuk teknologi alat bantu.

%2.1.4.1 Compatible%
\subsubsection{\textit{Compatible}}
\label{sec:compatible}
Maksimalkan kompatibilitas dengan agen pengguna saat ini maupun saat yang akan datang, termasuk teknologi alat bantu.

\paragraph{Kriteria Sukses 4.1.1 \textit{Parsing}}
\label{sec:kriteria_sukses_4.1.1}
(Level A)\\

Pada konten yang diimplementasikan dengan menggunakan bahasa markah, setiap elemen memiliki \textit{tag} awal dan akhir yang lengkap, setiap elemen disusun berlapis sesuai spesifikasi masing-masing, setiap elemen tidak mengandung atribut yang sama dua kali, dan setiap \textit{ID} bersifat unik kecuali jika ada spesifikasi yang mengizinkan.

\paragraph{Kriteria Sukses 4.1.2 \textit{Name, Role, Value}}
\label{sec:kriteria_sukses_4.1.2}
(Level A)\\

Untuk semua komponen antarmuka pengguna, nama dan peran dapat ditentukan melalui pemrograman. Keadaan, properti, dan nilai yang dapat ditentukan oleh pengguna juga harus dapat ditentukan melalui pemrograman. Notifikasi perubahan terhadap \textit{item-item} ini harus tersedia untuk agen pengguna termasuk juga teknologi alat bantu.

\paragraph{Kriteria Sukses 4.1.3 \textit{Status Messages}}
\label{sec:kriteria_sukses_4.1.3}
(Level AA)\\

Pada konten yang diimplementasikan dengan menggunakan bahasa markah, pesan status dapat ditentukan secara terprogram melalui peran atau sifat sedemikian rupa sehingga dapat disajikan kepada pengguna oleh teknologi bantuan tanpa perlu menerima fokus.
%End of 2.1.4.1 Compatible%

%End of 2.1.4 Robust%

%End of WCAG 2.1%

\section{BlueTape}
\label{sec:bluetape}
BlueTape adalah aplikasi dan \textit{framework} untuk membuat urusan-urusan berbasis kertas di Fakultas Teknologi Informasi dan Sains Universitas Katolik Parahyangan menjadi tanpa kertas. Aplikasi ini berbasis web dengan memanfaatkan CodeIgniter dan ZURB Foundation.

\subsection{Fitur-fitur}
Fitur-fitur yang tersedia pada aplikasi ini yaitu:
\begin{itemize}
	\item \textit{Framework} disediakan untuk menambah layanan baru. Menu sudah disediakan sehingga pengembang hanya tinggal menambahkan dalam bentuk modul.
	\item Layanan \textit{OAuth} ke Google, memungkinkan autentikasi pengguna dan menentukan hak akses yang bisa dilihat dari alamat surel pengguna.
\end{itemize}

\subsection{Layanan}
Layanan yang tersedia saat ini yaitu:
\begin{itemize}
	\item Permintaan dan kelola transkrip untuk melakukan permohonan serta pencetakan transkrip mahasiswa.
	\item Permintaan dan kelola perubahan kuliah untuk permohonan dan pencetakan perubahan jadwal kuliah oleh dosen.
\end{itemize}

\section{Template Skripsi FTIS UNPAR}
\label{sec:template}
 
Akan dipaparkan bagaimana menggunakan template ini, termasuk petunjuk singkat membuat referensi, gambar dan tabel.
Juga hal-hal lain yang belum terpikir sampai saat ini. 
 
\dtext{15-16}

\subsection{Tabel}  
Berikut adalah contoh pembuatan tabel. 
Penempatan tabel dan gambar secara umum diatur otomatis oleh \LaTeX{}, perhatikan contoh di file bab2.tex untuk melihat bagaimana cara memaksa tabel ditempatkan sesuai keinginan kita.

Perhatikan bawa berbeda dengan penempatan judul gambar gambar, keterangan tabel harus diletakkan di atas tabel!!
Lihat Tabel~\ref{tab:contoh1} berikut ini:

\begin{table}[H] %atau h saja untuk "kira kira di sini"
	\centering 
	\caption{Tabel contoh}
	\label{tab:contoh1}
	\begin{tabular}{cccc}
		\toprule
		& $v_{start}$ & $\mathcal{S}_{1}$ & $v_{end}$\\

		\midrule
		$\tau_{1}$ & 1 & 12& 20\\
		$\tau_{2}$ & 1 &  & 20\\
		$\tau_{3}$ & 1 & 9 & 20\\
		$\tau_{4}$ & 1 &  & 20\\

		\bottomrule
		
	\end{tabular} 
\end{table}
Tabel~\ref{tab:cthwarna1} dan Tabel~\ref{tab:cthwarna2} berikut ini adalah tabel dengan sel yang berwarna dan ada dua tabel yang bersebelahan. 
\begin{table}[H]
	\begin{minipage}[c]{0.49\linewidth}
		\centering
		\caption{Tabel bewarna(1)}
		\label{tab:cthwarna1}
		\begin{tabular}{ccccc}
			\toprule
			 & $v_{start}$ & $\mathcal{S}_{2}$ & $\mathcal{S}_{1}$ & $v_{end}$\\
			
			\midrule
			$\tau_{1}$ & 1 & 5 \cellcolor{green}& 12& 20\\
			$\tau_{2}$ & 1 & 8 \cellcolor{green}& & 20\\
			$\tau_{3}$ & 1 & 2/8/17 \cellcolor{green}& 9 & 20\\
			$\tau_{4}$ & 1 & \cellcolor{red}& & 20\\
			
			\bottomrule

		\end{tabular}
	\end{minipage}
	\begin{minipage}[c]{0.49\linewidth}
		
		\centering 
		\caption{Tabel bewarna(2)}
		\label{tab:cthwarna2}
		\begin{tabular}{ccccc}
			\toprule
			 & $v_{start}$ & $\mathcal{S}_{1}$ & $\mathcal{S}_{2}$ & $v_{end}$\\
			
			\midrule
			$\tau_{1}$ & 1 & 12& 5 \cellcolor{red} &20\\
			$\tau_{2}$ & 1 &  &  8 \cellcolor{green} &20\\
			$\tau_{3}$ & 1 & 9 & 2/8/17 \cellcolor{green} &20\\
			$\tau_{4}$ & 1 &   & \cellcolor{red} &20\\
			
			\bottomrule
		
		\end{tabular}
	\end{minipage}
\end{table}

 
\subsection{Kutipan}
\label{subs:kutipan} 
Berikut contoh kutipan dari berbagai sumber, untuk keterangan lebih lengkap, silahkan membaca file referensi.bib yang disediakan juga di template ini.
Contoh kutipan:
\begin{itemize}
	\item Buku:~\cite{berg:08:compgeom} 
	\item Bab dalam buku:~\cite{kreveld:04:GIS}
	\item Artikel dari Jurnal:~\cite{buchin:13:median}
	\item Artikel dari prosiding seminar/konferensi:~\cite{kreveld:11:median}
	\item Skripsi/Thesis/Disertasi:~\cite{lionov:02:animasi}~\cite{wiratma:10:following}~\cite{wiratma:22:later}
	\item Technical/Scientific Report:~\cite{kreveld:07:watertight}
	\item RFC (Request For Comments):~\cite{RFC1654}
	\item Technical Documentation/Technical Manual:~\cite{Z.500}~\cite{unicode:16:stdv9}~\cite{google:16:and7}
	\item Paten:~\cite{webb:12:comm}
	\item Tidak dipublikasikan:~\cite{wiratma:09:median}~\cite{lionov:11:cpoly}
	\item Laman web:~\cite{erickson:03:cgmodel}  
	\item Lain-lain:~\cite{agung:12:tango}
\end{itemize}    
  
\subsection{Gambar}

Pada hampir semua editor, penempatan gambar di dalam dokumen \LaTeX{} tidak dapat dilakukan melalui proses {\it drag and drop}.
Perhatikan contoh pada file bab2.tex untuk melihat bagaimana cara menempatkan gambar.
Beberapa hal yang harus diperhatikan pada saat menempatkan gambar:
\begin{itemize}
	\item Setiap gambar {\bf harus} diacu di dalam teks (gunakan {\it field} {\sc label})
	\item {\it Field} {\sc caption} digunakan untuk teks pengantar pada gambar. Terdapat dua bagian yaitu yang ada di antara tanda $[$ dan $]$ dan yang ada di antara tanda $\{$ dan $\}$. Yang pertama akan muncul di Daftar Gambar, sedangkan yang kedua akan muncul di teks pengantar gambar. Untuk skripsi ini, samakan isi keduanya.
	\item Jenis file yang dapat digunakan sebagai gambar cukup banyak, tetapi yang paling populer adalah tipe {\sc png} (lihat Gambar~\ref{fig:ularpng}), tipe {\sc jpg} (Gambar~\ref{fig:ularjpg}) dan tipe {\sc pdf} (Gambar~\ref{fig:ularpdf})
	\item Besarnya gambar dapat diatur dengan {\it field} {\sc scale}.
	\item Penempatan gambar diatur menggunakan {\it placement specifier} (di antara tanda  $[$ dan $]$ setelah deklarasi gambar.
	Yang umum digunakan adalah {\bf H} untuk menempatkan gambar {\bf sesuai} penempatannya di file .tex atau  {\bf h} yang berarti "kira-kira" di sini. \\
	Jika tidak menggunakan {\it placement specifier}, \LaTeX{} akan menempatkan gambar otomatis untuk menghindari bagian kosong pada dokumen anda.
	Walaupun cara ini sangat mudah, hindarkan terjadinya penempatan dua gambar secara berurutan. 	
	\begin{itemize}
		\item Gambar~\ref{fig:ularpng} ditempatkan di bagian atas halaman, walaupun penempatannya dilakukan setelah penulisan 3 paragraf setelah penjelasan ini.
		\item Gambar~\ref{fig:ularjpg} dengan skala 0.5 ditempatkan di antara dua buah paragraf. Perhatikan penulisannya di dalam file bab2.tex!
		\item Gambar~\ref{fig:ularpdf} ditempatkan menggunakan {\it specifier} {\bf h}.
	\end{itemize}
\end{itemize}
 
\dtext{17-18}
\begin{figure} 
	\centering  
	\includegraphics[scale=1]{ular-png}  
	\caption[Gambar {\it Serpentes} dalam format png]{Gambar {\it Serpentes} dalam format png} 
	\label{fig:ularpng} 
\end{figure} 

\dtext{19-20}
\begin{figure}[H]
	\centering  
	\includegraphics[scale=0.5]{ular-jpg}  
	\caption[Ular kecil]{Ular kecil} 
	\label{fig:ularjpg} 
\end{figure} 
\dtext{21-22}

\begin{figure}[ht] 
	\centering  
	\includegraphics[scale=1]{ular-pdf}  
	\caption[ {\it Serpentes} betina]{ {\it Serpentes} jantan} 
	\label{fig:ularpdf} 
\end{figure} 
 
