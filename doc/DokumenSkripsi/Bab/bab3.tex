\chapter{Analisis}
\label{chap:analisis}

\section{Tingkat Kepatuhan BlueTape Terhadap \textit{WCAG} 2.1}
Sukses atau tidaknya aplikasi BlueTape dalam mematuhi kriteria-kriteria sukses dalam \textit{WCAG} 2.1 dituliskan dalam tabel-tabel berikut.
\label{sec:kepatuhan_bluetape_terhadap_wcag_2.1}
\begin{table}[H]
    \centering 
    \caption{Tabel kepatuhan BlueTape terhadap prinsip \textit{Perceivable}}
    \label{tab:kepatuhan_bluetape_perceivable}
    \begin{tabular}{cccc}
        \toprule
        & Kriteria Sukses & Tingkat Kepatuhan & Hasil (sukses/tidak)\\

        \midrule
        & 1.1.1 & A & Tidak Sukses \\
        & 1.2.1 & A & Sukses \\
        & 1.2.2 & A & Sukses \\
        & 1.2.3 & A & Sukses \\
        & 1.2.4 & AA & Sukses \\
        & 1.2.5 & AA & Sukses \\
        & 1.2.6 & AAA & Sukses \\
        & 1.2.7 & AAA & Sukses \\
        & 1.2.8 & AAA & Sukses \\
        & 1.2.9 & AAA & Sukses \\
        & 1.3.1 & A & Tidak Sukses \\
        & 1.3.2 & A & Tidak Sukses \\
        & 1.3.3 & A & Tidak Sukses \\
        & 1.3.4 & AA & Sukses \\
        & 1.3.5 & AA & Sukses \\
        & 1.3.6 & AAA & Tidak Sukses \\
        & 1.4.1 & A & Sukses \\
        & 1.4.2 & A & Sukses \\
        & 1.4.3 & AA & Tidak Sukses \\
        & 1.4.4 & AA & Sukses \\
        & 1.4.5 & AA & Sukses \\
        & 1.4.6 & AAA & Tidak Sukses \\
        & 1.4.7 & AAA & Sukses \\
        & 1.4.8 & AAA & Tidak Sukses \\
        & 1.4.9 & AAA & Sukses \\
        & 1.4.10 & AA & Tidak Sukses \\
        & 1.4.11 & AA & \\
        & 1.4.12 & AA & \\
        & 1.4.13 & AA & Sukses \\
        
        \bottomrule

    \end{tabular}
\end{table}
\begin{table}[H]
    \centering 
    \caption{Tabel kepatuhan BlueTape terhadap prinsip \textit{Operable}}
    \label{tab:kepatuhan_bluetape_operable}
    \begin{tabular}{cccc}
        \toprule
        & Kriteria Sukses & Tingkat Kepatuhan & Hasil (sukses/tidak)\\

        \midrule
        & 2.1.1 & A & Tidak Sukses \\
        & 2.1.2 & A & Sukses \\
        & 2.1.3 & AAA & Tidak Sukses \\
        & 2.1.4 & A & Sukses \\
        & 2.2.1 & A & Sukses \\
        & 2.2.2 & A & Sukses \\
        & 2.2.3 & AAA & Sukses \\
        & 2.2.4 & AAA & Sukses \\
        & 2.2.5 & AAA & Tidak Sukses \\
        & 2.2.6 & AAA & Tidak Sukses \\
        & 2.3.1 & A & Sukses \\
        & 2.3.2 & AAA & Sukses \\
        & 2.3.3 & AAA & Sukses \\
        & 2.4.1 & A & Tidak Sukses \\
        & 2.4.2 & A & Sukses \\
        & 2.4.3 & A & Tidak Sukses \\
        & 2.4.4 & A & Tidak Sukses \\
        & 2.4.5 & AA & Tidak Sukses \\
        & 2.4.6 & AA & \\
        & 2.4.7 & AA & \\
        & 2.4.8 & AAA & \\
        & 2.4.9 & AAA & \\
        & 2.4.10 & AAA & \\
        & 2.5.1 & A & \\
        & 2.5.2 & A & \\
        & 2.5.3 & A & \\
        & 2.5.4 & A & \\
        & 2.5.5 & AAA & \\
        & 2.5.6 & AAA & \\

        \bottomrule
    
    \end{tabular}
\end{table}

\begin{table}[H]
    \centering 
    \caption{Tabel kepatuhan BlueTape terhadap prinsip \textit{Understandable}}
    \label{tab:kepatuhan_bluetape_understandable}
    \begin{tabular}{cccc}
        \toprule
        & Kriteria Sukses & Tingkat Kepatuhan & Hasil (sukses/tidak)\\

        \midrule
        & 3.1.1 & A & \\
        & 3.1.2 & AA & \\
        & 3.1.3 & AAA & \\
        & 3.1.4 & AAA & \\
        & 3.1.5 & AAA & \\
        & 3.1.6 & AAA & \\
        & 3.2.1 & A & \\
        & 3.2.2 & A & \\
        & 3.2.3 & AA & \\
        & 3.2.4 & AA & \\
        & 3.2.5 & AAA & \\
        & 3.3.1 & A & \\
        & 3.3.2 & A & \\
        & 3.3.3 & AA & \\
        & 3.3.4 & AA & \\
        & 3.3.5 & AAA & \\
        & 3.3.6 & AAA & \\

        \bottomrule

    \end{tabular}
\end{table}
\begin{table}[H]
    \centering 
    \caption{Tabel kepatuhan BlueTape terhadap prinsip \textit{Robust}}
    \label{tab:kepatuhan_bluetape_robust}
    \begin{tabular}{cccc}
        \toprule
        & Kriteria Sukses & Tingkat Kepatuhan & Hasil (sukses/tidak)\\

        \midrule
        & 4.1.1 & A & \\
        & 4.1.2 & A & \\
        & 4.1.3 & AA & \\

        \bottomrule
    
    \end{tabular} 
\end{table}

\subsection{\textit{Perceivable}}
\label{subsec:kepatuhan_bluetape_perceivable}

\subsubsection{\textit{Text Alternatives}}
\label{subsubsec:kepatuhan_bluetape_text_alternatives}

\paragraph{Kriteria Sukses 1.1.1 \textit{Non-text Content}}
\label{par:kepatuhan_bluetape_kriteria_sukses_1.1.1}
(Tidak Sukses)\\

Kriteria ini tidak sukses dipatuhi karena pada tampilan \textit{mobile}, di setiap halaman selain halaman \textit{login} terdapat tombol yang tidak memiliki alternatif teks yang dapat ditafsirkan oleh teknologi alat bantu. Tombol yang dimaksud hanya menampilkan tiga garis horizontal berwarna putih dan berfungsi untuk menampilkan bagian menu ketika ditekan.

\subsubsection{\textit{Time-based Media}}
\label{subsubsec:kepatuhan_bluetape_time_based_media}

\paragraph{Kriteria Sukses 1.2.1 \textit{Audio-only and Video-only (Prerecorded)}}
\label{par:kepatuhan_bluetape_kriteria_sukses_1.2.1}
(Sukses)\\

Kriteria ini sukses dipatuhi karena pada halaman web BlueTape tidak terdapat konten media berbasis waktu.

\paragraph{Kriteria Sukses 1.2.2 \textit{Captions (Prerecorded)}}
\label{par:kepatuhan_bluetape_kriteria_sukses_1.2.2}
(Sukses)\\

Kriteria ini sukses dipatuhi karena pada halaman web BlueTape tidak terdapat konten media berbasis waktu.

\paragraph{Kriteria Sukses 1.2.3 \textit{Audio Description or Media Alternative (Prerecorded)}}
\label{par:kepatuhan_bluetape_kriteria_sukses_1.2.3}
(Sukses)\\

Kriteria ini sukses dipatuhi karena pada halaman web BlueTape tidak terdapat konten media berbasis waktu.

\paragraph{Kriteria Sukses 1.2.4 \textit{Captions (Live)}}
\label{par:kepatuhan_bluetape_kriteria_sukses_1.2.4}
(Sukses)\\

Kriteria ini sukses dipatuhi karena pada halaman web BlueTape tidak terdapat konten media berbasis waktu.

\paragraph{Kriteria Sukses 1.2.5 \textit{Audio Description (Prerecorded)}}
\label{par:kepatuhan_bluetape_kriteria_sukses_1.2.5}
(Sukses)\\

Kriteria ini sukses dipatuhi karena pada halaman web BlueTape tidak terdapat konten media berbasis waktu.

\paragraph{Kriteria Sukses 1.2.6 \textit{Sign Language (Prerecorded)}}
\label{par:kepatuhan_bluetape_kriteria_sukses_1.2.6}
(Sukses)\\

Kriteria ini sukses dipatuhi karena pada halaman web BlueTape tidak terdapat konten media berbasis waktu.

\paragraph{Kriteria Sukses 1.2.7 \textit{Extended Audio Description (Prerecorded)}}
\label{par:kepatuhan_bluetape_kriteria_sukses_1.2.7}
(Sukses)\\

Kriteria ini sukses dipatuhi karena pada halaman web BlueTape tidak terdapat konten media berbasis waktu.

\paragraph{Kriteria Sukses 1.2.8 \textit{Media Alternative (Prerecorded)}}
\label{par:kepatuhan_bluetape_kriteria_sukses_1.2.8}
(Sukses)\\

Kriteria ini sukses dipatuhi karena pada halaman web BlueTape tidak terdapat konten media berbasis waktu.

\paragraph{Kriteria Sukses 1.2.9 \textit{Audio-only (Live)}}
\label{par:kepatuhan_bluetape_kriteria_sukses_1.2.9}
(Sukses)\\

Kriteria ini sukses dipatuhi karena pada halaman web BlueTape tidak terdapat konten media berbasis waktu.

\subsubsection{\textit{Adaptable}}
\label{subsubsec:kepatuhan_bluetape_adaptable}

\paragraph{Kriteria Sukses 1.3.1 \textit{Info and Relationships}}
\label{par:kepatuhan_bluetape_kriteria_sukses_1.3.1}
(Tidak Sukses)\\

Kriteria ini tidak sukses dipatuhi karena:
\begin{itemize}
    \item Terdapat penggunaan \textit{tag heading} yang tidak tepat secara struktur pada halaman cetak transkrip, manajemen cetak transkrip, perubahan kuliah, manajemen perubahan kuliah, dan entri jadwal dosen.
    \item Pada halaman manajemen cetak transkrip, kolom masukan NPM tidak memiliki label atau \textit{aria-label} yang bersangkutan dengan kolom masukan tersebut.
    \item Pada halaman entri jadwal dosen, setiap kolom masukkan tidak memiliki label atau \textit{aria-label} yang bersangkutan dengan kolom-kolom masukan tersebut.
    \item Pada halaman lihat jadwal dosen, terdapat elemen yang tidak tepat secara struktur pada bagian \textit{list}.
\end{itemize} 

\paragraph{Kriteria Sukses 1.3.2 \textit{Meaningful Sequence}}
\label{par:kepatuhan_bluetape_kriteria_sukses_1.3.2}
(Tidak Sukses)\\

Kriteria ini tidak sukses dipatuhi karena pada halaman perubahan kuliah di bagian "Permohonan Baru", urutan baca dan urutan navigasi kurang tepat. 

\paragraph{Kriteria Sukses 1.3.3 \textit{Sensory Characteristics}}
\label{par:kepatuhan_bluetape_kriteria_sukses_1.3.3}
(Tidak Sukses)\\

Kriteria ini tidak sukses dipatuhi karena pada halaman cetak transkrip, manajemen cetak transkrip, perubahan kuliah, dan manajemen perubahan kuliah terdapat satu atau lebih komponen untuk mengoperasikan konten yang hanya mengandalkan satu komponen karakteristik indra yaitu bentuk. Komponen-komponen ini terletak pada kolom "Aksi" dan hanya ditampilkan dalam rupa ikon tanpa memiliki keterangan lebih detail.

\paragraph{Kriteria Sukses 1.3.4 \textit{Orientation}}
\label{par:kepatuhan_bluetape_kriteria_sukses_1.3.4}
(Sukses)\\

Kriteria ini sukses dipatuhi karena konten pada halaman web BlueTape dapat disajikan dalam orientasi \textit{portrait} maupun \textit{landscape}.

\paragraph{Kriteria Sukses 1.3.5 \textit{Identify Input Purpose}}
\label{par:kepatuhan_bluetape_kriteria_sukses_1.3.5}
(Sukses)\\

Kriteria ini sukses dipatuhi karena setiap kolom masukan yang mengumpulkan informasi tentang pengguna sudah terisi otomatis.

\paragraph{Kriteria Sukses 1.3.6 \textit{Identify Purpose}}
\label{par:kepatuhan_bluetape_kriteria_sukses_1.3.6}
(Tidak Sukses)\\

Kriteria ini tidak sukses dipatuhi karena terdapat elemen \textit{HTML}5 yang seharusnya dapat digunakan namun tidak digunakan, contohnya pada bagian navigasi.

\subsubsection{\textit{Distinguishable}}
\label{subsubsec:kepatuhan_bluetape_distinguishable}

\paragraph{Kriteria Sukses 1.4.1 \textit{Use of Color}}
\label{par:kepatuhan_bluetape_kriteria_sukses_1.4.1}
(Sukses)\\

Kriteria ini sukses dipatuhi karena pada halaman web BlueTape, warna tidak digunakan sebagai satu-satunya cara untuk menyampaikan informasi secara visual, menandai suatu tindakan, meminta respons, atau membedakan elemen visual.

\paragraph{Kriteria Sukses 1.4.2 \textit{Audio Control}}
\label{par:kepatuhan_bluetape_kriteria_sukses_1.4.2}
(Sukses)\\

Kriteria ini sukses dipatuhi karena pada halaman web BlueTape tidak terdapat konten media berbasis waktu.

\paragraph{Kriteria Sukses 1.4.3 \textit{Contrast (Minimum)}}
\label{par:kepatuhan_bluetape_kriteria_sukses_1.4.3}
(Tidak Sukses)\\

Kriteria ini tidak sukses dipatuhi karena pada halaman web BlueTape terdapat beberapa teks dengan rasio kontras kurang dari 4,5:1 untuk teks yang berukuran kurang dari 24 piksel dan tidak \textit{bold}, antara lain:

\begin{itemize}
    \item Halaman \textit{login}: Teks "\textit{Login} dengan Google" memiliki rasio kontras 3.09:1 terhadap warna latar belakangnya. Teks "Petunjuk Penggunaan" memiliki rasio kontras 3.06:1 terhadap warna latar belakangnya.
    \item Halaman cetak transkrip: Teks "Kirim Permohonan" memiliki rasio kontras 3.09:1 terhadap warna latar belakangnya. Teks "Tercetak" di kolom status pada tabel memiliki rasio kontras 1,79:1 terhadap warna latar belakangnya. Teks "Ditolak" di kolom status pada tabel "Histori Permohonan" memiliki rasio kontras 3,44:1 terhadap warna latar belakangnya. Teks "Tunggu" di kolom status pada tabel "Histori Permohonan" memiliki rasio kontras 4,44:1 terhadap warna latar belakangnya.
    \item Halaman manajemen cetak transkrip: Teks "Cari" memiliki rasio kontras 3.09:1 terhadap warna latar belakangnya. Teks "Tercetak" di kolom status pada tabel memiliki rasio kontras 1,79:1 terhadap warna latar belakangnya. Teks "Menunggu" di kolom status pada tabel memiliki rasio kontras 1,84:1 terhadap warna latar belakangnya. Teks "Ditolak" di kolom status pada tabel memiliki rasio kontras 3,44:1 terhadap warna latar belakangnya. Teks "Hapus" pada bagian hapus permohonan memiliki rasio kontras 3,47:1 terhadap warna latar belakangnya.
    \item Halaman perubahan kuliah: Teks "Kirim Permohonan" memiliki rasio kontras 3.09:1 terhadap warna latar belakangnya. Teks "Tambah Pertemuan Ekstra" memiliki rasio kontras 4,47:1 terhadap warna latar belakangnya. Teks "Hapus" memiliki rasio kontras 4,47:1 terhadap warna latar belakangnya. Teks "Tunggu" di kolom status pada tabel "Histori Permohonan" memiliki rasio kontras 4,44:1 terhadap warna latar belakangnya. Teks "Ditolak" di kolom status pada tabel "Histori Permohonan" memiliki rasio kontras 3,44:1 terhadap warna latar belakangnya. Teks "Terkonfirmasi" di kolom status pada tabel "Histori Permohonan" memiliki rasio kontras 1,79:1 terhadap warna latar belakangnya.
    \item Halaman manajemen perubahan kuliah: Teks "Terkonfirmasi" di kolom status pada tabel memiliki rasio kontras 1,79:1 terhadap warna latar belakangnya. Teks "Menunggu" di kolom status pada tabel memiliki rasio kontras 1,84:1 terhadap warna latar belakangnya. Teks "Ditolak" di kolom status pada tabel memiliki rasio kontras 3,44:1 terhadap warna latar belakangnya. Teks "Hapus" pada bagian hapus permohonan memiliki rasio kontras 3,47:1 terhadap warna latar belakangnya.
    \item Halaman entri jadwal dosen: Teks "Tambah" memiliki rasio kontras 3.09:1 terhadap warna latar belakangnya. Teks "Delete All" memiliki rasio kontras 3.47:1 terhadap warna latar belakangnya. Teks "Ekspor ke XLS" memiliki rasio kontras 3.09:1 terhadap warna latar belakangnya.
    \item Halaman lihat jadwal dosen: Teks nama dosen di atas tabel yang sedang dipilih pengguna memiliki rasio kontras 2,47:1 terhadap warna latar belakangnya. Teks nama dosen di atas tabel yang sedang tidak dipilih pengguna memiliki rasio kontras 3,06:1 terhadap warna latar belakangnya. Teks "Ekspor ke XLS" memiliki rasio kontras 3.09:1 terhadap warna latar belakangnya.
\end{itemize}

\paragraph{Kriteria Sukses 1.4.4 \textit{Resize text}}
\label{par:kepatuhan_bluetape_kriteria_sukses_1.4.4}
(Sukses)\\

Kriteria ini sukses dipatuhi karena setiap halaman web BlueTape dapat tetap terbaca dan tidak kehilangan fungsionalitasnya ketika halaman diperbesar hingga 200 persen. 

\paragraph{Kriteria Sukses 1.4.5 \textit{Images of Text}}
\label{par:kepatuhan_bluetape_kriteria_sukses_1.4.5}
(Sukses)\\

Kriteria ini sukses dipatuhi karena pada halaman web BlueTape tidak terdapat gambar teks selain logo.

\paragraph{Kriteria Sukses 1.4.6 \textit{Contrast (Enhanced)}}
\label{par:kepatuhan_bluetape_kriteria_sukses_1.4.2}
(Tidak Sukses)\\

Kriteria ini tidak sukses dipatuhi karena pada halaman web BlueTape terdapat beberapa teks dengan rasio kontras kurang dari 7:1 untuk teks yang berukuran kurang dari 24 piksel dan tidak \textit{bold}, antara lain:

\begin{itemize}
    \item Halaman \textit{login}: Teks "\textit{Login} dengan Google" memiliki rasio kontras 3.09:1 terhadap warna latar belakangnya. Teks "Petunjuk Penggunaan" memiliki rasio kontras 3.06:1 terhadap warna latar belakangnya.
    \item Halaman cetak transkrip: Teks "Kirim Permohonan" memiliki rasio kontras 3.09:1 terhadap warna latar belakangnya. Teks "Tercetak" di kolom status pada tabel memiliki rasio kontras 1,79:1 terhadap warna latar belakangnya. Teks "Ditolak" di kolom status pada tabel "Histori Permohonan" memiliki rasio kontras 3,44:1 terhadap warna latar belakangnya. Teks "Tunggu" di kolom status pada tabel "Histori Permohonan" memiliki rasio kontras 4,44:1 terhadap warna latar belakangnya.
    \item Halaman manajemen cetak transkrip: Teks "Cari" memiliki rasio kontras 3.09:1 terhadap warna latar belakangnya. Teks "Tercetak" di kolom status pada tabel memiliki rasio kontras 1,79:1 terhadap warna latar belakangnya. Teks "Menunggu" di kolom status pada tabel memiliki rasio kontras 1,84:1 terhadap warna latar belakangnya. Teks "Ditolak" di kolom status pada tabel memiliki rasio kontras 3,44:1 terhadap warna latar belakangnya. Teks "Hapus" pada bagian hapus permohonan memiliki rasio kontras 3,47:1 terhadap warna latar belakangnya.
    \item Halaman perubahan kuliah: Teks "Kirim Permohonan" memiliki rasio kontras 3.09:1 terhadap warna latar belakangnya. Teks "Tambah Pertemuan Ekstra" memiliki rasio kontras 4,47:1 terhadap warna latar belakangnya. Teks "Hapus" memiliki rasio kontras 4,47:1 terhadap warna latar belakangnya. Teks "Tunggu" di kolom status pada tabel "Histori Permohonan" memiliki rasio kontras 4,44:1 terhadap warna latar belakangnya. Teks "Ditolak" di kolom status pada tabel "Histori Permohonan" memiliki rasio kontras 3,44:1 terhadap warna latar belakangnya. Teks "Terkonfirmasi" di kolom status pada tabel "Histori Permohonan" memiliki rasio kontras 1,79:1 terhadap warna latar belakangnya.
    \item Halaman manajemen perubahan kuliah: Teks "Terkonfirmasi" di kolom status pada tabel memiliki rasio kontras 1,79:1 terhadap warna latar belakangnya. Teks "Menunggu" di kolom status pada tabel memiliki rasio kontras 1,84:1 terhadap warna latar belakangnya. Teks "Ditolak" di kolom status pada tabel memiliki rasio kontras 3,44:1 terhadap warna latar belakangnya. Teks "Hapus" pada bagian hapus permohonan memiliki rasio kontras 3,47:1 terhadap warna latar belakangnya.
    \item Halaman entri jadwal dosen: Teks "Tambah" memiliki rasio kontras 3.09:1 terhadap warna latar belakangnya. Teks "Delete All" memiliki rasio kontras 3.47:1 terhadap warna latar belakangnya. Teks "Ekspor ke XLS" memiliki rasio kontras 3.09:1 terhadap warna latar belakangnya.
    \item Halaman lihat jadwal dosen: Teks nama dosen di atas tabel yang sedang dipilih pengguna memiliki rasio kontras 2,47:1 terhadap warna latar belakangnya. Teks nama dosen di atas tabel yang sedang tidak dipilih pengguna memiliki rasio kontras 3,06:1 terhadap warna latar belakangnya. Teks "Ekspor ke XLS" memiliki rasio kontras 3.09:1 terhadap warna latar belakangnya.
\end{itemize}

\paragraph{Kriteria Sukses 1.4.7 \textit{Low or No Background Audio}}
\label{par:kepatuhan_bluetape_kriteria_sukses_1.4.7}
(Sukses)\\

Kriteria ini sukses dipatuhi karena pada halaman web BlueTape tidak terdapat konten media berbasis waktu.

\paragraph{Kriteria Sukses 1.4.8 \textit{Visual Presentation}}
\label{par:kepatuhan_bluetape_kriteria_sukses_1.4.8}
(Tidak Sukses)\\

Kriteria ini tidak sukses dipatuhi karena pada halaman manajemen cetak transkrip bagian hapus permohonan, teks yang ditampilkan memiliki lebar lebih dari 80 karakter.

\paragraph{Kriteria Sukses 1.4.9 \textit{Images of Text (No Exception)}}
\label{par:kepatuhan_bluetape_kriteria_sukses_1.4.9}
(Sukses)\\

Kriteria ini sukses dipatuhi karena pada halaman web BlueTape tidak terdapat gambar teks selain logo.

\paragraph{Kriteria Sukses 1.4.10 \textit{Reflow}}
\label{par:kepatuhan_bluetape_kriteria_sukses_1.4.10}
(Tidak Sukses)\\

Kriteria ini tidak sukses dipatuhi karena pada halaman web BlueTape, bagian navigasi menu memerlukan \textit{scroll} secara horizontal ketika ditampilkan pada resolusi layar dengan lebar 1280 piksel dan diperbesar hingga 400 persen.

\paragraph{Kriteria Sukses 1.4.11 \textit{Non-text Contrast}}
\label{par:kepatuhan_bluetape_kriteria_sukses_1.4.11}
()\\

\paragraph{Kriteria Sukses 1.4.12 \textit{Text Spacing}}
\label{par:kepatuhan_bluetape_kriteria_sukses_1.4.12}
()\\

\paragraph{Kriteria Sukses 1.4.13 \textit{Content on Hover or Focus}}
\label{par:kepatuhan_bluetape_kriteria_sukses_1.4.13}
(Sukses)\\

Kriteria ini sukses dipatuhi karena setiap konten tambahan yang muncul sesaat ketika suatu elemen menerima penunjuk kursor atau fokus \textit{keyboard}, konten tambahan tersebut dapat disingkirkan, dapat ditunjuk, dan persisten.

\subsection{\textit{Operable}}
\label{subsec:kepatuhan_bluetape_operable}

\subsubsection{\textit{Keyboard Accessible}}
\label{subsubsec:kepatuhan_bluetape_keyboard_accessible}

\paragraph{Kriteria Sukses 2.1.1 \textit{Keyboard}}
\label{par:kepatuhan_bluetape_kriteria_sukses_2.1.1}
(Tidak Sukses)\\

Kriteria ini tidak sukses dipatuhi karena bagian navigasi menu tidak dapat dioperasikan dengan menggunakan \textit{keyboard}.

\paragraph{Kriteria Sukses 2.1.2 \textit{No Keyboard Trap}}
\label{par:kepatuhan_bluetape_kriteria_sukses_2.1.2}
(Sukses)\\

Kriteria ini sukses dipatuhi karena pengguna dapat bernavigasi dari satu komponen ke komponen lain pada setiap komponen yang dapat dinavigasikan pada halaman web BlueTape dengan menggunakan \textit{keyboard} tanpa terperangkap dalam suatu komponen tertentu.

\paragraph{Kriteria Sukses 2.1.3 \textit{Keyboard (No Exception)}}
\label{par:kepatuhan_bluetape_kriteria_sukses_2.1.3}
(Tidak Sukses)\\

Kriteria ini tidak sukses dipatuhi karena bagian navigasi menu tidak dapat dioperasikan dengan menggunakan \textit{keyboard}.

\paragraph{Kriteria Sukses 2.1.4 \textit{Character Key Shortcuts}}
\label{par:kepatuhan_bluetape_kriteria_sukses_2.1.4}
(Sukses)\\

Kriteria ini sukses dipatuhi karena pada halaman web BlueTape tidak terdapat pintasan \textit{keyboard} untuk konten yang disajikan.

\subsubsection{\textit{Enough Time}}
\label{subsubsec:kepatuhan_bluetape_enough_time}

\paragraph{Kriteria Sukses 2.2.1 \textit{Timing Adjustable}}
\label{par:kepatuhan_bluetape_kriteria_sukses_2.2.1}
(Sukses)\\

Kriteria ini sukses dipatuhi karena pada halaman web BlueTape tidak terdapat batas waktu bagi pengguna untuk membaca dan memanfaatkan konten.

\paragraph{Kriteria Sukses 2.2.2 \textit{Pause, Stop, Hide}}
\label{par:kepatuhan_bluetape_kriteria_sukses_2.2.2}
(Sukses)\\

Kriteria ini sukses dipatuhi karena pada halaman web BlueTape tidak terdapat konten yang bergerak, berkelip, bergulir, ataupun diperbarui otomatis.

\paragraph{Kriteria Sukses 2.2.3 \textit{No Timing}}
\label{par:kepatuhan_bluetape_kriteria_sukses_2.2.3}
(Sukses)\\

Kriteria ini sukses dipatuhi karena pada halaman web BlueTape tidak terdapat batas waktu bagi pengguna untuk membaca dan memanfaatkan konten.

\paragraph{Kriteria Sukses 2.2.4 \textit{Interruptions}}
\label{par:kepatuhan_bluetape_kriteria_sukses_2.2.4}
(Sukses)\\

Kriteria ini sukses dipatuhi karena setiap interupsi pada halaman web BlueTape dapat ditunda atau dihentikan oleh pengguna.

\paragraph{Kriteria Sukses 2.2.5 \textit{Re-authenticating}}
\label{par:kepatuhan_bluetape_kriteria_sukses_2.2.5}
(Tidak Sukses)\\

Kriteria ini tidak sukses dipatuhi karena ketika pengguna akan mengirim data dan sesi autentikasi berakhir, pengguna kehilangan data tersebut setelah melakukan autentikasi ulang.

\paragraph{Kriteria Sukses 2.2.6 \textit{Timeouts}}
\label{par:kepatuhan_bluetape_kriteria_sukses_2.2.6}
(Tidak Sukses)\\

Kriteria ini tidak sukses dipatuhi karena tidak terdapat peringatan untuk pengguna ketika sesi autentikasi akan berakhir yang dapat menyebabkan kehilangan data. Data pengguna tidak disimpan untuk bertahan lebih dari 20 jam ketika pengguna tidak melakukan tindakan apa pun.

\subsubsection{\textit{Seizures and Physical Reactions}}
\label{subsubsec:kepatuhan_bluetape_seizures_and_physical_reactions}

\paragraph{Kriteria Sukses 2.3.1 \textit{Three Flashes or Below Threshold}}
\label{par:kepatuhan_bluetape_kriteria_sukses_2.3.1}
(Sukses)\\

Kriteria ini sukses dipatuhi karena pada halaman web BlueTape tidak terdapat konten yang berkelip.

\paragraph{Kriteria Sukses 2.3.2 \textit{Three Flashes}}
\label{par:kepatuhan_bluetape_kriteria_sukses_2.3.2}
(Sukses)\\

Kriteria ini sukses dipatuhi karena pada halaman web BlueTape tidak terdapat konten yang berkelip.

\paragraph{Kriteria Sukses 2.3.3 \textit{Animation from Interactions}}
\label{par:kepatuhan_bluetape_kriteria_sukses_2.3.3}
(Sukses)\\

Kriteria ini sukses dipatuhi karena pada halaman web BlueTape tidak terdapat animasi gerak pada konten yang disajikan ketika pengguna melakukan interaksi dengan komponen-komponen yang ada.

\subsubsection{\textit{Navigable}}
\label{subsubsec:kepatuhan_bluetape_navigable}

\paragraph{Kriteria Sukses 2.4.1 \textit{Bypass Blocks}}
\label{par:kepatuhan_bluetape_kriteria_sukses_2.4.1}
(Tidak Sukses)\\

Kriteria ini tidak sukses dipatuhi karena pada halaman web BlueTape tidak tersedia mekanisme untuk melompati beberapa area konten yang berulang pada beberapa halaman web.

\paragraph{Kriteria Sukses 2.4.2 \textit{Page Titled}}
\label{par:kepatuhan_bluetape_kriteria_sukses_2.4.2}
(Sukses)\\

Kriteria ini sukses dipatuhi karena setiap halaman web BlueTape memiliki judul yang dapat menjelaskan topik atau tujuan dari halaman yang bersangkutan.

\paragraph{Kriteria Sukses 2.4.3 \textit{Focus Order}}
\label{par:kepatuhan_bluetape_kriteria_sukses_2.4.3}
(Tidak Sukses)\\

Kriteria ini tidak sukses dipatuhi karena pada halaman perubahan kuliah di bagian "Permohonan Baru", urutan baca dan urutan navigasi kurang tepat.

\paragraph{Kriteria Sukses 2.4.4 \textit{Link Purpose (In Context)}}
\label{par:kepatuhan_bluetape_kriteria_sukses_2.4.4}
(Tidak Sukses)\\

Kriteria ini tidak sukses dipatuhi karena pada halaman cetak transkrip, manajemen cetak transkrip, perubahan kuliah, dan manajemen perubahan kuliah terdapat tautan yang berisi konten bukan teks dan tidak terdapat teks yang dapat menjelaskan tujuan tautan tersebut.

\paragraph{Kriteria Sukses 2.4.5 \textit{Multiple Ways}}
\label{par:kepatuhan_bluetape_kriteria_sukses_2.4.5}
(Tidak Sukses)\\

Kriteria ini tidak sukses dipatuhi karena pada halaman web BlueTape hanya tersedia satu cara untuk menemukan suatu halaman web dalam sekumpulan halaman web yang tersedia.

\paragraph{Kriteria Sukses 2.4.6 \textit{Headings and Labels}}
\label{par:kepatuhan_bluetape_kriteria_sukses_2.4.6}
()\\



\subsection{\textit{Understandable}}
\label{subsec:kepatuhan_bluetape_understandable}

\subsection{\textit{Robust}}
\label{subsec:kepatuhan_bluetape_robust}