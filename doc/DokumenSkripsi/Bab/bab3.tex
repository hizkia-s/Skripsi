\chapter{Analisis}
\label{chap:analisis}

\section{Tingkat Kepatuhan BlueTape Terhadap \textit{WCAG} 2.1}
Sukses atau tidaknya aplikasi BlueTape dalam mematuhi kriteria-kriteria sukses dalam \textit{WCAG} 2.1 dituliskan dalam tabel-tabel berikut.
\label{sec:kepatuhan_bluetape_terhadap_wcag_2.1}
\begin{table}[H]
    \centering 
    \caption{Tabel kepatuhan BlueTape terhadap prinsip \textit{Perceivable}}
    \label{tab:kepatuhan_bluetape_perceivable}
    \begin{tabular}{cccc}
        \toprule
        & Kriteria Sukses & Tingkat Kepatuhan & Hasil (sukses/tidak)\\

        \midrule
        & 1.1.1 & A & Tidak Sukses \\
        & 1.2.1 & A & Sukses \\
        & 1.2.2 & A & Sukses \\
        & 1.2.3 & A & Sukses \\
        & 1.2.4 & AA & Sukses \\
        & 1.2.5 & AA & Sukses \\
        & 1.2.6 & AAA & Sukses \\
        & 1.2.7 & AAA & Sukses \\
        & 1.2.8 & AAA & Sukses \\
        & 1.2.9 & AAA & Sukses \\
        & 1.3.1 & A & Tidak Sukses \\
        & 1.3.2 & A & Sukses \\
        & 1.3.3 & A & Tidak Sukses \\
        & 1.3.4 & AA & Sukses \\
        & 1.3.5 & AA & Sukses \\
        & 1.3.6 & AAA & Tidak Sukses \\
        & 1.4.1 & A & \\
        & 1.4.2 & A & Sukses \\
        & 1.4.3 & AA & \\
        & 1.4.4 & AA & \\
        & 1.4.5 & AA & \\
        & 1.4.6 & AAA & \\
        & 1.4.7 & AAA & Sukses \\
        & 1.4.8 & AAA & \\
        & 1.4.9 & AAA & \\
        & 1.4.10 & AA & \\
        & 1.4.11 & AA & \\
        & 1.4.12 & AA & \\
        & 1.4.13 & AA & \\
        
        \bottomrule

    \end{tabular}
\end{table}
\begin{table}[H]
    \centering 
    \caption{Tabel kepatuhan BlueTape terhadap prinsip \textit{Operable}}
    \label{tab:kepatuhan_bluetape_operable}
    \begin{tabular}{cccc}
        \toprule
        & Kriteria Sukses & Tingkat Kepatuhan & Hasil (sukses/tidak)\\

        \midrule
        & 2.1.1 & A & \\
        & 2.1.2 & A & \\
        & 2.1.3 & AAA & \\
        & 2.1.4 & A & \\
        & 2.2.1 & A & \\
        & 2.2.2 & A & \\
        & 2.2.3 & AAA & \\
        & 2.2.4 & AAA & \\
        & 2.2.5 & AAA & \\
        & 2.2.6 & AAA & \\
        & 2.3.1 & A & \\
        & 2.3.2 & AAA & \\
        & 2.3.3 & AAA & \\
        & 2.4.1 & A & \\
        & 2.4.2 & A & \\
        & 2.4.3 & A & \\
        & 2.4.4 & A & \\
        & 2.4.5 & AA & \\
        & 2.4.6 & AA & \\
        & 2.4.7 & AA & \\
        & 2.4.8 & AAA & \\
        & 2.4.9 & AAA & \\
        & 2.4.10 & AAA & \\
        & 2.5.1 & A & \\
        & 2.5.2 & A & \\
        & 2.5.3 & A & \\
        & 2.5.4 & A & \\
        & 2.5.5 & AAA & \\
        & 2.5.6 & AAA & \\

        \bottomrule
    
    \end{tabular}
\end{table}

\begin{table}[H]
    \centering 
    \caption{Tabel kepatuhan BlueTape terhadap prinsip \textit{Understandable}}
    \label{tab:kepatuhan_bluetape_understandable}
    \begin{tabular}{cccc}
        \toprule
        & Kriteria Sukses & Tingkat Kepatuhan & Hasil (sukses/tidak)\\

        \midrule
        & 3.1.1 & A & \\
        & 3.1.2 & AA & \\
        & 3.1.3 & AAA & \\
        & 3.1.4 & AAA & \\
        & 3.1.5 & AAA & \\
        & 3.1.6 & AAA & \\
        & 3.2.1 & A & \\
        & 3.2.2 & A & \\
        & 3.2.3 & AA & \\
        & 3.2.4 & AA & \\
        & 3.2.5 & AAA & \\
        & 3.3.1 & A & \\
        & 3.3.2 & A & \\
        & 3.3.3 & AA & \\
        & 3.3.4 & AA & \\
        & 3.3.5 & AAA & \\
        & 3.3.6 & AAA & \\

        \bottomrule

    \end{tabular}
\end{table}
\begin{table}[H]
    \centering 
    \caption{Tabel kepatuhan BlueTape terhadap prinsip \textit{Robust}}
    \label{tab:kepatuhan_bluetape_robust}
    \begin{tabular}{cccc}
        \toprule
        & Kriteria Sukses & Tingkat Kepatuhan & Hasil (sukses/tidak)\\

        \midrule
        & 4.1.1 & A & \\
        & 4.1.2 & A & \\
        & 4.1.3 & AA & \\

        \bottomrule
    
    \end{tabular} 
\end{table}

\subsection{\textit{Perceivable}}
\label{subsec:kepatuhan_bluetape_perceivable}

\subsubsection{\textit{Text Alternatives}}
\label{subsubsec:kepatuhan_bluetape_text_alternatives}

\paragraph{Kriteria Sukses 1.1.1 \textit{Non-text Content}}
\label{par:kepatuhan_bluetape_kriteria_sukses_1.1.1}
(Tidak Sukses)\\

Kriteria ini tidak sukses dipatuhi karena pada tampilan \textit{mobile}, di setiap halaman selain halaman \textit{login} terdapat elemen tombol yang tidak memiliki alternatif teks yang dapat ditafsirkan oleh teknologi alat bantu. Elemen tombol yang dimaksud hanya menampilkan tiga garis horizontal berwarna putih dan berfungsi untuk menampilkan bagian menu ketika ditekan.

\subsubsection{\textit{Time-based Media}}
\label{subsubsec:kepatuhan_bluetape_time_based_media}

\paragraph{Kriteria Sukses 1.2.1 \textit{Audio-only and Video-only (Prerecorded)}}
\label{par:kepatuhan_bluetape_kriteria_sukses_1.2.1}
(Sukses)\\

Kriteria ini sukses dipatuhi karena pada website BlueTape tidak terdapat konten media berbasis waktu.

\paragraph{Kriteria Sukses 1.2.2 \textit{Captions (Prerecorded)}}
\label{par:kepatuhan_bluetape_kriteria_sukses_1.2.2}
(Sukses)\\

Kriteria ini sukses dipatuhi karena pada website BlueTape tidak terdapat konten media berbasis waktu.

\paragraph{Kriteria Sukses 1.2.3 \textit{Audio Description or Media Alternative (Prerecorded)}}
\label{par:kepatuhan_bluetape_kriteria_sukses_1.2.3}
(Sukses)\\

Kriteria ini sukses dipatuhi karena pada website BlueTape tidak terdapat konten media berbasis waktu.

\paragraph{Kriteria Sukses 1.2.4 \textit{Captions (Live)}}
\label{par:kepatuhan_bluetape_kriteria_sukses_1.2.4}
(Sukses)\\

Kriteria ini sukses dipatuhi karena pada website BlueTape tidak terdapat konten media berbasis waktu.

\paragraph{Kriteria Sukses 1.2.5 \textit{Audio Description (Prerecorded)}}
\label{par:kepatuhan_bluetape_kriteria_sukses_1.2.5}
(Sukses)\\

Kriteria ini sukses dipatuhi karena pada website BlueTape tidak terdapat konten media berbasis waktu.

\paragraph{Kriteria Sukses 1.2.6 \textit{Sign Language (Prerecorded)}}
\label{par:kepatuhan_bluetape_kriteria_sukses_1.2.6}
(Sukses)\\

Kriteria ini sukses dipatuhi karena pada website BlueTape tidak terdapat konten media berbasis waktu.

\paragraph{Kriteria Sukses 1.2.7 \textit{Extended Audio Description (Prerecorded)}}
\label{par:kepatuhan_bluetape_kriteria_sukses_1.2.7}
(Sukses)\\

Kriteria ini sukses dipatuhi karena pada website BlueTape tidak terdapat konten media berbasis waktu.

\paragraph{Kriteria Sukses 1.2.8 \textit{Media Alternative (Prerecorded)}}
\label{par:kepatuhan_bluetape_kriteria_sukses_1.2.8}
(Sukses)\\

Kriteria ini sukses dipatuhi karena pada website BlueTape tidak terdapat konten media berbasis waktu.

\paragraph{Kriteria Sukses 1.2.9 \textit{Audio-only (Live)}}
\label{par:kepatuhan_bluetape_kriteria_sukses_1.2.9}
(Sukses)\\

Kriteria ini sukses dipatuhi karena pada website BlueTape tidak terdapat konten media berbasis waktu.

\subsubsection{\textit{Adaptable}}
\label{subsubsec:kepatuhan_bluetape_adaptable}

\paragraph{Kriteria Sukses 1.3.1 \textit{Info and Relationships}}
\label{par:kepatuhan_bluetape_kriteria_sukses_1.3.1}
(Tidak Sukses)\\

Kriteria ini tidak sukses dipatuhi karena:
\begin{itemize}
    \item Terdapat penggunaan \textit{tag heading} yang tidak tepat secara struktur pada halaman cetak transkrip, manajemen cetak transkrip, perubahan kuliah, manajemen perubahan kuliah, dan entri jadwal dosen.
    \item Pada halaman manajemen cetak transkrip, kolom masukan NPM tidak memiliki label atau \textit{aria-label} yang bersangkutan dengan kolom masukan tersebut.
    \item Pada halaman entri jadwal dosen, setiap kolom masukkan tidak memiliki label atau \textit{aria-label} yang bersangkutan dengan kolom-kolom masukan tersebut.
    \item Pada halaman lihat jadwal dosen, terdapat elemen yang tidak tepat secara struktur pada bagian \textit{list}.
\end{itemize} 

\paragraph{Kriteria Sukses 1.3.2 \textit{Meaningful Sequence}}
\label{par:kepatuhan_bluetape_kriteria_sukses_1.3.2}
(Sukses)\\

Kriteria ini sukses dipatuhi karena setiap halaman memiliki urutan baca dan urutan navigasi yang benar. 

\paragraph{Kriteria Sukses 1.3.3 \textit{Sensory Characteristics}}
\label{par:kepatuhan_bluetape_kriteria_sukses_1.3.3}
(Tidak Sukses)\\

Kriteria ini tidak sukses dipatuhi karena pada halaman cetak transkrip, manajemen cetak transkrip, perubahan kuliah, dan manajemen perubahan kuliah terdapat satu atau lebih komponen untuk mengoperasikan konten yang hanya mengandalkan satu komponen karakteristik indra yaitu bentuk. Komponen-komponen ini terletak pada kolom "Aksi" dan hanya ditampilkan dalam rupa ikon tanpa memiliki keterangan lebih detail.

\paragraph{Kriteria Sukses 1.3.4 \textit{Orientation}}
\label{par:kepatuhan_bluetape_kriteria_sukses_1.3.4}
(Sukses)\\

Kriteria ini sukses dipatuhi karena konten pada website BlueTape dapat disajikan dalam orientasi \textit{portrait} maupun \textit{landscape}.

\paragraph{Kriteria Sukses 1.3.5 \textit{Identify Input Purpose}}
\label{par:kepatuhan_bluetape_kriteria_sukses_1.3.5}
(Sukses)\\

Kriteria ini sukses dipatuhi karena setiap kolom masukan yang mengumpulkan informasi tentang pengguna sudah terisi otomatis.

\paragraph{Kriteria Sukses 1.3.6 \textit{Identify Purpose}}
\label{par:kepatuhan_bluetape_kriteria_sukses_1.3.6}
(Tidak Sukses)\\

Kriteria ini tidak sukses dipatuhi karena terdapat elemen \textit{HTML}5 yang seharusnya dapat digunakan namun tidak digunakan, contohnya pada bagian navigasi.

\subsubsection{\textit{Distinguishable}}
\label{subsubsec:kepatuhan_bluetape_distinguishable}

\paragraph{Kriteria Sukses 1.4.1 \textit{Use of Color}}
\label{par:kepatuhan_bluetape_kriteria_sukses_1.4.1}
()\\

\paragraph{Kriteria Sukses 1.4.2 \textit{Audio Control}}
\label{par:kepatuhan_bluetape_kriteria_sukses_1.4.2}
(Sukses)\\

Kriteria ini sukses dipatuhi karena pada website BlueTape tidak terdapat konten media berbasis waktu.

\paragraph{Kriteria Sukses 1.4.3 \textit{Contrast (Minimum)}}
\label{par:kepatuhan_bluetape_kriteria_sukses_1.4.3}
()\\

\paragraph{Kriteria Sukses 1.4.4 \textit{Resize text}}
\label{par:kepatuhan_bluetape_kriteria_sukses_1.4.4}
()\\

\paragraph{Kriteria Sukses 1.4.5 \textit{Images of Text}}
\label{par:kepatuhan_bluetape_kriteria_sukses_1.4.5}
()\\

\paragraph{Kriteria Sukses 1.4.6 \textit{Contrast (Enhanced)}}
\label{par:kepatuhan_bluetape_kriteria_sukses_1.4.2}
()\\

\paragraph{Kriteria Sukses 1.4.7 \textit{Low or No Background Audio}}
\label{par:kepatuhan_bluetape_kriteria_sukses_1.4.7}
(Sukses)\\

Kriteria ini sukses dipatuhi karena pada website BlueTape tidak terdapat konten media berbasis waktu.

\paragraph{Kriteria Sukses 1.4.8 \textit{Visual Presentation}}
\label{par:kepatuhan_bluetape_kriteria_sukses_1.4.8}
()\\

\paragraph{Kriteria Sukses 1.4.9 \textit{Images of Text (No Exception)}}
\label{par:kepatuhan_bluetape_kriteria_sukses_1.4.9}
()\\

\paragraph{Kriteria Sukses 1.4.10 \textit{Reflow}}
\label{par:kepatuhan_bluetape_kriteria_sukses_1.4.10}
()\\

\paragraph{Kriteria Sukses 1.4.11 \textit{Non-text Contrast}}
\label{par:kepatuhan_bluetape_kriteria_sukses_1.4.11}
()\\

\paragraph{Kriteria Sukses 1.4.12 \textit{Text Spacing}}
\label{par:kepatuhan_bluetape_kriteria_sukses_1.4.12}
()\\

\paragraph{Kriteria Sukses 1.4.13 \textit{Content on Hover or Focus}}
\label{par:kepatuhan_bluetape_kriteria_sukses_1.4.13}
()\\

\subsection{\textit{Operable}}
\label{subsec:kepatuhan_bluetape_operable}

\subsection{\textit{Understandable}}
\label{subsec:kepatuhan_bluetape_understandable}

\subsection{\textit{Robust}}
\label{subsec:kepatuhan_bluetape_robust}