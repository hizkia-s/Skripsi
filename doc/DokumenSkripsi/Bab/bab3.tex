\chapter{Analisis}
\label{chap:analisis}

\section{Tingkat Kepatuhan BlueTape Terhadap \textit{WCAG} 2.1}
Sukses atau tidaknya aplikasi BlueTape dalam mematuhi kriteria-kriteria sukses dalam \textit{WCAG} 2.1 dituliskan dalam tabel-tabel berikut.
\label{sec:kepatuhan_bluetape_terhadap_wcag_2.1}
\begin{table}[H]
    \centering 
    \caption{Tabel kepatuhan BlueTape terhadap prinsip \textit{Perceivable}}
    \label{tab:kepatuhan_bluetape_perceivable}
    \begin{tabular}{cccc}
        \toprule
        & Kriteria Sukses & Tingkat Kepatuhan & Hasil (sukses/tidak)\\

        \midrule
        & 1.1.1 & A & Tidak Sukses \\
        & 1.2.1 & A & \\
        & 1.2.2 & A & \\
        & 1.2.3 & A & \\
        & 1.2.4 & AA & \\
        & 1.2.5 & AA & \\
        & 1.2.6 & AAA & \\
        & 1.2.7 & AAA & \\
        & 1.2.8 & AAA & \\
        & 1.2.9 & AAA & \\
        & 1.3.1 & A & \\
        & 1.3.2 & A & \\
        & 1.3.3 & A & \\
        & 1.3.4 & AA & \\
        & 1.3.5 & AA & \\
        & 1.3.6 & AAA & \\
        & 1.4.1 & A & \\
        & 1.4.2 & A & \\
        & 1.4.3 & AA & \\
        & 1.4.4 & AA & \\
        & 1.4.5 & AA & \\
        & 1.4.6 & AAA & \\
        & 1.4.7 & AAA & \\
        & 1.4.8 & AAA & \\
        & 1.4.9 & AAA & \\
        & 1.4.10 & AA & \\
        & 1.4.11 & AA & \\
        & 1.4.12 & AA & \\
        & 1.4.13 & AA & \\
        
        \bottomrule

    \end{tabular}
\end{table}
\begin{table}[H]
    \centering 
    \caption{Tabel kepatuhan BlueTape terhadap prinsip \textit{Operable}}
    \label{tab:kepatuhan_bluetape_operable}
    \begin{tabular}{cccc}
        \toprule
        & Kriteria Sukses & Tingkat Kepatuhan & Hasil (sukses/tidak)\\

        \midrule
        & 2.1.1 & A & \\
        & 2.1.2 & A & \\
        & 2.1.3 & AAA & \\
        & 2.1.4 & A & \\
        & 2.2.1 & A & \\
        & 2.2.2 & A & \\
        & 2.2.3 & AAA & \\
        & 2.2.4 & AAA & \\
        & 2.2.5 & AAA & \\
        & 2.2.6 & AAA & \\
        & 2.3.1 & A & \\
        & 2.3.2 & AAA & \\
        & 2.3.3 & AAA & \\
        & 2.4.1 & A & \\
        & 2.4.2 & A & \\
        & 2.4.3 & A & \\
        & 2.4.4 & A & \\
        & 2.4.5 & AA & \\
        & 2.4.6 & AA & \\
        & 2.4.7 & AA & \\
        & 2.4.8 & AAA & \\
        & 2.4.9 & AAA & \\
        & 2.4.10 & AAA & \\
        & 2.5.1 & A & \\
        & 2.5.2 & A & \\
        & 2.5.3 & A & \\
        & 2.5.4 & A & \\
        & 2.5.5 & AAA & \\
        & 2.5.6 & AAA & \\

        \bottomrule
    
    \end{tabular}
\end{table}

\begin{table}[H]
    \centering 
    \caption{Tabel kepatuhan BlueTape terhadap prinsip \textit{Understandable}}
    \label{tab:kepatuhan_bluetape_understandable}
    \begin{tabular}{cccc}
        \toprule
        & Kriteria Sukses & Tingkat Kepatuhan & Hasil (sukses/tidak)\\

        \midrule
        & 3.1.1 & A & \\
        & 3.1.2 & AA & \\
        & 3.1.3 & AAA & \\
        & 3.1.4 & AAA & \\
        & 3.1.5 & AAA & \\
        & 3.1.6 & AAA & \\
        & 3.2.1 & A & \\
        & 3.2.2 & A & \\
        & 3.2.3 & AA & \\
        & 3.2.4 & AA & \\
        & 3.2.5 & AAA & \\
        & 3.3.1 & A & \\
        & 3.3.2 & A & \\
        & 3.3.3 & AA & \\
        & 3.3.4 & AA & \\
        & 3.3.5 & AAA & \\
        & 3.3.6 & AAA & \\

        \bottomrule

    \end{tabular}
\end{table}
\begin{table}[H]
    \centering 
    \caption{Tabel kepatuhan BlueTape terhadap prinsip \textit{Robust}}
    \label{tab:kepatuhan_bluetape_robust}
    \begin{tabular}{cccc}
        \toprule
        & Kriteria Sukses & Tingkat Kepatuhan & Hasil (sukses/tidak)\\

        \midrule
        & 4.1.1 & A & \\
        & 4.1.2 & A & \\
        & 4.1.3 & AA & \\

        \bottomrule
    
    \end{tabular} 
\end{table}

\subsection{\textit{Perceivable}}
\label{subsec:kepatuhan_bluetape_perceivable}

\subsubsection{\textit{Text Alternatives}}
\label{subsubsec:kepatuhan_bluetap_text_alternatives}

\paragraph{Kriteria Sukses 1.1.1 \textit{Non-text Content}}
\label{par:kepatuhan_bluetape_kriteria_sukses_1.1.1}
\par Kriteria ini tidak sukses dipatuhi karena pada tampilan \textit{mobile}, di setiap halaman selain halaman \textit{login} terdapat elemen tombol yang tidak memiliki alternatif teks yang dapat ditafsirkan oleh teknologi alat bantu. Elemen tombol yang dimaksud hanya menampilkan tiga garis horizontal berwarna putih dan berfungsi untuk menampilkan bagian menu ketika ditekan.