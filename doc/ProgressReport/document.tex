\documentclass[a4paper,twoside]{article}
\usepackage[T1]{fontenc}
\usepackage[bahasa]{babel}
\usepackage{graphicx}
\usepackage{graphics}
\usepackage{float}
\usepackage[cm]{fullpage}
\pagestyle{myheadings}
\usepackage{etoolbox}
\usepackage{setspace} 
\usepackage{lipsum} 
\setlength{\headsep}{30pt}
\usepackage[inner=2cm,outer=2.5cm,top=2.5cm,bottom=2cm]{geometry} %margin
% \pagestyle{empty}
\usepackage[export]{adjustbox}
\usepackage[plainpages=false,pdfpagelabels,unicode]{hyperref}% untuk \autoref,\phantomsection & link 
\hypersetup{unicode=true,colorlinks=true,linkcolor=blue,citecolor=green,filecolor=magenta, urlcolor=cyan}
\usepackage[table]{xcolor} %for tables
\usepackage{booktabs,pgfplots} % for \toprule & \midrule
\usepackage{listings}%untuk penulisan source code
%listing khusus untuk penulisan kode program, menggunakan font Bera Mono
\lstset{numbers=left,stepnumber=1, numbersep=5pt, frame=leftline,
	tabsize=4, breaklines=true, basicstyle=\fontfamily{fvm}\selectfont\tiny, 
	commentstyle=\itshape\color{gray}, keywordstyle=\bfseries\color{blue}, 
	identifierstyle=\color{black}, stringstyle=\color{orange},
	literate={-}{-}1{-\,-}{--}1
}
\graphicspath{{./Gambar/}}% folder tempat gambar 

\makeatletter
\renewcommand{\@maketitle} {\begin{center} {\LARGE \textbf{ \textsc{\@title}} \par} \bigskip {\large \textbf{\textsc{\@author}} }\end{center} }
\renewcommand{\thispagestyle}[1]{}
\markright{\textbf{\textsc{Laporan Perkembangan Pengerjaan Skripsi\textemdash Sem. Genap 2015/2016}}}

\onehalfspacing
 
\begin{document}

\title{\@judultopik}
\author{\nama \textendash \@npm} 

%ISILAH DATA BERIKUT INI:
\newcommand{\nama}{Hizkia Steven}
\newcommand{\@npm}{2015730020}
\newcommand{\tanggal}{25/11/2019} %Tanggal pembuatan dokumen
\newcommand{\@judultopik}{Kepatuhan dan Rekomendasi Perbaikan Web Content Accessibility Guideline untuk Aplikasi BlueTape} % Judul/topik anda
\newcommand{\kodetopik}{PAN4702}
\newcommand{\jumpemb}{1} % Jumlah pembimbing, 1 atau 2
\newcommand{\pembA}{Pascal Alfadian Nugroho}
\newcommand{\pembB}{-}
\newcommand{\semesterPertama}{47 - Ganjil 19/20} % semester pertama kali topik diambil, angka 1 dimulai dari sem Ganjil 96/97
\newcommand{\lamaSkripsi}{1} % Jumlah semester untuk mengerjakan skripsi s.d. dokumen ini dibuat
\newcommand{\kulPertama}{Skripsi 1} % Kuliah dimana topik ini diambil pertama kali
\newcommand{\tipePR}{B} % tipe progress report :
% A : dokumen pendukung untuk pengambilan ke-2 di Skripsi 1
% B : dokumen untuk reviewer pada presentasi dan review Skripsi 1
% C : dokumen pendukung untuk pengambilan ke-2 di Skripsi 2

% Dokumen hasil template ini harus dicetak bolak-balik !!!!

\maketitle

\pagenumbering{arabic}

\section{Data Skripsi} %TIDAK PERLU MENGUBAH BAGIAN INI !!!
Pembimbing utama/tunggal: {\bf \pembA}\\
Pembimbing pendamping: {\bf \pembB}\\
Kode Topik : {\bf \kodetopik}\\
Topik ini sudah dikerjakan selama : {\bf \lamaSkripsi} semester\\
Pengambilan pertama kali topik ini pada : Semester {\bf \semesterPertama} \\
Pengambilan pertama kali topik ini di kuliah : {\bf \kulPertama} \\
Tipe Laporan : {\bf \tipePR} -
\ifdefstring{\tipePR}{A}{
			Dokumen pendukung untuk {\BF pengambilan ke-2 di Skripsi 1} }
		{
		\ifdefstring{\tipePR}{B} {
				Dokumen untuk reviewer pada presentasi dan {\bf review Skripsi 1}}
			{	Dokumen pendukung untuk {\bf pengambilan ke-2 di Skripsi 2}}
		}
		
\section{Latar Belakang}
BlueTape merupakan aplikasi berbasis web yang dibuat untuk memudahkan berbagai urusan administrasi di Fakultas Teknologi Informasi dan Sains Universitas Katolik Parahyangan. Konsep aplikasi ini yaitu membuat urusan-urusan administrasi dapat dikerjakan melalui situs web sehingga mengurangi penggunaan kertas. Aplikasi ini disediakan untuk digunakan oleh mahasiswa, staf tata usaha, dan dosen. Fitur-fitur yang tersedia pada BlueTape yaitu manajemen cetak transkrip dan manajemen perubahan jadwal kuliah.

\textit{Web Content Accessibility Guidelines (WCAG)} adalah panduan yang berisi rekomendasi-rekomendasi untuk membuat konten web lebih mudah diakses dan digunakan oleh orang-orang, termasuk mereka yang memiliki keterbatasan. Keterbatasan yang tercakup dalam panduan ini yaitu keterbatasan visual, keterbatasan pendengaran, keterbatasan gerak, keterbatasan berbicara dan berbahasa, keterbatasan belajar, fotosensitif, keterbatasan kognitif, dan kombinasi dari beberapa keterbatasan yang telah disebutkan. Dalam \textit{WCAG} terdapat 3 tingkat kriteria sukses yaitu A, AA, dan AAA. Kriteria sukses adalah pernyataan-pernyataan yang dapat diuji yang dijadikan acuan untuk menilai tingkat kepatuhan sebuah situs web terhadap \textit{WCAG}. Kepatuhan tingkat A adalah tingkat kepatuhan terendah yang diperoleh jika seluruh kriteria sukses tingkat A terpenuhi atau versi alternatif yang sesuai tersedia. Kepatuhan tingkat AA adalah tingkat kepatuhan yang diperoleh jika seluruh kriteria sukses tingkat A dan tingkat AA terpenuhi atau versi alternatif tingkat AA yang sesuai tersedia. Kepatuhan tingkat AAA adalah tingkat kepatuhan tertinggi yang diperoleh jika seluruh kriteria sukses tingkat A, tingkat AA, dan tingkat AAA terpenuhi atau versi alternatif tingkat AAA yang sesuai tersedia.

Pada skripsi ini, akan dilihat sejauh mana tingkat kepatuhan situs web BlueTape terhadap \textit{WCAG} 2.1 dan rekomendasi apa saja yang perlu dilakukan untuk menaikkan tingkat kepatuhannya. Selain itu, akan dilakukan pengujian pada situs web tersebut dengan beberapa kondisi keterbatasan yang terdapat dalam \textit{WCAG} 2.1 seperti keterbatasan visual, keterbatasan gerak, keterbatasan pendengaran, dan keterbatasan bahasa.

\section{Rumusan Masalah}
\begin{itemize}
	\item Bagaimana tingkat kepatuhan situs web BlueTape terhadap \textit{WCAG} 2.1?
	\item Bagaimana meningkatkan tingkat kepatuhan situs web BlueTape terhadap \textit{WCAG} 2.1?  
	\item Bagaimana pengalaman menggunakan situs web BlueTape yang telah diperbarui dengan berbagai kondisi keterbatasan seperti yang terdapat dalam \textit{WCAG} 2.1?
\end{itemize}

\section{Tujuan}
\begin{itemize}
	\item Mendapatkan tingkat kepatuhan situs web BlueTape terhadap \textit{WCAG} 2.1.
	\item Meningkatkan tingkat kepatuhan situs web BlueTape terhadap \textit{WCAG} 2.1.
	\item Mendapatkan pengalaman menggunakan situs web BlueTape yang telah diperbarui dengan berbagai kondisi keterbatasan seperti yang terdapat dalam \textit{WCAG} 2.1.
\end{itemize}

\section{Detail Perkembangan Pengerjaan Skripsi}
Detail bagian pekerjaan skripsi sesuai dengan rencana kerja/laporan perkembangan terkahir :
	\begin{enumerate}
		\item \textbf{Mempelajari situs web BlueTape saat ini.}\\
		{\bf Status :} Ada sejak rencana kerja skripsi.\\
		{\bf Hasil :} Sudah dilakukan. Hasil yang didapat dari mempelajari situs web BlueTape, antara lain:
		\begin{itemize}
			\item Penulis berhasil membuat aplikasi BlueTape dapat berjalan di komputer lokal milik penulis. 
			\item Penulis mengetahui halaman dan fitur apa saja yang terdapat pada aplikasi BlueTape.
			\item Penulis sudah melakukan uji coba berbagai kasus untuk fitur-fitur yang terdapat pada aplikasi BlueTape dengan berperan sebagai \textit{root} yang memiliki hak akses tak terbatas. 
		\end{itemize}
		
		\item \textbf{Melakukan studi literatur mengenai \textit{WCAG} 2.1.}\\
		{\bf Status :} Ada sejak rencana kerja skripsi.\\
		{\bf Hasil :} Sudah dilakukan. Hasil yang didapat dari mempelajari \textit{WCAG} 2.1 adalah sebagai berikut:

		%2.1 WCAG 2.1%
		\section*{\textit{WCAG 2.1}}
		\label{sec:wcag_2.1} 
		\textit{Web Content Accessibility Guidelines (WCAG)} 2.1 adalah versi ketiga dari \textit{WCAG} yang dirilis pada tanggal 5 Juni 2018 \cite{WCAG:2.1}. Versi pertama dari \textit{WCAG} adalah \textit{WCAG} 1.0 yang dirilis pada tanggal 5 Mei 1999 dan versi kedua adalah \textit{WCAG} 2.0 yang dirilis pada tanggal 11 Desember 2008. \textit{WCAG} 2.1 dikembangkan oleh World Wide Web Consortium (W3C) melalui kerja sama dengan individu dan organisasi di seluruh dunia, dengan tujuan memberikan standar bersama untuk aksesibilitas konten web yang memenuhi kebutuhan individu, organisasi, dan pemerintah internasional. 

		\textit{WCAG} 2.1 dibuat untuk meningkatkan versi sebelumnya yaitu \textit{WCAG} 2.0. Pada \textit{WCAG} 2.1 terdapat penambahan kriteria sukses baru beserta definisi-definisi pendukungnya, pedoman untuk mengatur penambahan, dan beberapa tambahan pada bagian tingkat kepatuhan. Dalam \textit{WCAG} 2.1 terdapat 3 tingkat kriteria sukses yaitu A, AA, dan AAA yang digunakan sebagai acuan untuk menilai tingkat kepatuhan sebuah situs web terhadap \textit{WCAG} 2.1.

		%2.1.1 Perceivable%
		\subsection*{\textit{Perceivable}}
		\label{sec:perceivable}
		Informasi dan komponen antarmuka pengguna harus dapat disajikan kepada pengguna dengan cara yang bisa dipahami.

		%2.1.1.1 Text Alternatives%
		\subsubsection*{\textit{Text Alternatives}}
		\label{sec:text_alternatives}
		Untuk setiap konten yang bukan merupakan teks perlu disediakan teks alternatif.

		\paragraph{Kriteria Sukses 1.1.1 \textit{Non-text Content}}
		\label{sec:kriteria_sukses_1.1.1}
		(Level A)\\

		Semua konten bukan teks yang disajikan ke pengguna mempunyai teks alternatif yang menyajikan informasi dengan tujuan yang sama, kecuali untuk situasi-situasi berikut:
		\begin{itemize}
			\item Kontrol, masukan: Bila konten bukan teks merupakan kontrol atau bila konten tersebut menerima masukan dari pengguna, maka konten tersebut harus mempunyai nama yang menjelaskan tujuannya.
			\item Media berbasis waktu: Jika konten bukan teks merupakan media berbasis waktu, maka setidaknya teks alternatif harus menyediakan identifikasi deskriptif dari konten tersebut.
			\item Tes: Jika konten bukan teks merupakan tes atau latihan, yang akan mengungkap jawabannya jika disajikan dalam bentuk teks, maka setidaknya teks alternatif harus menyajikan identifikasi deskriptif dari konten tersebut.
			\item Indra: Jika maksud utama konten bukan teks dibuat untuk menciptakan pengalaman sensorik tertentu, maka setidaknya teks alternatif harus menyediakan identifikasi deskriptif untuk konten tersebut.
			\item \textit{CAPTCHA}: Jika tujuan dari konten bukan teks adalah untuk mengonfirmasi bahwa konten sedang diakses oleh manusia dan bukan oleh komputer, maka tersedia teks alternatif yang mengidentifikasi dan menjelaskan tujuan dari konten tersebut, dan tersedia bentuk alternatif dari \textit{CAPTCHA} yang menggunakan mode keluaran untuk berbagai jenis persepsi sensoris agar dapat mengakomodasi berbagai disabilitas.
			\item Dekorasi, pemformatan, tak kentara: Jika konten bukan teks hanya merupakan dekorasi yang digunakan hanya untuk format visual atau tidak disajikan kepada pengguna, maka konten tersebut diimplementasikan dengan cara yang dapat diabaikan oleh teknologi alat bantu.
		\end{itemize}
		%End of 2.1.1.1 Text Alternatives%

		%2.1.1.2 Time-based Media%
		\subsubsection*{\textit{Time-based Media}}
		\label{sec:time_based_media}
		Tersedia alternatif untuk media berbasis waktu.

		\paragraph{Kriteria Sukses 1.2.1 \textit{Audio-only and Video-only (Prerecorded)}}
		\label{sec:kriteria_sukses_1.2.1}
		(Level A)\\

		Untuk media rekaman berupa audio saja dan video saja, ketentuan berikut ini berlaku, kecuali bila audio atau video tersebut merupakan media alternatif untuk teks dan dilabeli dengan jelas:
		\begin{itemize}
			\item Rekaman audio saja: Tersedia alternatif untuk media berbasis waktu yang isinya mewakili informasi yang sama dengan konten rekaman audio saja.
			\item Rekaman video saja: Tersedia alternatif untuk media berbasis waktu atau trek audio yang isinya mewakili informasi yang sama dengan konten rekaman video saja.
		\end{itemize}

		\paragraph{Kriteria Sukses 1.2.2 \textit{Captions (Prerecorded)}}
		\label{sec:kriteria_sukses_1.2.2}
		(Level A)\\

		Takarir disediakan untuk semua konten rekaman audio, kecuali bila media tersebut merupakan media alternatif untuk teks dan dilabeli dengan jelas.

		\paragraph{Kriteria Sukses 1.2.3 \textit{Audio Description or Media Alternative (Prerecorded)}}
		\label{sec:kriteria_sukses_1.2.3}
		(Level A)\\

		Tersedia alternatif untuk media berbasis waktu atau deskripsi audio dari konten video rekaman, kecuali bila media tersebut merupakan media alternatif untuk teks dan dilabeli dengan jelas.

		\paragraph{Kriteria Sukses 1.2.4 \textit{Captions (Live)}}
		\label{sec:kriteria_sukses_1.2.4}
		(Level AA)\\

		Takarir disediakan untuk semua konten audio siaran langsung.

		\paragraph{Kriteria Sukses 1.2.5 \textit{Audio Description (Prerecorded)}}
		\label{sec:kriteria_sukses_1.2.5}
		(Level AA)\\

		Deskripsi audio disediakan untuk semua konten rekaman video.

		\paragraph{Kriteria Sukses 1.2.6 \textit{Sign Language (Prerecorded)}}
		\label{sec:kriteria_sukses_1.2.6}
		(Level AAA)\\

		Penafsiran bahasa isyarat disediakan untuk semua konten rekaman audio. 

		\paragraph{Kriteria Sukses 1.2.7 \textit{Extended Audio Description (Prerecorded)}}
		\label{sec:kriteria_sukses_1.2.7}
		(Level AAA)\\

		Ketika jeda dalam audio latar depan tidak memadai bagi deskripsi audio untuk menyampaikan maksud video, deskripsi audio tambahan disediakan untuk semua konten rekaman video.

		\paragraph{Kriteria Sukses 1.2.8 \textit{Media Alternative (Prerecorded)}}
		\label{sec:kriteria_sukses_1.2.8}
		(Level AAA)\\

		Tersedia alternatif untuk media berbasis waktu untuk semua media rekaman dan untuk semua media rekaman video saja.

		\paragraph{Kriteria Sukses 1.2.9 \textit{Audio-only (Live)}}
		\label{sec:kriteria_sukses_1.2.9}
		(Level AAA)\\

		Tersedia alternatif untuk media berbasis waktu yang menyajikan informasi yang sama dengan konten siaran langsung audio saja.
		%End of 2.1.1.2 Time-based Media%

		%2.1.1.3 Adaptable%
		\subsubsection*{\textit{Adaptable}}
		\label{sec:adaptable}
		Buat konten yang dapat disajikan dalam berbagai cara (misalnya tata letak yang lebih sederhana) tanpa kehilangan informasi atau struktur konten tersebut.

		\paragraph{Kriteria Sukses 1.3.1 \textit{Info and Relationships}}
		\label{sec:kriteria_sukses_1.3.1}
		(Level A)\\

		Informasi, struktur, dan hubungan antar konten yang disampaikan melalui presentasi dapat ditentukan secara terprogram atau tersedia dalam bentuk teks. 

		\paragraph{Kriteria Sukses 1.3.2 \textit{Meaningful Sequence}}
		\label{sec:kriteria_sukses_1.3.2}
		(Level A)\\

		Ketika urutan konten yang disajikan memengaruhi maknanya, urutan membaca yang benar dapat ditentukan secara terprogram.

		\paragraph{Kriteria Sukses 1.3.3 \textit{Sensory Characteristics}}
		\label{sec:kriteria_sukses_1.3.3}
		(Level A)\\

		Instruksi yang disediakan untuk memahami maupun mengoperasikan konten, tidak hanya mengandalkan satu komponen karakteristik indra seperti bentuk, ukuran, lokasi visual, orientasi, atau suara.

		\paragraph{Kriteria Sukses 1.3.4 \textit{Orientation}}
		\label{sec:kriteria_sukses_1.3.4}
		(Level AA)\\

		Konten tidak membatasi tampilan dan operasinya hanya untuk satu orientasi tampilan, seperti \textit{portrait} atau \textit{landscape}, kecuali jika orientasi tampilan tertentu bersifat esensial.

		\paragraph{Kriteria Sukses 1.3.5 \textit{Identify Input Purpose}}
		\label{sec:kriteria_sukses_1.3.5}
		(Level AA)\\

		Tujuan dari setiap bidang masukan yang mengumpulkan informasi tentang pengguna dapat ditentukan secara terprogram ketika:
		\begin{itemize}
			\item Area masukan menyajikan tujuan yang dapat diidentifikasi oleh agen pengguna dan teknologi alat bantu sehingga memungkinkan lebih banyak orang untuk mengerti dan menggunakan konten.
			\item Konten diimplementasikan menggunakan teknologi dengan dukungan untuk mengidentifikasi makna yang diharapkan untuk formulir masukan data.
		\end{itemize}

		\paragraph{Kriteria Sukses 1.3.6 \textit{Identify Purpose}}
		\label{sec:kriteria_sukses_1.3.6}
		(Level AAA)\\

		Pada konten yang diimplementasikan dengan menggunakan bahasa markah, tujuan dari komponen antarmuka pengguna, ikon, dan bidang konten dapat ditentukan secara terprogram.
		%End of 2.1.1.3 Adaptable%

		%2.1.1.4 Distinguishable%
		\subsubsection*{\textit{Distinguishable}}
		\label{sec:distinguishable}
		Beri kemudahan bagi pengguna untuk melihat dan mendengar konten, termasuk memisahkan latar depan dari latar belakang.

		\paragraph{Kriteria Sukses 1.4.1 \textit{Use of Color}}
		\label{sec:kriteria_sukses_1.4.1}
		(Level A)\\

		Warna tidak digunakan sebagai satu-satunya cara yang digunakan untuk menyampaikan informasi secara visual, menandai suatu tindakan, meminta respons, atau membedakan elemen visual.

		\paragraph{Kriteria Sukses 1.4.2 \textit{Audio Control}}
		\label{sec:kriteria_sukses_1.4.2}
		(Level A)\\

		Jika ada audio apa pun di halaman web yang diputar otomatis selama lebih dari 3 detik, maka harus tersedia mekanisme untuk memberi jeda atau memberhentikan audio tersebut, atau tersedia mekanisme untuk mengendalikan volume audio yang terpisah dari tingkat volume sistem secara keseluruhan.

		\paragraph{Kriteria Sukses 1.4.3 \textit{Contrast (Minimum)}}
		\label{sec:kriteria_sukses_1.4.3}
		(Level AA)\\

		Presentasi visual dari teks dan gambar teks memiliki rasio kontras setidaknya 4,5:1, kecuali untuk ketentuan berikut:

		\begin{itemize}
			\item Teks berukuran besar: Teks berukuran besar dan gambar teks berukuran besar memiliki rasio kontras setidaknya 3:1.
			\item Insidental: Teks atau gambar teks yang hanya merupakan dekorasi yang tidak tampak kepada siapa pun atau bagian gambar yang mengandung konten visual lain yang signifikan, tidak wajib memenuhi persyaratan kontras apa pun.
			\item Berjenis logo: Teks yang merupakan bagian dari logo atau nama merek tidak wajib memiliki persyaratan kontras apa pun.
		\end{itemize}

		\paragraph{Kriteria Sukses 1.4.4 \textit{Resize text}}
		\label{sec:kriteria_sukses_1.4.4}
		(Level AA)\\

		Selain takarir dan gambar teks, teks dapat diubah ukurannya tanpa teknologi alat bantu hingga dengan 200 persen, tanpa kehilangan fungsionalitas ataupun konten.

		\paragraph{Kriteria Sukses 1.4.5 \textit{Images of Text}}
		\label{sec:kriteria_sukses_1.4.5}
		(Level AA)\\

		Jika teknologi yang digunakan dapat menyajikan presentasi visual, maka yang digunakan untuk menyampaikan informasi adalah teks, bukan gambar teks, kecuali untuk ketentuan berikut:

		\begin{itemize}
			\item Dapat disesuaikan: Gambar teks dapat disesuaikan secara visual sesuai kebutuhan pengguna;
			\item Esensial: Penyajian tertentu dari teks bersifat esensial terhadap informasi yang disampaikan.
		\end{itemize}

		\paragraph{Kriteria Sukses 1.4.6 \textit{Contrast (Enhanced)}}
		\label{sec:kriteria_sukses_1.4.6}
		(Level AAA)\\

		Presentasi visual dari teks dan gambar teks memiliki rasio kontras setidaknya 7:1, kecuali untuk ketentuan berikut:

		\begin{itemize}
			\item Teks berukuran besar: Teks berukuran besar dan gambar teks berukuran besar memiliki rasio kontras setidaknya 4,5:1.
			\item Insidental: Teks atau gambar teks yang hanya merupakan dekorasi yang tidak tampak kepada siapa pun atau bagian gambar yang mengandung konten visual lain yang signifikan, tidak wajib memenuhi persyaratan kontras apa pun.
			\item Berjenis logo: Teks yang merupakan bagian dari logo atau nama merek tidak wajib memiliki persyaratan kontras apa pun.
		\end{itemize}

		\paragraph{Kriteria Sukses 1.4.7 \textit{Low or No Background Audio}}
		\label{sec:kriteria_sukses_1.4.7}
		(Level AAA)\\

		Untuk konten rekaman audio yang (1) utamanya mengandung ucapan di latar depan, (2) bukan merupakan audio untuk \textit{CAPTCHA} atau audio untuk logo, dan (3) bukan merupakan vokalisasi untuk expresi musikal seperti nyanyian atau rap, setidaknya salah satu dari ketentuan berikut berlaku:

		\begin{itemize}
			\item Tanpa latar belakang: Audio tidak mengandung suara latar belakang.
			\item Dapat dimatikan: Suara latar belakang dapat dimatikan.
			\item 20 dB: Suara latar belakang setidaknya 20 desibel lebih rendah dari konten ucapan di latar depan, kecuali untuk suara-suara yang muncul sesekali dan hanya berdurasi satu atau dua detik.
		\end{itemize}

		\paragraph{Kriteria Sukses 1.4.8 \textit{Visual Presentation}}
		\label{sec:kriteria_sukses_1.4.8}
		(Level AAA)\\

		Untuk presentasi visual dari kumpulan teks, tersedia mekanisme untuk mencapai tujuan-tujuan berikut:

		\begin{itemize}
			\item Warna latar depan dan belakang dapat dipilih oleh pengguna.
			\item Lebarnya tidak lebih dari 80 karakter atau \textit{glyphs} (40 jika karakter \textit{CJK}).
			\item Teks tidak rata kiri-kanan.
			\item Awal paragraf menjorok masuk minimal satu setengah spasi, jarak setiap paragraf minimal satu setengah kali jarak setiap baris.
			\item Teks dapat diperbesar hingga 200 persen tanpa teknologi alat bantu, namun tetap memungkinkan pengguna untuk membaca baris teks tersebut tanpa perlu melakukan \textit{scroll} halaman secara horizontal.
		\end{itemize}

		\paragraph{Kriteria Sukses 1.4.9 \textit{Images of Text (No Exception)}}
		\label{sec:kriteria_sukses_1.4.9}
		(Level AAA)\\

		Gambar teks hanya digunakan untuk dekorasi semata atau ketika penyajian tertentu dari teks bersifat esensial dalam menyampaikan informasi.

		\paragraph{Kriteria Sukses 1.4.10 \textit{Reflow}}
		\label{sec:kriteria_sukses_1.4.10}
		(Level AA)\\

		Konten dapat disajikan tanpa kehilangan fungsionalitas ataupun konten, dan tanpa memerlukan \textit{scrolling} dalam dua dimensi untuk:

		\begin{itemize}
			\item Konten \textit{scrolling} vertikal dengan lebar setara 320 piksel \textit{CSS};
			\item Konten \textit{scrolling} horizontal dengan tinggi setara 256 piksel \textit{CSS}.
		\end{itemize}

		Kecuali untuk bagian-bagian konten yang memerlukan tata letak dua dimensi untuk makna atau penggunaan.

		\paragraph{Kriteria Sukses 1.4.11 \textit{Non-text Contrast}}
		\label{sec:kriteria_sukses_1.4.11}
		(Level AA)\\

		Presentasi visual pada poin-poin berikut memiliki rasio kontras setidaknya 3:1 terhadap warna yang bedekatan:

		\begin{itemize}
			\item Komponen antarmuka pengguna: Informasi visual dibutuhkan untuk mengidentifikasi keadaan dan komponen antarmuka pengguna, kecuali untuk komponen yang tidak aktif atau ketika penampilan komponen ditentukan oleh agen pengguna dan tidak dimodifikasi oleh penulis.
			\item Objek grafis: Bagian grafis dibutuhkan untuk memahami konten, kecuali ketika penyajian grafis tertentu bersifat esensial untuk informasi yang disampaikan.
		\end{itemize}

		\paragraph{Kriteria Sukses 1.4.12 \textit{Text Spacing}}
		\label{sec:kriteria_sukses_1.4.12}
		(Level AA)\\

		Pada konten yang diimplementasikan dengan menggunakan bahasa markah yang mendukung properti \textit{style} teks berikut, tidak ada fungsionalitas atau konten yang hilang ketika mengatur ketentuan-ketentuan berikut tanpa mengubah properti \textit{style} lainnya:

		\begin{itemize}
			\item Tinggi baris (jarak antara baris) setidaknya 1,5 kali ukuran tulisan;
			\item Jarak antara paragraf setidaknya 2 kali ukuran tulisan;
			\item Jarak antara huruf setidaknya 0,12 kali ukuran tulisan;
			\item Jarak antara kata setidaknya 0,16 kali ukuran tulisan.
		\end{itemize}

		Pengecualian: Konten teks yang tidak menggunakan satu atau lebih properti \textit{style} teks ini, dapat menyesuaikan dengan menggunakan properti yang tersedia untuk teks tersebut.

		\paragraph{Kriteria Sukses 1.4.13 \textit{Content on Hover or Focus}}
		\label{sec:kriteria_sukses_1.4.13}
		(Level AA)\\

		Ketika menerima lalu menghapus penunjuk kursor atau fokus \textit{keyboard} memicu konten tambahan untuk muncul sesaat, ketentuan berikut ini berlaku:

		\begin{itemize}
			\item Dapat disingkirkan: Tersedia mekanisme untuk menyingkirkan konten tambahan tanpa perlu memindahkan penunjuk kursor atau fokus \textit{keyboard}, kecuali jika konten tambahan menginformasikan kesalahan masukan atau tidak menghalangi maupun mengganti konten lain;
			\item Dapat ditunjuk: Jika penunjuk kursor dapat memicu konten tambahan, maka kursor dapat dipindahkan ke konten tambahan tanpa membuat konten tambahan tersebut menghilang;
			\item Persisten: Konten tambahan tetap tampak sampai penunjuk atau pemicu fokus dihapus, disingkirkan pengguna, atau informasinya sudah tidak valid.
		\end{itemize}

		Pengecualian: Presentasi visual pada konten tambahan diatur oleh agen pengguna dan tidak dimodifikasi oleh penulis.

		%End of 2.1.1.4 Distinguishable%

		%End of 2.1.1 Perceivable%

		%2.1.2 Operable%
		\subsection*{\textit{Operable}}
		\label{sec:operable}
		Komponen antarmuka pengguna dan navigasi harus bisa dioperasikan.

		%2.1.2.1 Keyboard Accessible%
		\subsubsection*{\textit{Keyboard Accessible}}
		\label{sec:keyboard_accessible}
		Pastikan semua fungsionalitas bisa diakses dengan \textit{keyboard}.

		\paragraph{Kriteria Sukses 2.1.1 \textit{Keyboard}}
		\label{sec:kriteria_sukses_2.1.1}
		(Level A)\\

		Semua fungsionalitas konten dapat dioperasikan melalui antarmuka \textit{keyboard} tanpa perlu mengatur jeda antar ketukan tombol, kecuali bila fungsi tersebut membutuhkan masukan yang bergantung pada jalur gerakan pengguna dan bukan hanya pada titik akhir.

		\paragraph{Kriteria Sukses 2.1.2 \textit{No Keyboard Trap}}
		\label{sec:kriteria_sukses_2.1.2}
		(Level A)\\

		Jika fokus \textit{keyboard} dapat dipindahkan ke komponen tertentu dengan menggunakan antarmuka \textit{keyboard}, maka fokus dapat dipindahkan dari komponen tersebut hanya dengan menggunakan antarmuka \textit{keyboard}. Jika dibutuhkan tindakan yang lebih dari sekadar menekan tombol panah atau \textit{tab} atau metode-metode keluar standar lainnya, maka pengguna akan diberi tahu tentang metode tersebut.

		\paragraph{Kriteria Sukses 2.1.3 \textit{Keyboard (No Exception)}}
		\label{sec:kriteria_sukses_2.1.3}
		(Level AAA)\\

		Semua fungsionalitas konten dapat dioperasikan melalui antarmuka \textit{keyboard} tanpa perlu mengatur jeda antar ketukan tombol.

		\paragraph{Kriteria Sukses 2.1.4 \textit{Character Key Shortcuts}}
		\label{sec:kriteria_sukses_2.1.4}
		(Level A)\\

		Jika pintasan \textit{keyboard} yang diterapkan dalam konten hanya menggunakan huruf (termasuk huruf besar dan kecil), tanda baca, angka, atau karakter simbol, maka setidaknya salah satu dari ketentuan berikut ini berlaku:
		\begin{itemize}
			\item Dapat dinonaktifkan: Tersedia mekanisme untuk menonaktifkan pintasan;
			\item Dipetakan kembali: Tersedia mekanisme untuk memetakan kembali pintasan untuk menggunakan satu atau lebih karakter \textit{keyboard} yang tidak dapat dicetak (misalnya: \textit{CTRL, ALT, SHIFT});
			\item Hanya aktif saat mendapat fokus: Pintasan \textit{keyboard} untuk komponen antarmuka pengguna hanya aktif ketika komponen tersebut mendapat fokus.
		\end{itemize}
		%End of 2.1.2.1 Keyboard Accessible%

		%2.1.2.2 Enough Time%
		\subsubsection*{\textit{Enough Time}}
		\label{sec:enough_time}
		Sediakan cukup waktu agar pengguna bisa membaca dan memanfaatkan konten.

		\paragraph{Kriteria Sukses 2.2.1 \textit{Timing Adjustable}}
		\label{sec:kriteria_sukses_2.2.1}
		(Level A)\\

		Untuk setiap batas waktu yang ditentukan oleh konten, setidaknya salah satu dari ketentuan berikut berlaku:
		\begin{itemize}
			\item Dapat dinonaktifkan: Pengguna dapat menonaktifkan batas waktu sebelum mencapai batas tersebut; atau
			\item Dapat disesuaikan: Pengguna diizinkan untuk menyesuaikan batas waktu sebelum mencapai batas tersebut, dengan waktu tambahan setidaknya sepuluh kali dari setelan batas waktu; atau
			\item Dapat diperpanjang: Pengguna diberi peringatan ketika batas waktu hampir habis dan diberikan waktu setidaknya dua puluh detik untuk memperpanjang batas waktu tersebut dengan tindakan yang sederhana (misalnya menekan tombol spasi), dan pengguna diizinkan untuk menambah batas waktu tersebut setidaknya sepuluh kali lipat; atau
			\item Perkecualian waktu riil: Batas waktu merupakan bagian yang wajib dari kejadian waktu riil (misalnya pelelangan), dan mustahil untuk menyediakan alternatif untuk batas waktu tersebut; atau
			\item Perkecualian penting: Batas waktu bersifat esensial dan jika diperpanjang maka akan menyalahi inti dari kegiatan tersebut; atau
			\item Perkecualian dua puluh jam: Batas waktu yang diberikan lebih dari dua puluh jam.
		\end{itemize}

		\paragraph{Kriteria Sukses 2.2.2 \textit{Pause, Stop, Hide}}
		\label{sec:kriteria_sukses_2.2.2}
		(Level A)\\

		Untuk informasi yang bergerak, berkelip, bergulir, atau diperbarui otomatis, semua ketentuan berikut berlaku:
		\begin{itemize}
			\item Bergerak, berkelip, bergulir: Untuk informasi apa pun yang bergerak, berkelip, atau bergulir yang (1) mulainya otomatis, (2) terjadi lebih dari lima detik, dan (3) disajikan paralel dengan konten lain, tersedia mekanisme bagi pengguna untuk memberi jeda, memberhentikan, atau menyembunyikan informasi tersebut; kecuali bila aktivitas bergerak, berkelip, atau bergulir tersebut merupakan bagian dari aktivitas yang esensial; dan
			\item Diperbarui otomatis: Untuk informasi apa pun yang diperbarui otomatis, yaitu yang (1) mulainya otomatis dan (2) disajikan paralel dengan konten lain, tersedia mekanisme bagi pengguna untuk memberi jeda, memberhentikan, atau menyembunyikan informasi tersebut; atau terdapat cara untuk mengendalikan frekuensi pembaruan tersebut, kecuali jika pembaruan otomatis tersebut merupakan bagian dari aktivitas yang esensial.
		\end{itemize}

		\paragraph{Kriteria Sukses 2.2.3 \textit{No Timing}}
		\label{sec:kriteria_sukses_2.2.3}
		(Level AAA)\\

		Waktu bukanlah bagian esensial dari kejadian atau aktivitas yang disajikan oleh konten, kecuali untuk \textit(synchronized media) yang tidak interaktif dan kejadian waktu riil.

		\paragraph{Kriteria Sukses 2.2.4 \textit{Interruptions}}
		\label{sec:kriteria_sukses_2.2.4}
		(Level AAA)\\

		Interupsi dapat ditunda atau dihentikan oleh pengguna, kecuali bila interupsi melibatkan keadaan darurat.

		\paragraph{Kriteria Sukses 2.2.5 \textit{Re-authenticating}}
		\label{sec:kriteria_sukses_2.2.5}
		(Level AAA)\\

		Ketika sesi autentikasi berakhir, pengguna dapat melanjutkan aktivitas tanpa kehilangan data setelah melakukan autentikasi ulang.

		\paragraph{Kriteria Sukses 2.2.6 \textit{Timeouts}}
		\label{sec:kriteria_sukses_2.2.6}
		(Level AAA)\\

		Pengguna diberi peringatan mengenai durasi ketidakaktifan pengguna yang dapat menyebabkan data hilang, kecuali jika data tersebut disimpan lebih dari dua puluh jam ketika pengguna tidak melakukan tindakan apa pun.
		%End of 2.1.2.2 Enough Time%

		%2.1.2.3 Seizures and Physical Reactions%
		\subsubsection*{\textit{Seizures and Physical Reactions}}
		\label{sec:seizures_and_physical_reactions}
		Jangan merancang konten yang dapat menyebabkan kejang atau reaksi fisik.

		\paragraph{Kriteria Sukses 2.3.1 \textit{Three Flashes or Below Threshold}}
		\label{sec:kriteria_sukses_2.3.1}
		(Level A)\\

		Halaman web tidak mengandung apa pun yang berkelip lebih dari tiga kali dalam jangka waktu satu detik, atau kelipan di bawah ambang batas kelipan biasa dan kelipan merah.

		\paragraph{Kriteria Sukses 2.3.2 \textit{Three Flashes}}
		\label{sec:kriteria_sukses_2.3.2}
		(Level AAA)\\

		Halaman web tidak mengandung apa pun yang berkelip lebih dari tiga kali dalam jangka waktu satu detik.

		\paragraph{Kriteria Sukses 2.3.3 \textit{Animation from Interactions}}
		\label{sec:kriteria_sukses_2.3.3}
		(Level AAA)\\

		Animasi gerak yang dipicu oleh interaksi dapat dinonaktifkan, kecuali jika animasi itu esensial untuk fungsionalitas atau informasi yang disampaikan.
		%End of 2.1.2.3 Seizures and Physical Reactions%

		%2.1.2.4 Navigable%
		\subsubsection*{\textit{Navigable}}
		\label{sec:navigable}
		Sediakan cara yang mudah untuk membantu pengguna bernavigasi, menemukan konten, dan menentukan di mana mereka berada.

		\paragraph{Kriteria Sukses 2.4.1 \textit{Bypass Blocks}}
		\label{sec:kriteria_sukses_2.4.1}
		(Level A)\\

		Tersedia mekanisme untuk melompati beberapa area konten yang berulang pada beberapa halaman web.

		\paragraph{Kriteria Sukses 2.4.2 \textit{Page Titled}}
		\label{sec:kriteria_sukses_2.4.2}
		(Level A)\\

		Halaman web memiliki judul yang menjelaskan topik atau tujuan.

		\paragraph{Kriteria Sukses 2.4.3 \textit{Focus Order}}
		\label{sec:kriteria_sukses_2.4.3}
		(Level A)\\

		Bila halaman web dapat dinavigasi secara berurutan dan urutan navigasi memengaruhi makna atau operasi, maka komponen yang memang dapat difokus akan menerima fokus sesuai urutan yang mempertahankan makna dan pengoperasian.

		\paragraph{Kriteria Sukses 2.4.4 \textit{Link Purpose (In Context)}}
		\label{sec:kriteria_sukses_2.4.4}
		(Level A)\\

		Tujuan tiap tautan dapat ditentukan dari teks pada tautan saja atau dari kombinasi teks pada tautan dengan konteks tautan yang ditentukan secara terprogram, kecuali bila tujuan tautan akan bersifat ambigu bagi pengguna secara umum.

		\paragraph{Kriteria Sukses 2.4.5 \textit{Multiple Ways}}
		\label{sec:kriteria_sukses_2.4.5}
		(Level AA)\\

		Tersedia lebih dari satu cara untuk menemukan halaman web dalam sekumpulan halaman web kecuali bila halaman web tersebut merupakan hasil dari, atau langkah ke-sekian dari suatu proses.

		\paragraph{Kriteria Sukses 2.4.6 \textit{Headings and Labels}}
		\label{sec:kriteria_sukses_2.4.6}
		(Level AA)\\
		Kepala tulisan dan label menjabarkan topik atau tujuan.

		\paragraph{Kriteria Sukses 2.4.7 \textit{Focus Visible}}
		\label{sec:kriteria_sukses_2.4.7}
		(Level AA)\\

		Setiap antarmuka pengguna yang dapat dioperasikan dengan \textit{keyboard} memiliki mode operasi yang memungkinkan indikator fokus dari \textit{keyboard} terlihat dengan jelas.

		\paragraph{Kriteria Sukses 2.4.8 \textit{Location}}
		\label{sec:kriteria_sukses_2.4.8}
		(Level AAA)\\

		Tersedia informasi mengenai lokasi pengguna dalam sekumpulan halaman web.

		\paragraph{Kriteria Sukses 2.4.9 \textit{Link Purpose (Link Only)}}
		\label{sec:kriteria_sukses_2.4.9}
		(Level AAA)\\

		Tersedia mekanisme untuk memungkinkan tujuan tiap tautan diidentifikasi dari teks pada tautan saja, kecuali bila tujuan tautan tersebut akan bersifat ambigu bagi pengguna secara umum.

		\paragraph{Kriteria Sukses 2.4.10 \textit{Section Headings}}
		\label{sec:kriteria_sukses_2.4.10}
		(Level AAA)\\

		Kepala tulisan tiap-tiap bagian digunakan untuk mengatur konten.
		%End of 2.1.2.4 Navigable%

		%2.1.2.5 Input Modalities%
		\subsubsection*{\textit{Input Modalities}}
		\label{sec:input_modalities}
		Permudah pengguna untuk mengoperasikan fungsionalitas melalui berbagai masukan di luar \textit{keyboard}.

		\paragraph{Kriteria Sukses 2.5.1 \textit{Pointer Gestures}}
		\label{sec:kriteria_sukses_2.5.1}
		(Level A)\\

		Semua fungsionalitas yang menggunakan \textit{multipoint} atau gestur berbasis \textit{path} dapat dioperasikan dengan kursor tunggal tanpa gestur berbasis \textit{path}, kecuali jika \textit{multipoint} atau gestur berbasis \textit{path} tersebut esensial.

		\paragraph{Kriteria Sukses 2.5.2 \textit{Pointer Cancellation}}
		\label{sec:kriteria_sukses_2.5.2}
		(Level A)\\

		Untuk fungsionalitas yang dapat dioperasikan dengan kursor tunggal, setidaknya salah satu dari ketentuan berikut berlaku:
		\begin{itemize}
			\item Tidak ada \textit{down-event}: \textit{Down-event} pada kursor tidak digunakan untuk menjalankan fungsi apa pun;
			\item Dapat dibatalkan atau dikembalikan: Penyelesaian fungsi terjadi pada \textit{up-event}, dan tersedia mekanisme untuk membatalkan fungsi jika fungsi tersebut belum selesai atau untuk membalikkan fungsi jika fungsi tersebut sudah selesai;
			\item \textit{Up reversal}: \textit{Up-event} membalikkan setiap hasil dari \textit{down-event} sebelumnya;
			\item Esensial: Menyelesaikan fungsi pada \textit{down-event} adalah hal yang esensial.
		\end{itemize}

		\paragraph{Kriteria Sukses 2.5.3 \textit{Label in Name}}
		\label{sec:kriteria_sukses_2.5.3}
		(Level A)\\

		Untuk komponen antarmuka pengguna dengan label yang menyertakan teks atau gambar teks, nama komponen tersebut mengandung teks yang disajikan secara visual. 

		\paragraph{Kriteria Sukses 2.5.4 \textit{Motion Actuation}}
		\label{sec:kriteria_sukses_2.5.4}
		(Level A)\\

		Fungsionalitas yang dapat dioperasikan oleh gerakan alat atau gerakan pengguna dapat juga dioperasikan oleh komponen antarmuka pengguna dan respon terhadap gerakan dapat dinonaktifkan untuk mencegah aksi yang tidak disengaja, kecuali saat:
		\begin{itemize}
			\item Antarmuka mendukung: Gerakan digunakan untuk mengoperasikan fungsionalitas melalui antarmuka yang mendukung aksesibilitas;
			\item Esensial: Gerakan adalah hal yang esensial untuk fungsi tersebut dan penonaktifan respon terhadap gerakan akan membatalkan aktivitas yang sedang berlangsung.
		\end{itemize}

		\paragraph{Kriteria Sukses 2.5.5 \textit{Target Size}}
		\label{sec:kriteria_sukses_2.5.5}
		(Level AAA)\\

		Ukuran target untuk masukan kursor tidak kurang dari 44 kali 44 piksel \textit{CSS} kecuali jika:

		\begin{itemize}
			\item Setara: Pada halaman yang sama tersedia kontrol atau tautan yang setara untuk target, dengan ukuran tidak kurang dari 44 kali 44 piksel \textit{CSS};  
			\item Terdapat dalam barisan: Target berada dalam kalimat atau blok teks;
			\item Kontrol agen pengguna: Ukuran target ditentukan oleh agen pengguna dan tidak dimodifikasi oleh penulis;
			\item Esensial: Penyajian khusus dari target bersifat esensial untuk informasi yang disampaikan.
		\end{itemize}

		\paragraph{Kriteria Sukses 2.5.6 \textit{Concurrent Input Mechanisms}}
		\label{sec:kriteria_sukses_2.5.6}
		(Level AAA)\\

		Konten web tidak membatasi penggunaan modalitas masukan yang tersedia pada platform kecuali jika pembatasan tersebut esensial dan diperlukan untuk memastikan keamanan konten atau untuk mematuhi pengaturan pengguna.
		%End of 2.1.2.5 Input Modalities%

		%End of 2.1.2 Operable%

		%2.1.3 Understandable%
		\subsection*{\textit{Understandable}}
		\label{sec:understandable}
		Informasi dan pengoperasian antarmuka pengguna harus dapat dimengerti.

		%2.1.3.1 Readable%
		\subsubsection*{\textit{Readable}}
		\label{sec:readable}
		Buat konten teks mudah dibaca dan dimengerti.

		\paragraph{Kriteria Sukses 3.1.1 \textit{Language of Page}}
		\label{sec:kriteria_sukses_3.1.1}
		(Level A)\\

		Bahasa manusia \textit{default} untuk setiap halaman web dapat ditentukan secara terprogram.

		\paragraph{Kriteria Sukses 3.1.2 \textit{Language of Parts}}
		\label{sec:kriteria_sukses_3.1.2}
		(Level AA)\\

		Bahasa manusia pada setiap bagian atau frasa yang terdapat dalam konten dapat ditentukan secara terprogram kecuali untuk nama diri, istilah teknis, kata dari bahasa yang tidak tentu, dan kata atau frasa yang telah menjadi bagian dari bahasa daerah dari teks yang ada di sekelilingnya.

		\paragraph{Kriteria Sukses 3.1.3 \textit{Unusual Words}}
		\label{sec:kriteria_sukses_3.1.3}
		(Level AAA)\\

		Tersedia mekanisme untuk mengidentifikasi definisi spesifik dari kata atau frasa yang digunakan dengan cara yang tidak lazim atau terbatas, termasuk idiom dan jargon.

		\paragraph{Kriteria Sukses 3.1.4 \textit{Abbreviations}}
		\label{sec:kriteria_sukses_3.1.4}
		(Level AAA)\\

		Tersedia mekanisme untuk mengidentifikasi kepanjangan dari singkatan.

		\paragraph{Kriteria Sukses 3.1.5 \textit{Reading Level}}
		\label{sec:kriteria_sukses_3.1.5}
		(Level AAA)\\

		Ketika teks yang tersaji cukup kompleks dan membutuhkan kemampuan membaca yang lebih tinggi dari rata-rata, versi konten yang lebih mudah dimengerti haruslah tersedia bagi pengguna.

		\paragraph{Kriteria Sukses 3.1.6 \textit{Pronunciation}}
		\label{sec:kriteria_sukses_3.1.6}
		(Level AAA)\\

		Tersedia mekanisme untuk mengidentifikasi pengucapan suatu kata apabila makna kata tersebut bersifat ambigu ketika cara mengucapkannya tidak diketahui.
		%End of 2.1.3.1 Readable%

		%2.1.3.2 Predictable%
		\subsubsection*{\textit{Predictable}}
		\label{sec:predictable}
		Pastikan halaman situs web tampak dan dapat dioperasikan dengan cara-cara yang mudah ditebak.

		\paragraph{Kriteria Sukses 3.2.1 \textit{On Focus}}
		\label{sec:kriteria_sukses_3.2.1}
		(Level A)\\

		Saat komponen antarmuka pengguna menerima fokus, komponen tersebut tidak menyebabkan perubahan konteks.

		\paragraph{Kriteria Sukses 3.2.2 \textit{On Input}}
		\label{sec:kriteria_sukses_3.2.2}
		(Level A)\\

		Mengubah setelan komponen antarmuka pengguna tidak otomatis menyebabkan perubahan konteks kecuali bila pengguna telah diperingati akan perilaku semacam ini sebelum menggunakan komponen tersebut.

		\paragraph{Kriteria Sukses 3.2.3 \textit{Consistent Navigation}}
		\label{sec:kriteria_sukses_3.2.3}
		(Level AA)\\

		Mekanisme navigasi yang muncul berulang pada tiap halaman web dalam sekumpulan halaman web, muncul dalam urutan relatif yang sama setiap kali tampak, kecuali jika ada perubahan yang dilakukan pengguna.

		\paragraph{Kriteria Sukses 3.2.4 \textit{Consistent Identification}}
		\label{sec:kriteria_sukses_3.2.4}
		(Level AA)\\

		Komponen-komponen yang memiliki fungsionalitas yang sama dalam sekumpulan halaman web diidentifikasikan secara konsisten.

		\paragraph{Kriteria Sukses 3.2.5 \textit{Change on Request}}
		\label{sec:kriteria_sukses_3.2.5}
		(Level AAA)\\

		Perubahan konteks hanya terjadi bila dilakukan oleh pengguna atau ada mekanisme yang tersedia untuk menonaktifkan perubahan tersebut.
		%End of 2.1.3.2 Predictable%

		%2.1.3.3 Input Assistance%
		\subsubsection*{\textit{Input Assistance}}
		\label{sec:input_assistance}
		Bantu pengguna menghindari kesalahan dan mengoreksi kesalahan tersebut.

		\paragraph{Kriteria Sukses 3.3.1 \textit{Error Identification}}
		\label{sec:kriteria_sukses_3.3.1}
		(Level A)\\

		Jika eror masukan terdeteksi otomatis, \textit{item} yang eror harus diidentifikasi dan eror harus dijabarkan kepada pengguna dalam bentuk teks.

		\paragraph{Kriteria Sukses 3.3.2 \textit{Labels or Instructions}}
		\label{sec:kriteria_sukses_3.3.2}
		(Level A)\\

		Label atau instruksi tersedia ketika konten membutuhkan masukan dari pengguna.

		\paragraph{Kriteria Sukses 3.3.3 \textit{Error Suggestion}}
		\label{sec:kriteria_sukses_3.3.3}
		(Level AA)\\

		Jika eror masukan terdeteksi otomatis dan saran untuk mengoreksi eror tersebut diketahui, maka saran disajikan kepada pengguna, kecuali bila saran tersebut akan mengacaukan keamanan atau tujuan dari konten.

		\paragraph{Kriteria Sukses 3.3.4 \textit{Error Prevention (Legal, Financial, Data)\\}}
		\label{sec:kriteria_sukses_3.3.4}
		(Level AA)\\

		Untuk halaman web yang mengirim tanggapan pengguna atau yang menyebabkan terjadinya komitmen hukum atau transaksi keuangan bagi pengguna, setidaknya salah satu dari ketentuan berikut berlaku:
		\begin{itemize}
			\item Bisa dibatalkan: Data yang akan dikirim bisa dibatalkan.
			\item Diperiksa: Data yang dimasukkan oleh pengguna diperiksa apa ada eror masukan atau tidak dan pengguna dipersilakan untuk mengoreksinya.
			\item Dikonfirmasi: Tersedia mekanisme untuk meninjau, mengonfirmasi, dan mengoreksi informasi sebelum informasi tersebut dikirim.
		\end{itemize}

		\paragraph{Kriteria Sukses 3.3.5 \textit{Help}}
		\label{sec:kriteria_sukses_3.3.5}
		(Level AAA)\\

		Tersedia bantuan terkait konteks yang sedang berjalan.

		\paragraph{Kriteria Sukses 3.3.6 \textit{Error Prevention (All)}}
		\label{sec:kriteria_sukses_3.3.6}
		(Level AAA)\\

		Untuk halaman web yang mewajibkan pengguna mengirim informasi, setidaknya salah satu dari ketentuan berikut berlaku:
		\begin{itemize}
			\item Bisa dibatalkan: Data yang akan dikirim bisa dibatalkan.
			\item Diperiksa: Data yang dimasukkan oleh pengguna diperiksa apa ada eror masukan atau tidak dan pengguna dipersilakan untuk mengoreksinya.
			\item Dikonfirmasi: Tersedia mekanisme untuk meninjau, mengonfirmasi, dan mengoreksi informasi sebelum informasi tersebut dikirim.
		\end{itemize}
		%End of 2.1.3.3 Input Assistance%

		%End of 2.1.3 Understandable%

		%2.1.4 Robust%
		\subsection*{\textit{Robust}}
		\label{sec:robust}
		Konten harus cukup andal sehingga dapat ditafsirkan oleh berbagai agen pengguna, termasuk teknologi alat bantu.

		%2.1.4.1 Compatible%
		\subsubsection*{\textit{Compatible}}
		\label{sec:compatible}
		Maksimalkan kompatibilitas dengan agen pengguna saat ini maupun saat yang akan datang, termasuk teknologi alat bantu.

		\paragraph{Kriteria Sukses 4.1.1 \textit{Parsing}}
		\label{sec:kriteria_sukses_4.1.1}
		(Level A)\\

		Pada konten yang diimplementasikan dengan menggunakan bahasa markah, setiap elemen memiliki \textit{tag} awal dan akhir yang lengkap, setiap elemen disusun berlapis sesuai spesifikasi masing-masing, setiap elemen tidak mengandung atribut yang sama dua kali, dan setiap \textit{ID} bersifat unik kecuali jika ada spesifikasi yang mengizinkan.

		\paragraph{Kriteria Sukses 4.1.2 \textit{Name, Role, Value}}
		\label{sec:kriteria_sukses_4.1.2}
		(Level A)\\

		Untuk semua komponen antarmuka pengguna, nama dan peran dapat ditentukan secara terprogram. Keadaan, properti, dan nilai yang dapat ditentukan oleh pengguna juga harus dapat ditentukan secara terprogram. Notifikasi perubahan terhadap \textit{item-item} ini harus tersedia untuk agen pengguna termasuk juga teknologi alat bantu.

		\paragraph{Kriteria Sukses 4.1.3 \textit{Status Messages}}
		\label{sec:kriteria_sukses_4.1.3}
		(Level AA)\\

		Pada konten yang diimplementasikan dengan menggunakan bahasa markah, pesan status dapat ditentukan secara terprogram melalui peran atau sifat sedemikian rupa sehingga dapat disajikan kepada pengguna oleh teknologi alat bantu tanpa perlu menerima fokus.
		%End of 2.1.4.1 Compatible%

		%End of 2.1.4 Robust%

		%End of WCAG 2.1%
		
		\item \textbf{Mengukur tingkat kepatuhan situs web BlueTape terhadap \textit{WCAG} 2.1.}\\
		{\bf Status :} Ada sejak rencana kerja skripsi.\\
		{\bf Hasil :} Sudah dilakukan. Hasil yang didapat dari pengukuran situs web BlueTape terhadap \textit{WCAG} 2.1. adalah sebagai berikut:

		Sukses atau tidaknya aplikasi BlueTape dalam mematuhi kriteria-kriteria sukses dalam \textit{WCAG} 2.1 dituliskan dalam tabel-tabel berikut.
		\label{sec:kepatuhan_bluetape_terhadap_wcag_2.1}
		\begin{table}[H]
			\centering 
			\caption{Tabel kepatuhan BlueTape terhadap prinsip \textit{Perceivable}}
			\label{tab:kepatuhan_bluetape_perceivable}
			\begin{tabular}{|c|c|c|}
				\toprule
				Kriteria Sukses & Hasil (sukses/tidak) & Tingkat Kepatuhan \\

				\midrule
				1.1.1 & Tidak Sukses & A \\
				1.2.1 & Sukses & A \\
				1.2.2 & Sukses & A \\
				1.2.3 & Sukses & A \\
				1.2.4 & Sukses & AA \\
				1.2.5 & Sukses & AA \\
				1.2.6 & Sukses & AAA \\
				1.2.7 & Sukses & AAA \\
				1.2.8 & Sukses & AAA \\
				1.2.9 & Sukses & AAA \\
				1.3.1 & Tidak Sukses & A \\
				1.3.2 & Sukses & A \\
				1.3.3 & Tidak Sukses & A \\
				1.3.4 & Sukses & AA \\
				1.3.5 & Tidak Sukses & AA \\
				1.3.6 & Tidak Sukses & AAA \\
				1.4.1 & Sukses & A \\
				1.4.2 & Sukses & A \\
				1.4.3 & Tidak Sukses & AA \\
				1.4.4 & Sukses & AA \\
				1.4.5 & Sukses & AA \\
				1.4.6 & Tidak Sukses & AAA \\
				1.4.7 & Sukses & AAA \\
				1.4.8 & Tidak Sukses & AAA \\
				1.4.9 & Sukses & AAA \\
				1.4.10 & Tidak Sukses & AA \\
				1.4.11 & Diabaikan & AA \\
				1.4.12 & Sukses & AA \\
				1.4.13 & Sukses & AA \\
				
				\bottomrule
				\multicolumn{2}{|c|}{Tingkat kepatuhan tertinggi yang dicapai} & - \\
				\bottomrule

			\end{tabular}
		\end{table}
		\begin{table}[H]
			\centering 
			\caption{Tabel kepatuhan BlueTape terhadap prinsip \textit{Operable}}
			\label{tab:kepatuhan_bluetape_operable}
			\begin{tabular}{|c|c|c|}
				\toprule
				Kriteria Sukses & Hasil (sukses/tidak) & Tingkat Kepatuhan\\

				\midrule
				2.1.1 & Tidak Sukses & A \\
				2.1.2 & Sukses & A \\
				2.1.3 & Tidak Sukses & AAA \\
				2.1.4 & Sukses & A \\
				2.2.1 & Sukses & A \\
				2.2.2 & Sukses & A \\
				2.2.3 & Sukses & AAA \\
				2.2.4 & Sukses & AAA \\
				2.2.5 & Tidak Sukses & AAA \\
				2.2.6 & Tidak Sukses & AAA \\
				2.3.1 & Sukses & A \\
				2.3.2 & Sukses & AAA \\
				2.3.3 & Sukses & AAA \\
				2.4.1 & Tidak Sukses & A \\
				2.4.2 & Sukses & A \\
				2.4.3 & Sukses & A \\
				2.4.4 & Tidak Sukses & A \\
				2.4.5 & Tidak Sukses & AA \\
				2.4.6 & Tidak Sukses & AA \\
				2.4.7 & Tidak Sukses & AA \\
				2.4.8 & Sukses & AAA \\
				2.4.9 & Tidak Sukses & AAA \\
				2.4.10 & Tidak Sukses & AAA \\
				2.5.1 & Sukses & A \\
				2.5.2 & Sukses & A \\
				2.5.3 & Tidak Sukses & A \\
				2.5.4 & Sukses & A \\
				2.5.5 & Tidak Sukses & AAA \\
				2.5.6 & Sukses & AAA \\

				\bottomrule
				\multicolumn{2}{|c|}{Tingkat kepatuhan tertinggi yang dicapai} & - \\
				\bottomrule

			\end{tabular}
		\end{table}

		\begin{table}[H]
			\centering 
			\caption{Tabel kepatuhan BlueTape terhadap prinsip \textit{Understandable}}
			\label{tab:kepatuhan_bluetape_understandable}
			\begin{tabular}{|c|c|c|}
				\toprule
				Kriteria Sukses & Hasil (sukses/tidak) & Tingkat Kepatuhan \\

				\midrule
				3.1.1 & Tidak Sukses & A \\
				3.1.2 & Tidak Sukses & AA \\
				3.1.3 & Tidak Sukses & AAA \\
				3.1.4 & Tidak Sukses & AAA \\
				3.1.5 & Sukses & AAA \\
				3.1.6 & Sukses & AAA \\
				3.2.1 & Sukses & A \\
				3.2.2 & Sukses & A \\
				3.2.3 & Sukses & AA \\
				3.2.4 & Tidak Sukses & AA \\
				3.2.5 & Sukses & AAA \\
				3.3.1 & Sukses & A \\
				3.3.2 & Tidak Sukses & A \\
				3.3.3 & Sukses & AA \\
				3.3.4 & Sukses & AA \\
				3.3.5 & Tidak Sukses & AAA \\
				3.3.6 & Sukses & AAA \\

				\bottomrule
				\multicolumn{2}{|c|}{Tingkat kepatuhan tertinggi yang dicapai} & - \\
				\bottomrule

			\end{tabular}
		\end{table}
		\begin{table}[H]
			\centering 
			\caption{Tabel kepatuhan BlueTape terhadap prinsip \textit{Robust}}
			\label{tab:kepatuhan_bluetape_robust}
			\begin{tabular}{|c|c|c|}
				\toprule
				Kriteria Sukses & Hasil (sukses/tidak) & Tingkat Kepatuhan\\

				\midrule
				4.1.1 & Tidak Sukses & A \\
				4.1.2 & Tidak Sukses & A \\
				4.1.3 & Tidak Sukses & AA \\

				\bottomrule
				\multicolumn{2}{|c|}{Tingkat kepatuhan tertinggi yang dicapai} & - \\
				\bottomrule

			\end{tabular} 
		\end{table}

		\subsection*{\textit{Perceivable}}
		\label{subsec:kepatuhan_bluetape_perceivable}

		\subsubsection*{\textit{Text Alternatives}}
		\label{subsubsec:kepatuhan_bluetape_text_alternatives}

		\paragraph{Kriteria Sukses 1.1.1 \textit{Non-text Content}}
		\label{par:kepatuhan_bluetape_kriteria_sukses_1.1.1}
		(Tidak Sukses)\\

		Kriteria ini tidak sukses dipatuhi karena pada tampilan \textit{mobile}, di setiap halaman selain halaman \textit{login} terdapat tombol yang berperan sebagai kontrol namun tidak memiliki nama yang menjelaskan tujuannya. Tombol yang dimaksud hanya menampilkan tiga garis horizontal berwarna putih dan berfungsi untuk menampilkan bagian menu ketika ditekan. Bagian yang bermasalah dapat dilihat pada potongan kode \ref{lst:1.1.1_tombol_tanpa_nama}. Tampilan pada halaman web dapat dilihat pada gambar \ref{fig:1.1.1_non_text_content}. Contoh tautan untuk halaman yang bermasalah dapat dilihat di \url{https://bluetape.azurewebsites.net/TranskripRequest}.

		\begin{lstlisting}[frame=single, label={lst:1.1.1_tombol_tanpa_nama}, language=HTML, caption=Kriteria Sukses 1.1.1 - Tombol Tanpa Nama]
			<div class="title-bar" data-responsive-toggle="navigation-menu" data-hide-for="medium">
				<button class="menu-icon" type="button" data-toggle></button>
				<div class="title-bar-title"><img src="/public/img/logo.png" class="textsized" alt="B"/></div>
			</div>
		\end{lstlisting}

		\begin{figure}[H]
			\centering  
			\includegraphics[scale=0.3, frame]{kriteria-sukses-1-1-1-non-text-content}  
			\caption[Kriteria Sukses 1.1.1 - Tombol Tanpa Nama]{Kriteria Sukses 1.1.1 - Tombol Tanpa Nama}
			\label{fig:1.1.1_non_text_content}  
		\end{figure} 

		\subsubsection*{\textit{Time-based Media}}
		\label{subsubsec:kepatuhan_bluetape_time_based_media}

		\paragraph{Kriteria Sukses 1.2.1 \textit{Audio-only and Video-only (Prerecorded)}}
		\label{par:kepatuhan_bluetape_kriteria_sukses_1.2.1}
		(Sukses)\\

		Kriteria ini sukses dipatuhi karena pada halaman web BlueTape tidak terdapat konten media berbasis waktu.

		\paragraph{Kriteria Sukses 1.2.2 \textit{Captions (Prerecorded)}}
		\label{par:kepatuhan_bluetape_kriteria_sukses_1.2.2}
		(Sukses)\\

		Kriteria ini sukses dipatuhi karena pada halaman web BlueTape tidak terdapat konten media berbasis waktu.

		\paragraph{Kriteria Sukses 1.2.3 \textit{Audio Description or Media Alternative (Prerecorded)}}
		\label{par:kepatuhan_bluetape_kriteria_sukses_1.2.3}
		(Sukses)\\

		Kriteria ini sukses dipatuhi karena pada halaman web BlueTape tidak terdapat konten media berbasis waktu.

		\paragraph{Kriteria Sukses 1.2.4 \textit{Captions (Live)}}
		\label{par:kepatuhan_bluetape_kriteria_sukses_1.2.4}
		(Sukses)\\

		Kriteria ini sukses dipatuhi karena pada halaman web BlueTape tidak terdapat konten media berbasis waktu.

		\paragraph{Kriteria Sukses 1.2.5 \textit{Audio Description (Prerecorded)}}
		\label{par:kepatuhan_bluetape_kriteria_sukses_1.2.5}
		(Sukses)\\

		Kriteria ini sukses dipatuhi karena pada halaman web BlueTape tidak terdapat konten media berbasis waktu.

		\paragraph{Kriteria Sukses 1.2.6 \textit{Sign Language (Prerecorded)}}
		\label{par:kepatuhan_bluetape_kriteria_sukses_1.2.6}
		(Sukses)\\

		Kriteria ini sukses dipatuhi karena pada halaman web BlueTape tidak terdapat konten media berbasis waktu.

		\paragraph{Kriteria Sukses 1.2.7 \textit{Extended Audio Description (Prerecorded)}}
		\label{par:kepatuhan_bluetape_kriteria_sukses_1.2.7}
		(Sukses)\\

		Kriteria ini sukses dipatuhi karena pada halaman web BlueTape tidak terdapat konten media berbasis waktu.

		\paragraph{Kriteria Sukses 1.2.8 \textit{Media Alternative (Prerecorded)}}
		\label{par:kepatuhan_bluetape_kriteria_sukses_1.2.8}
		(Sukses)\\

		Kriteria ini sukses dipatuhi karena pada halaman web BlueTape tidak terdapat konten media berbasis waktu.

		\paragraph{Kriteria Sukses 1.2.9 \textit{Audio-only (Live)}}
		\label{par:kepatuhan_bluetape_kriteria_sukses_1.2.9}
		(Sukses)\\

		Kriteria ini sukses dipatuhi karena pada halaman web BlueTape tidak terdapat konten media berbasis waktu.

		\subsubsection*{\textit{Adaptable}}
		\label{subsubsec:kepatuhan_bluetape_adaptable}

		\paragraph{Kriteria Sukses 1.3.1 \textit{Info and Relationships}}
		\label{par:kepatuhan_bluetape_kriteria_sukses_1.3.1}
		(Tidak Sukses)\\

		Kriteria ini tidak sukses dipatuhi karena:
		\begin{itemize}
			\item Terdapat penggunaan \textit{tag heading} yang tidak tepat secara struktur pada halaman cetak transkrip, manajemen cetak transkrip, perubahan kuliah, manajemen perubahan kuliah, dan entri jadwal dosen. Contoh kesalahan dapat dilihat pada potongan kode \ref{lst:1.3.1_heading_tidak_tepat} yang menampilkan kesalahan penggunaan \textit{tag heading} pada halaman cetak transkrip. Contoh tautan untuk halaman yang bermasalah dapat dilihat di \url{https://bluetape.azurewebsites.net/TranskripRequest}.
			\begin{lstlisting}[frame=single, label={lst:1.3.1_heading_tidak_tepat}, language=HTML, caption=Kriteria Sukses 1.3.1 - Penggunaan \textit{Heading} Tidak Tepat]
				<div class="callout">
					<h5>Permohonan Baru</h5>
					<form method="POST" action="/TranskripRequest/add">
			\end{lstlisting}

			\item Pada halaman manajemen cetak transkrip, kolom masukan NPM tidak memiliki label atau \textit{aria-label} yang bersangkutan dengan kolom masukan tersebut. Kesalahan dapat dilihat pada potongan kode \ref{lst:1.3.1_label_masukan_manajemen_cetak_transkrip}. Tautan untuk halaman yang bermasalah dapat dilihat di \url{https://bluetape.azurewebsites.net/TranskripManage}.
			\begin{lstlisting}[frame=single, label={lst:1.3.1_label_masukan_manajemen_cetak_transkrip}, language=HTML, caption=Kriteria Sukses 1.3.1 - Tidak Terdapat Label pada Kolom Masukan di Halaman Manajemen Cetak Transkrip]
				<span class="input-group-label">Cari NPM:</span>
				<input name="npm" class="input-group-field" type="text" placeholder="2013730013" maxlength="10" minlength="10"/>
				<div class="input-group-button">
			\end{lstlisting}

			\item Pada halaman entri jadwal dosen, setiap kolom masukkan tidak memiliki label atau \textit{aria-label} yang bersangkutan dengan kolom-kolom masukan tersebut. Contoh kesalahan dapat dilihat pada potongan kode \ref{lst:1.3.1_label_masukan_entri_jadwal_dosen}. Tautan untuk halaman yang bermasalah dapat dilihat di \url{https://bluetape.azurewebsites.net/EntriJadwalDosen}.
			\begin{lstlisting}[frame=single, label={lst:1.3.1_label_masukan_entri_jadwal_dosen}, language=HTML, caption=Kriteria Sukses 1.3.1 - Tidak Terdapat Label pada Kolom Masukan di Halaman Entri Jadwal Dosen]
				<input type="hidden" name="csrf_token" value="3c159eae7bc953dd591b679c080ed066"/>
				Hari
				<select name="hari">
			\end{lstlisting}
		\end{itemize} 

		\paragraph{Kriteria Sukses 1.3.2 \textit{Meaningful Sequence}}
		\label{par:kepatuhan_bluetape_kriteria_sukses_1.3.2}
		(Sukses)\\

		Kriteria ini sukses dipatuhi karena pada setiap halaman web BlueTape sudah memiiliki urutan membaca yang benar dan dapat ditentukan secara terprogram. 

		\paragraph{Kriteria Sukses 1.3.3 \textit{Sensory Characteristics}}
		\label{par:kepatuhan_bluetape_kriteria_sukses_1.3.3}
		(Tidak Sukses)\\

		Kriteria ini tidak sukses dipatuhi karena pada halaman cetak transkrip, manajemen cetak transkrip, perubahan kuliah, dan manajemen perubahan kuliah terdapat satu atau lebih komponen untuk mengoperasikan konten yang hanya mengandalkan satu komponen karakteristik indra yaitu bentuk. Komponen-komponen ini terletak pada kolom "Aksi" dan hanya ditampilkan dalam rupa ikon tanpa memiliki keterangan lebih detail. Contoh kesalahan dapat dilihat pada potongan kode \ref{lst:1.3.3_ikon_tanpa_keterangan}. Contoh tampilan pada halaman web dapat dilihat pada gambar \ref{fig:1.3.3_sensory_characteristics}. Contoh tautan untuk halaman yang bermasalah dapat dilihat di \url{https://bluetape.azurewebsites.net/PerubahanKuliahManage}.

		\begin{lstlisting}[frame=single, label={lst:1.3.3_ikon_tanpa_keterangan}, language=HTML, caption=Kriteria Sukses 1.3.3 - Ikon Tanpa Keterangan]
			<td>
				<a data-open="detail12">
					<i class="fi-eye"></i>
				</a>
				<a target="_blank" href="/PerubahanKuliahManage/printview/12">
					<i class="fi-print"></i>
				</a>
				<a data-open="konfirmasi12">
					<i class="fi-like"></i>
				</a>  
				<a data-open="tolak12">
					<i class="fi-dislike"></i>
				</a>
				<a data-open="hapus12">
					<i class="fi-trash"></i>
				</a>
			</td>
		\end{lstlisting}

		\begin{figure}[H]
			\centering  
			\includegraphics[scale=0.3, frame]{kriteria-sukses-1-3-3-sensory-characteristics}  
			\caption[Kriteria Sukses 1.3.3 - Ikon pada Bagian Aksi]{Kriteria Sukses 1.3.3 - Ikon pada Bagian Aksi}
			\label{fig:1.3.3_sensory_characteristics}  
		\end{figure} 

		\paragraph{Kriteria Sukses 1.3.4 \textit{Orientation}}
		\label{par:kepatuhan_bluetape_kriteria_sukses_1.3.4}
		(Sukses)\\

		Kriteria ini sukses dipatuhi karena konten pada halaman web BlueTape dapat disajikan dalam orientasi \textit{portrait} maupun \textit{landscape}.

		\paragraph{Kriteria Sukses 1.3.5 \textit{Identify Input Purpose}}
		\label{par:kepatuhan_bluetape_kriteria_sukses_1.3.5}
		(Tidak Sukses)\\

		Kriteria ini tidak sukses dipatuhi karena terdapat kolom-kolom masukan yang tidak memiliki label yang terasosiasi dengan kolom-kolom tersebut, meskipun setiap kolom masukan sudah memiliki atribut \textit{"autocomplete"} yang aktif. Bagian-bagian yang bermasalah, antara lain:
		\begin{itemize}
			\item Pada halaman manajemen cetak transkrip, kolom masukan NPM tidak memiliki label atau \textit{aria-label} yang bersangkutan dengan kolom masukan tersebut. Kesalahan dapat dilihat pada potongan kode \ref{lst:1.3.5_label_masukan_manajemen_cetak_transkrip}. Tautan untuk halaman yang bermasalah dapat dilihat di \url{https://bluetape.azurewebsites.net/TranskripManage}.
			\begin{lstlisting}[frame=single, label={lst:1.3.5_label_masukan_manajemen_cetak_transkrip}, language=HTML, caption=Kriteria Sukses 1.3.5 - Tidak Terdapat Label pada Kolom Masukan di Halaman Manajemen Cetak Transkrip]
				<span class="input-group-label">Cari NPM:</span>
				<input name="npm" class="input-group-field" type="text" placeholder="2013730013" maxlength="10" minlength="10"/>
				<div class="input-group-button">
			\end{lstlisting}
			
			\item Pada halaman entri jadwal dosen, setiap kolom masukkan tidak memiliki label atau \textit{aria-label} yang bersangkutan dengan kolom-kolom masukan tersebut. Contoh kesalahan dapat dilihat pada potongan kode \ref{lst:1.3.5_label_masukan_entri_jadwal_dosen}. Tautan untuk halaman yang bermasalah dapat dilihat di \url{https://bluetape.azurewebsites.net/EntriJadwalDosen}.
			\begin{lstlisting}[frame=single, label={lst:1.3.5_label_masukan_entri_jadwal_dosen}, language=HTML, caption=Kriteria Sukses 1.3.5 - Tidak Terdapat Label pada Kolom Masukan di Halaman Entri Jadwal Dosen]
				<input type="hidden" name="csrf_token" value="3c159eae7bc953dd591b679c080ed066" />
				Hari
				<select name="hari">
			\end{lstlisting}
		\end{itemize}

		\paragraph{Kriteria Sukses 1.3.6 \textit{Identify Purpose}}
		\label{par:kepatuhan_bluetape_kriteria_sukses_1.3.6}
		(Tidak Sukses)\\

		Kriteria ini tidak sukses dipatuhi karena terdapat elemen \textit{HTML}5 yang seharusnya dapat digunakan namun tidak digunakan, contohnya pada bagian navigasi. Kesalahan dapat dilihat pada potongan kode \ref{lst:1.3.6_navigasi}.

		\begin{lstlisting}[frame=single, label={lst:1.3.6_navigasi}, language=HTML, caption=Kriteria Sukses 1.3.6 - Navigasi]
			</div>
			<div class="top-bar" id="navigation-menu">
				<div class="top-bar-left">
		\end{lstlisting}

		\subsubsection*{\textit{Distinguishable}}
		\label{subsubsec:kepatuhan_bluetape_distinguishable}

		\paragraph{Kriteria Sukses 1.4.1 \textit{Use of Color}}
		\label{par:kepatuhan_bluetape_kriteria_sukses_1.4.1}
		(Sukses)\\

		Kriteria ini sukses dipatuhi karena pada halaman web BlueTape, warna tidak digunakan sebagai satu-satunya cara untuk menyampaikan informasi secara visual, menandai suatu tindakan, meminta respons, atau membedakan elemen visual.

		\paragraph{Kriteria Sukses 1.4.2 \textit{Audio Control}}
		\label{par:kepatuhan_bluetape_kriteria_sukses_1.4.2}
		(Sukses)\\

		Kriteria ini sukses dipatuhi karena pada halaman web BlueTape tidak terdapat konten media berbasis waktu.

		\paragraph{Kriteria Sukses 1.4.3 \textit{Contrast (Minimum)}}
		\label{par:kepatuhan_bluetape_kriteria_sukses_1.4.3}
		(Tidak Sukses)\\

		Kriteria ini tidak sukses dipatuhi karena pada halaman web BlueTape terdapat beberapa teks dengan rasio kontras kurang dari 4,5:1 untuk teks yang berukuran kurang dari 24 piksel dan tidak \textit{bold}, antara lain:

		\begin{itemize}
			\item Halaman \textit{login}: Teks "\textit{Login} dengan Google" memiliki rasio kontras 3.09:1 terhadap warna latar belakangnya. Teks "Petunjuk Penggunaan" memiliki rasio kontras 3.06:1 terhadap warna latar belakangnya. Tampilan pada halaman web dapat dilihat pada gambar \ref{fig:1.4.3_contrast_minimum_1}. Tautan untuk halaman yang bermasalah dapat dilihat di \url{https://bluetape.azurewebsites.net}.
			\begin{figure}[H]
				\centering  
				\includegraphics[scale=0.3, frame]{kriteria-sukses-1-4-3-contrast-minimum-1}  
				\caption[Kriteria Sukses 1.4.3 - Kontras Elemen pada Halaman \textit{login}]{Kriteria Sukses 1.4.3 - Kontras Elemen pada Halaman \textit{login}}
				\label{fig:1.4.3_contrast_minimum_1}  
			\end{figure} 
			
			\item Halaman cetak transkrip: Teks "Kirim Permohonan" memiliki rasio kontras 3.09:1 terhadap warna latar belakangnya. Teks "Tercetak" di kolom status pada tabel memiliki rasio kontras 1,79:1 terhadap warna latar belakangnya. Teks "Ditolak" di kolom status pada tabel "Histori Permohonan" memiliki rasio kontras 3,44:1 terhadap warna latar belakangnya. Teks "Tunggu" di kolom status pada tabel "Histori Permohonan" memiliki rasio kontras 4,44:1 terhadap warna latar belakangnya. Tampilan pada halaman web dapat dilihat pada gambar \ref{fig:1.4.3_contrast_minimum_2}. Tautan untuk halaman yang bermasalah dapat dilihat di \url{https://bluetape.azurewebsites.net/TranskripRequest}.
			\begin{figure}[H]
				\centering  
				\includegraphics[scale=0.3, frame]{kriteria-sukses-1-4-3-contrast-minimum-2}  
				\caption[Kriteria Sukses 1.4.3 - Kontras Elemen pada Halaman Cetak Transkrip]{Kriteria Sukses 1.4.3 - Kontras Elemen pada Halaman Cetak Transkrip}
				\label{fig:1.4.3_contrast_minimum_2}  
			\end{figure} 
			
			\item Halaman manajemen cetak transkrip: Teks "Cari" memiliki rasio kontras 3.09:1 terhadap warna latar belakangnya. Teks "Tercetak" di kolom status pada tabel memiliki rasio kontras 1,79:1 terhadap warna latar belakangnya. Teks "Menunggu" di kolom status pada tabel memiliki rasio kontras 1,84:1 terhadap warna latar belakangnya. Teks "Ditolak" di kolom status pada tabel memiliki rasio kontras 3,44:1 terhadap warna latar belakangnya. Teks "Hapus" pada bagian hapus permohonan memiliki rasio kontras 3,47:1 terhadap warna latar belakangnya. Tampilan pada halaman web dapat dilihat pada gambar \ref{fig:1.4.3_contrast_minimum_3}. Tautan untuk halaman yang bermasalah dapat dilihat di \url{https://bluetape.azurewebsites.net/TranskripManage}.
			\begin{figure}[H]
				\centering  
				\includegraphics[scale=0.3, frame]{kriteria-sukses-1-4-3-contrast-minimum-3}  
				\caption[Kriteria Sukses 1.4.3 - Kontras Elemen pada Halaman Manajemen Cetak Transkrip]{Kriteria Sukses 1.4.3 - Kontras Elemen pada Halaman Manajemen Cetak Transkrip}
				\label{fig:1.4.3_contrast_minimum_3}  
			\end{figure} 
			
			\item Halaman perubahan kuliah: Teks "Kirim Permohonan" memiliki rasio kontras 3.09:1 terhadap warna latar belakangnya. Teks "Tambah Pertemuan Ekstra" memiliki rasio kontras 4,47:1 terhadap warna latar belakangnya. Teks "Hapus" memiliki rasio kontras 4,47:1 terhadap warna latar belakangnya. Teks "Tunggu" di kolom status pada tabel "Histori Permohonan" memiliki rasio kontras 4,44:1 terhadap warna latar belakangnya. Teks "Ditolak" di kolom status pada tabel "Histori Permohonan" memiliki rasio kontras 3,44:1 terhadap warna latar belakangnya. Teks "Terkonfirmasi" di kolom status pada tabel "Histori Permohonan" memiliki rasio kontras 1,79:1 terhadap warna latar belakangnya. Tampilan pada halaman web dapat dilihat pada gambar \ref{fig:1.4.3_contrast_minimum_4}. Tautan untuk halaman yang bermasalah dapat dilihat di \url{https://bluetape.azurewebsites.net/PerubahanKuliahRequest}.
			\begin{figure}[H]
				\centering  
				\includegraphics[scale=0.3, frame]{kriteria-sukses-1-4-3-contrast-minimum-4}  
				\caption[Kriteria Sukses 1.4.3 - Kontras Elemen pada Halaman Perubahan Kuliah]{Kriteria Sukses 1.4.3 - Kontras Elemen pada Halaman Perubahan Kuliah}
				\label{fig:1.4.3_contrast_minimum_4}  
			\end{figure} 
			
			\item Halaman manajemen perubahan kuliah: Teks "Terkonfirmasi" di kolom status pada tabel memiliki rasio kontras 1,79:1 terhadap warna latar belakangnya. Teks "Menunggu" di kolom status pada tabel memiliki rasio kontras 1,84:1 terhadap warna latar belakangnya. Teks "Ditolak" di kolom status pada tabel memiliki rasio kontras 3,44:1 terhadap warna latar belakangnya. Teks "Hapus" pada bagian hapus permohonan memiliki rasio kontras 3,47:1 terhadap warna latar belakangnya. Tampilan pada halaman web dapat dilihat pada gambar \ref{fig:1.4.3_contrast_minimum_5}. Tautan untuk halaman yang bermasalah dapat dilihat di \url{https://bluetape.azurewebsites.net/PerubahanKuliahManage}.
			\begin{figure}[H]
				\centering  
				\includegraphics[scale=0.3, frame]{kriteria-sukses-1-4-3-contrast-minimum-5}  
				\caption[Kriteria Sukses 1.4.3 - Kontras Elemen pada Halaman Manajemen Perubahan Kuliah]{Kriteria Sukses 1.4.3 - Kontras Elemen pada Halaman Manajemen Perubahan Kuliah}
				\label{fig:1.4.3_contrast_minimum_5}  
			\end{figure} 

			\item Halaman entri jadwal dosen: Teks "Tambah" memiliki rasio kontras 3.09:1 terhadap warna latar belakangnya. Teks "Delete All" memiliki rasio kontras 3.47:1 terhadap warna latar belakangnya. Teks "Ekspor ke XLS" memiliki rasio kontras 3.09:1 terhadap warna latar belakangnya. Teks "Save" pada bagian "Edit Jadwal" memiliki rasio kontras 3.09:1 terhadap warna latar belakangnya. Teks "Delete" memiliki rasio kontras 3.47:1 terhadap warna latar belakangnya. Tampilan pada halaman web dapat dilihat pada gambar \ref{fig:1.4.3_contrast_minimum_6}. Tautan untuk halaman yang bermasalah dapat dilihat di \url{https://bluetape.azurewebsites.net/EntriJadwalDosen}.	
			\begin{figure}[H]
				\centering  
				\includegraphics[scale=0.3, frame]{kriteria-sukses-1-4-3-contrast-minimum-6}  
				\caption[Kriteria Sukses 1.4.3 - Kontras Elemen pada Halaman Entri Jadwal Dosen]{Kriteria Sukses 1.4.3 - Kontras Elemen pada Halaman Entri Jadwal Dosen}
				\label{fig:1.4.3_contrast_minimum_6}  
			\end{figure} 
			
			\item Halaman lihat jadwal dosen: Teks nama dosen di atas tabel yang sedang dipilih pengguna memiliki rasio kontras 2,47:1 terhadap warna latar belakangnya. Teks nama dosen di atas tabel yang sedang tidak dipilih pengguna memiliki rasio kontras 3,06:1 terhadap warna latar belakangnya. Teks "Ekspor ke XLS" memiliki rasio kontras 3.09:1 terhadap warna latar belakangnya. Tampilan pada halaman web dapat dilihat pada gambar \ref{fig:1.4.3_contrast_minimum_7}. Tautan untuk halaman yang bermasalah dapat dilihat di \url{https://bluetape.azurewebsites.net/LihatJadwalDosen}.
			\begin{figure}[H]
				\centering  
				\includegraphics[scale=0.3, frame]{kriteria-sukses-1-4-3-contrast-minimum-7}  
				\caption[Kriteria Sukses 1.4.3 - Kontras Elemen pada Halaman Lihat Jadwal Dosen]{Kriteria Sukses 1.4.3 - Kontras Elemen pada Halaman Lihat Jadwal Dosen}
				\label{fig:1.4.3_contrast_minimum_7}  
			\end{figure} 
		\end{itemize}

		\paragraph{Kriteria Sukses 1.4.4 \textit{Resize text}}
		\label{par:kepatuhan_bluetape_kriteria_sukses_1.4.4}
		(Sukses)\\

		Kriteria ini sukses dipatuhi karena setiap halaman web BlueTape dapat tetap terbaca dan tidak kehilangan fungsionalitasnya ketika halaman diperbesar hingga 200 persen. 

		\paragraph{Kriteria Sukses 1.4.5 \textit{Images of Text}}
		\label{par:kepatuhan_bluetape_kriteria_sukses_1.4.5}
		(Sukses)\\

		Kriteria ini sukses dipatuhi karena pada halaman web BlueTape tidak terdapat gambar teks selain logo.

		\paragraph{Kriteria Sukses 1.4.6 \textit{Contrast (Enhanced)}}
		\label{par:kepatuhan_bluetape_kriteria_sukses_1.4.6}
		(Tidak Sukses)\\

		Kriteria ini tidak sukses dipatuhi karena pada halaman web BlueTape terdapat beberapa teks dengan rasio kontras kurang dari 7:1 untuk teks yang berukuran kurang dari 24 piksel dan tidak \textit{bold}, antara lain:

		\begin{itemize}
			\item Halaman \textit{login}: Teks "\textit{Login} dengan Google" memiliki rasio kontras 3.09:1 terhadap warna latar belakangnya. Teks "Petunjuk Penggunaan" memiliki rasio kontras 3.06:1 terhadap warna latar belakangnya. Tampilan pada halaman web dapat dilihat pada gambar \ref{fig:1.4.6_contrast_enchanced_1}. Tautan untuk halaman yang bermasalah dapat dilihat di \url{https://bluetape.azurewebsites.net}.
			\begin{figure}[H]
				\centering  
				\includegraphics[scale=0.3, frame]{kriteria-sukses-1-4-6-contrast-enchanced-1}  
				\caption[Kriteria Sukses 1.4.6 - Kontras Elemen pada Halaman \textit{login}]{Kriteria Sukses 1.4.6 - Kontras Elemen pada Halaman \textit{login}}
				\label{fig:1.4.6_contrast_enchanced_1}  
			\end{figure} 

			\item Halaman cetak transkrip: Teks "Kirim Permohonan" memiliki rasio kontras 3.09:1 terhadap warna latar belakangnya. Teks "Tercetak" di kolom status pada tabel memiliki rasio kontras 1,79:1 terhadap warna latar belakangnya. Teks "Ditolak" di kolom status pada tabel "Histori Permohonan" memiliki rasio kontras 3,44:1 terhadap warna latar belakangnya. Teks "Tunggu" di kolom status pada tabel "Histori Permohonan" memiliki rasio kontras 4,44:1 terhadap warna latar belakangnya. Tampilan pada halaman web dapat dilihat pada gambar \ref{fig:1.4.6_contrast_enchanced_2}. Tautan untuk halaman yang bermasalah dapat dilihat di \url{https://bluetape.azurewebsites.net/TranskripRequest}.
			\begin{figure}[H]
				\centering  
				\includegraphics[scale=0.3, frame]{kriteria-sukses-1-4-6-contrast-enchanced-2}  
				\caption[Kriteria Sukses 1.4.6 - Kontras Elemen pada Halaman Cetak Transkrip]{Kriteria Sukses 1.4.6 - Kontras Elemen pada Halaman Cetak Transkrip}
				\label{fig:1.4.6_contrast_enchanced_2}  
			\end{figure} 
			
			\item Halaman manajemen cetak transkrip: Teks "Cari" memiliki rasio kontras 3.09:1 terhadap warna latar belakangnya. Teks "Tercetak" di kolom status pada tabel memiliki rasio kontras 1,79:1 terhadap warna latar belakangnya. Teks "Menunggu" di kolom status pada tabel memiliki rasio kontras 1,84:1 terhadap warna latar belakangnya. Teks "Ditolak" di kolom status pada tabel memiliki rasio kontras 3,44:1 terhadap warna latar belakangnya. Teks "Hapus" pada bagian hapus permohonan memiliki rasio kontras 3,47:1 terhadap warna latar belakangnya. Tampilan pada halaman web dapat dilihat pada gambar \ref{fig:1.4.6_contrast_enchanced_3}. Tautan untuk halaman yang bermasalah dapat dilihat di \url{https://bluetape.azurewebsites.net/TranskripManage}.
			\begin{figure}[H]
				\centering  
				\includegraphics[scale=0.3, frame]{kriteria-sukses-1-4-6-contrast-enchanced-3}  
				\caption[Kriteria Sukses 1.4.6 - Kontras Elemen pada Halaman Manajemen Cetak Transkrip]{Kriteria Sukses 1.4.6 - Kontras Elemen pada Halaman Manajemen Cetak Transkrip}
				\label{fig:1.4.6_contrast_enchanced_3}  
			\end{figure} 
			
			\item Halaman perubahan kuliah: Teks "Kirim Permohonan" memiliki rasio kontras 3.09:1 terhadap warna latar belakangnya. Teks "Tambah Pertemuan Ekstra" memiliki rasio kontras 4,47:1 terhadap warna latar belakangnya. Teks "Hapus" memiliki rasio kontras 4,47:1 terhadap warna latar belakangnya. Teks "Tunggu" di kolom status pada tabel "Histori Permohonan" memiliki rasio kontras 4,44:1 terhadap warna latar belakangnya. Teks "Ditolak" di kolom status pada tabel "Histori Permohonan" memiliki rasio kontras 3,44:1 terhadap warna latar belakangnya. Teks "Terkonfirmasi" di kolom status pada tabel "Histori Permohonan" memiliki rasio kontras 1,79:1 terhadap warna latar belakangnya. Tampilan pada halaman web dapat dilihat pada gambar \ref{fig:1.4.6_contrast_enchanced_4}. Tautan untuk halaman yang bermasalah dapat dilihat di \url{https://bluetape.azurewebsites.net/PerubahanKuliahRequest}.
			\begin{figure}[H]
				\centering  
				\includegraphics[scale=0.3, frame]{kriteria-sukses-1-4-6-contrast-enchanced-4}  
				\caption[Kriteria Sukses 1.4.6 - Kontras Elemen pada Halaman Perubahan Kuliah]{Kriteria Sukses 1.4.6 - Kontras Elemen pada Halaman Perubahan Kuliah}
				\label{fig:1.4.6_contrast_enchanced_4}  
			\end{figure} 
			
			\item Halaman manajemen perubahan kuliah: Teks "Terkonfirmasi" di kolom status pada tabel memiliki rasio kontras 1,79:1 terhadap warna latar belakangnya. Teks "Menunggu" di kolom status pada tabel memiliki rasio kontras 1,84:1 terhadap warna latar belakangnya. Teks "Ditolak" di kolom status pada tabel memiliki rasio kontras 3,44:1 terhadap warna latar belakangnya. Teks "Hapus" pada bagian hapus permohonan memiliki rasio kontras 3,47:1 terhadap warna latar belakangnya. Tampilan pada halaman web dapat dilihat pada gambar \ref{fig:1.4.6_contrast_enchanced_5}. Tautan untuk halaman yang bermasalah dapat dilihat di \url{https://bluetape.azurewebsites.net/PerubahanKuliahManage}.
			\begin{figure}[H]
				\centering  
				\includegraphics[scale=0.3, frame]{kriteria-sukses-1-4-6-contrast-enchanced-5}  
				\caption[Kriteria Sukses 1.4.6 - Kontras Elemen pada Halaman Manajemen Perubahan Kuliah]{Kriteria Sukses 1.4.6 - Kontras Elemen pada Halaman Manajemen Perubahan Kuliah}
				\label{fig:1.4.6_contrast_enchanced_5}  
			\end{figure} 
			
			\item Halaman entri jadwal dosen: Teks "Tambah" memiliki rasio kontras 3.09:1 terhadap warna latar belakangnya. Teks "Delete All" memiliki rasio kontras 3.47:1 terhadap warna latar belakangnya. Teks "Ekspor ke XLS" memiliki rasio kontras 3.09:1 terhadap warna latar belakangnya. Teks "Save" pada bagian "Edit Jadwal" memiliki rasio kontras 3.09:1 terhadap warna latar belakangnya. Teks "Delete" memiliki rasio kontras 3.47:1 terhadap warna latar belakangnya. Tampilan pada halaman web dapat dilihat pada gambar \ref{fig:1.4.6_contrast_enchanced_6}. Tautan untuk halaman yang bermasalah dapat dilihat di \url{https://bluetape.azurewebsites.net/EntriJadwalDosen}.
			\begin{figure}[H]
				\centering  
				\includegraphics[scale=0.3, frame]{kriteria-sukses-1-4-6-contrast-enchanced-6}  
				\caption[Kriteria Sukses 1.4.6 - Kontras Elemen pada Halaman Entri Jadwal Dosen]{Kriteria Sukses 1.4.6 - Kontras Elemen pada Halaman Entri Jadwal Dosen}
				\label{fig:1.4.6_contrast_enchanced_6}  
			\end{figure} 
			
			\item Halaman lihat jadwal dosen: Teks nama dosen di atas tabel yang sedang dipilih pengguna memiliki rasio kontras 2,47:1 terhadap warna latar belakangnya. Teks nama dosen di atas tabel yang sedang tidak dipilih pengguna memiliki rasio kontras 3,06:1 terhadap warna latar belakangnya. Teks "Ekspor ke XLS" memiliki rasio kontras 3.09:1 terhadap warna latar belakangnya. Tampilan pada halaman web dapat dilihat pada gambar \ref{fig:1.4.6_contrast_enchanced_7}. Tautan untuk halaman yang bermasalah dapat dilihat di \url{https://bluetape.azurewebsites.net/LihatJadwalDosen}.
			\begin{figure}[H]
				\centering  
				\includegraphics[scale=0.3, frame]{kriteria-sukses-1-4-6-contrast-enchanced-7}  
				\caption[Kriteria Sukses 1.4.6 - Kontras Elemen pada Halaman Lihat Jadwal Dosen]{Kriteria Sukses 1.4.6 - Kontras Elemen pada Halaman Lihat Jadwal Dosen}
				\label{fig:1.4.6_contrast_enchanced_7}  
			\end{figure} 
		\end{itemize}

		\paragraph{Kriteria Sukses 1.4.7 \textit{Low or No Background Audio}}
		\label{par:kepatuhan_bluetape_kriteria_sukses_1.4.7}
		(Sukses)\\

		Kriteria ini sukses dipatuhi karena pada halaman web BlueTape tidak terdapat konten media berbasis waktu.

		\paragraph{Kriteria Sukses 1.4.8 \textit{Visual Presentation}}
		\label{par:kepatuhan_bluetape_kriteria_sukses_1.4.8}
		(Tidak Sukses)\\

		Kriteria ini tidak sukses dipatuhi karena pada halaman manajemen cetak transkrip bagian hapus permohonan, teks yang ditampilkan memiliki lebar lebih dari 80 karakter. Tampilan pada halaman web dapat dilihat pada gambar \ref{fig:1.4.8_visual_presentation}. Tautan untuk halaman yang bermasalah dapat dilihat di \url{https://bluetape.azurewebsites.net/LihatJadwalDosen}.

		\begin{figure}[H]
			\centering  
			\includegraphics[scale=0.3, frame]{kriteria-sukses-1-4-8-visual-presentation}  
			\caption[Kriteria Sukses 1.4.8 - Teks Terlalu Panjang]{Kriteria Sukses 1.4.8 - Teks Terlalu Panjang}
			\label{fig:1.4.8_visual_presentation}  
		\end{figure} 
		
		\paragraph{Kriteria Sukses 1.4.9 \textit{Images of Text (No Exception)}}
		\label{par:kepatuhan_bluetape_kriteria_sukses_1.4.9}
		(Sukses)\\

		Kriteria ini sukses dipatuhi karena pada halaman web BlueTape tidak terdapat gambar teks selain logo.

		\paragraph{Kriteria Sukses 1.4.10 \textit{Reflow}}
		\label{par:kepatuhan_bluetape_kriteria_sukses_1.4.10}
		(Tidak Sukses)\\

		Kriteria ini tidak sukses dipatuhi karena pada halaman web BlueTape, bagian navigasi menu memerlukan \textit{scroll} secara horizontal ketika ditampilkan pada resolusi layar dengan lebar 1280 piksel dan diperbesar hingga 400 persen. Tampilan pada halaman web dapat dilihat pada gambar \ref{fig:1.4.8_visual_presentation}.

		\begin{figure}[H]
			\centering  
			\includegraphics[scale=0.3, frame]{kriteria-sukses-1-4-10-reflow}  
			\caption[Kriteria Sukses 1.4.10 - Horizontal \textit{Scroll} pada Navigasi]{Kriteria Sukses 1.4.10 - Horizontal \textit{Scroll} pada Navigasi}
			\label{fig:1.4.10_reflow}  
		\end{figure} 

		\paragraph{Kriteria Sukses 1.4.11 \textit{Non-text Contrast}}
		\label{par:kepatuhan_bluetape_kriteria_sukses_1.4.11}
		(Diabaikan)\\

		Kriteria ini diabaikan karena terlalu sulit untuk melakukan pengukuran dan mendapatkan hasilnya. Tampilan pada halaman web dapat dilihat pada gambar \ref{fig:1.4.11_non_text_contrast}. Contoh tautan untuk halaman yang bermasalah dapat dilihat di \url{https://bluetape.azurewebsites.net/PerubahanKuliahManage}.

		\begin{figure}[H]
			\centering  
			\includegraphics[scale=0.3, frame]{kriteria-sukses-1-4-11-non-text-contrast}  
			\caption[Kriteria Sukses 1.4.11 - Ikon pada Bagian Aksi]{Kriteria Sukses 1.4.11 - Ikon pada Bagian Aksi}
			\label{fig:1.4.11_non_text_contrast}  
		\end{figure} 
		
		\paragraph{Kriteria Sukses 1.4.12 \textit{Text Spacing}}
		\label{par:kepatuhan_bluetape_kriteria_sukses_1.4.12}
		(Sukses)\\

		Kriteria ini sukses dipatuhi karena penulis sudah melakukan uji coba dengan mengubah beberapa bagian pada halaman perubahan kuliah sesuai kriteria ini dan tidak terjadi masalah. Oleh karena itu, penulis menduga tidak akan ada masalah ketika ukuran-ukuran tertentu diubah pada halaman lain. Tampilan pada halaman web dapat dilihat pada gambar \ref{fig:1.4.12_text_spacing}.

		\begin{figure}[H]
			\centering  
			\includegraphics[scale=0.3, frame]{kriteria-sukses-1-4-12-text-spacing}  
			\caption[Kriteria Sukses 1.4.12 - Uji Coba Mengubah Ukuran Elemen]{Kriteria Sukses 1.4.12 - Uji Coba Mengubah Ukuran Elemen}
			\label{fig:1.4.12_text_spacing}  
		\end{figure} 

		\paragraph{Kriteria Sukses 1.4.13 \textit{Content on Hover or Focus}}
		\label{par:kepatuhan_bluetape_kriteria_sukses_1.4.13}
		(Sukses)\\

		Kriteria ini sukses dipatuhi karena setiap konten tambahan yang muncul sesaat ketika suatu elemen menerima penunjuk kursor atau fokus \textit{keyboard}, konten tambahan tersebut dapat disingkirkan, dapat ditunjuk, dan persisten.

		\subsection*{\textit{Operable}}
		\label{subsec:kepatuhan_bluetape_operable}

		\subsubsection*{\textit{Keyboard Accessible}}
		\label{subsubsec:kepatuhan_bluetape_keyboard_accessible}

		\paragraph{Kriteria Sukses 2.1.1 \textit{Keyboard}}
		\label{par:kepatuhan_bluetape_kriteria_sukses_2.1.1}
		(Tidak Sukses)\\

		Kriteria ini tidak sukses dipatuhi karena terdapat fungsionalitas konten yang tidak dapat dioperasikan melalui \textit{keyboard}, antara lain:

		\begin{itemize}
			\item Pada bagian navigasi menu, pengguna tidak dapat memilih halaman yang diinginkan. Tampilan pada halaman web dapat dilihat pada gambar \ref{fig:2.1.1_keyboard_1}.
			\begin{figure}[H]
				\centering  
				\includegraphics[scale=0.3, frame]{kriteria-sukses-2-1-1-keyboard-1}  
				\caption[Kriteria Sukses 2.1.1 - Penggunaan \textit{Keyboard} pada Menu Navigasi]{Kriteria Sukses 2.1.1 - Penggunaan \textit{Keyboard} pada Menu Navigasi}
				\label{fig:2.1.1_keyboard_1}  
			\end{figure} 

			\item Pada bagian tabel "Daftar Jadwal" di halaman entri jadwal dosen. Tampilan pada halaman web dapat dilihat pada gambar \ref{fig:2.1.1_keyboard_2}. Tautan untuk halaman yang bermasalah dapat dilihat di \url{https://bluetape.azurewebsites.net/EntriJadwalDosen}.
			\begin{figure}[H]
				\centering  
				\includegraphics[scale=0.3, frame]{kriteria-sukses-2-1-1-keyboard-2}  
				\caption[Kriteria Sukses 2.1.1 - Penggunaan \textit{Keyboard} pada Halaman Entri Jadwal Dosen]{Kriteria Sukses 2.1.1 - Penggunaan \textit{Keyboard} pada Halaman Entri Jadwal Dosen}
				\label{fig:2.1.1_keyboard_2}  
			\end{figure} 
		\end{itemize}

		\paragraph{Kriteria Sukses 2.1.2 \textit{No Keyboard Trap}}
		\label{par:kepatuhan_bluetape_kriteria_sukses_2.1.2}
		(Sukses)\\

		Kriteria ini sukses dipatuhi karena pengguna dapat bernavigasi dari satu komponen ke komponen lain pada setiap komponen yang dapat dinavigasikan pada halaman web BlueTape dengan menggunakan \textit{keyboard} tanpa terperangkap dalam suatu komponen tertentu.

		\paragraph{Kriteria Sukses 2.1.3 \textit{Keyboard (No Exception)}}
		\label{par:kepatuhan_bluetape_kriteria_sukses_2.1.3}
		(Tidak Sukses)\\

		Kriteria ini tidak sukses dipatuhi karena terdapat fungsionalitas konten yang tidak dapat dioperasikan melalui \textit{keyboard}, antara lain:

		\begin{itemize}
			\item Pada bagian navigasi menu, pengguna tidak dapat memilih halaman yang diinginkan. Tampilan pada halaman web dapat dilihat pada gambar \ref{fig:2.1.3_keyboard_no_exception_1}.
			\begin{figure}[H]
				\centering  
				\includegraphics[scale=0.3, frame]{kriteria-sukses-2-1-3-keyboard-no-exception-1}  
				\caption[Kriteria Sukses 2.1.3 - Penggunaan \textit{Keyboard} pada Menu Navigasi]{Kriteria Sukses 2.1.3 - Penggunaan \textit{Keyboard} pada Menu Navigasi}
				\label{fig:2.1.3_keyboard_no_exception_1}  
			\end{figure} 

			\item Pada bagian tabel "Daftar Jadwal" di halaman entri jadwal dosen. Tampilan pada halaman web dapat dilihat pada gambar \ref{fig:2.1.3_keyboard_no_exception_2}. Tautan untuk halaman yang bermasalah dapat dilihat di \url{https://bluetape.azurewebsites.net/EntriJadwalDosen}.
			\begin{figure}[H]
				\centering  
				\includegraphics[scale=0.3, frame]{kriteria-sukses-2-1-3-keyboard-no-exception-2}  
				\caption[Kriteria Sukses 2.1.3 - Penggunaan \textit{Keyboard} pada Halaman Entri Jadwal Dosen]{Kriteria Sukses 2.1.3 - Penggunaan \textit{Keyboard} pada Halaman Entri Jadwal Dosen}
				\label{fig:2.1.3_keyboard_no_exception_2}  
			\end{figure} 
		\end{itemize}

		\paragraph{Kriteria Sukses 2.1.4 \textit{Character Key Shortcuts}}
		\label{par:kepatuhan_bluetape_kriteria_sukses_2.1.4}
		(Sukses)\\

		Kriteria ini sukses dipatuhi karena pada halaman web BlueTape tidak terdapat pintasan \textit{keyboard} untuk konten yang disajikan.

		\subsubsection*{\textit{Enough Time}}
		\label{subsubsec:kepatuhan_bluetape_enough_time}

		\paragraph{Kriteria Sukses 2.2.1 \textit{Timing Adjustable}}
		\label{par:kepatuhan_bluetape_kriteria_sukses_2.2.1}
		(Sukses)\\

		Kriteria ini sukses dipatuhi karena pada halaman web BlueTape tidak terdapat batas waktu bagi pengguna untuk membaca dan memanfaatkan konten.

		\paragraph{Kriteria Sukses 2.2.2 \textit{Pause, Stop, Hide}}
		\label{par:kepatuhan_bluetape_kriteria_sukses_2.2.2}
		(Sukses)\\

		Kriteria ini sukses dipatuhi karena pada halaman web BlueTape tidak terdapat konten yang bergerak, berkelip, bergulir, ataupun diperbarui otomatis.

		\paragraph{Kriteria Sukses 2.2.3 \textit{No Timing}}
		\label{par:kepatuhan_bluetape_kriteria_sukses_2.2.3}
		(Sukses)\\

		Kriteria ini sukses dipatuhi karena pada halaman web BlueTape tidak terdapat batas waktu bagi pengguna untuk membaca dan memanfaatkan konten.

		\paragraph{Kriteria Sukses 2.2.4 \textit{Interruptions}}
		\label{par:kepatuhan_bluetape_kriteria_sukses_2.2.4}
		(Sukses)\\

		Kriteria ini sukses dipatuhi karena setiap interupsi pada halaman web BlueTape dapat ditunda atau dihentikan oleh pengguna.

		\paragraph{Kriteria Sukses 2.2.5 \textit{Re-authenticating}}
		\label{par:kepatuhan_bluetape_kriteria_sukses_2.2.5}
		(Tidak Sukses)\\

		Kriteria ini tidak sukses dipatuhi karena ketika pengguna akan mengirim data dan sesi autentikasi berakhir, pengguna kehilangan data tersebut setelah melakukan autentikasi ulang.

		\paragraph{Kriteria Sukses 2.2.6 \textit{Timeouts}}
		\label{par:kepatuhan_bluetape_kriteria_sukses_2.2.6}
		(Tidak Sukses)\\

		Kriteria ini tidak sukses dipatuhi karena tidak terdapat peringatan untuk pengguna ketika sesi autentikasi akan berakhir yang dapat menyebabkan kehilangan data. Data pengguna tidak disimpan untuk bertahan lebih dari 20 jam ketika pengguna tidak melakukan tindakan apa pun. Pada potongan kode \ref{lst:2.2.6_sesi_autentikasi} ditunjukkan bahwa sesi auntentikasi pengguna hanya bertahan selama 2 jam saja ketika pengguna tidak melakukan tindakan apa pun.

		\begin{lstlisting}[frame=single, label={lst:2.2.6_sesi_autentikasi}, language=PHP, caption=Kriteria Sukses 2.2.6 - Sesi Autentikasi]
			$config['sess_cookie_name'] = 'ci_session';
			$config['sess_expiration'] = 7200;
			$config['sess_save_path'] = NULL;
		\end{lstlisting}

		\subsubsection*{\textit{Seizures and Physical Reactions}}
		\label{subsubsec:kepatuhan_bluetape_seizures_and_physical_reactions}

		\paragraph{Kriteria Sukses 2.3.1 \textit{Three Flashes or Below Threshold}}
		\label{par:kepatuhan_bluetape_kriteria_sukses_2.3.1}
		(Sukses)\\

		Kriteria ini sukses dipatuhi karena pada halaman web BlueTape tidak terdapat konten yang berkelip.

		\paragraph{Kriteria Sukses 2.3.2 \textit{Three Flashes}}
		\label{par:kepatuhan_bluetape_kriteria_sukses_2.3.2}
		(Sukses)\\

		Kriteria ini sukses dipatuhi karena pada halaman web BlueTape tidak terdapat konten yang berkelip.

		\paragraph{Kriteria Sukses 2.3.3 \textit{Animation from Interactions}}
		\label{par:kepatuhan_bluetape_kriteria_sukses_2.3.3}
		(Sukses)\\

		Kriteria ini sukses dipatuhi karena pada halaman web BlueTape tidak terdapat animasi gerak pada konten yang disajikan ketika pengguna melakukan interaksi dengan komponen-komponen yang ada.

		\subsubsection*{\textit{Navigable}}
		\label{subsubsec:kepatuhan_bluetape_navigable}

		\paragraph{Kriteria Sukses 2.4.1 \textit{Bypass Blocks}}
		\label{par:kepatuhan_bluetape_kriteria_sukses_2.4.1}
		(Tidak Sukses)\\

		Kriteria ini tidak sukses dipatuhi karena pada halaman web BlueTape tidak tersedia mekanisme untuk melompati beberapa area konten yang berulang pada beberapa halaman web.

		\paragraph{Kriteria Sukses 2.4.2 \textit{Page Titled}}
		\label{par:kepatuhan_bluetape_kriteria_sukses_2.4.2}
		(Sukses)\\

		Kriteria ini sukses dipatuhi karena setiap halaman web BlueTape memiliki judul yang dapat menjelaskan topik atau tujuan dari halaman yang bersangkutan.

		\paragraph{Kriteria Sukses 2.4.3 \textit{Focus Order}}
		\label{par:kepatuhan_bluetape_kriteria_sukses_2.4.3}
		(Sukses)\\

		Kriteria ini sukses dipatuhi karena setiap halaman web BlueTape memiliki elemen dengan urutan fokus yang benar. 

		\paragraph{Kriteria Sukses 2.4.4 \textit{Link Purpose (In Context)}}
		\label{par:kepatuhan_bluetape_kriteria_sukses_2.4.4}
		(Tidak Sukses)\\

		Kriteria ini tidak sukses dipatuhi karena pada halaman cetak transkrip, manajemen cetak transkrip, perubahan kuliah, dan manajemen perubahan kuliah terdapat tautan yang berisi konten bukan teks dan tidak terdapat teks yang dapat menjelaskan tujuan tautan tersebut. Kesalahan dapat dilihat pada potongan kode \ref{lst:2.4.4_tautan_tanpa_keterangan}. Tampilan pada halaman web dapat dilihat pada gambar \ref{fig:2.4.4_link_purpose_in_context}. Contoh tautan untuk halaman yang bermasalah dapat dilihat di \url{https://bluetape.azurewebsites.net/TranskripRequest}.

		\begin{lstlisting}[frame=single, label={lst:2.4.4_tautan_tanpa_keterangan}, language=HTML, caption=Kriteria Sukses 2.4.4 - Tautan Tanpa Keterangan]
				</div>
				<a data-open="detail6">
					<i class="fi-eye"></i>
				</a>
			</td>
		\end{lstlisting}

		\begin{figure}[H]
			\centering  
			\includegraphics[scale=0.3, frame]{kriteria-sukses-2-4-4-link-purpose-in-context}  
			\caption[Kriteria Sukses 2.4.4 - Tautan Tanpa Keterangan]{Kriteria Sukses 2.4.4 - Tautan Tanpa Keterangan}
			\label{fig:2.4.4_link_purpose_in_context}  
		\end{figure} 

		\paragraph{Kriteria Sukses 2.4.5 \textit{Multiple Ways}}
		\label{par:kepatuhan_bluetape_kriteria_sukses_2.4.5}
		(Tidak Sukses)\\

		Kriteria ini tidak sukses dipatuhi karena pada halaman web BlueTape hanya tersedia satu cara untuk menemukan suatu halaman web dalam sekumpulan halaman web yang tersedia, yaitu melalui menu navigasi. Tampilan pada halaman web dapat dilihat pada gambar \ref{fig:2.4.5_multiple_ways}.

		\begin{figure}[H]
			\centering  
			\includegraphics[scale=0.3, frame]{kriteria-sukses-2-4-5-multiple-ways}  
			\caption[Kriteria Sukses 2.4.5 - Menemukan Halaman Web pada Navigasi]{Kriteria Sukses 2.4.5 - Menemukan Halaman Web pada Navigasi}
			\label{fig:2.4.5_multiple_ways}  
		\end{figure}

		\paragraph{Kriteria Sukses 2.4.6 \textit{Headings and Labels}}
		\label{par:kepatuhan_bluetape_kriteria_sukses_2.4.6}
		(Tidak Sukses)\\

		Kriteria ini tidak sukses dipatuhi karena pada halaman entri jadwal dosen, setiap elemen masukan tidak memiliki label yang menjelaskan tujuan dari elemen tersebut. Contoh kesalahan dapat dilihat pada potongan kode \ref{lst:2.4.6_label_masukan_entri_jadwal_dosen}. Tautan untuk halaman yang bermasalah dapat dilihat di \url{https://bluetape.azurewebsites.net/EntriJadwalDosen}.

		\begin{lstlisting}[frame=single, label={lst:2.4.6_label_masukan_entri_jadwal_dosen}, language=HTML, caption=Kriteria Sukses 2.4.6 - Tidak Terdapat Label pada Kolom Masukan di Halaman Entri Jadwal Dosen]
			<input type="hidden" name="csrf_token" value="3c159eae7bc953dd591b679c080ed066"/>
			Hari
			<select name="hari">
		\end{lstlisting}
		
		\paragraph{Kriteria Sukses 2.4.7 \textit{Focus Visible}}
		\label{par:kepatuhan_bluetape_kriteria_sukses_2.4.7}
		(Tidak Sukses)\\

		Kriteria ini tidak sukses dipatuhi karena setiap halaman web BlueTape memiliki komponen tombol dengan warna latar belakang yang sama dengan warna indikator fokus dari \textit{keyboard}. Tampilan pada halaman web dapat dilihat pada gambar \ref{fig:2.4.7_focus_visible}. Contoh tautan untuk halaman yang bermasalah dapat dilihat di \url{https://bluetape.azurewebsites.net/TranskripRequest}. 

		\begin{figure}[H]
			\centering  
			\includegraphics[scale=0.3, frame]{kriteria-sukses-2-4-7-focus-visible}  
			\caption[Kriteria Sukses 2.4.7 - Indikator Fokus]{Kriteria Sukses 2.4.7 - Indikator Fokus}
			\label{fig:2.4.7_focus_visible}  
		\end{figure}
		
		\paragraph{Kriteria Sukses 2.4.8 \textit{Location}}
		\label{par:kepatuhan_bluetape_kriteria_sukses_2.4.8}
		(Sukses)\\

		Kriteria ini sukses dipatuhi karena pada menu navigasi terdapat indikator yang menunjukkan halaman yang sedang dipilih oleh pengguna, indikator ini berupa penebalan ukuran tulisan.

		\paragraph{Kriteria Sukses 2.4.9 \textit{Link Purpose (Link Only)}}
		\label{par:kepatuhan_bluetape_kriteria_sukses_2.4.9}
		(Tidak Sukses)\\

		Kriteria ini tidak sukses dipatuhi karena pada halaman cetak transkrip, manajemen cetak transkrip, perubahan kuliah, dan manajemen perubahan kuliah terdapat tautan yang berisi konten bukan teks dan tidak terdapat teks yang dapat menjelaskan tujuan tautan tersebut. Kesalahan dapat dilihat pada potongan kode \ref{lst:2.4.9_tautan_tanpa_keterangan}. Tampilan pada halaman web dapat dilihat pada gambar \ref{fig:2.4.9_link_purpose_link_only}. Contoh tautan untuk halaman yang bermasalah dapat dilihat di \url{https://bluetape.azurewebsites.net/TranskripRequest}.

		\begin{lstlisting}[frame=single, label={lst:2.4.9_tautan_tanpa_keterangan}, language=HTML, caption=Kriteria Sukses 2.4.9 - Tautan Tanpa Keterangan]
				</div>
				<a data-open="detail6">
					<i class="fi-eye"></i>
				</a>
			</td>
		\end{lstlisting}

		\begin{figure}[H]
			\centering  
			\includegraphics[scale=0.3, frame]{kriteria-sukses-2-4-9-link-purpose-link-only}  
			\caption[Kriteria Sukses 2.4.9 - Tautan Tanpa Keterangan]{Kriteria Sukses 2.4.9 - Tautan Tanpa Keterangan}
			\label{fig:2.4.9_link_purpose_link_only}  
		\end{figure} 
		
		\paragraph{Kriteria Sukses 2.4.10 \textit{Section Headings}}
		\label{par:kepatuhan_bluetape_kriteria_sukses_2.4.10}
		(Tidak Sukses)\\

		Kriteria ini tidak sukses dipatuhi karena terdapat penggunaan \textit{tag heading} yang tidak tepat secara struktur pada halaman cetak transkrip, manajemen cetak transkrip, perubahan kuliah, manajemen perubahan kuliah, dan entri jadwal dosen. Contoh kesalahan dapat dilihat pada potongan kode \ref{lst:1.3.1_heading_tidak_tepat} yang menampilkan kesalahan penggunaan \textit{tag heading} pada halaman cetak transkrip. Contoh tautan untuk halaman yang bermasalah dapat dilihat di \url{https://bluetape.azurewebsites.net/TranskripRequest}.

		\begin{lstlisting}[frame=single, label={lst:2.4.10_heading_tidak_tepat}, language=HTML, caption=Kriteria Sukses 2.4.10 - Penggunaan \textit{Heading} Tidak Tepat]
			<div class="callout">
				<h5>Permohonan Baru</h5>
				<form method="POST" action="/TranskripRequest/add">
		\end{lstlisting}
	
		\subsubsection*{\textit{Input Modalities}}
		\label{subsubsec:kepatuhan_bluetape_input_modalities}

		\paragraph{Kriteria Sukses 2.5.1 \textit{Pointer Gestures}}
		\label{par:kepatuhan_bluetape_kriteria_sukses_2.5.1}
		(Sukses)\\

		Kriteria ini sukses dipatuhi karena pada halaman web BlueTape tidak terdapat fungsionalitas yang harus dijalankan dengan menggunakan \textit{multipoint} atau gestur berbasis \textit{path}.

		\paragraph{Kriteria Sukses 2.5.2 \textit{Pointer Cancellation}}
		\label{par:kepatuhan_bluetape_kriteria_sukses_2.5.2}
		(Sukses)\\

		Kriteria ini sukses dipatuhi karena setiap fungsionalitas yang dapat dioperasikan dengan kursor tunggal tidak memiliki \textit{down-event}.

		\paragraph{Kriteria Sukses 2.5.3 \textit{Label in Name}}
		\label{par:kepatuhan_bluetape_kriteria_sukses_2.5.3}
		(Tidak Sukses)\\

		Kriteria ini tidak sukses dipatuhi karena setiap komponen masukan yang memiliki label yang menyertakan teks, nama komponen tersebut tidak mengandung teks yang disajikan secara visual. Contoh kesalahan dapat dilihat pada potongan kode \ref{lst:2.5.3_teks_label_pada_nama}. Contoh tautan untuk halaman yang bermasalah dapat dilihat di \url{https://bluetape.azurewebsites.net/TranskripRequest}.

		\begin{lstlisting}[frame=single, label={lst:2.5.3_teks_label_pada_nama}, language=HTML, caption=Kriteria Sukses 2.5.3 - Teks Label pada Atribut Nama]
			<div class="large-4 column">
				<label>Yang memohon:
					<input type="email" name="requestByEmail" value="7315020@student.unpar.ac.id" readonly="readonly"/>
				</label>
			</div>
		\end{lstlisting}

		\paragraph{Kriteria Sukses 2.5.4 \textit{Motion Actuation}}
		\label{par:kepatuhan_bluetape_kriteria_sukses_2.5.4}
		(Sukses)\\

		Kriteria ini sukses dipatuhi karena tidak terdapat fungsionalitas yang dapat dioperasikan oleh gerakan perangkat atau gerakan pengguna.

		\paragraph{Kriteria Sukses 2.5.5 \textit{Target Size}}
		\label{par:kepatuhan_bluetape_kriteria_sukses_2.5.5}
		(Tidak Sukses)\\

		Kriteria ini tidak sukses dipatuhi karena terdapat banyak elemen yang dapat menerima masukan kursor dan ukurannya diatur oleh penulis namun memiliki ukuran kurang dari 44 kali 44 piksel \textit{CSS}. Contoh tampilan pada halaman web dapat dilihat pada gambar \ref{fig:2.5.5_target_size}. Contoh tautan untuk halaman yang bermasalah dapat dilihat di \url{https://bluetape.azurewebsites.net/TranskripRequest}.

		\begin{figure}[H]
			\centering  
			\includegraphics[scale=0.3, frame]{kriteria-sukses-2-5-5-target-size}  
			\caption[Kriteria Sukses 2.5.5 - Ukuran Elemen]{Kriteria Sukses 2.5.5 - Ukuran Elemen}
			\label{fig:2.5.5_target_size}  
		\end{figure} 

		\paragraph{Kriteria Sukses 2.5.6 \textit{Concurrent Input Mechanisms}}
		\label{par:kepatuhan_bluetape_kriteria_sukses_2.5.6}
		(Sukses)\\

		Kriteria ini sukses dipatuhi karena konten web BlueTape tidak membatasi penggunaan modalitas masukan yang tersedia pada platform.

		\subsection*{\textit{Understandable}}
		\label{subsec:kepatuhan_bluetape_understandable}

		\subsubsection*{\textit{Readable}}
		\label{subsubsec:kepatuhan_bluetape_readable}

		\paragraph{Kriteria Sukses 3.1.1 \textit{Language of Page}}
		\label{par:kepatuhan_bluetape_kriteria_sukses_3.1.1}
		(Tidak Sukses)\\

		Kriteria ini tidak sukses dipatuhi karena bahasa manusia \textit{default} yang digunakan pada halaman web BlueTape tidak sesuai dengan konten yang disajikan pada halaman tersebut. Bahasa manusia \textit{default} yang digunakan pada halaman web BlueTape adalah bahasa Inggris sementara konten disajikan dalam bahasa Indonesia. Contoh kesalahan dapat dilihat pada potongan kode \ref{lst:3.1.1_bahasa_halaman}. Contoh tautan untuk halaman yang bermasalah dapat dilihat di \url{https://bluetape.azurewebsites.net/TranskripRequest}.

		\begin{lstlisting}[frame=single, label={lst:3.1.1_bahasa_halaman}, language=HTML, caption=Kriteria Sukses 3.1.1 - Bahasa yang Tidak Sesuai]
			// Setelan bahasa diatur menjadi bahasa Inggris
			<!DOCTYPE html>
				<html class="no-js" lang="en">
			<head>

			// Konten disajikan dalam bahasa Indonesia
			<div class="large-4 column">
				<label>Yang memohon:
					<input type="email" name="requestByEmail" value="7315020@student.unpar.ac.id" readonly="readonly"/>
				</label>
			</div>
			<div class="large-4 column">
				<label>NPM:
					<input type="text" value="2015730020" readonly="readonly"/>
				</label>
			</div>
			<div class="large-4 column">
				<label>Nama:
					<input type="text" name="requestByName" value="HIZKIA STEVEN" readonly="readonly"/>
				</label>
			</div>
		\end{lstlisting}

		\paragraph{Kriteria Sukses 3.1.2 \textit{Language of Parts}}
		\label{par:kepatuhan_bluetape_kriteria_sukses_3.1.2}
		(Tidak Sukses)\\

		Kriteria ini tidak sukses dipatuhi karena pada halaman web BlueTape terdapat konten teks dalam bahasa Inggris namun konten tersebut tidak memiliki atribut \textit{lang}. Konten yang dimaksud adalah:

		\begin{itemize}
			\item Pada bagian navigasi menu terdapat elemen dengan teks "Logout". Letak kesalahan dapat dilihat pada potongan kode \ref{lst:3.1.2_bahasa_pada_navigasi}. Contoh tampilan pada halaman web dapat dilihat pada gambar \ref{fig:3.1.2_language_of_parts_1}. 
			\begin{lstlisting}[frame=single, label={lst:3.1.2_bahasa_pada_navigasi}, language=HTML, caption=Kriteria Sukses 3.1.2 - Bahasa yang Tidak Sesuai pada Menu Navigasi]
				<ul class="menu">
					<li><a href="/auth/logout">Logout</a></li>
				</ul>
			\end{lstlisting}
			
			\begin{figure}[H]
				\centering  
				\includegraphics[scale=0.3, frame]{kriteria-sukses-3-1-2-language-of-parts-1}  
				\caption[Kriteria Sukses 3.1.2 - Bahasa pada Menu Navigasi]{Kriteria Sukses 3.1.2 - Bahasa pada Menu Navigasi}
				\label{fig:3.1.2_language_of_parts_1}  
			\end{figure}

			\item Pada halaman entri jadwal dosen terdapat tombol dengan teks "Delete All". Letak kesalahan dapat dilihat pada potongan kode \ref{lst:3.1.2_bahasa_pada_halaman_entri_jadwal_dosen_1}. Contoh tampilan pada halaman web dapat dilihat pada gambar \ref{fig:3.1.2_language_of_parts_2}. Tautan untuk halaman yang bermasalah dapat dilihat di \url{https://bluetape.azurewebsites.net/EntriJadwalDosen}.
			\begin{lstlisting}[frame=single, label={lst:3.1.2_bahasa_pada_halaman_entri_jadwal_dosen_1}, language=HTML, caption=Kriteria Sukses 3.1.2 - Bahasa yang Tidak Sesuai pada Halaman Entri Jadwal Dosen 1]
				<a href="/EntriJadwalDosen/deleteAll/export/" class="alert button" onClick="return konfirmasi();">
					Delete All
				</a>
			\end{lstlisting}
			
			\begin{figure}[H]
				\centering  
				\includegraphics[scale=0.3, frame]{kriteria-sukses-3-1-2-language-of-parts-2}  
				\caption[Kriteria Sukses 3.1.2 - Bahasa pada Halaman Entri Jadwal Dosen 1]{Kriteria Sukses 3.1.2 - Bahasa pada Halaman Entri Jadwal Dosen 1}
				\label{fig:3.1.2_language_of_parts_2}  
			\end{figure}
			
			\item Pada halaman entri jadwal dosen bagian "Edit Jadwal" terdapat tombol dengan teks "Save" dan tombol dengan teks "Delete". Letak kesalahan dapat dilihat pada potongan kode \ref{lst:3.1.2_bahasa_pada_halaman_entri_jadwal_dosen_2}. Contoh tampilan pada halaman web dapat dilihat pada gambar \ref{fig:3.1.2_language_of_parts_3}. Tautan untuk halaman yang bermasalah dapat dilihat di \url{https://bluetape.azurewebsites.net/EntriJadwalDosen}.
			\begin{lstlisting}[frame=single, label={lst:3.1.2_bahasa_pada_halaman_entri_jadwal_dosen_2}, language=HTML, caption=Kriteria Sukses 3.1.2 - Bahasa yang Tidak Sesuai pada Halaman Entri Jadwal Dosen 2]
				// Tombol "Save"
				<div class="large-2 column">
					<input type="submit" name="submitId8" class="button" value="Save">
					</form>
				</div>

				// Tombol "Delete"
				<form name="formDelete8" method="POST" action="/EntriJadwalDosen/delete/8">    
					<input type="hidden" name="csrf_token" value="3c159eae7bc953dd591b679c080ed066"/>
					<input id="deletebtn8" name="8" type="submit" class="alert button" value="Delete">
				</form><div>
			\end{lstlisting}
			
			\begin{figure}[H]
				\centering  
				\includegraphics[scale=0.3, frame]{kriteria-sukses-3-1-2-language-of-parts-3}  
				\caption[Kriteria Sukses 3.1.2 - Bahasa pada Halaman Entri Jadwal Dosen 2]{Kriteria Sukses 3.1.2 - Bahasa pada Halaman Entri Jadwal Dosen 2}
				\label{fig:3.1.2_language_of_parts_3}  
			\end{figure}
		\end{itemize}

		\paragraph{Kriteria Sukses 3.1.3 \textit{Unusual Words}}
		\label{par:kepatuhan_bluetape_kriteria_sukses_3.1.3}
		(Tidak Sukses)\\

		Kriteria ini tidak sukses dipatuhi karena terdapat kata-kata yang digunakan dengan cara yang tidak lazim atau terbatas namun tidak tersedia mekanisme untuk mengidentifikasi definisi spesifik dari kata-kata tersebut. Kata-kata yang dimaksud adalah:

		\begin{itemize}
			\item Pada halaman perubahan kuliah, tombol dengan teks "Tambah Pertemuan Ekstra". Letak kesalahan dapat dilihat pada potongan kode \ref{lst:3.1.3_kata_pada_halaman_perubahan_kuliah}. Contoh tampilan pada halaman web dapat dilihat pada gambar \ref{fig:3.1.3_unusual_words_1}. Tautan untuk halaman yang bermasalah dapat dilihat di \url{https://bluetape.azurewebsites.net/PerubahanKuliahRequest}.
			\begin{lstlisting}[frame=single, label={lst:3.1.3_kata_pada_halaman_perubahan_kuliah}, language=HTML, caption=Kriteria Sukses 3.1.3 - Kata yang Tak Lazim pada Halaman Perubahan Kuliah]
					<a href="#" id="addToButton" class="button secondary">
						Tambah Pertemuan Ekstra
					</a>
			\end{lstlisting}
			
			\begin{figure}[H]
				\centering  
				\includegraphics[scale=0.3, frame]{kriteria-sukses-3-1-3-unusual-words-1}  
				\caption[Kriteria Sukses 3.1.3 - Kata yang Tak Lazim pada Halaman Perubahan Kuliah]{Kriteria Sukses 3.1.3 - Kata yang Tak Lazim pada Halaman Perubahan Kuliah}
				\label{fig:3.1.3_unusual_words_1}  
			\end{figure}
			
			\item Pada halaman entri jadwal dosen, tombol dengan teks "Ekspor ke XLS". Letak kesalahan dapat dilihat pada potongan kode \ref{lst:3.1.3_kata_pada_halaman_entri_jadwal_dosen}. Contoh tampilan pada halaman web dapat dilihat pada gambar \ref{fig:3.1.3_unusual_words_2}. Tautan untuk halaman yang bermasalah dapat dilihat di \url{https://bluetape.azurewebsites.net/EntriJadwalDosen}.
			\begin{lstlisting}[frame=single, label={lst:3.1.3_kata_pada_halaman_entri_jadwal_dosen}, language=HTML, caption=Kriteria Sukses 3.1.3 - Kata yang Tak Lazim pada Halaman Entri Jadwal Dosen]
				<a href="/EntriJadwalDosen/export/" class="button">
					Ekspor ke XLS
				</a>
			\end{lstlisting}
			
			\begin{figure}[H]
				\centering  
				\includegraphics[scale=0.3, frame]{kriteria-sukses-3-1-3-unusual-words-2}  
				\caption[Kriteria Sukses 3.1.3 - Kata yang Tak Lazim pada Halaman Entri Jadwal Dosen]{Kriteria Sukses 3.1.3 - Kata yang Tak Lazim pada Halaman Entri Jadwal Dosen}
				\label{fig:3.1.3_unusual_words_2}  
			\end{figure}

			\item Pada halaman lihat jadwal dosen, tombol dengan teks "Ekspor ke XLS". Letak kesalahan dapat dilihat pada potongan kode \ref{lst:3.1.3_kata_pada_halaman_lihat_jadwal_dosen}. Contoh tampilan pada halaman web dapat dilihat pada gambar \ref{fig:3.1.3_unusual_words_3}. Tautan untuk halaman yang bermasalah dapat dilihat di \url{https://bluetape.azurewebsites.net/LihatJadwalDosen}.
			\begin{lstlisting}[frame=single, label={lst:3.1.3_kata_pada_halaman_lihat_jadwal_dosen}, language=HTML, caption=Kriteria Sukses 3.1.3 - Kata yang Tak Lazim pada Halaman Lihat Jadwal Dosen]
				<a href="/LihatJadwalDosen/export/" class="button">
					Ekspor ke XLS
				</a>
			\end{lstlisting}
			
			\begin{figure}[H]
				\centering  
				\includegraphics[scale=0.3, frame]{kriteria-sukses-3-1-3-unusual-words-3}  
				\caption[Kriteria Sukses 3.1.3 - Kata yang Tak Lazim pada Halaman Lihat Jadwal Dosen]{Kriteria Sukses 3.1.3 - Kata yang Tak Lazim pada Halaman Lihat Jadwal Dosen}
				\label{fig:3.1.3_unusual_words_3}  
			\end{figure}
			
		\end{itemize}

		\paragraph{Kriteria Sukses 3.1.4 \textit{Abbreviations}}
		\label{par:kepatuhan_bluetape_kriteria_sukses_3.1.4}
		(Tidak Sukses)\\

		Kriteria ini tidak sukses dipatuhi karena tidak tersedia mekanisme untuk mengidentifikasi kepanjangan dari singkatan. Singkatan-singkatan yang terdapat pada halaman web BlueTape, antara lain:

		\begin{itemize}
			\item Pada halaman cetak transkrip terdapat singkatan "NPM" dan "DPS". Letak kesalahan dapat dilihat pada potongan kode \ref{lst:3.1.4_singkatan_pada_halaman_cetak_transkrip}. Contoh tampilan pada halaman web dapat dilihat pada gambar \ref{fig:3.1.4_abbreviations_1}. Tautan untuk halaman yang bermasalah dapat dilihat di \url{https://bluetape.azurewebsites.net/TranskripRequest}.
			\begin{lstlisting}[frame=single, label={lst:3.1.4_singkatan_pada_halaman_cetak_transkrip}, language=HTML, caption=Kriteria Sukses 3.1.4 - Singkatan pada Halaman Cetak Transkrip]
				// Singkatan "NPM"
				<div class="large-4 column">
					<label>NPM:
						<input type="text" value="2015730020" readonly="readonly"/>
					</label>
				</div>
				
				// Singkatan "DPS"
				<select name="requestType">
					<option value="DPS">
						DPS (Seluruh Semester, Bilingual)
					</option>
				</select>
			\end{lstlisting}

			\begin{figure}[H]
				\centering  
				\includegraphics[scale=0.3, frame]{kriteria-sukses-3-1-4-abbreviations-1}  
				\caption[Kriteria Sukses 3.1.4 - Singkatan pada Halaman Cetak Transkrip]{Kriteria Sukses 3.1.4 - Singkatan pada Halaman Cetak Transkrip}
				\label{fig:3.1.4_abbreviations_1}  
			\end{figure}

			\item Pada halaman manajemen cetak transkrip terdapat singkatan "NPM". Letak kesalahan dapat dilihat pada potongan kode \ref{lst:3.1.4_singkatan_pada_halaman_manajemen_cetak_transkrip}. Contoh tampilan pada halaman web dapat dilihat pada gambar \ref{fig:3.1.4_abbreviations_2}. Tautan untuk halaman yang bermasalah dapat dilihat di \url{https://bluetape.azurewebsites.net/TranskripManage}.
			\begin{lstlisting}[frame=single, label={lst:3.1.4_singkatan_pada_halaman_manajemen_cetak_transkrip}, language=HTML, caption=Kriteria Sukses 3.1.4 - Singkatan pada Halaman Manajemen Cetak Transkrip]
				<div class="input-group">
					<span class="input-group-label">Cari NPM:</span>
					<input name="npm" class="input-group-field" type="text" placeholder="2013730013" maxlength="10" minlength="10"/>
			\end{lstlisting}

			\begin{figure}[H]
				\centering  
				\includegraphics[scale=0.3, frame]{kriteria-sukses-3-1-4-abbreviations-2}  
				\caption[Kriteria Sukses 3.1.4 - Singkatan pada Halaman Manajemen Cetak Transkrip]{Kriteria Sukses 3.1.4 - Singkatan pada Halaman Manajemen Cetak Transkrip}
				\label{fig:3.1.4_abbreviations_2}  
			\end{figure}
			
			\item Pada halaman perubahan kuliah terdapat singkatan "MK". Letak kesalahan dapat dilihat pada potongan kode \ref{lst:3.1.4_singkatan_pada_halaman_perubahan_kuliah}. Contoh tampilan pada halaman web dapat dilihat pada gambar \ref{fig:3.1.4_abbreviations_3}. Tautan untuk halaman yang bermasalah dapat dilihat di \url{https://bluetape.azurewebsites.net/PerubahanKuliahRequest}.
			\begin{lstlisting}[frame=single, label={lst:3.1.4_singkatan_pada_halaman_perubahan_kuliah}, language=HTML, caption=Kriteria Sukses 3.1.4 - Singkatan pada Halaman Perubahan Kuliah]
				<div class="large-2 column">
					<label>Kode MK:
						<input type="text" name="mataKuliahCode" required maxlength="9" pattern="[A-Z]{3}[0-9]{3}([0-9]{3})?" title="Kode MK dalam format XYZ123"/>
			\end{lstlisting}

			\begin{figure}[H]
				\centering  
				\includegraphics[scale=0.3, frame]{kriteria-sukses-3-1-4-abbreviations-3}  
				\caption[Kriteria Sukses 3.1.4 - Singkatan pada Halaman Perubahan Kuliah]{Kriteria Sukses 3.1.4 - Singkatan pada Halaman Perubahan Kuliah}
				\label{fig:3.1.4_abbreviations_3}  
			\end{figure}
			
			\item Pada halaman manajemen perubahan kuliah terdapat singkatan "MK". Letak kesalahan dapat dilihat pada potongan kode \ref{lst:3.1.4_singkatan_pada_halaman_manajemen_perubahan_kuliah}. Contoh tampilan pada halaman web dapat dilihat pada gambar \ref{fig:3.1.4_abbreviations_4}. Tautan untuk halaman yang bermasalah dapat dilihat di \url{https://bluetape.azurewebsites.net/PerubahanKuliahManage}.
			\begin{lstlisting}[frame=single, label={lst:3.1.4_singkatan_pada_halaman_manajemen_perubahan_kuliah}, language=HTML, caption=Kriteria Sukses 3.1.4 - Singkatan pada Halaman Manajemen Perubahan Kuliah]
				<th>Tanggal Permohonan</th>
				<th>Kode MK</th>
				<th>Perubahan</th>
			\end{lstlisting}

			\begin{figure}[H]
				\centering  
				\includegraphics[scale=0.3, frame]{kriteria-sukses-3-1-4-abbreviations-4}  
				\caption[Kriteria Sukses 3.1.4 - Singkatan pada Halaman Manajemen Perubahan Kuliah]{Kriteria Sukses 3.1.4 - Singkatan pada Halaman Manajemen Perubahan Kuliah}
				\label{fig:3.1.4_abbreviations_4}  
			\end{figure}
			
			\item Pada halaman entri jadwal dosen terdapat singkatan "XLS". Letak kesalahan dapat dilihat pada potongan kode \ref{lst:3.1.4_singkatan_pada_halaman_entri_jadwal_dosen}. Contoh tampilan pada halaman web dapat dilihat pada gambar \ref{fig:3.1.4_abbreviations_5}. Tautan untuk halaman yang bermasalah dapat dilihat di \url{https://bluetape.azurewebsites.net/EntriJadwalDosen}.
			\begin{lstlisting}[frame=single, label={lst:3.1.4_singkatan_pada_halaman_entri_jadwal_dosen}, language=HTML, caption=Kriteria Sukses 3.1.4 - Singkatan pada Halaman Entri Jadwal Dosen]
				<a href="/EntriJadwalDosen/export/" class="button">
					Ekspor ke XLS
				</a>
			\end{lstlisting}

			\begin{figure}[H]
				\centering  
				\includegraphics[scale=0.3, frame]{kriteria-sukses-3-1-4-abbreviations-5}  
				\caption[Kriteria Sukses 3.1.4 - Singkatan pada Halaman Entri Jadwal Dosen]{Kriteria Sukses 3.1.4 - Singkatan pada Halaman Entri Jadwal Dosen}
				\label{fig:3.1.4_abbreviations_5}  
			\end{figure}
			
			\item Pada halaman lihat jadwal dosen terdapat singkatan "XLS". Letak kesalahan dapat dilihat pada potongan kode \ref{lst:3.1.4_singkatan_pada_halaman_lihat_jadwal_dosen}. Contoh tampilan pada halaman web dapat dilihat pada gambar \ref{fig:3.1.4_abbreviations_6}. Tautan untuk halaman yang bermasalah dapat dilihat di \url{https://bluetape.azurewebsites.net/LihatJadwalDosen}.
			\begin{lstlisting}[frame=single, label={lst:3.1.4_singkatan_pada_halaman_lihat_jadwal_dosen}, language=HTML, caption=Kriteria Sukses 3.1.4 - Singkatan pada Halaman Lihat Jadwal Dosen]
				<a href="/LihatJadwalDosen/export/" class="button">
					Ekspor ke XLS
				</a>
			\end{lstlisting}

			\begin{figure}[H]
				\centering  
				\includegraphics[scale=0.3, frame]{kriteria-sukses-3-1-4-abbreviations-6}  
				\caption[Kriteria Sukses 3.1.4 - Singkatan pada Halaman Lihat Jadwal Dosen]{Kriteria Sukses 3.1.4 - Singkatan pada Halaman Lihat Jadwal Dosen}
				\label{fig:3.1.4_abbreviations_6}  
			\end{figure}
		\end{itemize}

		\paragraph{Kriteria Sukses 3.1.5 \textit{Reading Level}}
		\label{par:kepatuhan_bluetape_kriteria_sukses_3.1.5}
		(Sukses)\\

		Kriteria ini sukses dipatuhi karena pada halaman web BlueTape tidak terdapat teks yang cukup kompleks dan membutuhkan kemampuan membaca yang lebih tinggi dari rata-rata.

		\paragraph{Kriteria Sukses 3.1.6 \textit{Pronunciation}}
		\label{par:kepatuhan_bluetape_kriteria_sukses_3.1.6}
		(Sukses)\\

		Kriteria ini sukses dipatuhi karena pada halaman web BlueTape setiap kata dapat dimengerti artinya tanpa pengguna perlu mengetahui cara mengucapkan kata tersebut.

		\subsubsection*{\textit{Predictable}}
		\label{subsubsec:kepatuhan_bluetape_predictable}

		\paragraph{Kriteria Sukses 3.2.1 \textit{On Focus}}
		\label{par:kepatuhan_bluetape_kriteria_sukses_3.2.1}
		(Sukses)\\

		Kriteria ini sukses dipatuhi karena setiap komponen antarmuka pengguna yang menerima fokus, tidak menyebabkan perubahan konteks.

		\paragraph{Kriteria Sukses 3.2.2 \textit{On Input}}
		\label{par:kepatuhan_bluetape_kriteria_sukses_3.2.2}
		(Sukses)\\

		Kriteria ini sukses dipatuhi karena ketika pengguna mengubah setelan komponen antarmuka pengguna tidak terjadi perubahan konteks otomatis.

		\paragraph{Kriteria Sukses 3.2.3 \textit{Consistent Navigation}}
		\label{par:kepatuhan_bluetape_kriteria_sukses_3.2.3}
		(Sukses)\\

		Kriteria ini sukses dipatuhi karena bagian navigasi menu yang muncul berulang pada tiap halaman web BlueTape, muncul dalam urutan relatif yang sama setiap kali tampak.

		\paragraph{Kriteria Sukses 3.2.4 \textit{Consistent Identification}}
		\label{par:kepatuhan_bluetape_kriteria_sukses_3.2.4}
		(Tidak Sukses)\\

		Kriteria ini tidak sukses dipatuhi karena terdapat tombol yang memiliki fungsionalitas yang sama yaitu untuk menghapus namun diidentifikasikan dengan nama yang berbeda. Tombol yang dimaksud adalah tombol "Delete" pada bagian "Edit Jadwal" di halaman entri jadwal dosen. Tombol dengan fungsionalitas yang sama di halaman lain diidentifikasi dengan nama "Hapus". Contoh tampilan pada halaman web dapat dilihat pada gambar \ref{fig:3.2.4_consistent_identification}. Tautan untuk halaman yang bermasalah dapat dilihat di \url{https://bluetape.azurewebsites.net/EntriJadwalDosen}.

		\begin{figure}[H]
			\centering  
			\includegraphics[scale=0.3, frame]{kriteria-sukses-3-2-4-consistent-identification}  
			\caption[Kriteria Sukses 3.2.4 - Identifikasi pada Tombol]{Kriteria Sukses 3.2.4 - Identifikasi pada Tombol}
			\label{fig:3.2.4_consistent_identification}  
		\end{figure}

		\paragraph{Kriteria Sukses 3.2.5 \textit{Change on Request}}
		\label{par:kepatuhan_bluetape_kriteria_sukses_3.2.5}
		(Sukses)\\

		Kriteria ini sukses dipatuhi karena setiap perubahan konteks hanya terjadi bila dilakukan oleh pengguna.

		\subsubsection*{\textit{Input Assistance}}
		\label{subsubsec:kepatuhan_bluetape_input_assistance}

		\paragraph{Kriteria Sukses 3.3.1 \textit{Error Identification}}
		\label{par:kepatuhan_bluetape_kriteria_sukses_3.3.1}
		(Sukses)\\

		Kriteria ini sukses dipatuhi karena ketika terdapat eror masukan yang terdeteksi otomatis, \textit{item} yang eror diidentifikasi dan eror dijabarkan kepada pengguna dalam bentuk teks.

		\paragraph{Kriteria Sukses 3.3.2 \textit{Labels or Instructions}}
		\label{par:kepatuhan_bluetape_kriteria_sukses_3.3.2}
		(Tidak Sukses)\\

		Kriteria ini tidak sukses dipatuhi karena pada halaman entri jadwal dosen, setiap elemen masukan tidak memiliki label yang menjelaskan tujuan dari elemen tersebut. Contoh kesalahan dapat dilihat pada potongan kode \ref{lst:3.3.2_label_masukan_entri_jadwal_dosen}. Tautan untuk halaman yang bermasalah dapat dilihat di \url{https://bluetape.azurewebsites.net/EntriJadwalDosen}.

		\begin{lstlisting}[frame=single, label={lst:3.3.2_label_masukan_entri_jadwal_dosen}, language=HTML, caption=Kriteria Sukses 3.3.2 - Tidak Terdapat Label pada Kolom Masukan di Halaman Entri Jadwal Dosen]
			<input type="hidden" name="csrf_token" value="3c159eae7bc953dd591b679c080ed066"/>
			Hari
			<select name="hari">
		\end{lstlisting}

		\paragraph{Kriteria Sukses 3.3.3 \textit{Error Suggestion}}
		\label{par:kepatuhan_bluetape_kriteria_sukses_3.3.3}
		(Sukses)\\

		Kriteria ini sukses dipatuhi karena ketika terdapat eror masukan yang terdeteksi otomatis, saran untuk mengoreksi eror tersebut disajikan kepada pengguna.

		\paragraph{Kriteria Sukses 3.3.4 \textit{Error Prevention (Legal, Financial, Data)\\}}
		\label{par:kepatuhan_bluetape_kriteria_sukses_3.3.4}
		(Sukses)\\

		Kriteria ini sukses dipatuhi karena pada halaman yang mengirim tanggapan pengguna, data yang dimasukkan oleh pengguna diperiksa terkait eror masukan dan pengguna dipersilakan untuk mengoreksinya.

		\paragraph{Kriteria Sukses 3.3.5 \textit{Help}}
		\label{par:kepatuhan_bluetape_kriteria_sukses_3.3.5}
		(Tidak Sukses)\\

		Kriteria ini tidak sukses dipatuhi karena pada halaman entri jadwal dosen, setiap elemen masukan tidak memiliki label yang menjelaskan tujuan dari elemen tersebut. Contoh kesalahan dapat dilihat pada potongan kode \ref{lst:3.3.5_label_masukan_entri_jadwal_dosen}. Tautan untuk halaman yang bermasalah dapat dilihat di \url{https://bluetape.azurewebsites.net/EntriJadwalDosen}.

		\begin{lstlisting}[frame=single, label={lst:3.3.5_label_masukan_entri_jadwal_dosen}, language=HTML, caption=Kriteria Sukses 3.3.5 - Tidak Terdapat Label pada Kolom Masukan di Halaman Entri Jadwal Dosen]
			<input type="hidden" name="csrf_token" value="3c159eae7bc953dd591b679c080ed066"/>
			Hari
			<select name="hari">
		\end{lstlisting}
		
		\paragraph{Kriteria Sukses 3.3.6 \textit{Error Prevention (All)}}
		\label{par:kepatuhan_bluetape_kriteria_sukses_3.3.6}
		(Sukses)\\

		Kriteria ini sukses dipatuhi karena pada halaman yang mewajibkan pengguna mengirim informasi, data yang dimasukkan oleh pengguna diperiksa terkait eror masukan dan pengguna dipersilakan untuk mengoreksinya.

		\subsection*{\textit{Robust}}
		\label{subsec:kepatuhan_bluetape_robust}

		\subsubsection*{\textit{Compatible}}
		\label{subsubsec:kepatuhan_bluetape_compatible}

		\paragraph{Kriteria Sukses 4.1.1 \textit{Parsing}}
		\label{par:kepatuhan_bluetape_kriteria_sukses_4.1.1}
		(Tidak Sukses)\\

		Kriteria ini tidak sukses dipatuhi karena terdapat beberapa kesalahan, antara lain:

		\begin{itemize}
			\item Pada halaman cetak transkrip, terdapat penggunaan \textit{tag "time"} yang kurang tepat yaitu tidak terdapat \textit{tag} akhir. Letak kesalahan dapat dilihat pada potongan kode \ref{lst:4.1.1_parsing_halaman_cetak_transkrip}. Tautan untuk halaman yang bermasalah dapat dilihat di \url{https://bluetape.azurewebsites.net/TranskripRequest}.
			\begin{lstlisting}[frame=single, label={lst:4.1.1_parsing_halaman_cetak_transkrip}, language=HTML, caption=Kriteria Sukses 4.1.1 - Kesalahan Elemen pada Halaman Cetak Transkrip]
				<td>
					<time datetime="2019-11-17 17:55:16">
					Minggu, 17 November 2019
				</td>
			\end{lstlisting}
			
			\item Pada halaman manajemen cetak transkrip, terdapat penggunaan atribut \textit{"readonly"} yang kurang tepat yaitu terdapat nilai \textit{"true"} untuk atribut tersebut. Letak kesalahan dapat dilihat pada potongan kode \ref{lst:4.1.1_parsing_halaman_manajemen_cetak_transkrip}. Tautan untuk halaman yang bermasalah dapat dilihat di \url{https://bluetape.azurewebsites.net/TranskripManage}.
			\begin{lstlisting}[frame=single, label={lst:4.1.1_parsing_halaman_manajemen_cetak_transkrip}, language=HTML, caption=Kriteria Sukses 4.1.1 - Kesalahan Elemen pada Halaman Manajemen Cetak Transkrip]
				<label>Email penjawab:
					<input type="text" value="7315020@student.unpar.ac.id" readonly="true"/>
				</label>
			\end{lstlisting}
			
			\item Pada halaman manajemen perubahan kuliah, terdapat penggunaan atribut \textit{"readonly"} yang kurang tepat yaitu terdapat nilai \textit{"true"} untuk atribut tersebut. Letak kesalahan dapat dilihat pada potongan kode \ref{lst:4.1.1_parsing_halaman_manajemen_perubahan_kuliah}. Tautan untuk halaman yang bermasalah dapat dilihat di \url{https://bluetape.azurewebsites.net/PerubahanKuliahManage}.
			\begin{lstlisting}[frame=single, label={lst:4.1.1_parsing_halaman_manajemen_perubahan_kuliah}, language=HTML, caption=Kriteria Sukses 4.1.1 - Kesalahan Elemen pada Halaman Manajemen Perubahan Kuliah]
				<label>Email penjawab:
					<input type="text" value="7315020@student.unpar.ac.id" readonly="true"/>
				</label>
			\end{lstlisting}

			\item Pada halaman entri jadwal dosen, terdapat elemen \textit{"HTML"} yang penempatannya tidak tepat yaitu \textit{tag "div"} berada sebelum \textit{tag "body"}. Letak kesalahan dapat dilihat pada potongan kode \ref{lst:4.1.1_parsing_halaman_entri_jadwal_dosen}. Tautan untuk halaman yang bermasalah dapat dilihat di \url{https://bluetape.azurewebsites.net/EntriJadwalDosen}.
			\begin{lstlisting}[frame=single, label={lst:4.1.1_parsing_halaman_entri_jadwal_dosen}, language=HTML, caption=Kriteria Sukses 4.1.1 - Kesalahan Elemen pada Halaman Entri Jadwal Dosen]
				</head>
				<div class="row"></div>        
				<body>
			\end{lstlisting}
			
			\item Pada halaman lihat jadwal dosen, terdapat \textit{tag "a"} yang memiliki atribut yang tidak tepat yaitu atribut \textit{"a"}. Letak kesalahan dapat dilihat pada potongan kode \ref{lst:4.1.1_parsing_halaman_lihat_jadwal_dosen}. Selain itu, terdapat \textit{tag "div"} buka yang tidak disertai dengan \textit{tag "div"} tutup karena jumlah keduanya tidak sama. Tautan untuk halaman yang bermasalah dapat dilihat di \url{https://bluetape.azurewebsites.net/LihatJadwalDosen}.
			\begin{lstlisting}[frame=single, label={lst:4.1.1_parsing_halaman_lihat_jadwal_dosen}, language=HTML, caption=Kriteria Sukses 4.1.1 - Kesalahan Elemen pada Halaman Lihat Jadwal Dosen]
				<li class="tabs-title">
					<a a href="#hal1">Hizkia Steven</a>
				</li>
				<li class="tabs-title">
					<a a href="#hal2">Hizkia Steven</a>
				</li>
			\end{lstlisting}
		\end{itemize}

		\paragraph{Kriteria Sukses 4.1.2 \textit{Name, Role, Value}}
		\label{par:kepatuhan_bluetape_kriteria_sukses_4.1.2}
		(Tidak Sukses)\\

		Kriteria ini tidak sukses dipatuhi karena pada halaman lihat jadwal dosen terdapat \textit{tag "a"} yang belum memiliki atribut \textit{role} yang tepat, yaitu pada bagian nama dosen yang sedang dipilih untuk ditampilkan jadwalnya. Letak kesalahan dapat dilihat pada potongan kode \ref{lst:4.1.2_role_halaman_lihat_jadwal_dosen}. Tautan untuk halaman yang bermasalah dapat dilihat di \url{https://bluetape.azurewebsites.net/LihatJadwalDosen}.

		\begin{lstlisting}[frame=single, label={lst:4.1.2_role_halaman_lihat_jadwal_dosen}, language=HTML, caption=Kriteria Sukses 4.1.1 - Atribut \textit{Role} pada Halaman Lihat Jadwal Dosen]
			<li class="tabs-title is-active">
				<a href="#hal0" aria-selected="true">HIZKIA STEVEN</a>
			</li> 
		\end{lstlisting}

		\paragraph{Kriteria Sukses 4.1.3 \textit{Status Messages}}
		\label{par:kepatuhan_bluetape_kriteria_sukses_4.1.3}
		(Tidak Sukses)\\

		Kriteria ini tidak sukses dipatuhi karena setiap pesan status yang ditampilkan kepada pengguna tidak memiliki atribute \textit{"role"} dengan nilai \textit{"alert"} sehingga teknologi alat bantu tidak dapat mengidentifikasi serta menginformasikan pesan tersebut kepada pengguna. Letak kesalahan dapat dilihat pada potongan kode \ref{lst:4.1.3_pesan_status}.

		\begin{lstlisting}[frame=single, label={lst:4.1.3_pesan_status}, language=PHP, caption=Kriteria Sukses 4.1.3 - Atribut pada Pesan Status]
			<?php if (isset($_SESSION['error'])): ?>
				<div class="callout alert"><?= $_SESSION['error'] ?></div>
			<?php endif; ?>
			<?php if (isset($_SESSION['info'])): ?>
				<div class="callout primary"><?= $_SESSION['info'] ?></div>
			<?php endif; ?>
		\end{lstlisting}

		\item \textbf{Menulis sebagian dokumen skripsi yaitu bab 1, 2 dan 3.}\\
		{\bf Status :} Ada sejak rencana kerja skripsi.\\
		{\bf Hasil :} Sudah dilakukan. Dokumen skripsi yang sudah dikerjakan yaitu bab 1, bab 2, dan sebagian dari bab 3. Bab 1 berisi latar belakang, rumusan masalah, tujuan, batasan masalah, metodologi, dan sistematika penulisan. Bab 2 berisi dasar teori mengenai \textit{WCAG} 2.1 dan BlueTape. Bab 3 berisi hasil analisis mengenai tingkat kepatuhan BlueTape terhadap \textit{WCAG} 2.1.

		\item \textbf{Memodifikasi situs web BlueTape sehingga level kepatuhan terhadap \textit{WCAG} 2.1 meningkat.}\\
		{\bf Status :} Ada sejak rencana kerja skripsi.\\
		{\bf Hasil :} Belum dilakukan. Bagian ini akan dikerjakan pada Skripsi 2.

		\item \textbf{Melakukan pengujian dan eksperimen pada situs web BlueTape dengan berbagai kondisi keterbatasan seperti yang terdapat dalam \textit{WCAG} 2.1.}\\
		{\bf Status :} Ada sejak rencana kerja skripsi.\\
		{\bf Hasil :} Belum dilakukan. Bagian ini akan dikerjakan pada Skripsi 2.

		\item \textbf{Menulis dokumen skripsi untuk bab 4, 5 dan 6.}\\
		{\bf Status :} Ada sejak rencana kerja skripsi.\\
		{\bf Hasil :} Belum dilakukan. Bagian ini akan dikerjakan pada Skripsi 2.

	\end{enumerate}

\section{Pencapaian Rencana Kerja}
Langkah-langkah kerja yang berhasil diselesaikan dalam Skripsi 1 ini adalah sebagai berikut:
\begin{enumerate}
	\item Mempelajari situs web BlueTape saat ini
	\item Melakukan studi literatur mengenai \textit{WCAG} 2.1
	\item Mengukur tingkat kepatuhan situs web BlueTape terhadap \textit{WCAG} 2.1
	\item Menulis dokumen skripsi yaitu bab 1, bab 2, dan sebagian dari bab 3
\end{enumerate}



\section{Kendala yang Dihadapi}
%TULISKAN BAGIAN INI JIKA DOKUMEN ANDA TIPE A ATAU C
Kendala - kendala yang dihadapi selama mengerjakan skripsi :
\begin{itemize}
	\item Rasa malas mengerjakan skripsi
	\item Terdapat beberapa poin dalam \textit{WCAG} 2.1 yang agak sulit untuk dipahami
	
\end{itemize}

\vspace{1cm}
\centering Bandung, \tanggal\\
\vspace{2cm} \nama \\ 
\vspace{1cm}

Menyetujui, \\
\ifdefstring{\jumpemb}{2}{
\vspace{1.5cm}
\begin{centering} Menyetujui,\\ \end{centering} \vspace{0.75cm}
\begin{minipage}[b]{0.45\linewidth}
% \centering Bandung, \makebox[0.5cm]{\hrulefill}/\makebox[0.5cm]{\hrulefill}/2013 \\
\vspace{2cm} Nama: \pembA \\ Pembimbing Utama
\end{minipage} \hspace{0.5cm}
\begin{minipage}[b]{0.45\linewidth}
% \centering Bandung, \makebox[0.5cm]{\hrulefill}/\makebox[0.5cm]{\hrulefill}/2013\\
\vspace{2cm} Nama: \pembB \\ Pembimbing Pendamping
\end{minipage}
\vspace{0.5cm}
}{
% \centering Bandung, \makebox[0.5cm]{\hrulefill}/\makebox[0.5cm]{\hrulefill}/2013\\
\vspace{2cm} Nama: \pembA \\ Pembimbing Tunggal
}
\end{document}

