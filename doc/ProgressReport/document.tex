\documentclass[a4paper,twoside]{article}
\usepackage[T1]{fontenc}
\usepackage[bahasa]{babel}
\usepackage{graphicx}
\usepackage{graphics}
\usepackage{float}
\usepackage[cm]{fullpage}
\pagestyle{myheadings}
\usepackage{etoolbox}
\usepackage{setspace} 
\usepackage{lipsum} 
\setlength{\headsep}{30pt}
\usepackage[inner=2cm,outer=2.5cm,top=2.5cm,bottom=2cm]{geometry} %margin
% \pagestyle{empty}
\usepackage{listings}%untuk penulisan source code
%listing khusus untuk penulisan kode program, menggunakan font Bera Mono
\lstset{numbers=left,stepnumber=1, numbersep=5pt, frame=leftline,
	tabsize=4, breaklines=true, basicstyle=\fontfamily{fvm}\selectfont\tiny, 
	commentstyle=\itshape\color{gray}, keywordstyle=\bfseries\color{blue}, 
	identifierstyle=\color{black}, stringstyle=\color{orange},
	literate={-}{-}1{-\,-}{--}1
}
\graphicspath{{./Gambar/}}% folder tempat gambar 

\makeatletter
\renewcommand{\@maketitle} {\begin{center} {\LARGE \textbf{ \textsc{\@title}} \par} \bigskip {\large \textbf{\textsc{\@author}} }\end{center} }
\renewcommand{\thispagestyle}[1]{}
\markright{\textbf{\textsc{Laporan Perkembangan Pengerjaan Skripsi\textemdash Sem. Genap 2015/2016}}}

\onehalfspacing
 
\begin{document}

\title{\@judultopik}
\author{\nama \textendash \@npm} 

%ISILAH DATA BERIKUT INI:
\newcommand{\nama}{Hizkia Steven}
\newcommand{\@npm}{2015730020}
\newcommand{\tanggal}{22/11/2019} %Tanggal pembuatan dokumen
\newcommand{\@judultopik}{Kepatuhan dan Rekomendasi Perbaikan Web Content Accessibility Guideline untuk Aplikasi BlueTape} % Judul/topik anda
\newcommand{\kodetopik}{PAN4702}
\newcommand{\jumpemb}{1} % Jumlah pembimbing, 1 atau 2
\newcommand{\pembA}{Pascal Alfadian Nugroho}
\newcommand{\pembB}{-}
\newcommand{\semesterPertama}{47 - Ganjil 19/20} % semester pertama kali topik diambil, angka 1 dimulai dari sem Ganjil 96/97
\newcommand{\lamaSkripsi}{1} % Jumlah semester untuk mengerjakan skripsi s.d. dokumen ini dibuat
\newcommand{\kulPertama}{Skripsi 1} % Kuliah dimana topik ini diambil pertama kali
\newcommand{\tipePR}{B} % tipe progress report :
% A : dokumen pendukung untuk pengambilan ke-2 di Skripsi 1
% B : dokumen untuk reviewer pada presentasi dan review Skripsi 1
% C : dokumen pendukung untuk pengambilan ke-2 di Skripsi 2

% Dokumen hasil template ini harus dicetak bolak-balik !!!!

\maketitle

\pagenumbering{arabic}

\section{Data Skripsi} %TIDAK PERLU MENGUBAH BAGIAN INI !!!
Pembimbing utama/tunggal: {\bf \pembA}\\
Pembimbing pendamping: {\bf \pembB}\\
Kode Topik : {\bf \kodetopik}\\
Topik ini sudah dikerjakan selama : {\bf \lamaSkripsi} semester\\
Pengambilan pertama kali topik ini pada : Semester {\bf \semesterPertama} \\
Pengambilan pertama kali topik ini di kuliah : {\bf \kulPertama} \\
Tipe Laporan : {\bf \tipePR} -
\ifdefstring{\tipePR}{A}{
			Dokumen pendukung untuk {\BF pengambilan ke-2 di Skripsi 1} }
		{
		\ifdefstring{\tipePR}{B} {
				Dokumen untuk reviewer pada presentasi dan {\bf review Skripsi 1}}
			{	Dokumen pendukung untuk {\bf pengambilan ke-2 di Skripsi 2}}
		}
		
\section{Latar Belakang}
BlueTape merupakan aplikasi berbasis web yang dibuat untuk memudahkan berbagai urusan administrasi di Fakultas Teknologi Informasi dan Sains Universitas Katolik Parahyangan. Konsep aplikasi ini yaitu membuat urusan-urusan administrasi dapat dikerjakan melalui situs web sehingga mengurangi penggunaan kertas. Aplikasi ini disediakan untuk digunakan oleh mahasiswa, staf tata usaha, dan dosen. Fitur-fitur yang tersedia pada BlueTape yaitu manajemen cetak transkrip dan manajemen perubahan jadwal kuliah.

\textit{Web Content Accessibility Guidelines (WCAG)} adalah panduan yang berisi rekomendasi-rekomendasi untuk membuat konten web lebih mudah diakses dan digunakan oleh orang-orang, termasuk mereka yang memiliki keterbatasan. Keterbatasan yang tercakup dalam panduan ini yaitu keterbatasan visual, keterbatasan pendengaran, keterbatasan gerak, keterbatasan berbicara dan berbahasa, keterbatasan belajar, fotosensitif, keterbatasan kognitif, dan kombinasi dari beberapa keterbatasan yang telah disebutkan. Dalam \textit{WCAG} terdapat 3 tingkat kriteria sukses yaitu A, AA, dan AAA. Kriteria sukses adalah pernyataan-pernyataan yang dapat diuji yang dijadikan acuan untuk menilai tingkat kepatuhan sebuah situs web terhadap \textit{WCAG}. Kepatuhan tingkat A adalah tingkat kepatuhan terendah yang diperoleh jika seluruh kriteria sukses tingkat A terpenuhi atau versi alternatif yang sesuai tersedia. Kepatuhan tingkat AA adalah tingkat kepatuhan yang diperoleh jika seluruh kriteria sukses tingkat A dan tingkat AA terpenuhi atau versi alternatif tingkat AA yang sesuai tersedia. Kepatuhan tingkat AAA adalah tingkat kepatuhan tertinggi yang diperoleh jika seluruh kriteria sukses tingkat A, tingkat AA, dan tingkat AAA terpenuhi atau versi alternatif tingkat AAA yang sesuai tersedia.

Pada skripsi ini, akan dilihat sejauh mana tingkat kepatuhan situs web BlueTape terhadap \textit{WCAG} 2.1 dan rekomendasi apa saja yang perlu dilakukan untuk menaikkan tingkat kepatuhannya. Selain itu, akan dilakukan pengujian pada situs web tersebut dengan beberapa kondisi keterbatasan yang terdapat dalam \textit{WCAG} 2.1 seperti keterbatasan visual, keterbatasan gerak, keterbatasan pendengaran, dan keterbatasan bahasa.

\section{Rumusan Masalah}
\begin{itemize}
	\item Bagaimana tingkat kepatuhan situs web BlueTape terhadap \textit{WCAG} 2.1?
	\item Bagaimana meningkatkan tingkat kepatuhan situs web BlueTape terhadap \textit{WCAG} 2.1?  
	\item Bagaimana pengalaman menggunakan situs web BlueTape yang telah diperbarui dengan berbagai kondisi keterbatasan seperti yang terdapat dalam \textit{WCAG} 2.1?
\end{itemize}

\section{Tujuan}
\begin{itemize}
	\item Mendapatkan tingkat kepatuhan situs web BlueTape terhadap \textit{WCAG} 2.1.
	\item Meningkatkan tingkat kepatuhan situs web BlueTape terhadap \textit{WCAG} 2.1.
	\item Mendapatkan pengalaman menggunakan situs web BlueTape yang telah diperbarui dengan berbagai kondisi keterbatasan seperti yang terdapat dalam \textit{WCAG} 2.1.
\end{itemize}

\section{Detail Perkembangan Pengerjaan Skripsi}
Detail bagian pekerjaan skripsi sesuai dengan rencana kerja/laporan perkembangan terkahir :
	\begin{enumerate}
		\item \textbf{Mempelajari situs web BlueTape saat ini.}\\
		{\bf Status :} Ada sejak rencana kerja skripsi.\\
		{\bf Hasil :} Hasil dari mempelajari situs web BlueTape, antara lain:
		\begin{itemize}
			\item Penulis berhasil membuat aplikasi BlueTape dapat berjalan di komputer lokal milik penulis. 
			\item Penulis mengetahui halaman dan fitur apa saja yang terdapat pada aplikasi BlueTape.
			\item Penulis sudah melakukan uji coba berbagai kasus untuk fitur-fitur yang terdapat pada aplikasi BlueTape dengan berperan sebagai \textit{root} yang memiliki hak akses tak terbatas. 
		\end{itemize}
		
		\item \textbf{Melakukan studi literatur mengenai \textit{WCAG} 2.1.}\\
		{\bf Status :} Ada sejak rencana kerja skripsi.\\
		{\bf Hasil :}

		\item \textbf{Mengukur tingkat kepatuhan situs web BlueTape terhadap \textit{WCAG} 2.1.}\\
		{\bf Status :} Ada sejak rencana kerja skripsi.\\
		{\bf Hasil :}

		\item \textbf{Menulis sebagian dokumen skripsi yaitu bab 1, 2, dan 3}\\
		{\bf Status :} Ada sejak rencana kerja skripsi.\\
		{\bf Hasil :} Dokumen skripsi sudah dikerjakan hingga bab 3. Bab 1 berisi dengan latar belakang, rumusan masalah, tujuan, batasan masalah, metodologi, dan sistematika penulisan. Bab 2 berisi dasar teori mengenai \textit{WCAG} 2.1 dan BlueTape. Bab 3 berisi hasil analisis mengenai tingkat kepatuhan BlueTape terhadap \textit{WCAG} 2.1.

	\end{enumerate}

\section{Pencapaian Rencana Kerja}
Langkah-langkah kerja yang berhasil diselesaikan dalam Skripsi 1 ini adalah sebagai berikut:
\begin{enumerate}
	\item Mempelajari situs web BlueTape saat ini
	\item Melakukan studi literatur mengenai \textit{WCAG} 2.1
	\item Mengukur tingkat kepatuhan situs web BlueTape terhadap \textit{WCAG} 2.1
	\item Menulis sebagian dokumen skripsi yaitu bab 1, 2, dan 3
\end{enumerate}



\section{Kendala yang Dihadapi}
%TULISKAN BAGIAN INI JIKA DOKUMEN ANDA TIPE A ATAU C
Kendala - kendala yang dihadapi selama mengerjakan skripsi :
\begin{itemize}
	\item Rasa malas mengerjakan skripsi
	\item Terdapat beberapa poin dalam \textit{WCAG} 2.1 yang agak sulit untuk dipahami
	
\end{itemize}

\vspace{1cm}
\centering Bandung, \tanggal\\
\vspace{2cm} \nama \\ 
\vspace{1cm}

Menyetujui, \\
\ifdefstring{\jumpemb}{2}{
\vspace{1.5cm}
\begin{centering} Menyetujui,\\ \end{centering} \vspace{0.75cm}
\begin{minipage}[b]{0.45\linewidth}
% \centering Bandung, \makebox[0.5cm]{\hrulefill}/\makebox[0.5cm]{\hrulefill}/2013 \\
\vspace{2cm} Nama: \pembA \\ Pembimbing Utama
\end{minipage} \hspace{0.5cm}
\begin{minipage}[b]{0.45\linewidth}
% \centering Bandung, \makebox[0.5cm]{\hrulefill}/\makebox[0.5cm]{\hrulefill}/2013\\
\vspace{2cm} Nama: \pembB \\ Pembimbing Pendamping
\end{minipage}
\vspace{0.5cm}
}{
% \centering Bandung, \makebox[0.5cm]{\hrulefill}/\makebox[0.5cm]{\hrulefill}/2013\\
\vspace{2cm} Nama: \pembA \\ Pembimbing Tunggal
}
\end{document}

