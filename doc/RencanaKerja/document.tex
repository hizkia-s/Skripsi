\documentclass[a4paper,twoside]{article}
\usepackage[T1]{fontenc}
\usepackage[bahasa]{babel}
\usepackage{graphicx}
\usepackage{graphics}
\usepackage{float}
\usepackage[cm]{fullpage}
\pagestyle{myheadings}
\usepackage{etoolbox}
\usepackage{setspace} 
\usepackage{lipsum} 
\setlength{\headsep}{30pt}
\usepackage[inner=2cm,outer=2.5cm,top=2.5cm,bottom=2cm]{geometry} %margin
% \pagestyle{empty}

\makeatletter
\renewcommand{\@maketitle} {\begin{center} {\LARGE \textbf{ \textsc{\@title}} \par} \bigskip {\large \textbf{\textsc{\@author}} }\end{center} }
\renewcommand{\thispagestyle}[1]{}
\markright{\textbf{\textsc{AIF401/AIF402 \textemdash Rencana Kerja Skripsi \textemdash Sem. Ganjil 2019/2020}}}

\newcommand{\HRule}{\rule{\linewidth}{0.4mm}}
\renewcommand{\baselinestretch}{1}
\setlength{\parindent}{0 pt}
\setlength{\parskip}{6 pt}

\onehalfspacing
 
\begin{document}

\title{\@judultopik}
\author{\nama \textendash \@npm} 

%tulis nama dan NPM anda di sini:
\newcommand{\nama}{Hizkia Steven}
\newcommand{\@npm}{2015730020}
\newcommand{\@judultopik}{Kepatuhan dan Rekomendasi Perbaikan Web Content Accessibility Guideline untuk Aplikasi BlueTape} % Judul/topik anda
\newcommand{\jumpemb}{1} % Jumlah pembimbing, 1 atau 2
\newcommand{\tanggal}{01/01/1900}

% Dokumen hasil template ini harus dicetak bolak-balik !!!!

\maketitle

\pagenumbering{arabic}

\section{Deskripsi}
BlueTape merupakan aplikasi berbasis web yang dibuat untuk memudahkan berbagai urusan administrasi di Fakultas Teknologi Informasi dan Sains Universitas Katolik Parahyangan. Konsep aplikasi ini yaitu membuat urusan-urusan administrasi dapat dikerjakan melalui situs web sehingga mengurangi penggunaan kertas. Aplikasi ini disediakan untuk digunakan oleh mahasiswa, staf tata usaha, dan dosen. Fitur-fitur yang tersedia pada BlueTape yaitu manajemen cetak transkrip dan manajemen perubahan jadwal kuliah.

Aplikasi BlueTape yang berbasis web mungkin saja digunakan oleh orang-orang yang memiliki keterbatasan. Oleh karena itu perlu dipastikan bahwa aplikasi ini dapat digunakan dengan mudah oleh setiap subjek yang mengaksesnya. W3C (World Wide Web Consortium) memiliki rekomendasi bernama \textit{WCAG} 2.1 (\textit{Web Content Accessibility Guidelines}) yang merupakan panduan untuk membuat konten web yang dapat diakses dengan mudah oleh semua orang termasuk mereka yang memiliki keterbatasan.

Pada skripsi ini, akan dilihat sejauh mana tingkat kepatuhan situs web BlueTape terhadap \textit{WCAG} 2.1 dan rekomendasi apa saja yang perlu dilakukan untuk menaikkan tingkat kepatuhannya. Selain itu, akan dilakukan pengujian pada situs web tersebut dengan beberapa kondisi keterbatasan yang terdapat dalam \textit{WCAG} 2.1 seperti keterbatasan visual, keterbatasan gerak, keterbatasan pendengaran, dan keterbatasan bahasa.

\section{Rumusan Masalah}
\begin{itemize}
	\item Bagaimana tingkat kepatuhan situs web BlueTape terhadap \textit{WCAG} 2.1?
	\item Rekomendasi apa saja (dalam bentuk perubahan kode) yang perlu dilakukan terhadap situs web BlueTape untuk menaikkan tingkat kepatuhannya?  
	\item Bagaimana pengalaman menggunakan situs web BlueTape dengan berbagai kondisi keterbatasan seperti yang terdapat dalam \textit{WCAG} 2.1?
\end{itemize}

\section{Tujuan}
\begin{itemize}
	\item Mendapatkan tingkat kepatuhan situs web Bluetape terhadap \textit{WCAG} 2.1.
	\item Meningkatkan level kepatuhan situs web BlueTape terhadap \textit{WCAG} 2.1.
	\item Mendapatkan pengalaman menggunakan situs web BlueTape dengan berbagai kondisi keterbatasan seperti yang terdapat dalam \textit{WCAG} 2.1.
\end{itemize}

\section{Deskripsi Perangkat Lunak}
Aplikasi akhir yang akan dibuat memiliki fitur minimal sebagai berikut:
\begin{itemize}
	\item Memiliki fitur yang sama atau lebih dari aplikasi yang sudah ada saat ini.
	\item Memiliki tingkat kepatuhan terhadap \textit{WCAG} 2.1 yang lebih tinggi dari aplikasi yang sudah ada saat ini.
\end{itemize}

\section{Detail Pengerjaan Skripsi}
Bagian-bagian pekerjaan skripsi ini adalah sebagai berikut :
\begin{enumerate}
	\item Mempelajari situs web BlueTape saat ini
	\item Melakukan studi literatur mengenai \textit{WCAG} 2.1
	\item Mengukur tingkat kepatuhan situs web BlueTape terhadap \textit{WCAG} 2.1
	\item Memodifikasi situs web BlueTape sehingga level kepatuhan terhadap \textit{WCAG} 2.1 meningkat
	\item Melakukan pengujian dan eksperimen pada situs web BlueTape dengan berbagai kondisi keterbatasan seperti yang terdapat dalam \textit{WCAG} 2.1
	\item Menulis dokumen skripsi
\end{enumerate}

\section{Rencana Kerja}
Rincian capaian yang direncanakan di Skripsi 1 adalah sebagai berikut:
\begin{enumerate}
\item Mempelajari situs web BlueTape saat ini
\item Melakukan studi literatur mengenai \textit{WCAG} 2.1
\item Mengukur tingkat kepatuhan situs web BlueTape terhadap \textit{WCAG} 2.1
\item Menulis sebagian dokumen skripsi yaitu bab 1, 2 dan 3
\end{enumerate}

Sedangkan yang akan diselesaikan di Skripsi 2 adalah sebagai berikut:
\begin{enumerate}
\item Memodifikasi situs web BlueTape sehingga level kepatuhan terhadap \textit{WCAG} 2.1 meningkat
\item Melakukan pengujian dan eksperimen pada situs web BlueTape dengan berbagai kondisi keterbatasan seperti yang terdapat dalam \textit{WCAG} 2.1
\item Menulis dokumen skripsi untuk bab 4, 5 dan 6
\end{enumerate}

\vspace{1cm}
\centering Bandung, \tanggal\\
\vspace{2cm} \nama \\ 
\vspace{1cm}

Menyetujui, \\
\ifdefstring{\jumpemb}{2}{
\vspace{1.5cm}
\begin{centering} Menyetujui,\\ \end{centering} \vspace{0.75cm}
\begin{minipage}[b]{0.45\linewidth}
% \centering Bandung, \makebox[0.5cm]{\hrulefill}/\makebox[0.5cm]{\hrulefill}/2013 \\
\vspace{2cm} Nama: \makebox[3cm]{\hrulefill}\\ Pembimbing Utama
\end{minipage} \hspace{0.5cm}
\begin{minipage}[b]{0.45\linewidth}
% \centering Bandung, \makebox[0.5cm]{\hrulefill}/\makebox[0.5cm]{\hrulefill}/2013\\
\vspace{2cm} Nama: \makebox[3cm]{\hrulefill}\\ Pembimbing Pendamping
\end{minipage}
\vspace{0.5cm}
}{
% \centering Bandung, \makebox[0.5cm]{\hrulefill}/\makebox[0.5cm]{\hrulefill}/2013\\
\vspace{2cm} Nama: \makebox[3cm]{\hrulefill}\\ Pembimbing Tunggal
}
\end{document}

